\subsection{Elliptic Curves over Local Fields and Reduction Types}

Now assume that $K$ is a local field of characteristic $0$ with a discrete valuation $\nu$, ring of integers $R$ and residue field $\kappa$. We denote by $a\mapsto \tilde{a}$ for the natural quotient map $K\to\kappa$. We say that \eqref{eqn_gen_elliptic} is a \textbf{minimal Weierstrass equation} if $a_2,a_4,a_6\in R$ and $\nu(\Delta)$ is minimal among all such equations. When this is the case, we have a well-defined associated curve $\tilde{E}$ over $\kappa$ defined by the equation $y^2=x^3+\tilde{a_2}x^2+\tilde{a_4}x+\tilde{a_6}$ and the associated \textbf{reduction map}
\begin{equation}\label{eqn_reduction}
    \widetilde{(\cdot)}:E(K)\longrightarrow \tilde{E}(\kappa),
\end{equation}

obtained by reducing the coordinates of a point $P\in E(K)$ modulo $\kappa$. One needs to have some care defining the reduction map. For a detailed construction, see \cite[\S1 VII.2]{S1}. We remark that $\tilde{E}$ may be a singular curve, and the \textbf{reduction type} of $E$ over $K$ describes the behaviour of $\tilde{E}$ as a curve over $\kappa$.

\begin{defn}
    Let $E/K$ and $\tilde{E}/\kappa$ be as above. Then we say that 
    \begin{enumerate}[label={(\alph*)}]
        \item $E/K$ has good (or stable) reduction if $\tilde{E}$ is non-singular.
        \item $E/K$ has multiplicative (or semistable) reduction if $\tilde{E}$ has a node.
        \item $E/K$ has additive (or unstable) reduction if $\tilde{E}$ has a cusp.
    \end{enumerate}
    In cases (b) and (c) we say that $E/K$ has bad reduction. Moreover, if $E/K$ has multiplicative reduction, we say that the reduction is split if the slopes of the tangent lines at the node are in $K$, and non-split otherwise.
\end{defn}

By Proposition \ref*{prop_nodecusp} we immediately see that $E/K$ has good reduction if $\nu(\Delta)=0$ and bad otherwise. In that case, $E/K$ has multiplicative reduction if $\nu(c_4)=0$ and additive otherwise.

An important question which will be of interest for us is to understand how the reduction type of an elliptic curve $E$ changes over a finite field extension $F/K$. The following proposition gathers this information.

\begin{prop}[Semistable Reduction Theorem]\label{prop_semi_red}
    Let $E$ be an elliptic curve over a local field $K$ of characteristic $0$. 
    \begin{enumerate}[label={(\roman*)}]
        \item Let $F/K$ be an unramified extension. Then the reduction type of $E$ over $K$ (good multiplicative or additive) is the same as the reduction type of $E$ over $F$.
        \item Let $F/K$ be a finite extension. If $E$ has good or multiplicative reduction over $K$, then it has the same reduction type over $F$. This also applies specfically to split multiplicative reduction.
        \item If $E$ has non-split multiplicative reduction over $K$ and $F/K$ is a finite extension with even residual degree, then $E$ has split multiplicative reduction over $F$. 
        \item There exists a finite extension $F/K$ such that $E$ has either good or spit multiplicative over $F$.
    \end{enumerate}
\end{prop}
\begin{proof}
    \cite[\S VII Proposition 5.4]{S1} 
\end{proof}

Given an unstable elliptic curve $E$ over $K$, we say that it has potentially good (resp. multiplicative) reduction if it has good (resp. multiplicative) reduction over a finite field extension of $K$. This behaviour is chacterized by the value of the $j$-invariant.

\begin{prop}\label{prop_j_inv}
    Let $E$ be an elliptic curve of characteristic $0$. Then
    \begin{itemize}
        \item If $\nu(j)\geq0$, then $E$ has good or potentially good reduction.
        \item If $\nu(j)<0$, then $E$ has multiplicative or potentially multiplicative reduction.
    \end{itemize}
\end{prop}
