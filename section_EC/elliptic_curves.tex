Having discussed the relevant aspects of representation theory that we will require, we now introduce elliptic curves, our main object to study. Our discussion will be rather informal and brief, and will avoid most proofs. We will spend some time discussing the reduction type of elliptic curves and how this can change over finite extensions. Nevertheless, we assume good familiarity with elliptic curves. There is great material available for elliptic curves, and \cite{S1} gives a complete discussion.

\begin{defn}
An \textbf{elliptic curve} $E$ over a field $K$ is a genus one smooth projective curve with a specified $K$-rational point. Any such curve can be written as the locus on $\PP^2$ of a \textbf{Weierstrass equation} (written in non-homogeneous coordinates)
\begin{equation}\label{eqn_gen_elliptic}
    E: y^2+a_1xy+a_3y=x^3+a_2x^2+a_4x+a_6,\quad a_i\in K,
\end{equation}
together with the specified $K$-rational point $[0:1:0]$ at infinity.
\end{defn}
Given a curve $E$ defined by the Weierstrass equation \eqref{eqn_gen_elliptic}, define the following
\begin{table}[H]
	\centering\vspace{-1em}
	\begin{tabular}{l  l  l  l}
		$b_2$ &  $= a_1^2 + 4 a_2,$ & $b_4$ & $=2 a_4 + a_1 a_3,$\\
		$b_6$ & $=a_3^2 + 4 a_6,$ & $b_8$ & $= a_1^2 a_6 + 4 a_2 a_6 - a_1 a_3 a_4 + a_2 a_3^2 - a_4^2.$\\
	\end{tabular}
    \vspace{-1em}
\end{table}
\noindent Then one can define the following invariants associated to $E:$
$$c_4 =b_2^2 - 24 b_4, \qquad \Delta = -b_2^2 b_8 - 8 b_4^3 - 27b_6^2 + 9b_2b_4b_6, \qquad $$
where $\Delta$ is known as the \textbf{discriminant} associated to the Weierstrass equation \eqref{eqn_gen_elliptic}. The following proposition characterizes the behaviour of $E$:
%When $\Char(K)\neq2$, which will always be the case, we can further simplify the equation to
%\begin{equation}\label{eqn_elliptic}
 %   E: y^2=f(x)=x^3+a_2x^2+a_4x+a_6,\quad a_2,a_4,a_6\in K,
%\end{equation}
%so unless stated otherwise we will assume that $E$ has an equation of this form. We remark that if $\Char(K)=2$, then \eqref{eqn_elliptic} always defines a singular curve. Associated to this equation there are constants 
%$$c_4=16(a_2^2-3a_4) \quad\text{and}\quad \Delta=16(a_2^2a_4^2-4a_2^3a_6-4a_4^3-27a_6^2+18a_2a_4a_6)$$
%and differential 
%\begin{equation}\label{eqn_differential}
%    w=\frac{dx}{2y}=\frac{dy}{3x^2+2a_2x+a_4}.
%\end{equation}
%One can also define these constants for the general Wieirstrass equation \eqref{eqn_gen_elliptic}, but we omit the description. A complete description is given in \cite[\S III.1]{S1}
%The curve defined by \eqref{eqn_elliptic} is singular if and only if the polynomial $f(x)$ has repeated roots over $\bar{K}$. If it has a double and a simple root, then we say it has a node; if it has a triple root, then it has a cusp. The following proposition characterizes this behaviour in terms of $c_4$ and $\Delta$.

\begin{prop}\label{prop_nodecusp}
    Let $E$ be a curve given by the Weierstrass equation \eqref{eqn_gen_elliptic}. Then
    \begin{enumerate}
        \setlength\itemsep{0em}
        \item $E$ is non-singular if and only if $\Delta\neq0$.
        \item $E$ has a nodal-singularity if and only if $\Delta=0$ and $c_4 \neq 0$.
        \item $E$ has a cuspidal-singularity if and only if $\Delta= c_4 = 0$. 
    \end{enumerate}
\end{prop}
\begin{proof}
    \cite[\S \text{III} Proposition 1.4]{S1}. See \cite[\S \text{III}.1]{S1} for the definition of a nodal and cuspidal singularity.
\end{proof}
Thus when $\Delta\neq0$ the equation defines an elliptic curve. %A fundamental property is that the set of $K$-rational points of an elliptic curve forms an abelian group, denoted by $(E(K),\oplus)$ (\cite[\S III.2]{S1}). When $K$ is a number field, the Mordell-Weil theorem shows that this group is also finitely generated, and therefore 
%$$E(K)=E(K)_{\tors}\times\ZZ^r,$$
%where $E(K)_{\tors}$ is a finite abelian group and $r$ is denoted the rank of $E$.
To an elliptic curve $E / K$ one can associate the \textbf{invariant differential} 
$$ \omega = \frac{dx}{2y + a_1 x + a_3} \in \Omega_E,$$
which has neither zeros nor poles, and is named the invariant differential due to its invariance under translation (\cite[\S\text{III}.5, Proposition 5.1]{S1}). Such a differential is unique up to multiplication by a scalar.

\subsection{Elliptic Curves over Local Fields and Reduction Types}

Now assume that $K$ is a non-Archimedean local field of characteristic $0$ with a discrete valuation $\nu$, ring of integers $R$, uniformizer $\pi$ and residue field $\kappa = R / (\pi)$. We denote by $a\mapsto \tilde{a}$ for the natural quotient map $K\to\kappa$.

We say that \eqref{eqn_gen_elliptic} is a \textbf{minimal Weierstrass equation} if $a_i \in R$ and $\nu(\Delta)$ is minimal among all such equations. Note that $a_i \in R$ implies $\nu(\Delta) \geq 0$, and since $\nu$ is discrete it follows that a minimal Weierstrass equation exists. When this is the case, we call $\Delta$ a \textbf{minimal discriminant} for $E$. Then we have a well-defined associated curve $\tilde{E}$ over $\kappa$ defined by \eqref{eqn_gen_elliptic} with the $a_i$ replaced by $\tilde{a_i}$, as well as an associated \textbf{reduction map}
\begin{equation}\label{eqn_reduction}
    \widetilde{(\cdot)}:E(K)\longrightarrow \tilde{E}(\kappa),
\end{equation}
obtained by reducing the coordinates of a point $P\in E(K)$ modulo $\pi$. One needs to take some care defining the reduction map. For a detailed construction, see \cite[\S1 VII.2]{S1}. 

We remark that $\tilde{E}$ may be a singular curve, and the \textbf{reduction type} of $E$ over $K$ describes the behaviour of $\tilde{E}$ as a curve over $\kappa$.

\begin{defn}
    Let $E/K$ and $\tilde{E}/\kappa$ be as above. Let $\Delta$ be the minimal discriminant for $E$. Then we say that 
    \begin{enumerate}[label={(\alph*)}]
        \setlength\itemsep{0em}
        \item $E/K$ has \textbf{good} (or \textbf{stable}) \textbf{reduction} if $\tilde{E}$ is non-singular (equivalently $\nu(\Delta) = 0$), 
        \item $E/K$ has \textbf{multiplicative} (or \textbf{semistable}) \textbf{reduction} if $\tilde{E}$ has a node (equivalently $\nu(\Delta) >0$ and $\nu(c_4) = 0$),
        \item $E/K$ has \textbf{additive} (or \textbf{unstable}) \textbf{reduction} if $\tilde{E}$ has a cusp (equivalently $\nu(\Delta) >0$ and $\nu(c_4) > 0$).
    \end{enumerate}
    In cases (b) and (c) we say that $E/K$ has \textbf{bad reduction}. Moreover, if $E/K$ has multiplicative reduction, we say that the reduction is \textbf{split} if the slopes of the tangent lines at the node are in $\kappa$, and \textbf{non-split} otherwise.
\end{defn}

%By Proposition \ref*{prop_nodecusp} we immediately see that $E/K$ has good reduction if $\nu(\Delta)=0$ and bad otherwise. In that case, $E/K$ has multiplicative reduction if $\nu(c_4)=0$ and additive otherwise.
An important question which will be of interest for us is to understand how the reduction type of an elliptic curve $E$ changes over a finite field extension $F/K$. The following proposition gathers this information.

\begin{prop}[Semistable reduction theorem]
    Let $E$ be an elliptic curve over a non-Archimedean local field $K$ of characteristic $0$. Let $F / K$ be a finite extension, with ramification degree $e_{F / K}$ and residue degree $f_{F / K}$.
    \begin{enumerate}[label={(\roman*)}]
        \setlength\itemsep{0em}
         \item If $e_{F / K} = 1$, then the reduction type of $E$ over $K$ (good, multiplicative or additive) is the same as the reduction type of $E$ over $F$. 
         \item If $E$ has good or split multiplicative reduction over $K$, then it has the same reduction type over $F$.
        \item If $E$ has non-split multiplicative reduction over $K$ and $f_{F / K}$ is even, then $E$ has split multiplicative reduction over $F$. 
        \item There exists a finite extension $F'/K$ such that $E$ has either good or split multiplicative over $F'$.
    \end{enumerate}
\end{prop}
\begin{proof}
    \cite[\S VII Proposition 5.4]{S1} 
\end{proof}

The last property of the above proposition tells us that if $E / K$ has additive reduction, then there is some finite extension $F' / K$ over which $E$ attains good/multiplicative reduction. 

\begin{defn}
    Let $E / K$ have additive reduction. 
    \begin{enumerate}
        \setlength\itemsep{0em}
        \item $E / K$ has \textbf{potentially good reduction} if there exists a finite extension $F ' / K$ where $E / F'$ has good reduction,
        \item Otherwise, $E / K$ has \textbf{potentially multiplicative reduction}.
    \end{enumerate}
\end{defn}

Given an elliptic curve $E / K$, one can use Tate's algorithm $(\cite[\S \text{IV}.9]{S2})$ to determine the reduction type of $E$. The reduction types are encoded by \textbf{Kodaira symbols}. These are described in the following table, assuming that the characteristic of $k$ is not equal to $2$ or $3$ when $E / K$ has additive reduction.
\begin{table}[H]
    \centering
    \begin{tabular}{ c | l }
        $\I_0:$ & $E / K$ has good reduction, \\ 
        $\I_n:$ & $E / K$ has multiplicative reduction, $\nu(\Delta) = n$,\\
        $\I_n^* :$ & $E / K$ has potentially multiplicative reduction, $\nu(\Delta) = 6 + n$, \\
        $\I_0^*:$ & $E / K$ has potentially good reduction, $\nu(\Delta) = 6$, \\
        $\II$, $\III$, $\IV:$ & $E / K$ has potentially good reduction, $\nu(\Delta) = 2$, $3$, $4$, respectively,\\
        $\IV^*$, $\III^*$, $\II^*:$ & $E / K$ has potentially good reduction, $\nu(\Delta) = 8$, $9$, $10$ respectively. 
    \end{tabular}
    %\caption{Kodaira symbols for $E / K$, where $\Delta$ is the minimal discriminant, and $\text{char}(k) \not= 2, 3$ when $E / K$ has additive reduction.}
\end{table}


\subsection{Tamagawa Numbers} \label{subs_tamagawa}

Recall from the previous section that if $E/K$ has bad reduction, then $\tilde{E}$ is not a smooth curve and therefore its $\kappa$-rational points may not form a group. However, the set $\tilde{E}_{ns}(\kappa)$ of non-singular points of $\tilde{E}(\kappa)$ does indeed form a group. The reduction map \eqref{eqn_reduction} is in general not surjective, but it does surject onto $\tilde{E}_{ns}(\kappa)$. It is natural therefore to define $E_0(K)=\{P\in E(K):\widetilde{P}\in\tilde{E}_{ns}(\kappa)\}$, which is also a subgroup of $E(K)$. Importantly, the resulting reduction map 
$$\widetilde{(\cdot)}:E_0(K)\longrightarrow \tilde{E}_{ns}(\kappa)$$
is a surjective homomorphism of abelian groups.
\begin{defn}
    The \textbf{Tamagawa number} of $E/K$ is defined as
    \begin{equation}
        c(E/K):=|E(K)/E_0(K)|.
    \end{equation}
\end{defn}
In later sections we will be concerned with computing Tamagawa numbers. Note that if $E/K$ has good reduction, then $E_0(K)=E(K)$ and therefore $c(E/K)=1$. %However, when $E/K$ has bad reduction, this is a hard question to answer in general. Fortunately, this question can always be resolved using Tate's Algorithm (see \cite[\S IV.9]{S2}), and for semistable reduction, Tamagawa numbers have a simple explicit description.

Tate's algorithm can also be used to obtain the Tamagawa number of $E / K$. For semistable reduction, one has the following description:

\begin{prop}
    Let $E/K$ have multiplicative reduction, and let $n=\nu(\Delta)$ be the valuation of the minimal discriminant. Then
    \begin{align*}
        c(E/K)=
        \begin{cases}
            n \quad\text{if $E/K$ has split reduction,}\\
            1 \quad\text{if $n$ is odd and $E/K$ is non-split,}\\
            2 \quad\text{if $n$ is even and $E/K$ is non-split}.
        \end{cases}    
    \end{align*}
\end{prop}

In the case of additive reduction, if we assume that the residue characteristic is $\geq 5$, then one can determine the Tamagawa number from the shortened Weierstrass equation of $E$.

\begin{lemma}\label{tamagawa-num}
    Let $F /K / \bQ_p$ be finite extensions and $p \geq 5$. Let
    $$E \colon  y^2 = x^3 + Ax + B, \qquad A, B \in K$$
    be an elliptic curve over $K$ with additive reduction. One has $\Delta = -16(4A^3 + 27 B^2)$. Let $\delta=v_K(\Delta)$ be the valuation of the minimal discriminant, and $e$ the ramification index of $F/K$.
    If $E$ has potentially good reduction, then 
        \[
        \begin{array}{l l l l}
            \gcd(\delta e, 12) = 2 & \implies & c(E / F) = 1, & \quad (\II, \II^*) \\
            \gcd(\delta e, 12) = 3 & \implies & c(E / F) = 2, & \quad (\III, \III^*) \\
            \gcd(\delta e, 12) = 4 & \implies & c(E / F) = \begin{cases} 1, & \sqrt{B} \notin F
                                \\ 3, & \sqrt{B} \in F \end{cases}, & \quad (\IV, \IV^*) \\
            \gcd(\delta e, 12) = 6 & \implies & c(E / F) = \begin{cases} 2, & \sqrt{\Delta} \notin F
                \\ 1 \ \text{or} \ 4, & \sqrt{\Delta} \in F \end{cases}, & \quad (\I_0^*) 
        \end{array}
        \]
    If $E$ has potentially multiplicative reduction of type $\I_n^*$ over $K$,
    and $e$ is even, then it attains multiplicative reduction over $F$ of type $\I_{en}$. If $e$ is odd the reduction type remains potentially multiplicative of type $\I_{en}^*$. Moreover, 
    \[
        \begin{array}{l l l l}
        2 \nmid e, 2 \nmid n & \implies & c(E / F) = \begin{cases} 2, & \sqrt{B} \not\in F, \\ 4, & \sqrt{B} \in F. \end{cases} & \quad (\I_{en}^*) \\

        2 \nmid e, 2 \mid n & \implies & c(E / F) = \begin{cases} 2 & \sqrt{\Delta} \not\in F, \\ 4 & \sqrt{\Delta} \in F \end{cases} & \quad ({\I_{en}^*}) \\

        2 \mid e, \sqrt{-6 B} \not\in F & \implies & c(E / F ) = 2 & \quad (\I_{en}, \text{ non-split}) \\
        2 \mid e, \sqrt{-6 B} \in F & \implies & c(E / F) = en & \quad (\I_{en}, \text{ split})
        \end{array} 
    \]
\end{lemma}

\begin{proof}
    \cite[Lemma 3.22]{reg-const}
\end{proof}

\subsection{Elliptic Curves over Global Fields}

The topics we have discussed so far, such as the reduction type of an elliptic curve and the Tamagawa number, are intrinsically local objects. We now briefly discuss how we can associate these objects to global fields. For simplicity, assume that $E$ is an elliptic curve over a number field $K$, let $\pp$ be a finite place of $K$ and denote $K_\pp$ by the completion of $K$ at $\pp$ with residue field $\kappa_\pp$. Clearly, we have that $E(K)\subseteq E(K_\pp)$ and therefore we can apply the previous description to the curve $E/K_\pp$.

In particular, the reduction type of $E/K$ at $\pp$ is the reduction type of $E/K_\pp$ and the Tamagawa number of $E/K$ at $\pp$ is defined as 
$$c_\pp(E/K):=c(E/K_\pp),$$
and we also define 
$$c(E/K):=\prod_\pp c_\pp(E/K).$$
Finally, we say that a Weierstrass equation \eqref{eqn_gen_elliptic} is a \textbf{global minimal equation} if it is a minimal equation for all finite places $\pp$ of $K$. Even though such an equation does not always exists for any $K$, it does hold for $\QQ$.

\begin{prop}\label{prop_globmin}
    Let $E/\QQ$ be an elliptic curve. Then $E$ has a global minimal Weierstrass equation.
\end{prop}

Throughout the document, we will work with elliptic curves over $\QQ$, so unless stated otherwise we will assume the defining equation is global minimal.