
\subsection{Tamagawa Numbers} \label{subs_tamagawa}

Recall from the previous section that if $E/K$ has bad reduction, then $\tilde{E}$ is not a smooth curve and therefore its $\kappa$-rational points may not form a group. However, the set $\tilde{E}_{ns}(\kappa)$ of non-singular points of $\tilde{E}(\kappa)$ does indeed form a group. The reduction map is in general not surjective, but it does surject onto $\tilde{E}_{ns}(\kappa)$. It is natural therefore to define $E_0(K)=\{P\in E(K):\widetilde{P}\in\tilde{E}_{ns}(\kappa)\}$, which is also a subgroup of $E(K)$. Importantly, the resulting reduction map 
$$\widetilde{(\cdot)}:E_0(K)\longrightarrow \tilde{E}_{ns}(\kappa)$$
is a surjective homomorphism of abelian groups.
\begin{defn}
    The \textbf{Tamagawa number} of $E/K$ is defined as
    \begin{equation*}
        c(E/K):=|E(K)/E_0(K)|.
    \end{equation*}
\end{defn}
In later sections we will be concerned in computing Tamagawa numbers. Note that if $E/K$ has good reduction, then $E_0(K)=E(K)$ and therefore $c(E/K)=1$. However, when $E/K$ has bad reduction, this is a hard question to answer in general. Fortunately, this question can always be resolved using Tate's Algorithm (see \cite[\S IV.9]{S2}), and for multiplicative reduction, Tamagawa numbers have a simple explicit description.

\begin{lemma}\label{lem_mult_tam}
    Let $E/K$ have multiplicative reduction, and let $n=\nu(\Delta)$ be the valuation of the minimal discriminant. Then
    \begin{align*}
        c(E/K)=
        \begin{cases}
            n \quad\text{if $E/K$ has split reduction,}\\
            1 \quad\text{if $n$ is odd and $E/K$ is non-split,}\\
            2 \quad\text{if $n$ is even and $E/K$ is non-split}.
        \end{cases}    
    \end{align*}
\end{lemma}

In the case of additive reduction, if we assume that the residue characteristic is $\geq 5$, then one can determine the Tamagawa number from the shortened Weierstrass equation of $E$. The following result is a consequence of Tate's algorithm.

\begin{lemma}\label{lem_add_tam}\label{tamagawa-num}\cite[Lemma 3.22]{reg-const}
    Let $F /K / \bQ_p$ be finite extensions and $p \geq 5$. Let
    $$E \colon  y^2 = x^3 + Ax + B, \qquad A, B \in K$$
    be an elliptic curve over $K$ with additive reduction. One has $\Delta = -16(4A^3 + 27 B^2)$. Let $\delta=v_K(\Delta)$ be the valuation of the minimal discriminant, and $e$ the ramification index of $F/K$.
    If $E$ has potentially good reduction, then 
        \[
        \begin{array}{l l l l}
            \gcd(\delta e, 12) = 2 & \implies & c(E / F) = 1, & \quad (\II, \II^*) \\
            \gcd(\delta e, 12) = 3 & \implies & c(E / F) = 2, & \quad (\III, \III^*) \\
            \gcd(\delta e, 12) = 4 & \implies & c(E / F) = \begin{cases} 1, & \sqrt{B} \notin F
                                \\ 3, & \sqrt{B} \in F \end{cases}, & \quad (\IV, \IV^*) \\
            \gcd(\delta e, 12) = 6 & \implies & c(E / F) = \begin{cases} 2, & \sqrt{\Delta} \notin F
                \\ 1 \ \text{or} \ 4, & \sqrt{\Delta} \in F \end{cases}, & \quad (\I_0^*) 
        \end{array}
        \]
     
        If $E$ has potentially multiplicative reduction of type $\I_n^*$ over $K$,
        and $e$ is even, then it attains multiplicative reduction over $F$ of type $\I_{en}$. If $e$ is odd the reduction type remains potentially multiplicative of type $\I_{en}^*$. Moreover, 
        \[
            \begin{array}{l l l l}
                2 \nmid e, 2 \nmid n & \implies & c(E / F) = \begin{cases} 2, & \sqrt{B} \not\in F, \\ 4, & \sqrt{B} \in F. \end{cases} & \quad (\I_{en}^*) \\
                2 \nmid e, 2 \mid n & \implies & c(E / F) = \begin{cases} 2 & \sqrt{\Delta} \not\in F, \\ 4 & \sqrt{\Delta} \in F \end{cases} & \quad ({\I_{en}^*}) \\
                2 \mid e, \sqrt{-6 B} \not\in F & \implies & c(E / F ) = 2 & \quad (\I_{en}, \text{ non-split}) \\
                2 \mid e, \sqrt{-6 B} \in F & \implies & c(E / F) = en & \quad (\I_{en}, \text{ split})
                \end{array} 
            \]
\end{lemma}
