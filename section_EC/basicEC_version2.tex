
Having discussed the relevant aspects of representation theory that we will require, we now introduce elliptic curves, our main object to study. Our discussion will be rather informal and brief, and will avoid most proofs. We will spend some time discussing the reduction type of elliptic curves and how this can change over finite extensions. Nevertheless, we assume good familiarity with elliptic curves. There is great material available for elliptic curves, and \cite{S1} gives a complete discussion.

An elliptic curve $E$ over a field $K$ is a genus one smooth projective curve with a specified $K$-rational point. Any such curve can be written as the locus on $\PP^2$ of a \textbf{Weierstrass equation}
\begin{equation}\label{eqn_gen_elliptic}
    E: y^2+a_1xy+a_3y=x^3+a_2x^2+a_4x+a_6,\quad a_2,a_4,a_6\in K,
\end{equation}
together with the specified $K$-rational point $[0:1:0]$ at infinity.

Associated to this equation, there are constants 
$$c_4=\phi(a_1,a_2,a_3,a_4,a_6)\quad\text{and}\quad \Delta=\psi(a_1,a_2,a_3,a_4,a_6)$$
and differential
$$\omega=\frac{dx}{2y+a_1x+a_3}=\frac{dy}{3x^2+2a_2x+a_4-a_1y},$$
where $\phi$ and $\psi$ are explicit polynomials with coefficients over $K$. The explicit description is lenghty and we omit it. The relevant proposition is the following one.

\begin{prop}\label{prop_nodecusp}
    The curve given by a Weierstrass equation satisfies:
    \begin{enumerate}
        \item It is nonsingular if and only if $\Delta\neq0$.
        \item It has a node (singular point with two tangent directions) if and only if $\Delta=0$ and $c_4 \neq 0$.
        \item It has a cusp (singular point with one tangent direction) if and only if $\Delta= c_4 = 0$. 
    \end{enumerate}
\end{prop}
For a complete discussion and proof of the above proposition, see \cite[\S III.1]{S1}.

When $\Delta\neq0$, the equation defines an elliptic curve. A fundamental property is that the set of $K$-rational points of an elliptic curve forms an abelian group, denoted by $(E(K),\oplus)$ (\cite[\S III.2]{S1}). When $K$ is a number field, the Mordell-Weil theorem shows that this group is also finitely generated, and therefore 
$$E(K)=E(K)_{\tors}\times\ZZ^r,$$
where $E(K)_{\tors}$ is a finite abelian group and $r$ is denoted the rank of $E$.
