
In this preliminary section, we introduce elliptic curves, our main object of study throughout this document. Our discussion will be rather informal and brief, and will avoid most proofs. Therefore, even though we introduce most notions that will be relevant to us, we assume significant familiarity with the topic. For the unfamiliar reader, there is great material available for elliptic curves, such as \cite{S1}, where a complete discussion is given. Whenever material is readily available, we give the references. Nevertheless, we will spend some time discussing reduction types of elliptic curves and some relevant results that are less well-known and harder to find in the literature. 

%Associated to this equation, there are constants
%$$c_4=\phi(a_1,a_2,a_3,a_4,a_6),\quad\text{discriminant } \Delta=\psi(a_1,a_2,a_3,a_4,a_6)\quad \text{and $j$-invariant }j=c_4^3/\Delta,$$
%where $\phi$ and $\psi$ are explicit polynomials with coefficients over $K$. 
%Furthermore, there is an associated differential $$\omega=\frac{dx}{2y+a_1x+a_3}=\frac{dy}{3x^2+2a_2x+a_4-a_1y}.$$
%The explicit description of these polynomials is lengthy and we omit it, but we refer to the following result. For a complete discussion and proof this proposition, see \cite[\S III.1]{S1}.

%\begin{prop}\label{prop_nodecusp}
%    The curve given by a Weierstrass equation satisfies:
%    \begin{enumerate}
%        \item It is nonsingular if and only if $\Delta\neq0$.
%        \item It has a node (singular point with two tangent directions) if and only if $\Delta=0$ and $c_4 \neq 0$.
%        \item It has a cusp (singular point with one tangent direction) if and only if $\Delta= c_4 = 0$. 
%    \end{enumerate}
%\end{prop}

%Hence, when $\Delta\neq0$, the equation defines an elliptic curve. A fundamental property is that the set of $K$-rational points of an elliptic curve forms an abelian group, denoted by $(E(K),\oplus)$ (\cite[\S III.2]{S1}). When $K$ is a number field, the Mordell-Weil theorem shows that this group is also finitely generated, and therefore 
%$$E(K)=E(K)_{\tors}\times\ZZ^r,$$
%where $E(K)_{\tors}$ is a finite abelian group and $r$ is denoted the rank of $E$.
