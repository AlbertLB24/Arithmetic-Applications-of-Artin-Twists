\documentclass{beamer}
\usepackage{amsmath}
\usepackage{xcolor}
\usepackage{multimedia}
\usepackage{bbm}
\usetheme{Copenhagen}
\definecolor{purple}{rgb}{0.3,0,0.4}
\definecolor{aqua}{rgb}{0,0.85,0.8}
\definecolor{grey}{rgb}{60,60,60}
\setbeamercolor*{palette primary}{use=structure, fg=white, bg=purple}
\setbeamercolor*{background canvas}{bg=purple!5}
\setbeamercolor*{block title example}{use=structure,fg=white,bg=purple}
\setbeamercolor*{block body example}{fg=black,use=block title,bg=purple!20}
\setbeamercolor*{block title}{use=example text,fg=white,bg=aqua!60!black}
\setbeamercolor*{block body}{fg=black,use=block title example,bg=aqua!10}

\usepackage{graphicx}
\usepackage{tikz-cd}

\newcommand{\Gal}{\mathrm{Gal}}
\newcommand{\Cl}{\mathrm{Cl}}
\newcommand{\BSD}{\mathrm{BSD}}
\newcommand{\tors}{\mathrm{tors}}
\newcommand{\Reg}{\mathrm{Reg}}
\newcommand{\rk}{\mathrm{rk}}
\newcommand{\trivial}{\mathbbm{1}}
\newcommand{\Ind}{\mathrm{Ind}}
\newcommand{\Rep}{\mathrm{Rep}}
\newcommand{\Tr}{\mathrm{Tr}}
\newcommand{\Stab}{\mathrm{Stab}} 
\newcommand{\Res}{\mathrm{Res}}

\newcommand{\CC}{\mathbb{C}}
\newcommand{\FF}{\mathbb{F}}
\newcommand{\NN}{\mathbb{N}}
\newcommand{\PP}{\mathfrak{P}}
\newcommand{\QQ}{\mathbb{Q}}
\newcommand{\RR}{\mathbb{R}}
\newcommand{\ZZ}{\mathbb{Z}}
\newcommand{\GG}{\mathbb{G}}
\newcommand{\adele}{\mathbb{A}}
\newcommand{\pp}{\mathfrak{p}}
\newcommand{\qq}{\mathfrak{q}}
\newcommand{\rr}{\mathfrak{r}}
\newcommand{\af}{\mathfrak{a}}
\newcommand{\fg}{\mathfrak{g}}
\newcommand{\bQ}{\mathbb{Q}}
\newcommand{\bC}{\mathbb{C}}
\newcommand{\bZ}{\mathbb{Z}}
\newcommand{\fP}{\mathfrak{P}}
\newcommand{\fp}{\mathfrak{p}}
\newcommand{\fw}{\mathfrak{w}}
\newcommand{\fN}{\mathfrak{N}}
\newcommand{\fq}{\mathfrak{q}}
\newcommand{\cO}{\mathcal{O}}
\newcommand{\repnorm}[1]{\fN_{\bQ(#1) / \bQ}(#1)}

%Sha:
\usepackage[OT2,T1]{fontenc}
\DeclareSymbolFont{cyrletters}{OT2}{wncyr}{m}{n}
\DeclareMathSymbol{\Sha}{\mathalpha}{cyrletters}{"58}
%end of Sha

\theoremstyle{plain}
\newtheorem{thm}{Theorem}[section]
\newtheorem{rem}[thm]{Remark}
\newtheorem{proposition}[thm]{Proposition}
\newtheorem{conjecture}[thm]{Conjecture}
\newtheorem{test}[thm]{Test}


\graphicspath{ }


\title[Artin Formalism]{A Study of the Arithmetic Consequences of Artin Formalism for
Predicting Positive Rank}
\author{Edwina Aylward, Albert Lopez Bruch}
%\institute{LSGNT}
\date{21 May, 2024}

\begin{document}

%THINGS TO SAY AT THE BEGINNING. E IS ALWAYS AN ELLIPTIC CURVE. ALL EXTENSIONS ARE ASSUMED TO BE FINITE. K, F, L ARE FIELDS


\frame{\titlepage}

\begin{frame}
    \frametitle{Plan for today}
    \tableofcontents
    \end{frame}

%\begin{section}{Norm relations test}

\begin{frame}
    \frametitle{BSD term}
    We discard some of the BSD terms for $E / K$
    and define  

    \begin{definition}
        $$\BSD(E / K) = \frac{\Reg_{E / K}|\Sha_{E / K}| C_{E / K}}{|E(K)_{\tors}|^2 }$$
    \end{definition}
    \pause

    Suppose $\rk E / K = 0$. Then $\Reg_{E / K} = 1$. Assuming finiteness of $\Sha_{E / K}$, Cassels proved that $|\Sha_{E / K}|$ is a square. \pause Thus one has
    $$ \BSD(E / K) \equiv C_{E / K} \mod (\QQ^{\times})^2. $$
\end{frame}
%\end{section}

\begin{frame}
    \frametitle{Artin formalism for BSD}
    Consider $E / \QQ$, and $F / \QQ$ with $G = \Gal(F / \QQ)$.  \pause
    As in the $L$-function case, we would like to be able to break up $\BSD(E / F)$ according to representations of $G$. \pause
    
    \begin{conjecture}[BSD-term conjecture]
        For each representation $\rho$ of $G$, there exists an invariant $\BSD(E, \rho) \in \CC^{\times}$ such that:\pause
        \begin{enumerate}
            \item$\BSD(E / F^H) = \BSD(E, \Ind_H^G \trivial)$ for $H \leq G$, \pause
            \item $\BSD(E, \rho \oplus \tau) = \BSD(E, \rho)\BSD(E , \tau)$, where $\tau \in \Rep(G)$.\pause
        \end{enumerate}
    \end{conjecture}

    Therefore, if $\Ind_H^G \trivial \simeq \trivial \oplus \bigoplus_i \rho_i$, one has
    $$ \BSD(E / F^H) = \BSD(E / \QQ)\prod_i \BSD(E, \rho_i). $$ 
\end{frame}

\begin{frame}
    \frametitle{Artin formalism for BSD}
This conjecture is studied in \cite{DEW1}. They show it is true assuming some well-known conjectures. \pause They also conjecture (and justify) an additional useful property: 

\begin{conjecture}[BSD-term conjecture]
    For each representation $\rho$ of $G$, there exists an invariant $\BSD(E, \rho) \in \CC^{\times}$, satisfying the previous properties, \pause such that if $\langle \rho, E(F)_{\CC} \rangle = 0$, then
    \begin{enumerate}\setcounter{enumi}{2}
        \item Galois equivariance: $\BSD(E, \rho) \in \QQ(\rho)^{\times}$ and $$\BSD(E, \rho^{\mathfrak{g}}) = \BSD(E, \rho)^{\mathfrak{g}} \text{ for } \mathfrak{g} \in \Gal(\bQ(\rho) / \bQ).$$ 
    \end{enumerate}

\end{conjecture}
\end{frame}

%\begin{frame}
%    \frametitle{Artin formalism for BSD}
%    \begin{conjecture}
%        For each representation $\rho$ of $G$, there exists an invariant $\BSD(E, \rho) \in \CC^{\times}$ (as before)  such that if $\langle \rho, E(F)_{\CC} \rangle = 0$, then
%        \begin{enumerate}\setcounter{enumi}{2}
%            \item Galois equivariance: $\BSD(E, \rho) \in \QQ(\rho)^{\times}$ and $$\BSD(E, \rho^{\mathfrak{g}}) = \BSD(E, \rho)^{\mathfrak{g}} \text{ for } \mathfrak{g} \in \Gal(\bQ(\rho) / \bQ).$$ 
%        \end{enumerate}
    
%    \end{conjecture}
%    \pause
%    Here 
%    \begin{itemize}
%        \item $E(F)_{\bC} = E(F) \otimes_{\bZ} \bC$ is viewed as a $G$-representation, \pause
%        \item $\langle -, - \rangle$ is the usual inner product of characters of $G$, \pause
%       \item $\bQ(\rho)$ is an abelian extension of $\bQ$ obtained by adding $\{ \Tr \rho(\fg) \colon \fg \in G \}$ to $\bQ$, \pause
%        \item $\rho^{\fg}$ is the representation of $G$ with character $\Tr \rho^{\fg} = \fg \circ \Tr \rho $.
%    \end{itemize}

%\end{frame}

\begin{frame}
    \frametitle{Norm relations test}
    Again, consider $E/ \bQ$, $F / \bQ$ and $G = \Gal(F / \bQ)$. \pause Take a representation $\rho$ of $G$, and assume $\langle \rho, E(F)_{\bC} \rangle = 0$. Then 
    $$ \repnorm{\rho} := \bigoplus_{\fg \in \Gal(\bQ(\rho) / \bQ)} \rho^{\fg}$$
    has rational character. \pause 
    The BSD-term conjecture implies that
    $$\BSD(E, \bigoplus_{\fg \in \Gal(\bQ(\rho) / \bQ)} \rho^{\fg}) = \prod_{\fg \in \Gal(\bQ(\rho) / \bQ)} \BSD(E, \rho)^{\fg}$$\pause
    is the norm of an element from $\bQ(\rho)^{\times}$. 
\end{frame}

\begin{frame}
    \frametitle{Norm relations test}
    There exists some $m \geq 1$ such that $$\bigg(\bigoplus_{\fg \in \Gal(\bQ(\rho) / \bQ)} \rho^{\fg}\bigg)^{\oplus m}$$
    can be written in terms of permutation representations, i.e. \pause
    $$  \bigg(\bigoplus_{\fg \in \Gal(\bQ(\rho) / \bQ)} \rho^{\fg}\bigg)^{\oplus m} \simeq \bigoplus_{i} \Ind_{H_i}^G \trivial \ominus \bigoplus_{j} \Ind_{H_j'}^G \trivial$$
    for some $H_i$, $H_j' \leq G$. 
\end{frame}

\begin{frame}
    \frametitle{Norm relations test}
    $$\bigg(\bigoplus_{\fg \in \Gal(\bQ(\rho) / \bQ)} \rho^{\fg}\bigg)^{\oplus m} \simeq \bigoplus_{i} \Ind_{H_i}^G \trivial \ominus \bigoplus_{j} \Ind_{H_j'}^G \trivial$$ \pause
    On the level of BSD terms we get
    $$\bigg(\prod_{\fg \in \Gal(\bQ(\rho) / \bQ)} \BSD(E, \rho)^{\fg}\bigg)^m = \frac{\prod_i \BSD(E / F^{H_i})}{\prod_j \BSD(E / F^{H_j'})} ,$$
    \pause so the left hand side is the norm of an element of $\bQ(\rho)$. \pause It is also the norm of an element from any subfield of $\bQ(\rho)$. 
\end{frame}

\begin{frame}
    \frametitle{Norm relations test}
    Now assume $\rk E / F = 0$. Then for all $H \leq G$, 
    $$ \BSD(E / F^{H}) \equiv C_{E / F^H} \mod (\bQ)^{\times 2}.$$ \pause
    Squares are norms from quadratic fields. \pause Thus if $\bQ(\sqrt{D}) \subset \bQ(\rho)$, then 
   $$\frac{\prod_i \BSD(E / F^{H_i})}{\prod_j \BSD(E / F^{H_j'})} \in N_{\bQ(\sqrt{D}) / \bQ}(\bQ(\sqrt{D})^{\times})$$ \pause $$\implies 
   \frac{\prod_i C_{E / F^{H_i}}}{\prod_j C_{E / F^{H_j'}}} \in N_{\bQ(\sqrt{D}) / \bQ}(\bQ(\sqrt{D}^{\times})). $$\pause
   \textbf{so if this product isn't a norm...\pause our rank assumption was wrong!}
\end{frame}

\begin{frame}
    \frametitle{Norm relations test}
    %We have deduced the following:
    \begin{theorem}[{Norm relations test, \cite[Theorem 33]{DEW1}}]\label{thm_positive_rank}
    \small{  Suppose our conjecture for the BSD terms holds. Consider $E/\QQ$, and $F/\QQ$  with $G = \Gal(F / \bQ)$. Let $\rho$ be a representation of $G$, and let $\bQ(\sqrt{D}) \subset \bQ(\rho)$. Suppose we have  
        $$\left(\bigoplus_{\mathfrak{g}\in\Gal(\QQ(\rho)/\QQ)}\rho^{\mathfrak{g}}\right)^{\oplus m}=\bigoplus_i\Ind_{H_i}^G\mathbbm{1}\ominus\bigoplus_j\Ind_{H'_j}^G\mathbbm{1}$$
        for some $m\geq 1$ and $H_i, H_j' \leq G$. If 
        $$\frac{\prod_i C_{E/F^{H_i}}}{\prod_j C_{E/F^{H_j'}}} \not\in
        \begin{cases}
            (\bQ)^{\times 2} & m \text{ even,}\\
            N_{\bQ(\sqrt{D}) / \bQ}(\bQ(\sqrt{D})^{\times}) & m \text{ odd,}
        \end{cases}
        $$
        then $\rk E / F > 0.$ }
    \end{theorem}
\end{frame}

\begin{frame}{Parity conjecture and parity tests}
    A more common method of forcing positive rank for an elliptic curve $E / K$ is by use of the parity conjecture:
    \begin{conjecture}[Parity conjecture]\label{parity}
            The \textit{global root number} $w(E / K) \in \{ \pm 1 \}$ satisfies 
            $$(-1)^{\rk E / K} = w(E / K)$$
    \end{conjecture} \pause
    
    Thus one computes $w(E / K)$ and if it is $-1$, then $\rk E / K$ is odd and so $> 0$. 

\end{frame}

\begin{frame}
    \frametitle{Parity tests vs. norm relations tests}

    Originally we hoped to find an example where our norm relations test forced rank growth, and root number computations didn't. \pause But we couldn't. \pause \text{:(} \pause

    \begin{conjecture}
        Let $E / \bQ$, and $F / \bQ$ Galois. Whenever our norm relations test forces $\rk E / F >0$ then there is some intermediate field $\bQ \subset K \subset F$ such that $w(E / K)w(E / \bQ) = -1$.
    \end{conjecture}
\end{frame}


%\begin{section}{Representation theoretic methods}
\begin{frame}
    \frametitle{Brauer relations}
    Let $G$ be a finite group. \pause There is a bijection 
    \[ \{ \text{transitive finite } G \text{-sets}\}/(\text{isom.})\leftrightarrow \{ \text{subgroups } H \leq G \}/(\text{conj.})\] 
    given by sending a transitive finite $G$-set $X$ to $H = \Stab_{G} (x)$ for some $x \in X$. \pause The permutation representation of $G$ acting on $X$ (denoted $\bC[X]$) is isomorphic to $\Ind_{H}^G \trivial  = \bC[G / H]$. 
\end{frame}

\begin{frame}{Brauer relations}
    Let $\{ X_i \colon i \in I\}$, $\{X_j \colon j \in J \}$ be transitive finite $G$-sets. If $G$ is non-cyclic, it can happen that
    \[ \coprod_{i \in I} X_i \not\simeq \coprod_{j \in J} X_j  \]
    as $G$-sets, but \pause 
    \[ \bigoplus_{i \in I}\bC[ X_i] \simeq \bigoplus_{j \in J} \bC[ X_j]\] 
    as representations of $G$. \pause

    \textbf{Slogan:} Non-isomorphic $G$-sets can give rise to isomorphic permutation representations.

\end{frame}

\begin{frame}{Brauer relations}
    This corresponds to  
    \[ \bigoplus_{i \in I} \Ind_{H_i}^G \trivial \ominus \bigoplus_{j \in J} \Ind_{H_j}^G \trivial = 0 \] 
    as a virtual permutation representation (i.e. the character is zero). 

    We call this a \textbf{Brauer relation}.\pause

    \begin{example}
        Consider $G = S_3$. \pause 
        Then 
        \[ \bC[G / C_1] \oplus \bC[G / G]^{\oplus 2} \simeq \bC[G / C_2]^{\oplus 2} \oplus \bC[G / C_3] .\] 
    \end{example}
\end{frame}

\begin{frame}{Brauer relations : Applications}
    
    Brauer relations provide interesting relations. \pause Let $F / K$ be an extension of number fields with $\Gal(F / K) = G$ and consider 
    \[ \bigoplus_{i \in I} \Ind_{H_i}^G \trivial \ominus \bigoplus_{j \in J} \Ind_{H_j}^G \trivial = 0.\] \pause
    \begin{itemize}
        
        \item \only<3>{(Dedekind-zeta functions)
        \[ \prod_i \zeta_{F^{H_i}}(s) = \prod_j \zeta_{F^{H_j}}(s)\]}
        \only<4-6>{ (L-functions) If $E / K$ is an elliptic curve one has 
        \[ \prod_{i \in I} L(E / F^{H_i}, s)= \prod_{j \in J} L(E / F^{H_j}, s). \] \pause\pause
        \item Let $V$ be a representation of $G$. Then 
        \[ \sum_i \dim V^{H_i} = \sum_j \dim V^{H_j}. \] \pause 
        For example when $V = E(F) \otimes_{\bZ} \bC$ then $\dim V^{H} = \rk E / F^H$.} 
    \end{itemize}
\end{frame}


\begin{frame}{Brauer relations and the norm relations test}
    
    Now consider 
    \[ \bigg(\bigoplus_{\mathfrak{g}\in\Gal(\QQ(\rho)/\QQ)}\rho^{\mathfrak{g}}\bigg)^{\oplus m}=\bigoplus_i\Ind_{H_i}^G\mathbbm{1}\ominus\bigoplus_j\Ind_{H'_j}^G\mathbbm{1}\] 
    for $\rho$ a representation of $G$, $H_i, H_j' \leq G$. \pause 
    If we have a Brauer relation
    \[ \bigoplus_{k \in K} \Ind_{F_k}^G \trivial \ominus \big{(}\bigoplus_{l \in L} \Ind_{F_l}^G \trivial\big{)} = 0 \] 
    for $F_k, F_l \leq G$, \pause then 
    \small{\[ \bigg(\bigoplus_{\mathfrak{g}\in\Gal(\QQ(\rho)/\QQ)}\rho^{\mathfrak{g}}\bigg)^{\oplus m}=\bigoplus_i\Ind_{H_i}^G\mathbbm{1}\ominus\bigoplus_j\Ind_{H'_j}^G\mathbbm{1} \oplus \bigoplus_{k \in K} \Ind_{F_k}^G \trivial \ominus \bigoplus_{l \in L} \Ind_{F_l}^G \trivial \]}
\end{frame}

\begin{frame}{Brauer relations and the norm relations test}
\textbf{The point:} A relation  
\[ \bigg(\bigoplus_{\mathfrak{g}\in\Gal(\QQ(\rho)/\QQ)}\rho^{\mathfrak{g}}\bigg)^{\oplus m}=\bigoplus_i\Ind_{H_i}^G\mathbbm{1}\ominus\bigoplus_j\Ind_{H'_j}^G\mathbbm{1}\]
is not unique, and is only defined up to Brauer relations.
\end{frame}

\begin{frame}{Counting primes}
    Let $F / K$ be a finite Galois extension of number fields, with $G = \Gal(F / K)$. \pause Let $\fp$ be a prime of $K$, $\fq$ a prime of $F$ lying over $\fp$. Let $D$, $I \leq G$ be the decomposition group and inertia group of $\fp$. \pause 
    
    \begin{lemma}    
    For $H \leq G$, the primes of $L = F^H$ above $\fp$ are in one-to-one correspondence with the double cosets $H  \backslash G / D$. 
    \end{lemma} \pause

    \vspace{1em}
    Mackey's theorem tells us that 
    \[ \Res_D^G \Ind_H^G = \sum_{x \in H\backslash G / D} \Ind_{D \cap H^{x^{-1}}}^D \trivial, \]
    where $H^x := x H x^{-1}$. \pause Therefore
    \[ \langle \trivial, \Res_D^G \Ind_H^G \trivial \rangle = | H \backslash G / D|.\]
\end{frame}

\begin{frame}{Ramification index and residue degree}
    \[ \Res_D^G \Ind_H^G = \sum_{x \in H\backslash G / D} \Ind_{D \cap H^{x^{-1}}}^D \trivial. \]
    Let a prime $\fP$ of $L$ above $\fp$ correspond to the double coset $D_p x H$. \pause Then 
    \[ e_{\fP \mid \fp} = \frac{|I|}{|H \cap I^{x}|}, \quad  f_{\fP \mid \fp} = \frac{[D : I]}{[H \cap D^{x} : H \cap I^x]}.\] \pause 
    Let $\psi_n \colon D \to D / I \to \bC^{\times}$ be a character mapping Frobenius (which generates $D/ I$) to $\zeta_n$. Then 
    \[ \# (\text{ primes $\fP$ of $L = F^H$ above $\fp$ with $n \mid f_{\fP \mid \fp}$ }) = \langle \psi_n , \Res_D^G \Ind_H^G \trivial\rangle.  \] \pause 
    \textbf{Remark:}
        A representation-theoretic formula for counting the number of primes of given ramification degree does not exist. 
\end{frame}

\begin{frame}{D-local functions}
    Consider $E / K$, and $F / K$ Galois with $G = \Gal(F / K)$. 
    
    As we saw, the reduction types of elliptic curves at primes change in field extensions according to ramification and residue degrees.\pause
\end{frame}


%\end{section}

%\begin{section}{Odd degree extensions}
\begin{frame}
    \frametitle{Odd degree extensions}
    Now consider $E / \bQ$, and $F / \bQ$ Galois with $G = \Gal(F / \bQ)$ of odd order. \pause
    We proved that
    {\begin{thm}\label{odd-exts}
       Assume $E$ has good or multiplicative reduction at $2$ and $3$. 
       Take any representation $\rho$ of $G$ with $\bQ(\sqrt{D}) \subset \bQ(\rho)$ and relation
       \begin{equation*}\label{odd-exp}
        \left(\bigoplus_{\mathfrak{g}\in\Gal(\QQ(\rho)/\QQ)}\rho^{\mathfrak{g}}\right)^{\oplus m }=\bigoplus_i\Ind_{H_i}^G\mathbbm{1}\ominus\bigoplus_j\Ind_{H_j'}^G\mathbbm{1}
       \end{equation*}
         Then
        \[ \frac{\prod_i C_{E/F^{H_i}}}{\prod_j C_{E/F^{H_j'}}}  \in 
           \begin{cases} 
            (\bQ^{\times})^2 & m \ \text{even},\\
               N_{\bQ(\sqrt{D}) / \bQ}(\bQ(\sqrt{D})^{\times}) & m \ \text{odd}.
           \end{cases} \] 
       \end{thm}}
       %\uncover<3>{\vspace{6em} \begin{center}i.e. our Norm relations test never forces $\rk E / F > 0$ when $[F : \bQ]$ is odd. \end{center}}
\end{frame}
%\end{section}

\begin{frame}
    \frametitle{Why would we expect this?}
        Therefore the norm relations test can never predict positive growth in the case of odd degree Galois extensions. \pause We expected this because root number computations never force rank growth in this case either. \pause
        \begin{lemma}\label{lem_oddroot}
            Let $F / \bQ$ be an odd Galois extension with $G = \Gal(F / \bQ)$. Then $w(E / \bQ) = w(E / F^H)$ for all $H \leq G$. 
        \end{lemma}

\end{frame}

\begin{frame}
    \frametitle{Towards proving...}
    Recall that 
    $$ C_{E / K} = \prod_{\fP \subset \cO_K} 
    c_{\fP}(E / K) \cdot d_{\fP}(E / K) = \prod_p C_{\fP \mid p}(E / K)
    $$
    where $C_{\fP \mid p}(E / K) = \prod_{\fP \mid p} c_{\fP}(E / K) \cdot d_{\fP}(E / K, \omega)$.\pause 

    The local term $$\frac{\prod_i C_{\fP \mid p}(E / F^{H_i})}{\prod_j C_{\fP \mid p}(E / F^{H_j'})}$$ 
    depends on the choice of decomposition group $D_{p}$ and inertia group $I_{p}$ of $p$ in $F / \bQ$.
\end{frame}

\begin{frame}{Minimal $m$}
    
\end{frame}

\begin{frame}{Residue degree one}
    
    \begin{lemma}
       Let $E / K$ be an elliptic curve. Let $K' / K$ be an odd degree extension of number fields, unramified at the prime $\fp$ of $K$.\pause Then 
       \[  c_{\fP}(E / K')\cdot d_{\fP}(E / K') \equiv c_{\fp}(E / K)\cdot d_{\fp}(E / K) \mod (\bQ)^{\times 2} \]
       for any prime $\fP$ of $K'$ with $\fP \mid \fp$.
    \end{lemma}\pause

    \begin{corollary}
        We may assume that $D_p = I_p$ when computing 
        $$\frac{\prod_i C_{\fP \mid p}(E / F^{H_i})}{\prod_j C_{\fP \mid p}(E / F^{H_j'})}.$$ 
    \end{corollary}\pause

    \textbf{Upshot:} $D_p = I_p = P_p \ltimes C_l$, $P_P \triangleleft I_p$ is a $p$-group and $l \mid p - 1$.  
\end{frame}

\begin{frame}{Only certain $D_p$}
    \begin{lemma}
        Consider $\bQ(\sqrt{D}) \subset \bQ(\rho)$. Let $f$ be the smallest integer such that $\bQ(\sqrt{D}) \subset \bQ(\zeta_{f})$. \pause 
        Consider $p$ with decomposition group $D_p$ of exponent $b$ ( $=$ lcm of orders of elements of $G$.) \pause 

        If $f \nmid b$, then 
        $$\frac{\prod_i C_{\fP \mid p}(E / F^{H_i})}{\prod_j C_{\fP \mid p}(E / F^{H_j'})} \in (\bQ)^{\times 2}.$$\pause (Regardless of the reduction type of $E$ at $p$!)
    \end{lemma}\pause

    \begin{proof}
        Representation theory. \pause
    \end{proof}
    
    This narrowed down the decomposition groups we had to consider. 
\end{frame}

\begin{frame}{Proof strategy}
    \begin{itemize}
        \item We looked at the possible reduction types of $E / \bQ_p$ and proved in each case that 
    $$\frac{\prod_i C_{\fP \mid p}(E / F^{H_i})}{\prod_j C_{\fP \mid p}(E / F^{H_j'})}$$
    is a norm from all required subfields. \pause 
    \item 
    To calculate these up to squares, we reinterpreted the factors $C_{E / K}$ as representation-theoretic functions. \pause 
    \item
    To determine that the values were norms, we used some class field theory, for example using the genus class fields of quadratic fields to see when a prime is the norm of a principal ideal. 
    \end{itemize}
 \end{frame}

%\begin{section}{Outlook}
\begin{frame}{Outlook : Adding in regulators}
The following is a refinement of our norm relations test:

\begin{proposition}
    Let $G$ be a finite group and $\rho$ a representation of $G$ containing a quadratic subfield $\bQ(\sqrt{D}) \subset \bQ(\rho)$. \pause Suppose the BSD-term conjecture holds for $E / \bQ$ and $F / \bQ$ with $\Gal(F / \bQ) = G$. \pause If    \[ \frac{\prod_i \Reg_{E / F^{H_i}} \cdot C_{E/F^{H_i}}}{\prod_j \Reg_{E / f^{H_j'}} \cdot C_{E/F^{H_j'}}} \not\in
    \begin{cases}
        (\bQ)^{\times 2} & m \text{ even,}\\
        N_{\bQ(\sqrt{D}) / \bQ}(\bQ(\sqrt{D})^{\times}) & m \text{ odd,}
    \end{cases}\]  one has 
    \[ \langle \rho, E(F)_{\bC} \rangle > 0 .\] 
\end{proposition}
\end{frame}

\begin{frame}{Outlook: future work}
\begin{itemize}
    \item We avoided additive reduction at $2$ and $3$ because the terms $c_{\fP}(E / K) \cdot d_{\fP}(E / K)$ are harder to calculate. But to be complete we should check what happens in these cases. \pause 
    \item Explore the proposition on the previous slide and see if it provides some interesting examples / prove some results about it. 
\end{itemize}
\end{frame}

%\begin{frame}{Outlook : Adding in regulators}
%    Regulators are not rational numbers in general. \pause But one has 
%    \begin{lemma}
%        Let $E / K$ be an elliptic curve, $F / K$ a finite field extension of degree $d$. \pause If $\rk E / F = \rk E / K$, then 
%        \[ \Reg_{E / F} = \square \cdot d^{\rk E / K} \Reg_{E / K} . \]\pause
%    \end{lemma} 
%    \textbf{Idea: } Use this property to compute the product of regulators in our relation up to squares. 
%\end{frame}

%\begin{frame}{Outlook : Other future work}
%    \begin{itemize}
%        \item We avoided additive reduction at $2$ and $3$ because the terms $c_{\fP}(E / K) \cdot d_{\fP}(E / K)$ are harder to calculate. \pause 
%        \item There should be an analogous norm relations test for hyperelliptic curves. Perhaps we could find more interesting examples here.
%    \end{itemize}
%\end{frame}
%\end{section} 

\begin{frame}
    \vspace{4em}
    \begin{center}
        \textbf{Thank you!}
    \end{center}
\end{frame}

\begin{frame} 
    \bibliographystyle{amsalpha}
    \bibliography{../references.bib}
\end{frame}


\end{document}