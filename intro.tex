The Birch--Swinnerton-Dyer conjecture has long been exploited to obtain arithmetical information about elliptic curves. Firstly, it relates the rank of an elliptic curve over a number field $F$ to the order of vanishing of its $L$-function $L(E / F, s)$ at $s = 1$. Secondly, and more astonishingly, it relates the leading term of the Taylor series of $L(E / F, s)$ at $s = 1$ to a myriad of quantities associated to $E$. In particular this leading term has the BSD-quotient as a factor:

\[
 \BSD(E / F) = \frac{\Reg_{E/F}|\Sha_{E/F}|C_{E/F}}{|E(F)_{\tors}|^2},
\]
which is described in detail in $\S$\ref{sec-explain-terms}.

Consider a rational elliptic curve $E / \bQ$ and an Artin representation $\rho$ of $G_{\bQ}$ which factors through a finite group $G = \Gal(F / \bQ)$. One can define a \textit{twisted} $L$-function, denoted $L(E, \rho, s)$ ({\color{red} ref section}). There is an analogue for the first part of BSD for twists, which states that 
\[ \ord_{s = 1} L(E, \rho, s) = \langle \rho, E(F)_{\bC} \rangle, \]
where $E(F)_{\bC} = E(F) \otimes_{\bZ} \bC$ is viewed as a representation of $G$ and $\langle \cdot , \cdot \rangle$ is the usual inner-product of characters of $G$. The analogue for the second part of BSD, that is describing the leading term of the Taylor series of $L(E, \rho, s)$ at $s = 1$, is a lot more difficult to specify. Let us call this leading term $\BSD(E, \rho)$.

The authors of \cite{DEW1} put forth a proposed description of $\BSD(E, \rho)$ and study applications of its existence. Subject to some well-known conjectures, they show that this should satisfy the following properties.

\begin{conj*}[ = Conjecture \ref{conj_4}, {\cite[Conjecture 4]{DEW1}})]\label{intro-conj_4}
    Let $\rho$, $\tau$ be Artin representations over $\bQ$ that factor through $G = \Gal(F / \bQ)$, and denote by $\bQ(\rho)$ the abelian extension generated by the values of the character of $\rho$. Then,
\begin{enumerate}
    \setlength\itemsep{0em}
    \item $\BSD(E, \Ind_{K / \bQ} \trivial) = \BSD(E / K)$ for a number field $K$,
    \item $\BSD(E, \rho \oplus \tau ) = \BSD(E, \rho)\BSD(E, \tau)$,
    \item When $\langle \rho, E(F)_{\bC} \rangle = 0$, $\BSD(E, \rho) \in \bQ(\rho)$ and $\BSD(E, \rho^{\fg}) = \BSD(E, \rho)^{\fg}$ for all $\fg \in \Gal(\bQ(\rho) / \bQ)$. 
\end{enumerate}
\end{conj*}

Artin representations over $\bQ$ that factor through $G$ are just representations of $G$. If such a representation $\tau$ has rational character, then $\BSD(E, \tau)$ is determined by the above conjecture. Indeed, there exists\footnote{see Remark \ref{image-of-burnside}} some $m \geq 1$ such that
\begin{equation}\label{intro-reln}
     \tau^{\oplus m} = \bigoplus_i \Ind_{F_i / \bQ} \trivial \ominus \bigoplus_j\Ind_{F_j' / \bQ} \trivial
\end{equation}
for subfields $F_i$, $F_j'$ of $F / \bQ$. Then the first two parts of the conjecture imply that 
\[
\BSD(E, \tau)^{m} = \frac{\prod_i \BSD(E / F_i)}{\prod_j \BSD(E / F_j')} .
\]
We also want to use the third part of the conjecture. Consider the representation $$\tau = \bigoplus_{\fg \in \Gal(\bQ(\rho) / \bQ)} \rho^{\fg},$$
where $\rho$ is an arbitrary representation of $G$. This has rational character, and if we assume that $\langle \rho, E(F)_{\bC} \rangle = 0$, then when \eqref{intro-reln} is satisfied one additionally has that $\prod_i \BSD(E / F_i) / \prod_j \BSD(E / F_j')$ is a norm from the field $\bQ(\rho)$, i.e. an element of $N_{\bQ(\rho) / \bQ}(\bQ(\rho)^{\times})$.
This restriction on the values of the BSD terms of intermediate subfields of $F / \bQ$ has some very interesting consequences that are explored in \cite[\S3]{DEW1}.

In this report, we consider when $\bQ(\rho)$ is quadratic, so that the square terms $|E(F)_{\tors}|^2$ and $|\Sha_{E / F}|$ (assuming finiteness!) of $\BSD(E / F)$ (as well as for intermediate subfields) are norms of elements of $\bQ(\rho)$. Suppose in addition that $\rk E / F = 0$, so that the regulator terms for all intermediate sub-fields vanish. Then the conjecture implies the following:

\begin{thm*}[= Theorem \ref{thm_positive_rank}, 
    {\cite[Theorem 33]{DEW1}}]
    Assume the above conjecture holds. Let $E/\QQ$ be an elliptic curve, $F/\QQ$ a finite Galois extension with Galois group $G$. Let $\rho$ be a representation of $G$ with $\bQ(\rho)$ quadratic and let $\tau = \rho \oplus \sigma(\rho)$, where $\sigma$ generates $\Gal(\bQ(\rho) / \bQ)$. 
    
    Suppose that $\rk E / F = 0$, so that in particular $\langle \rho, E(F)_{\bC} \rangle = 0$.
    Then, for any relation of the form \eqref{intro-reln}, one has that if 
    $$\frac{\prod_i C_{E/F_i}}{\prod_j C_{E/F'_j}} \not\in \begin{cases} N_{\bQ(\rho) / \bQ}(\bQ(\rho)^{\times}) & m \text{ odd}, \\
        \bQ^{\times 2} & m \text{ even}.\end{cases}$$
    then $E$ has a point of infinite order over $F$.
\end{thm*}

{\color{red} wrap it up.}


\subsection*{Structure of report}

{\color{red} brief overview of each section}



\subsection*{Notation}
Let $G$ be a finite group, $K$ a field of characteristic $0$, and $\rho$ a representation of $G$. We use the following notation for characters and representations of $G$:

\bigskip

\begin{tabular}{l | l}
     $\R_{K}(G)$ & the ring generated by the characters of representations of $G$ over $K$,\\
    $\Perm(G)$ & the ring of characters of virtual permutations representations of $G$, \\
    $\Char_{\bQ}(G)$ & the ring of rationally-valued characters of $G$,\\ 
    $\C(G)$ & the finite abelian group $\Char_{\bQ}(G) / \Perm(G)$, \\ 
    $\bQ(\rho)$ & the abelian extension of $\bQ$ generated by the character values of $\rho$, \\
    $\rho^{\fg}$ & the representation of $G$ which acts by $\fg \circ \rho$, where $\fg \in \Gal\left(\bQ(\rho) / \bQ\right)$,\\
    $\repnorm{\rho}$ &  $ = \sum_{\fg \in \Gal(\bQ(\rho) / \bQ)} \rho^{\fg},$ (Definition \ref{rho-norm}),\\
    $\rho^*$ & the dual representation of $\rho$,
    \\
    $\B(G)$ & the Burnside ring of $G$ (Definition \ref{burnside}), \\
\end{tabular}
\vspace{1em}

For a subgroup $H \leq G$ and $x \in F$, let $H^{x} = x H x^{-1}$.  

For an odd prime $p$, let $p^* = p$ if $p \equiv 1 \pmod 4$, and $p^* = -p$ if $p \equiv 3 \pmod 4$. 

$\Ind_{F / K} \trivial  = \Ind_{G_F}^{G_K} \trivial$ for a finite extension of number fields $F / K$?
%Given an elliptic curve $E / \bQ$ and a number field $F$, we define

%\[ C_{E / F} = \prod_v c_v(E / F) \left| \frac{\omega}{\omega_v^{\min}} \right|_v. \]
%The product is taken over the finite places of $F$, $\omega$ is a global minimal differential for $E / \bQ$, and $\omega_v^{\min}$ is a minimal differential at $v$. In terms of minimal discriminants, if $E$ is of the form
%\[ y^2 + a_1 x y + a_3 y = x^3 + a_2 x^2 + a_4 x + a_6 \]
%with minimal discriminant $\Delta_E$ and $\omega = \frac{dx}{2 y + a_1 x + a_3}$, then
%\[ \left| \frac{\omega}{\omega_v^{\min}} \right|_v^{-12} = \left| \frac{\Delta_E}{\Delta_{E, v}^{\min}} \right|_v . \]