\subsection{Representations of finite groups}\label{rep}

Let $G$ be a finite group, $K$ a field of characteristic zero. Recall that a \textbf{representation} of $G$ over $K$ is a group homomorphism $\rho \colon G \to \GL(V)$ where $V$ is a $K$-vector space. Associated to a representation $\rho$ is a \textbf{character} $\chi \colon G \to K^{\times}$, defined by letting $\chi(g) = \Tr \rho(g)$ for $g \in G$. For complex representations, $\rho$ is determined by its character; if $\rho$, $\rho'$ are representations with identical characters, then $\rho$ and $\rho'$ are isomorphic as representations. 

\begin{defn}
Let $\chi_1, \ldots \chi_h$ be the distinct characters of the complex irreducible representations of $G$. 
Then the \textbf{representation ring} of $G$ is 
\[ R(G) = \bZ \chi_1 \oplus \cdots \oplus \bZ \chi_h .\]
%We can also view this as the Grothendieck group of the category of finitely generated $\bC[G]$-modules.
\end{defn}

Since we take differences of characters in $R(G)$, we sometimes call elements of $R(G)$ \textbf{virtual representations}. 

Let $K$ be a number field. Denote by $R_K(G)$ the group generated by characters of the representations of $G$ over $K$. This is a subring of $R(G)$.
When $K = \bQ$ this is called the \textbf{rational representation ring}.
The characters of the distinct irreducible representations of $G$ over $K$ form an orthogonal basis of $R_K(G)$ (\cite[Proposition 32]{Serre}).
Let $m$ be the exponent of $G$. If $K$ contains the $m$-th roots of unity, then $R_K(G) = R(G)$ (\cite[Theorem 24]{Serre}). This implies every representation of $G$ can be realized over such $K$. 
\vspace{1em}

%The rank of $R_K(G)$ is discussed in {\color{red} Serre 12.4}.
Let $\Perm(G)$ be the ring of virtual permutation representations of $G$ (i.e. the ring generated by the characters of $\bC[G / H]$ for $H \leq G$). Let $\Char_{\bQ}(G)$ be the ring of rationally valued characters of $G$. Then we have inclusions 
\[ \Perm(G) \to R_{\bQ}(G) \to \Char_{\bQ}(G). \]
Each of these groups have equal $\bZ$-rank, equal to the number of conjugacy classes of cyclic subgroups of $G$ {\color{red} ref}. Moreover the cokerenels of these maps are finite.

It is of interest to obtain characters of $\Perm(G)$ from characters of $R(G)$.
For $\rho \in R(G)$ one obtains an element of $\Char_{\bQ}(G)$ by taking 
$$ {\widetilde{\rho}} = \sum_{\sigma \in \Gal(\bQ(\rho)/\bQ)}  \rho^\sigma \quad .$$
Here $\bQ(\rho)$ means the smallest Galois field extension that contains the values of $\rho(g)$ for all $g \in G$, and $\rho^\sigma$ is the character such that $\rho^\sigma(g) = \sigma(\rho(g))$. Conversely, if a representation has a rationally valued character, then any complex irreducible constituent must occur along with all its Galois conjugates with equal multiplicity. Therefore our map $R(G) \to \Char_{\bQ}(G)$ is surjective.

Such a character may not be in $R_{\bQ}(G)$, however. That is, it has rational character, but the corresponding representation cannot be realized over $\bQ$. The quotient $\Char_{\bQ}(G) / R_{\bQ}(G)$ is the study of Schur indices.  If $\rho \in R(G)$ is an irreducible representation, the \textbf{Schur index} is the smallest integer $m(\rho)$ such that 
\[ \sum_{\sigma \in \Gal(\bQ(\rho)/\bQ)}m(\rho) \cdot \rho^\sigma \quad \in R_{\bQ}(G). \]

We are more interested in the group $$C(G) = \Char_{\bQ}(G) / \Perm(G).$$ This is a finite abelian group. It has exponent dividing $|G|$ by Artin's induction theorem {\color{red} ref}. The study of this group is quite subtle, see for example \cite{Tim-Alex}. For us, it's enough to know that there exists an integer $m$ dividing $|G|$ such that $\widetilde{\rho}^{\ \oplus m } \in \Perm(G)$, where $m$ is the order of $\widetilde{\rho}$ in $C(G)$. Thus, we have a map $\Irr(G) \to \Perm(G)$. We extend this additively to a map $R(G) \to \Perm(G)$.

%\begin{notn}
%For $\rho \in R_{\bC}(G)$ an irreducible character let 
%\[  \widetilde{\rho} = \sum_{\sigma \in \Gal(\bQ(\rho)/\bQ)}m(\rho)\cdot \rho^\sigma \quad \in R_{\bQ}(G) , \]
%where $m(\rho) \in \bZ$ is the Schur index of $\rho$.
%\end{notn}
%Then $\widetilde{\rho}$ is the character of an irreducible rational representation. Every irreducible rational representation can be obtained this way. We can extend this map additively to a surjective map $R_{\bC}(G) \to R_{\bQ}(G)$. 

%Let $\overline{R_K(G)}$ be the subring of elements of $R(G)$ which have values in $K$. Then $R_K(G) \subset \overline{R_K(G)}$ and this inclusion is of finite index. 
%Given a character $\chi$ of $G$, let $\bQ(\chi)$ be the smallest subfield of $\bC$ containing $\{ \chi(g) \mid g \in G \}$.
%Let $R_{\bC}(G)$ denote the ring of characters of complex representations of $G$. The number of complex irreducible representations of $G$ is equal to the number of conjugacy classes of $G$. Let $R_{\bQ}(G)$ be the ring of characters of rational valued representations of $G$.
%The number of irreducible $\bQ G$-representations up to isomorphism is equal to the number of conjugacy classes of cyclic subgroups of $G$. %(\cite[$\mathsection 13.1$, Cor. 1]{Serre})
%Induction, Restriction\dots
%\begin{thm}[Mackey Decomposition] 
%\end{thm} 

\subsection{The Burnside ring and \color{red} permutation relations}

Let $G$ be a finite group. Recall that there is a bijection between the isomorphism classes of transitive finite $G$-sets and the conjugacy classes of subgroups $H \leq G$, given by sending a transitive $G$-set $X$ to $H = \Stab_{G}(x)$ for some $x \in X$. Then the action of $G$ on $X$ is equivalent to the action of $G$ on $G / H$. 

\begin{defn}
Let $[X]$ denote the isomorphism class of a $G$-set $X$. 
The \textbf{Burnside ring} $B(G)$ is the free abelian group on isomorphism classes of finite $G$-sets, modulo the relations  $[S] + [T] = [S \sqcup T]$. This is a ring; multiplication is given by $[S] \cdot [T] = [S \times T]$. Using the identification of finite $G$-sets with subgroups of $G$, we write elements of $B(G)$ as $\sum_i n_i H_i$ for $n_i \in \bZ$, $H_i \leq G$. 
\end{defn}

\begin{notn}
There is a homomorphism from the Burnside ring to the rational representation ring $R_{\bQ}(G)$ of $G$ given by taking the corresponding permutation representation:
\[ \bC[-] \colon B(G) \to \Perm(G),  \qquad \Theta = \sum_i n_i H_i \ \mapsto \ \bC[\Theta ] = \sum_i n_i \Ind_{H_i}^G \trivial_{H_i}. \]
\end{notn}
Elements in the kernel of this map are known as \textbf{Brauer relations}. These show instances of non-isomorphic $G$-sets giving rise to isomorphic permutation representations. 

\begin{example}
    {\color{red} $S_3$ example}
\end{example}

\begin{example}
Cyclic groups have no Brauer relations. 
\end{example}

In the last section, we constructed a character in $\Perm(G)$ for $\rho \in R(G)$. We are interested in when this is an image of an element from the Burnside ring.

\begin{defn}
We call $\Theta = \sum_i n_i H_i \in B(G)$ a \textbf{$\rho$-relation} if $\bC[\Theta] \simeq \widetilde{\rho}^{\ \oplus m}$, where $m$ is the order of $\widetilde{\rho}$ in $C(G)$.
\end{defn}
There are $\#$(Brauer relations) $+ 1$ such elements $\Theta$ {\color{red} expand on why?}. Of course, when $\rho = 0$ these are Brauer relations. 


\begin{example}\label{cyclic-relns}
    Let $G = C_n$. For each $d \mid n$, let $\chi_d = \widetilde{\varphi_d}$, where $\varphi_d$ is an irreducible complex character of $G$ with field of values $\bQ(\zeta_d)$ and kernel of index $d$.
    Then $\{ \chi_d \colon d\mid n \}$ form an orthogonal basis for the irreducible rational-valued representations of $G$. Note that $\Ind_{C_{n/ d}}^G \trivial$ is the direct sum of irreducible complex representations of $G$ contain $C_{n / d}$ in their kernel. Thus, $\Ind_{C_{n/ d}}^G \trivial \simeq \sum_{d' \mid d} \chi_{d'}$. Applying M\"{o}bius inversion, we obtain the \textit{unique} $\varphi_d$-relation for each $d \mid n$:
    \[ \chi_d = \sum_{d' \mid d} \mu(d / d') \cdot \Ind_{C_{n/ d}}^G \trivial. \]
    \end{example}

\begin{notn}
    For $D \leq G$, define maps $\Res_D \colon B(G) \to B(D)$ and $\Ind_D \colon B(D) \to B(G)$ by
    \[  \Res_D H = \sum_{x \in H \backslash G / D} D \cap H^{x^{-1}}, \qquad \quad \Ind_D H = H. \]
    These correspond to the representation theory side, where $\Res_D \Ind_{H}^G \trivial = \sum_{x \in H \backslash G / D} \Ind_{D \cap H^{x^{-1}}}^D \trivial$ (Mackey's decomposition), and $\Ind_{D}^G\Ind_{H}^D \trivial = \Ind_{H}^G \trivial$.
\end{notn}