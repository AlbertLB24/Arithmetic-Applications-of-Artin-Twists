\subsection{Norm Relations Tests}
We are concerned with the case of predicting positive rank for families of elliptic curves over certain number fields. We illustrate the proof of the main result that predicts positive rank conditional on Conjecture \ref{conj_4}. Let $F$ be a Galois extension over $\QQ$ and let $G=\Gal(F/\QQ)$. Let $E/\QQ$ be an elliptic curve and let $\rho$ be a representation over $G$, which we view as an Artin representation. Then the representation 
$$\bigoplus_{\mathfrak{g}\in\Gal(\QQ(\rho)/\QQ)}\rho^{\mathfrak{g}}$$
has $\QQ$-valued character and therefore\footnote{see Remark \ref{image-of-burnside}} there is some $m\geq 1$ and subfields $F_i, F_j'$ such that 
$$\left(\bigoplus_{\mathfrak{g}\in\Gal(\QQ(\rho)/\QQ)}\rho^{\mathfrak{g}}\right)^m\oplus\bigoplus_j\Ind_{F'_j/\QQ}\mathds{1}=\bigoplus_i\Ind_{F_i/\QQ}\mathds{1}.$$

Assume that $\rk E/F=0$ so that in particular $\langle \rho, E(F)_\CC\rangle_G=0$. Therefore Conjecture \ref{conj_4} implies that 
\begin{equation}\label{eqn_rank}\tag{\textdagger}
    \frac{\prod_i\BSD(E/F_i)}{\prod_j\BSD(E/F'_j)}=\frac{\prod_i\BSD(E,\Ind_{F_i/\QQ}\mathds{1})}{\prod_j\BSD(E, \Ind_{F'_j/\QQ}\mathds{1})}=\left(\prod_{\mathfrak{g}\in\Gal(\QQ(\rho)/\QQ)}\BSD(E,\rho)^{\mathfrak{g}}\right)^m
\end{equation}
and the right-hand side is clearly the $m$-th power of a norm of an element in $\QQ(\rho)$. 

The product of BSD terms on the LHS of \eqref{eqn_rank} involves regulators, the torsion subgroups, the Tate-Shafarevich groups and the terms $C_{E/F}$ which are the product of local factors. Under the assumption that $\rk E/F=0$, the regulators vanish from the product. In general, it is very difficult to deal with the size of the Tate-Shafarevich group for families of elliptic curves, and therefore very difficult to know if the LHS is an $m$-th power the norm of an element in $\QQ(\rho)$. However, not all hope is lost, since Cassels proved the following.

\begin{thm}
    Let $E$ be an elliptic curve over a number field $K$. If $\Sha_{E/K}$ is finite, then $|\Sha_{E/K}|$ is a square.
\end{thm}

Rational squares are not necessarily the norms of general number fields, but they are always norms of quadratic number fields. Furthermore, if $\QQ(\sqrt{D})$ is a quadratic subfield of $\QQ(\rho)$, then the RHS of \eqref{eqn_rank} is also the norm of an element of $\QQ(\sqrt{D})$, and a rational square if $m$ is even. Under the assumption of finiteness of $\Sha$, we know that $|\Sha_{E/F}|$ and $|E(F)_{\tors}|^2$ are rational squares and therefore norms from $\QQ(\sqrt{D})$. The only remaining terms on the LHS of \eqref{eqn_rank} are the product of local factors $C_{E/F_i}$ and $C_{E/F'_j}$. We have therefore proven the following.

\begin{thm}\cite[Theorem 33]{DEW1} \label{thm_positive_rank}
    Suppose Conjecture \ref{conj_4} holds. Let $E/\QQ$ be an elliptic curve, $F/\QQ$ a finite Galois extension with Galois group $G$, $\rho$ an Artin representation over $\bQ$ that factors through $G$ and 
    $$\left(\bigoplus_{\mathfrak{g}\in\Gal(\QQ(\rho)/\QQ)}\rho^{\mathfrak{g}}\right)^m=\bigoplus_i\Ind_{F_i/\QQ}\mathds{1}\ominus\bigoplus_j\Ind_{F'_j/\QQ}\mathds{1}$$
    for some $m\geq 1$ and subfields $F_i,F'_j\subseteq F$. If either $\frac{\prod_i C_{E/F_i}}{\prod_j C_{E/F'_j}}$ is not a norm from some quadratic subfield $\QQ(\sqrt{D})\subseteq\QQ(\rho)$, or if it is not a rational square when $m$ is even, then $E$ has a point of infinite order over $F$.
\end{thm}

This is a remarkable result, since it can predict positive rank for general families of elliptic curves based solely on local data. Let us call applying this theorem a \textbf{norm relations test}.
Following this result, we introduce some notation that will be very useful to compute the local factors $C_{E/F}$.

\begin{notation}\label{not_contr}
    Let $E$ be an elliptic curve defined over $\QQ$ and let $F/K$ be a finite extension of number fields. For each finite place $\pp$ of $K$, we write the \textbf{local contribution of $\pp$} as 
    %$$T_{\mathfrak{P}\mid\pp}(E/F)=\prod_{\mathfrak{P}\mid\pp}c_\mathfrak{P}(E/F)\quad\text{and}\quad D_{\mathfrak{P}\mid\pp}(E/F)=\prod_{\mathfrak{P}\mid\pp}\left|\frac{\Delta_{E,\mathfrak{P}}^{\min}}{\Delta_E}\right|_\mathfrak{P}^{\frac{1}{12}},$$
    $$T_{\mathfrak{P}\mid\pp}(E/F)=\prod_{\mathfrak{P}\mid\pp}c_\mathfrak{P}(E/F),\quad D_{\mathfrak{P}\mid\pp}(E/F)=\prod_{\mathfrak{P}\mid\pp}\left|\frac{\Delta_{E,\mathfrak{P}}^{\min}}{\Delta_E}\right|_\mathfrak{P}^{\frac{1}{12}}$$ 
    and $C_{\fP\mid\fp}(E/F)=T_{\fP\mid\fp}(E/F)D_{\fP\mid\fp}(E/F)$, where the product ranges over primes $\fP$ of $F$ over $\pp$. The local Tamagawa number is defined in $\S$\ref{subs_tamagawa}, $\Delta_E$ is the global minimal discriminant of $E / \bQ$, and $\Delta_{E, \fP}^{\min}$ is the minimal discriminant of $E$ at $\fP$. 

    Thus given $F / K$, one can obtain the global contributions from the terms above by taking the product over all primes $\fp$ of $K$. We denote the \textbf{global contribution over $F$} of the Tamagawa numbers and the discriminant terms as 
    $$T(E/F)=\prod_\pp T_{\fP\mid\pp}(E/F)=\prod_\fP c_\fP(E/F)\quad\text{and}\quad D(E/F)=\prod_\pp D_{\fP\mid\pp}(E/F)=\prod_\fP \left|\frac{\Delta_{E,\mathfrak{P}}^{\min}}{\Delta_E}\right|_\mathfrak{P}^{\frac{1}{12}}.$$ 
An immediate consequence of this notation is the fact that $C_{E/F}=T(E/F)D(E/F)$.
%Observe that if $E$ has good reduction over $\pp$, then $T_{\mathfrak{P}\mid\pp}(E/F)=D_{\mathfrak{P}\mid\pp}(E/F)=1$ for any finite extension $F$ of $K$. 
\end{notation}

We have now seen that both Theorem \ref{thm_positive_rank} and Conjecture \ref{parity} can force the rank of an elliptic curve to be positive. One can observe that for the examples in the next subsection, whenever our norm relations test forces rank growth, a root number computation for the subfields appearing in our relation also implies positive rank. 
It would be a lot more interesting if we could find an example where the norm relations test forces positive rank and root numbers do not. 
We suspect however that such an example does not exist. 

Consider $G = \Gal(L / K)$. Then changes in parity between subfields of $L / K$ correspond to twisted root numbers being equal to $-1$. Indeed by Proposition \ref{compute-root-twist}, for $H \leq G$, 
\[ \Ind_{H}^G \trivial \simeq \trivial\oplus \bigoplus_i \rho_i  \implies w(E / L^H) = w(E / K)\prod_i w(E / K, \rho_i) \] for some representations $\rho_i $ of $G $. Hence a change in parity from $w(E / K)$ to $w(E / L^H)$ is determined by $\prod_i w(E / K, \rho_i)$. On the premise that a failure of our norms relations test is always explained by root numbers, we conjecture the following:
%{\color{red} On the other hand ... root numbers should help us determine what the regulators look like ... and we could put them into our factors... maybe talk about this in a conclusion / outlook?}

\begin{conj}
Consider an elliptic curve $E / \bQ$, $F / \bQ$ a finite Galois extension, and relation 
$$\left(\bigoplus_{\mathfrak{g}\in\Gal(\QQ(\rho)/\QQ)}\rho^{\mathfrak{g}}\right)^m=\bigoplus_i\Ind_{F_i/\QQ}\mathds{1}\ominus\bigoplus_j\Ind_{F'_j/\QQ}\mathds{1}$$
as in Theorem \ref{thm_positive_rank}. If the product $\frac{\prod_i C_{E/F_i}}{\prod_j C_{E/F'_j}}$ is not a norm from some quadratic subfield $\QQ(\sqrt{D})\subseteq\QQ(\rho)$, or if it is not a rational square when $m$ is even, then there exists a self-dual Artin representation $\tau$ of $G$ such that $w(E / \bQ, \tau) = -1 $. 
\end{conj}

The case of odd order Galois groups inspires some confidence in this conjecture. In $\S$\ref{subsec-odd}, we prove that one cannot use Theorem \ref{thm_positive_rank} to conclude that $\rk E / F > 0$, i.e. that our norms relation test never "fails".
The reason one would expect that our norm relations test never forces rank growth in this case is because root number computations do not, as we now show.

\begin{lemma}
 Let $F / \bQ$ be an odd Galois extension with $G = \Gal(F / \bQ)$. Then $w(E / \bQ) = w(E / F^H)$ for all $H \leq G$. 
\end{lemma}

\begin{proof}
Consider $H \leq G$ and intermediate field $F^H$. Then 
$\Ind_H^G \trivial \simeq \trivial \oplus 
\rho \oplus \rho^*$ for some non self-dual representation $\rho$ of $G$, since the only self-dual representation of an odd-order group is the trivial representation. Therefore by Proposition \ref{compute-root-twist}, $w(E / F^H) = w(E / F)w(E, \rho \oplus \rho^*)$ and $w(E, \rho \oplus \rho^*) = 1$. Hence $w(E / \bQ) = w(E / F^H)$ for all $H \leq G$. 
\end{proof}

Therefore once we assume that $\rk E / \bQ = 0$, the parity conjecture tells us that $\rk E / F^H$ is even for all $H \leq G$, which does not force it to be non-zero. 

