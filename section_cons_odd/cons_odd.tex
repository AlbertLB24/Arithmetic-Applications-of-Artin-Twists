%\subsection{Norm Relations in Odd Order Extensions}

In this section we prove the analogous result to the previous section when $F / \bQ$ is any Galois extension of odd order. The reason one would expect that our norm relations test never forces rank growth in this case is because root number computations do not, as we now show.

\begin{lemma}
 Let $F / \bQ$ be an odd Galois extension with $G = \Gal(F / \bQ)$. Then $w(E / \bQ) = w(E / F^H)$ for all $H \leq G$. 
\end{lemma}

\begin{proof}
Consider $H \leq G$ and intermediate field $F^H$. Then 
$\Ind_H^G \trivial \simeq \trivial \oplus 
\rho \oplus \rho^*$ for some non self-dual representation $\rho$ of $G$, since the only self-dual representation of an odd-order group is the trivial representation. Therefore by Proposition \ref{compute-root-twist}, $w(E / F^H) = w(E / F)w(E, \rho \oplus \rho^*)$ and $w(E, \rho \oplus \rho^*) = 1$. Hence $w(E / \bQ) = w(E / F^H)$ for all $H \leq G$. 
\end{proof}

Therefore once we assume that $\rk E / \bQ = 0$, the parity conjecture tells us that $\rk E / F^H$ is even for all $H \leq G$, which does not force it to be non-zero. 


\begin{thm}\label{odd-exts}
 Let $E / \bQ$ be an elliptic curve, $F / \bQ$ be an extension of odd order with Galois group $G$. 
 
Assume that $E$ has good or multiplicative reduction at $2$ and $3$. 
Take any representation $\rho$ of $G$ with quadratic subfield $\bQ(\sqrt{D}) \subset \bQ(\rho)$ and relation
\begin{equation*}\label{odd-exp} \repnorm{\rho}^{\oplus m} =
 \left(\bigoplus_{\mathfrak{g}\in\Gal(\QQ(\rho)/\QQ)}\rho^{\mathfrak{g}}\right)^{\oplus m }=\bigoplus_i\Ind_{F_i/\QQ}\mathds{1}\ominus\bigoplus_j\Ind_{F'_j/\QQ}\mathds{1}
\end{equation*}
 as in Theorem \ref{thm_positive_rank}. Then
 \[ \frac{\prod_i C_{E/F_i}}{\prod_j C_{E/F_j'}}  \in 
    \begin{cases}
        N_{\bQ(\sqrt{D}) / \bQ}(\bQ(\sqrt{D})^{\times}) & m \ \text{odd}, \\
        (\bQ^{\times})^2 & m \ \text{even}.
    \end{cases} \] 
    In other words, one cannot use Theorem \ref{thm_positive_rank} to conclude that $E / F$ must have positive rank. 
\end{thm}

As in Remark \ref{rephrase-thm}, replacing $\rho$ by the sum of its conjugates by elements of $ \Gal(\bQ(\rho) / \bQ(\sqrt{D}))$, we may assume that $\bQ(\rho) = \bQ(\sqrt{D})$. We take $D \in \bZ \backslash \{0,1 \}$ to be squarefree.

The product of terms we are computing is $C(\Theta)$, where $C \colon \B(G) \to \bQ^{\times}$ is given by $C \colon H \mapsto C_{E / F^H}$, and $\Theta$ is any $\rho$-relation.
We break up the function $C$ into $C = \prod_p C_{\fP \mid p} = \prod_p T_{\fP \mid p} \cdot D_{\fP \mid p}$, ranging over primes $p \in \bQ$,
as defined in Notation \ref{not_contr_fns}.
We showed that
\begin{equation*}\label{Dp-loc}
C_{\fP \mid p} = (D_p, C_v)
\end{equation*}
where $C_v$ is a function on $\B(D_p)$ sending $H \mapsto C_v(E / F_w^H)$, for $D_p = \Gal(F_w / \bQ_p)$. The following result will allow us to apply some results from $\S$\ref{sec-norm-rels}.

\begin{thm}\label{odd-c-brauer}
    When $D_p$ has odd order, $C_v(\Psi) \in (\bQ^{\times})^2$ for any Brauer relation $\Psi \in \B(D_p)$. 
\end{thm}

\begin{proof}
    This follows from \cite[Theorem 2.47]{reg-const} and \cite[Theorem 3.2  (Tam)]{reg-const}.
\end{proof}

\begin{cor}
It is enough to prove Theorem \ref{odd-exts} when $m$ is the order of $\repnorm{\rho}$ in $\C(G)$. Thus we need to prove that, given any $
\Theta \in \B(G)$ such that $\bC[\Theta] \simeq \repnorm{\rho}^{\oplus m}$, $\Theta$ is a norm relation for the function $C$. 
\end{cor}

\begin{proof}
    Since $\bQ(\rho)$ is quadratic, we have that rational squares are norms from $\bQ(\rho)$. As $G$ is odd, any choice of $D_p$ is odd. It follows from Theorem \ref{odd-c-brauer} that $C(\Psi) \in (\bQ^{\times})^2$ for all Brauer relations $\Psi \in \B(G)$. Therefore by Proposition \ref{min-to-all}, it is enough to prove Theorem \ref{odd-exts} when $m$ is the order of $\repnorm{\rho}$ in $\C(G)$. Then $m$ divides $|G|$, hence is odd.
\end{proof}

Let $\tau$ be the generator of $\Gal(\bQ(\sqrt{D}) / \bQ)$.
Let $\ff$ be the smallest integer such that $\bQ(\sqrt{D}) \subset \bQ(\zeta_{\ff})$. Then $\ff \mid |G|$, hence is odd. By Remark \ref{conductor}, $\ff = |D|$ and $D \equiv 1 \pmod 4$. The following shows that it is only of interest to consider decomposition groups of exponent divisible by $\ff$.

\begin{cor}\label{rational-res-2}
    Let the exponent of $D_p$ be $b$. If $\ff \nmid b$, then $C_{\fP \mid p}(\Theta) \in (\bQ^{\times})^2$.
\end{cor}

\begin{proof}
    One has $\bQ(\rho) \subset \bQ(\zeta_b) \implies \ff \mid b$ by minimality of $\ff$. Since $\ff \nmid b$, we have $\bQ(\rho) \not\subset \bQ(\zeta_b)$, so $\bQ(\Res_{D_p} \rho) = \bQ$. The corollary then follows from Proposition \ref{rational-res} and Theorem \ref{odd-c-brauer}, noting that since $\C(D_p)$ is odd, multiplication by $2$ is injective. 
\end{proof}

Fix $\Theta = \sum_i n_i H_i \in \B(G)$ with $\bC[\Theta] \simeq \repnorm{\rho}^{\oplus m}$. We prove that at each prime $p$, $C_{\fP \mid p}(\Theta)$ is the norm of an element of $\bQ(\rho)$.  As observed, this depends on $D_p$ and $I_p$. As we deal with each local factor individually, we argue that one can take $D_p = I_p$.

\begin{lemma}\label{tam-up-to-square}
    Let $E / K$ be an elliptic curve. Let $K' / K$ be an extension of number fields odd degree, unramified at the place $v$ of $K$. Then $C_w(E / K') \equiv C_v(E / K) \mod (\bQ^{\times})^2$ for any place $w$ of $K'$ with $ w \mid v$. 
\end{lemma}

\begin{proof}
This is automatic for good reduction and split multiplicative reduction. It is also clear for non-split multiplicative reduction since the residue degree cannot be even (so the reduction type remains non-split at $w$). For additive reduction, see \cite[Lemma 3.12]{reg-const}.
\end{proof}

\begin{lemma}\label{DeqI}
    At a prime $p$, we may assume that $D_p = I_p$ when computing $C_{\fP \mid p}(\Theta)$. 
\end{lemma}

\begin{proof}
Let $p$ have residue degree $f_p$. Let $L / \bQ$ be a Galois extension of degree $f_p$ with cyclic Galois group, such that $p$ is inert in $L$. Further ensure that $F \cap L = \bQ$. Then $\Gal(FL / L) = G$. Let $F_i = F^{H_i}$ and $L_i = F_i L$.

Let $v$ be a place over $p$ in $F_i$. The extension $L_i / F_i$ is Galois, so $v$ is either split or inert in $L_i$.
We claim that $C_v(E / F_i) \equiv \prod_{w | v} C_w(E / L_i) \mod (\bQ^{\times})^2$. Indeed, the number of terms in the product on the right is odd, and by Lemma \ref{tam-up-to-square} $C_v(E / F_i) \equiv C_w(E / L_i) \mod (\bQ^{\times})^2$. 
Letting $C_{\fP \mid p}'$ be the function on $\B(G)$ defined as in \eqref{not_contr_fns} but with $\bQ$, $F$ replaced by $L$, $FL$, we see that $C_{\fP \mid p}'(\Theta) \equiv C_{\fP \mid p}(\Theta) \mod (\bQ^{\times})^2$. 
Thus it is equivalent to do our computation in $FL / L$, but here $p$ has residue degree $1$.
\end{proof}

To prove Theorem \ref{odd-exts}, we proceed by computing $C_{\fP \mid p}(\Theta)$ for each reduction type.

\subsubsection*{Good reduction}
If $E / \bQ$ has good reduction at $p$, then by Proposition \ref{prop_local_fns}(i), $C_{\fP \mid p} = 1$.

\subsubsection*{Multiplicative reduction}

\begin{lemma}
Let $E / \bQ_p$ have non-split multiplicative reduction. Then $C_{\fP \mid p}(\Theta) \in (\bQ^{\times})^2$.
\end{lemma}

\begin{proof}
Since $D_p = I_p$, all primes above $p$ have residue degree $1$. Moreover, the ramification degrees are always odd. Thus by Proposition \ref{prop_local_fns}(iii), $C_{\fP \mid p}  = (D_p, \alpha)$
where $\alpha$ is the constant function on $\B(D_p)$ with $\alpha \in \{1, 2\}$, depending on $v_p(\Delta)$ being even or odd. By Proposition \ref{const-fns}, it follows that $C_{\fP \mid p}(\Theta) \in (\bQ^{\times})^2$.
\end{proof}

Now suppose $E / \bQ_p$ has split multiplicative reduction. The reduction type remains split at all places above $p$ within sub-extensions of $F / \bQ$. Let $v_p(\Delta) = n$. Then by Proposition \ref{prop_local_fns}(ii), 
\[ C_{\fP \mid p} = (D_p, D_p, en). \]
Since the $n$ factor is constant, $(D_p, D_p, en)(\Theta) \equiv (D_p, D_p, e)(\Theta) \mod (\bQ^{\times}) ^2$ by Proposition \ref{const-fns} .

We have $D_p = I_p = P_p \ltimes C_l$, where $P_p \triangleleft I_p$ is wild inertia, and $C_l = I_p / P_p$ is the tame quotient. $C_l$ is a cyclic group, with $l \mid p^f - 1 = p - 1$. By Corollary \ref{rational-res-2}, we may consider such $D_p$ with exponent  $p^u l$ for some $u \geq 0$ such that $\ff \mid p^u l$. To compute $C_{\fP \mid p}(\Theta)$, we reduce to the tame quotient.

\begin{lemma}
Let $g \colon \B(C_l) \to \bQ^{\times}$ be defined by $H \mapsto [C_l \colon H] = \dim \bC[C_l / H]$. Let $\Psi = P_p \cdot \Res_{D_p} \Theta / P_p \in \B(C_l)$ Then $(D_p, D_p, e)(\Theta)$ and $g(\Psi)$ differ by a (possible) factor of $p$. 
\end{lemma}

\begin{proof}
%Now, $(D_p, D_p, e)(\Theta)$ is the product of ramification indices at primes above $p$. We separate the $p$-part and tame part of this expression.
Recall that the ramification index of a place $w$ above $p$ corresponding to the double coset $H_i x D_p$ has ramification degree $e_w = \frac{|I_p|}{|H_i \cap I_p^x|} =\frac{|I_p|}{|I_p \cap H^{x^{-1}}|}$.
This is the dimension of the permutation representation $\bC[D_p / D_p \cap H^{x^{-1}}]$.
Let  $D_p \cap H^{x^{-1}} = P' \ltimes C_a$ where $P' \leq P$ and $a | l$. Then the ramification index is $\frac{|P|}{|P'|}\cdot \frac{l}{a}$. 
Taking fixed points under wild inertia, one has the following isomorphism of $D_p$-representations  $$\bC[D_p / D_p \cap H^{x^{-1}}]^{P_p} \simeq \bC[D_p / P_p (D_p \cap H^{x^{-1}})] \simeq \bC[D_p / P_p \ltimes C_a].$$ This permutation representation has dimension $\frac{l}{a}$, so we've killed off the $p$-part. 
Then $$\bC[\Res_{D_p} \Theta]^{P_p} \simeq \left(\Res_{D_p} \rho^{\oplus m} \oplus \tau\left(\Res_{D_p}\rho^{\oplus m}\right)\right)^{P_p},$$
and we can consider these as representations of $D_p / P_p = C_l$.
It follows that $(D_p, D_p, e)(\Theta)$ differs from $g(\Psi)$ up to a (possible) factor of $p$. 
\end{proof}

Now we show that this factor of $p$ is a norm from $\fieldnorm{\rho}$.
Note that if $p = 2$ then $P_p = 1$ since $|P_p| \mid |G|$ which is odd. So we only need to consider this factor of $p$ for $p$ odd.

\begin{lemma}\label{p-norm-odd}
    Let $K = \bQ(\sqrt{D})$, with $\ff$ the smallest positive integer such that $K \subset \bQ(\zeta_{\ff})$. Suppose that $\ff$ is odd. Let $\ff \mid p^u l $, for some odd prime $p$, $u \geq 0$ and $l$ such that $p \equiv 1 \pmod l$. Then $p$ is the norm of an element from $K^{\times}$.
\end{lemma}

\begin{proof}
    Since $\ff$ is odd, one has $D = \prod_{q | \ff} q^*$, the product being taken over primes dividing $\ff$. Note that if $q \not= p$, then since $q \mid l$, we have $p \equiv 1 \pmod l \implies p \equiv 1 \pmod q$. By Corollary \ref{p-one-mod-disc},  $p$ is the norm of a principal fractional ideal of $K$, and by Theorem \ref{p-norm-elem-1} or Theorem \ref{p-norm-elem-2}, it is the norm of an element of $K$.
    %We show that $p$ has residue degree $1$ in the extended genus field $E^{+} = K(\{\sqrt{q^*} \colon q | \ff \})$ of $K$ ({\color{red} cf. appendix}).
    %If $q \not= p$ then $q \mid l$, so $p \equiv 1 \pmod l$. Therefore $p$ splits in any quadratic subfield of $E^{+}$ of discriminant not divisible by $p$. Else, $p$ ramifies in any quadratic subfield with discriminant divisible by $p$. Thus it is clear that $p$ has residue degree $1$ in $E^{+}$, hence also in the genus field $E$, and it follows from theorem \ref{p-principal} that $p$ is the norm of a principal ideal.  Else, we invoke theorem \ref{minus-one-norm}.
\end{proof}

\begin{cor}
Let $E / \bQ_p$ have split multiplicative reduction. Then $C_{\fP \mid p}(\Theta) \in \fieldnorm{\rho}$.
\end{cor}

\begin{proof}
By the previous two results, it is sufficient to show that $g(\Psi) \in \fieldnorm{\rho}$. Let $\phi = (\Res_{D_p} \rho)^{P_p}$, viewed as a representation on $D_p / P_p = C_l$. Then $\Psi$ is a $\phi$-relation. If $\ff \nmid l$ then $\bQ(\phi) = \bQ$. Therefore $\bC[C_l / \Psi] \simeq \phi^{\oplus 2}$, implying that $\Psi = 2\Psi'$ for some $\Psi' \in \B(C_l)$ with $\bC[C_l / \Psi'] = \phi$. Then $g(\Psi) = g(\Psi')^2 \in \fieldnorm{\rho}$. Otherwise, suppose that $\bQ(\phi) = \bQ(\rho)$. It follows from Proposition \ref{index-fn-trivial} that $g(\Psi) \in \fieldnorm{\rho}$.
\end{proof}

\subsubsection*{Additive reduction}

Now suppose that $E / \bQ_p$ has additive reduction. In this case, assume that $p \geq 5$
Write $D_p = \Gal(F_w / \bQ_p)$ for $w \mid p$ a place of $F$.

Again we have $D_p = P_p \ltimes C_l$ with $ l \mid p - 1$, and may assume that $\ff \mid p^u l$ where $p^u l $ is the exponent of $D_p$ by Corollary \ref{rational-res-2}. Let $n = v_p(\Delta_E)$. 

We will compute $D_{\fP \mid p}(\Theta)$ and $T_{\fP \mid p}(\Theta)$ separately. 
By Proposition \ref{prop_local_fns}(iv), (v)
\[ D_{\fP \mid p} = 
    \begin{cases}
        (D_p, D_p,\ p^{\floor{e_ /2}}) & \text{if } E / \bQ_p \text{ has potentially multiplicative reduction}, \\
        (D_p, D_p,\ p^{\floor{en /12}}) & \text{if } E / \bQ_p \text { has potentially good reduction}.
    \end{cases}
    \]
In either case, $D_{\fP \mid p}(\Theta) \in N_{\bQ(\rho) / \bQ}(\bQ(\rho)^{\times})$. Indeed, this takes values $1$ or $p$ in $\bQ^{\times} / (\bQ^{\times})^2$. But $p$ is a norm from $\bQ(\rho)$ by Lemma \ref{p-norm-odd}.

%\[ \left|\frac{\Delta_{E}}{\Delta_{E, w}^\min} \right|_w = p^{f_w 12 \cdot \floor{e_w n / 12}} \implies 
%       \left|\frac{\omega}{\omega_{w}^\min} \right|_w = p \]
\vspace{1em}

To compute $T_{\fP \mid p}(\Theta)$, since $p \geq 5$ we may write $E / \bQ_p$ as $E \colon y^2 = x^3 + A x + B$ and use the description from \cite{reg-const} for computing Tamagawa numbers, as detailed in Lemma \ref{tamagawa-num}. The discriminant of $E / \bQ_p$ is $\Delta = -16(4A^3 + 27 B^2)$. The case of potentially multiplicative reduction is almost immediate:

\begin{lemma}[Potentially multiplicative reduction]
    If $E / \bQ_p$ has reduction type $\I_{n}^{*}$ then $T_{\fP \mid p}(\Theta) \in (\bQ^{\times})^2$. 
\end{lemma}

\begin{proof}
Since we assume $D_p = I_p$, i.e. the residue degree is one, it follows that any subextension $L'$ of $F_{w} / \bQ_p$ satisfies $\sqrt{B} \in L' \iff \sqrt{B} \in \bQ_p$ and $\sqrt{\Delta} \in L' \iff \sqrt{B} \in \bQ_p$. 
Therefore $T_{\fP \mid p} = (D_p, \alpha)$ where $\alpha \in \{2, 4\}$ by Lemma \ref{lem_add_tam}. But then $(D_p, \alpha)(\Theta) \in (\bQ^{\times})^2$ by Proposition \ref{const-fns}.
\end{proof}

Now suppose that $E / \bQ_p$ has potentially good reduction. Recall from Lemma \ref{lem_add_tam} that if $L' / \bQ_p$ has ramification degree $e$, then the Kodaira type of $E / L'$ depends on $\gcd(e n, 12)$. Thus in a ramified extension of degree coprime to $12$, the Kodaira type is unchanged, and further if this extension is totally ramified (so the residue degree is $1$), the Tamagawa number is unchanged also. Thus when $3 \nmid |D_p|$, $T_{\fP \mid p} = (D_p, \alpha)$ for some constant $\alpha$. If we have type $\III$ or $\III^*$ or $\I_0^*$ then the Tamagawa number is still unchanged in any totally ramified extension of odd degree extension, even when the degree is divisible by $3$. Then the Proposition \ref{const-fns} implies the following lemma:

\begin{lemma}
    \
    \begin{enumerate}[(i)]
        \setlength\itemsep{0em}
        \item If $E / \bQ_p$ has potentially good reduction and $3 \nmid |D_p|$, then $T_{\fP \mid p}(\Theta) \in (\bQ^{\times})^2$.
        \item If $E / \bQ_p$ has potentially good reduction of type $\III$, $\III^*$, or $\I_0^*$, then $T_{\fP \mid p}(\Theta) \in (\bQ^{\times})^2$.
    \end{enumerate}
\end{lemma}

Thus we assume that $3 \mid |D_p|$. Since we assumed $p \geq 5$, we have $D_p = I_p = P_p \ltimes C_l$ with $3 \mid l$ and $p \equiv 1 \pmod l$.

\begin{lemma}
If $E / \bQ_p$ has Type $\II$ or Type $\II^*$ additive reduction and $3 \mid |D_p|$, then $T_{\fP \mid p}(\Theta) \in (\bQ^{\times})^2$. 
\end{lemma}

\begin{proof}
If $3 \mid |D_p|$ then there is a subextension $F'$ of $F_w / \bQ_p$ with $\Gal(F_w / F') = C_3$. Then Lemma \ref{lem_nottwo} implies that $\sqrt{\Delta} \in F'$. But $F' / \bQ_p$ has residue degree $1$, hence $\sqrt{\Delta} \in \bQ_p$. 

If $L' / \bQ_p$ is an odd degree extension that is divisible by $3$, then $E / L'$ has reduction type $I_0^*$. By Lemma \ref{tamagawa-num} the Tamagawa number of $E / L'$ is $2$ if $\sqrt{\Delta} \not\in \bQ_p$ and $1$ or $4$ if $\sqrt{\Delta} \in \bQ_p$. Therefore the Tamagawa number will be $1$ or $4$, which is a square.
On the other hand if $L' / \bQ_p$ is an extension of odd degree then the reduction type over $L'$ is $\II$ or $\II^*$ and the Tamagawa number is $1$. It follows that $T_{\fP \mid p}(\Theta)$ is a product of square terms, so is itself square.  
\end{proof} 

Now, if $E /\bQ_p$ has additive reduction of type $\IV$ or $\IV^*$, it attains good reduction over any totally ramified cyclic extension of degree divisible by $3$. This could result with $3$ coming up an odd number of times in $T_{\fP \mid p}(\Theta)$, when $\sqrt{B} \not\in \bQ_p$. 
%We show that for both types, one has $\sqrt{B} \in \bQ_p$. 
%Indeed, if $\delta = 4$, then $v_p(B) = 2$, and $v_p(A) \geq 2$. 
%\vspace{1em}
%In summary, 
%\begin{equation}
 %   \prod_{d ' \mid d} C(E / F_{\fp}^{D_{d'}})^{\mu(d / d')}
  %  = 
   % \begin{cases}
    %    1 & 3 \nmid d, \\
    %   1 & 3 \mid d, \delta \in \{0, 3, 6, 9\}, \\
    %    1 \cdot \square & 3 \mid d, \delta \in \{2, 10\}, \\
    %    3^a \cdot\square, a \in \{0,1\} & 3 \mid d, \delta \in \{4,8\}.
    %\end{cases}
%\end{equation}
%\begin{rem}
%   There's no reason why we can't get 3; see elliptic curve 441b1 with additive reduction at $7$ of type IV and Tamagawa number equal to $3$
%\end{rem}

%\textbf{If $D_p = C_l$ then we are able to finish our argument.} As in the proof of Proposition \ref{semi-stable-gd}, there exists $a_{l'} \in \bZ$ such that $\Res_{D_p}\Theta = \sum_{l' \mid l} a_{l'} \Psi_{l'}$ where $\Psi_{l'} \in \B(G)$ is such that $\bC[\Psi_{l'}] \simeq \chi_{l'}$, as in Example \ref{cyclic-relns}.

Recall from the proof of Proposition \ref{const-fns} that $\Res_{D_p} \Theta = \sum_i n_i \sum_{x \in H_i \backslash G / D_p} D_p \cap H^{x^{-1}}$, with $\sum_i n_i | H_i \backslash G / D_p|$ even. If $D_p = \Gal(F_w / \bQ_p)$, then the number of subextensions divisible by $3$ (i.e. the number of subextensions where we obtain good reduction ) corresponds to the number of subgroups with index divisible by $3$ in $\Res_{D_p}\Theta$. We compute this number to determine $\ord_3(T_{\fP \mid p}(\Theta))$ modulo squares.

Similarly to the split multiplicative case, we may pass to the tame quotient $C_p / P_p = C_l$. Indeed 

\[ 3 \mid  [D_p : D_p \cap H^{x^{-1}}] = \dim \bC [ D_p / D_p \cap H^{x^{-1}}] \iff 3 \mid \dim \bC[ D_p / D_p \cap H^{x^{-1}} ]^{P_p},\] 
since $3 \nmid|P_p|$. Therefore we may compute the number of subgroups divisible by $3$ in $\Psi = P_p \cdot \Res_{D_p} \Theta / P_p \in \B(C_l)$.  Let $h \colon \B(C_l) \to \bQ^{\times}$ be the function given by $H \mapsto \begin{cases} 3 & 3 \mid [C_l : H], \\ 1 & 3 \nmid [C_l : H]. \end{cases}$

\begin{prop}
   Suppose that $E / \bQ_p$ has additive reduction of Type $\IV$ or $\IV^*$, with $c_v(E / \bQ_p) = 3$.  Then $T_{\fP \mid p}(\Theta) \equiv h(\Psi) \mod (\bQ^{\times})^2$ and $T_{\fP \mid p}(\Theta) \in \fieldnorm{\rho}$. 
\end{prop}

\begin{proof}
The fact that $T_{\fP \mid p}(\Theta) \equiv h(\Psi) \mod (\bQ^{\times})^2$ has been observed above.
Let $\psi_3$ be an irreducible character of $D_p$ of order $3$. One has that $ \langle \Ind_{C_{l / l'}}^{C_l} \trivial , \psi_3 \rangle =  1$ when $3 \mid l'$ and is zero when  $3 \nmid l'$. 
Thus $$h(\Psi) = 3^{\langle \bC[C_l / \Psi], \psi_3 \rangle}.$$ As in the proof of Proposition \ref{index-fn-trivial}, write $\bC[C_l / \Psi] = \sum_{l' \mid l} a_{l'} \chi_{l'}$, where $\chi_{l'}$ is an irreducible rational character, of $C_l$ with kernel of index $l'$. Observe that $\langle \chi_{l'}, \psi_3 \rangle = 0$ unless $l' = 3$, in which case it is $1$. Therefore $h(\Psi) \equiv 3^{a_3} \mod (\bQ^{\times})^2$. In the proof of Proposition \ref{index-fn-trivial}, we showed that $a_3$ is even unless $\ff \mid 3$,  i.e. that $\bQ(\rho) = \bQ(\sqrt{-3})$. But then $3$ is a norm in $\bQ(\rho)$. Thus we see that in all cases $T_{\fP \mid p}(\Theta) \in N_{\bQ(\rho) / \bQ}(\bQ(\rho)^{\times})$. 
\end{proof}

We have observed that for all reduction types of $E / \bQ_p$, one has $C_{\fP \mid p} (\Theta) \in \fieldnorm{\rho}$, and so $C(\Theta) \in \fieldnorm{\rho}$, completing the proof of Theorem \ref{odd-exts}.

\qed

\begin{cor}\label{cor-odd-decomp}
    Let $G$ be a finite group, $\rho$ a character of $G$ with $\bQ(\rho)$ quadratic. Let $\Theta \in \B(G)$ be a $\rho$-relation. If $D_p \leq G$ is of odd order, then $C_{\fP \mid p}(\Theta) \in \fieldnorm{\rho}$.
\end{cor}

\begin{proof}
 Throughout this section our results only depended on the oddness of $D_p$.
\end{proof}
