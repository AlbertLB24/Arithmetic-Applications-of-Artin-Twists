\subsection{Representation theory of finite groups}\label{rep}

Let $G$ be a finite group. Recall that a \textbf{representation} of $G$ is a group homomorphism $\rho \colon G \to \GL(V)$ where $V$ is a complex vector space. Associated to a representation $\rho$ is a \textbf{character} $\chi \colon G \to \bC^{\times}$, defined by letting $\chi(g) = \Tr \rho(g)$ for $g \in G$. For complex represenations, $\rho$ is determined by its character; if $\rho$, $\rho'$ are representations with identical characters, then $\rho$ and $\rho'$ are isomorphic as representations. 

%Given a character $\chi$ of $G$, let $\bQ(\chi)$ be the smallest subfield of $\bC$ containing $\{ \chi(g) \mid g \in G \}$.
%Let $R_{\bC}(G)$ denote the ring of characters of complex representations of $G$. The number of complex irreducible representations of $G$ is equal to the number of conjugacy classes of $G$. Let $R_{\bQ}(G)$ be the ring of characters of rational valued representations of $G$.
%The number of irreducible $\bQ G$-representations up to isomorphism is equal to the number of conjugacy classes of cyclic subgroups of $G$. %(\cite[$\mathsection 13.1$, Cor. 1]{Serre})

Given an irreducible $\bQ G$-representation with character $\psi$, we have that 
\[\psi =  \sum_{\sigma \in \Gal(\bQ(\rho)/\bQ)}m(\rho)\cdot \rho^\sigma \]
for $\rho$ the character of an irreducible $\bC G$-representation, and $m(\rho)$ the Schur index.

In particular, the map $R_{\bC}(G) \to R_{\bQ}(G)$ given by sending an irreducible complex character $\rho$ to 
$\tilde{\rho} = \sum_{\sigma \in \Gal(\bQ(\rho)/\bQ)}m(\rho)\cdot \rho^\sigma$ is surjective.


Induction, Restriction\dots

\begin{thm}[Mackey Decomposition] 


\end{thm} 


\subsubsection{Permutation representations and the Burnside ring}

Let $G$ be a finite group. The \textbf{Burnside ring} $B(G)$ is the ring of formal sums of isomorphism classes of finite $G$-sets. We have addition by disjoint union: $[S] + [T] = [S \sqcup T]$, and multiplication by Cartesian product: $[S] \times [T] = [S \times T]$ for $S$, $T$ finite $G$-sets. 

There exists a bijection between the isomorphism classes of transitive $G$-sets and the conjugacy classes of subgroups $H \leq G$, where $H$ is the stabilizer of a point on which $G$ acts. Then any transitive $G$-set $X$ is isomorphic to the action of $G$ on $G/H$ for $H \leq G$, so that we can consider $B(G)$ to be a $\bZ$-module generated by the orbits of the action of $G$ on the elements $\{G/H : H \leq G\}$, where we consider $H$ up to conjugacy. For notational purposes, we then write elements $\Theta \in B(G)$ as $\Theta = \sum_i n_i H_i$ with $n_i \in \bZ$, $H_i \leq G$.

Given a transitive $G$-set $G/H$ for $H\leq G$, we can look at the permutation representation $\bC[G/H]$ . This defines a homomorphism from the Burnside ring to the rational representation ring $R_{\bQ}(G)$ of $G$: 
\[ a \colon B(G) \to R_{\bQ}(G),  \ \ \ \ \ \sum_i n_i H_i \mapsto \sum_i n_i \Ind_{H_i}^G \trivial_{H_i}. \]

Elements in the kernel of this map are known as \textbf{Brauer relations}