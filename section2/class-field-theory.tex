\subsection{Class field theory}

Consider an odd prime $p$. Let $\bQ(\sqrt{p^*}) = \begin{cases} \bQ(\sqrt{p}), & p \equiv 1 \pmod 4,\\ \bQ(\sqrt{-p}), & p \equiv 1 \pmod 3  
\end{cases}$.

\begin{prop}
$\bQ(\sqrt{p^*})$ has odd narrow class number.
\end{prop}    

\begin{cor}\label{p-norm}
The prime $p \in \bQ$ is the norm of an element in $\bQ(\sqrt{p^*})^{\times}$.
\end{cor}