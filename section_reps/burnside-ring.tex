\subsection{The Burnside ring and relations for permutation representations}

Let $G$ be a finite group. Recall that there is a bijection  
\[ \{ \text{transitive finite }G\text{-sets } X  \text{ up to isomorphism}\}\leftrightarrow  \{ \text{subgroups } H \leq G \text{ up to conjugacy} \} \] 
given by sending a transitive finite $G$-set $X$ to $H = \Stab_{G}(x)$ for some $x \in X$.  The action of $G$ on $X$ is equivalent to the action of $G$ on $G / H$. 

\begin{defn}\label{burnside}
Let $[X]$ denote the isomorphism class of a $G$-set $X$. 
The \textbf{Burnside ring} $\B(G)$ is the free abelian group on isomorphism classes of finite $G$-sets, modulo the relations  $[S] + [T] = [S \sqcup T]$ for $S$, $T$ finite $G$-sets. This is a ring; multiplication is given by $[S] \cdot [T] = [S \times T]$.
\end{defn}

We only need that $\B(G)$ is a group, and do not use its multiplicative structure. $\B(G)$ is generated by $\{ [X] \colon X \text{ finite transitive } G\text{-set}\}$. Using the identification of finite transitive $G$-sets with subgroups of $G$, we write elements of $\B(G)$ as $\sum_i n_i H_i$ for $n_i \in \bZ$, $H_i \leq G$. 

\begin{defn}
    Given a $G$-set $X$, one obtains a representation of $G$ by considering its permutation representation $\bC[X]$. We extend this to $\B(G)$; given $\Theta  = \sum_i n_i H_i \in \B(G)$, define 
    \[ \bC[G / \Theta ] = \sum_i n_i \Ind_{H_i}^G \trivial. \]
    Let $\chi_{\bC[G / \Theta]}$ be the character of $\bC[G / \Theta]$. Then  $\Theta \mapsto \chi_{\bC[G / \Theta]}$ defines a homomorphism $\B(G) \to \Perm(G)$.
\end{defn}

\begin{defn}
If $\bC[G / \Theta] = 0$ as a virtual permutation representation (i.e. $\chi_{\bC[G / \Theta]} = 0$), then $\Theta$ is called a  \textbf{Brauer relation}.
\end{defn}

Non-trivial Brauer relations are instances of non-isomorphic $G$-sets giving rise to isomorphic permutation representations. 

\begin{example}
    The irreducible representations of $G = S_3$ are the trivial representation $\trivial$, the sign representation $\epsilon$ and the $2$-dimensional representation $\rho$.
    We have
    \begin{table}[H]
        \centering
    \begin{tabular}{l l l l l l l}
        $\bC[G / C_1]$ & $=$ & $\trivial \oplus \epsilon \oplus \rho^{\oplus 2},$ & $\qquad$ &
        $\bC[G / C_2]$ & $=$ & $\trivial \oplus \rho,$\\ 
        $\bC[G / C_3]$ & $=$ & $\trivial \oplus \epsilon,$ & $\qquad$ &
        $\bC[G / G]$ & $=$ & $\trivial.$  
    \end{tabular}
\end{table}
    Then $\Psi = C_1  - 2 C_2 - C_3 + 2S_3$ is the unique Brauer relation for $G$.
\end{example}

\begin{example}\label{cyclic-no-brauer}
Cyclic groups have no Brauer relations. Indeed, if $G = C_n$, the $\bZ$-rank of $\Perm(G)$ is the number of cyclic subgroups of $C_n$, i.e the number of subgroups of $C_n$, which is the $\bZ$-rank of $\B(G)$. Hence the rank of the kernel of the map $\B(G) \to \Perm(G)$ is zero.
\end{example}

In the last section, we described how to obtain a virtual permutation representation from an arbitrary representation of $G$. We are interested in when this is an image of an element from the Burnside ring.

\begin{defn}
Let $\rho$ be a representation of $G$.    
We call $\Theta = \sum_i n_i H_i \in \B(G)$ a \textbf{$\rho$-relation} if $\bC[G / \Theta] \simeq \repnorm{\rho}^{\ \oplus m}$, for some $m \geq 1$.
\end{defn}

If $D \leq G$, then one can pass from virtual permutation representations of $G$ to virtual permutation representations of $D$ via restriction, and in the other direction via induction. We define analogous maps for the Burnside ring.    

\begin{defn}
    For $D \leq G$, define maps $\Res_D \colon \B(G) \to \B(D)$ and $\Ind_D \colon \B(D) \to \B(G)$ by
    \[  \Res_D H = \sum_{x \in H \backslash G / D} D \cap H^{x^{-1}}, \qquad \quad \Ind_D H = H. \]
    These correspond to the representation theory side, where $\Res_D \Ind_{H}^G \trivial = \sum_{x \in H \backslash G / D} \Ind_{D \cap H^{x^{-1}}}^D \trivial$ (Mackey's decomposition, cf. \cite[Chapter 7, \S 7.3]{Serre}, $H^{x^{-1}} = x^{-1}H x$), and $\Ind_{D}^G\Ind_{H}^D \trivial = \Ind_{H}^G \trivial$.
\end{defn}
%\begin{rem}
%If such a relation exists, then $m$ is a multiple of the order of $\repnorm{\rho}$ in $C(G)$. Note that if $\Theta$ is a $\rho$-relation and $\Psi \in B(G)$ is a Brauer relation, then $\Theta + \Psi$ is also a $\rho$-relation. It follows that for a fixed $m \geq 1$, if $\repnorm{\rho}^{\oplus m} \in \Perm(G)$ then there are $\#$(Brauer relations of $G$) $+ 1$ elements $\Theta \in \B(G)$ with $\bC[G / \Theta] = \repnorm{\rho}^{\oplus m}$.
%\end{rem}

The following are some elementary properties of these relations:

\begin{prop} Let $\rho$ be a representation of $G$, $\Theta = \sum_i n_i H_i \in \B(G)$ a $\rho$-relation. Then,
    \begin{enumerate}
        \item $n \Theta$ is a $\rho$-relation for all $n \geq 1$.
        \item $\bC[G / \Theta] \simeq \repnorm{\rho}^{\oplus m}$ where $m$ is a multiple of the order of the character of $\repnorm{\rho}$ in $\C(G)$.
        \item If $\Psi \in B(G)$ is a Brauer relation, then $\Theta + \Psi$ is also a $\rho$-relation. 
        \item For a fixed $m \geq 1$, if $\repnorm{\rho}^{\oplus m}$ is a virtual permutation representation then there are $\#$(Brauer relations of $G$) $+ 1$ elements $\Theta \in \B(G)$ with $\bC[G / \Theta] = \repnorm{\rho}^{\oplus m}$.
        
        \item (Projection) If $N \trianglelefteq G$ then $(N \cdot \Theta) / N = \sum_i n_i N H_i / N$ is a $\rho^N$-relation, viewing this relation as an isomorphism of representations of $G / N$.
        \item (Restriction) For $D \leq G$, $\Res_D \Theta$ is a $\Res_D \rho$-relation.
    \end{enumerate}
\end{prop}

\begin{proof}
    All but (5) are clear. For (5), observe that for $H \leq G$, $\bC[G / H]^N \simeq \bC[G / NH]$ as $G$-representations for $N$ normal (see proof of \cite[Theorem 2.8]{reg-const}). We also need to show that $\repnorm{\rho}^N \simeq \repnorm{\rho^N}^{\oplus k}$ for some $k \geq 1$. This is the case; 
    $$\repnorm{\rho}^N \simeq 
    \bigoplus_{\fg \in \Gal(\bQ(\rho) / \bQ)} (\rho^{\fg})^N \simeq \bigoplus_{\fg \in \Gal(\bQ(\rho) / \bQ)} (\rho^N)^{\fg} = \repnorm{\rho^N}^{\oplus k},$$
    where $k = [ \bQ(\rho) : \bQ(\rho^N)]$. 
\end{proof}


\begin{example}\label{cyclic-relns}[\cite[Exercise 13.1]{Serre}]
    Let $G = C_n$. For each $d \mid n$, let $\chi_d = \repnorm{\varphi_d}$, where $\varphi_d$ is an irreducible complex character of $G$ with field of values $\bQ(\zeta_d)$ and kernel of index $d$.
    Then $\{ \chi_d \colon d\mid n \}$ form an orthogonal basis for the irreducible rational-valued representations of $G$. Since $C_{n / d} \trianglelefteq G$, $\Ind_{C_{n/ d}}^G \trivial$ is the direct sum of irreducible complex representations of $G$ containing $C_{n / d}$ in their kernel. Thus, $\Ind_{C_{n/ d}}^G \trivial \simeq \sum_{d' \mid d} \chi_{d'}$. Applying M\"{o}bius inversion, we obtain a $\varphi_d$-relation for each $d \mid n$:
    \[ \chi_d = \sum_{d' \mid d} \mu(d / d') \cdot \Ind_{C_{n/ d'}}^G \trivial. \]
    Note that this is the only way of writing $\chi_d$ as a sum of permutation representations, since cyclic groups have no Brauer relations (Example \ref{cyclic-no-brauer}). Similarly, there is a unique $\Theta \in B(G)$ such that $\bC[G / \Theta] \simeq \chi_d^{m}$ for all $m \geq 1$.
    \end{example}
