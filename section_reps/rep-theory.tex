Hopefully the last section will have convinced the reader that knowledge of representation theory of finite groups $G$, in particular relations between permutation representations of $G$, has useful applications to the study of elliptic curves.  Specifically, in this section we introduce some theory and notation that will  help us prove general results about when we obtain norms from expressions in Theorem \ref{thm_positive_rank}. 

The first two subsections discuss rational representations of $G$, and writing these as a sum of virtual permutation representations.
In the third subsection, we consider functions defined on subgroups of $G$, corresponding to subfields of $F / \bQ$. Our main example of interest is the function sending $H \mapsto C_{E / F^H}$ for $H \leq G$ (when $E / \bQ$ is an elliptic curve). We also describe \textit{$D$-local functions}, which are functions that depend on a decomposition group, borrowing definitions that appear in \cite[Section 2.iii]{reg-const}.

We have attempted to make this section mainly representation-theoretic, in case the results within can be applied in other contexts.

\subsection{Rational Characters and Permutation Representations}\label{rep}
%\begin{defn}     
%\begin{enumerate}
%    \setlength\itemsep{0em}
%    \item A \textbf{representation} of $G$ over $K$ is a group homomorphism $\rho \colon G \to \GL(V)$ where $V$ is a $K$-vector space.
%    \item  Associated to a representation $\rho$ is a \textbf{character} $\chi \colon G \to K^{\times}$, defined by letting $\chi(g) = \Tr \rho(g)$ for $g \in G$. 
%\end{enumerate}
%\end{defn}

%If we don't specify the field a representation is defined over, assume it is defined over $\bC$. For complex representations, $\rho$ is determined by its character; if $\rho$, $\rho'$ are representations with identical characters, then $\rho$ and $\rho'$ are isomorphic as representations (cf. \cite[Chapter 2, \S 2.3]{Serre}). 

%\begin{defn}\label{rep-ring}
%Let $\chi_1, \ldots,  \chi_h$ be the distinct characters of the complex irreducible representations of $G$, where $h = \#$ conjugacy classes of $G$. 
%Then the \textbf{representation ring} of $G$ is 
%\[ \R(G) = \bZ \chi_1 \oplus \cdots \oplus \bZ \chi_h .\]
%We can also view this as the Grothendieck group of the category of finitely generated $\bC[G]$-modules.
%\end{defn}
%Since we take differences of characters in $\R(G)$, we call elements of $\R(G)$ \textbf{virtual representations}. 
Let $G$ be a finite group, $K$ a number field.
Denote by $\R_K(G)$ the group generated by characters of the representations of $G$ over $K$. A representation of $G$ is then defined over $K$ if and only if its character is in $\R_K(G)$ (\cite[Proposition 33]{Serre}).
$\R_K(G)$ is a subring of $\R_{\bC}(G)$, where $\R_{\bC}(G)$ is finitely generated by $\Irr(G)$; the irreducible characters of $G$ over $\bC$.
When $K = \bQ$ this is called the \textbf{rational representation ring}.
The characters of the distinct irreducible representations of $G$ over $K$ form an orthogonal basis of $\R_K(G)$ with respect to the usual inner product of characters of $G$ (\cite[Proposition 32]{Serre}).
Let $m$ be the exponent of $G$. If $K$ contains the $m$-th roots of unity, then $\R_K(G) = \R_{\bC}(G)$ (\cite[Theorem 24]{Serre}). This implies every representation of $G$ can be realized over such $K$. 
\vspace{1em}

%The rank of $R_K(G)$ is discussed in {\color{red} Serre 12.4}.
Let $\Perm(G)$ be the ring of virtual permutation representations of $G$ (i.e. the ring generated by the characters of $\bC[G / H] = \Ind_{H}^G \trivial$ for $H \leq G$). Let $\Char_{\bQ}(G)$ be the ring of rationally valued characters of $G$. Then we have inclusions 
\[ \Perm(G) \to \R_{\bQ}(G) \to \Char_{\bQ}(G). \]
Each of these groups have equal $\bZ$-rank, equal to the number of conjugacy classes of cyclic subgroups of $G$ (\cite[Chapter 13, \S13.1]{Serre}). Moreover the cokernels of these maps are finite.

\begin{defn}\label{rho-norm}
    Let $\rho$ be a representation of $G$. We define the norm of $\rho$, denoted $\repnorm{\rho}$, by 
    \[
    \repnorm{\rho} \defeq \sum_{\fg \in \Gal(\bQ(\rho)/\bQ)}  \rho^\fg \quad,
    \]
    where $\bQ(\rho)$ is the abelian extension of $\bQ$ generated by $\{ \Tr \rho(g) \colon g \in G \}$, and $\rho^\fg$ is the representation of $G$ such that $\Tr \rho^{\fg}(g) = \fg(\Tr \rho(g))$ for $g \in G$. 
\end{defn}

It's clear that $\Char_{\bQ}(G)$ is generated by the characters of $\repnorm{\rho}$ as $\rho$ ranges over the complex irreducible representations of $G$. Indeed, if a representation has a rationally valued character, then any complex irreducible constituent must occur along with all its Galois conjugates with equal multiplicity.

\begin{rem}
This is not additive, i.e. one does not have \[\repnorm{\rho + \tau} = \repnorm{\rho} + \repnorm{\tau}. \] 
This does hold when $\bQ(\rho) = \bQ(\tau)$. 
\end{rem}

\begin{example}
    Let $G = C_p$ and $\psi_p$ a character of order $p$. Then $\bQ(\psi_p) = \bQ(\zeta_p)$ and $\repnorm{\psi_p}$ is the sum over the $p - 1$ non-trivial characters of $G$. But $\repnorm{\psi_p + \trivial} = \trivial^{\oplus (p - 1)} + \repnorm{\psi_p} \not= \repnorm{\trivial} + \repnorm{\psi_p}$.
\end{example}

%Conversely,  Therefore our map $R(G) \to \Char_{\bQ}(G)$ is surjective.

%Such a character may not be in $R_{\bQ}(G)$, however. That is, it has rational character, but the corresponding representation cannot be realized over $\bQ$. The quotient $\Char_{\bQ}(G) / R_{\bQ}(G)$ is the study of Schur indices.  If $\rho \in R(G)$ is an irreducible representation, the \textbf{Schur index} is the smallest integer $m(\rho)$ such that 
%\[ \sum_{\sigma \in \Gal(\bQ(\rho)/\bQ)}m(\rho) \cdot \rho^\sigma \quad \in R_{\bQ}(G). \]

\begin{rem}\label{image-of-burnside}
The group $$\C(G) \defeq \frac{\Char_{\bQ}(G)}{\Perm(G)}$$ is a finite abelian group, of exponent dividing $|G|$ (this follows from Artin's induction theorem, \cite[Theorem 17]{Serre}). The study of this group is quite subtle. For us, it's enough to know that given a representation $\rho$ of $G$, there exists a minimum integer $m$, depending on $\rho$ and dividing $|G|$, such that 
\[ \repnorm{\rho}^{\ \oplus m } = \bigoplus_i \Ind_{H_i}^G \trivial \ominus \bigoplus_j \Ind_{H_j'}^{G} \trivial \]
for some subgroups $H_i$, $H_j' \leq G$, i.e. that the character of $\repnorm{\rho}^{\ \oplus m }$ is in $\Perm(G)$. This minimum integer $m$ is the order of the character of $\repnorm{\rho}$ in $\C(G)$. 
\end{rem}
%Thus, we have a map $\Irr(G) \to \Perm(G)$. We extend this additively to a map $R(G) \to \Perm(G)$.
\begin{example}
If $G = C_n$ then $\C(G)$ is trivial (see Example \ref{cyclic-relns}). $\C(G)$ is also trivial for the symmetric groups $G = S_n$. 
\end{example}

 \begin{example}
    $G = Q_8$, the quaternion group, has $\C(G) = \bZ / 2 \bZ$. Let $\rho$ be the faithful irreducible representation of $G$ of dimension $2$. Its character $\chi$ is rational and one has 
    \[ \rho^{\oplus 2} = \Ind_{C_1}^G \trivial \ominus \Ind_{C_2}^G \trivial, \]
    but one cannot write $\rho$ as a virtual permutation representation ($\chi$ has Schur index $2$ so $\chi \not\in \R_{\bQ}(G)$).  
 \end{example}

%\begin{notn}
%For $\rho \in R_{\bC}(G)$ an irreducible character let 
%\[  \repnorm{\rho} = \sum_{\sigma \in \Gal(\bQ(\rho)/\bQ)}m(\rho)\cdot \rho^\sigma \quad \in R_{\bQ}(G) , \]
%where $m(\rho) \in \bZ$ is the Schur index of $\rho$.
%\end{notn}
%Then $\repnorm{\rho}$ is the character of an irreducible rational representation. Every irreducible rational representation can be obtained this way. We can extend this map additively to a surjective map $R_{\bC}(G) \to R_{\bQ}(G)$. 

%Let $\overline{R_K(G)}$ be the subring of elements of $R(G)$ which have values in $K$. Then $R_K(G) \subset \overline{R_K(G)}$ and this inclusion is of finite index. 
%Given a character $\chi$ of $G$, let $\bQ(\chi)$ be the smallest subfield of $\bC$ containing $\{ \chi(g) \mid g \in G \}$.
%Let $R_{\bC}(G)$ denote the ring of characters of complex representations of $G$. The number of complex irreducible representations of $G$ is equal to the number of conjugacy classes of $G$. Let $R_{\bQ}(G)$ be the ring of characters of rational valued representations of $G$.
%The number of irreducible $\bQ G$-representations up to isomorphism is equal to the number of conjugacy classes of cyclic subgroups of $G$. %(\cite[$\mathsection 13.1$, Cor. 1]{Serre})
%Induction, Restriction\dots
%\begin{thm}[Mackey Decomposition] 
%\end{thm} 
