\subsection{Functions on the Burnside ring and norm relations}

Consider a multiplicative function $f \colon B(G) \to A$, where $A$ is an abelian group. As in \cite{reg-const}, say that $f$ is \textbf{representation theoretic} if $f$ is trivial on Brauer relations. This means that for a $G$-set $X$, $f$ only depends on the representation $\bC[X]$. 

\begin{example}
  Let $V$ be a representation of $G$. The function $\psi(H) = \dim V^H$ is trivial on Brauer relations, as $\dim V^H = \langle \Res_{H} V , \trivial_H \rangle = \langle V, \Ind_{H}^G \trivial \rangle$ by Frobenius reciprocity.
\end{example}

We want to extend this notion and consider functions that are trivial on $\rho$-relations.
Take a multiplicative function on the Burnside ring of the form $\psi \colon B(G) \to \bQ^{\times}$. Given $\rho \in R_{\bC}(G)$ we can extend such functions from the Burnside ring to $\overline{\psi} \colon B(G) \to \bQ^{\times} / N_{\bQ(\rho) / \bQ}(\bQ(\rho)^\times)$. {\color{red} try motivate this a bit better, e.g. why do we expect functions to give norms... try relate this back to introduction}

\begin{defn}
If $\Theta \in \ker \overline{\psi}$, then $\psi(\Theta)$ is the norm of an element from $\bQ(\rho)^{\times}$. We call an instance of this a \textbf{norm relation}.
\end{defn}

\begin{defn}
 We say two functions $\psi$, $\psi'$ are \textbf{$\rho$-equivalent}, written $\psi \sim_{\rho} \psi'$, if $\overline{\psi /\psi'}$ is trivial on all $\rho$-relations. Equivalently, $\psi(\Theta) / \psi'(\Theta)$ is a norm relation for all $\rho$-relations $\Theta$. 
\end{defn}

If a function $f$ satisfies $f \sim_{\rho} 1$, we say $f$ is trivial on $\rho$-relations. 

\begin{example}
    Let $G = C_p$ for $p$ a prime. Let $\rho$ be a character of degree $p$. There is a unique $\rho$-relation given by $\Theta = C_1 - C_p$. Let $\psi(H) = [G \colon H]$. Then $\psi(\Theta) = p$, which is a norm from $\bQ(\sqrt{p^*}) \subset \bQ(\zeta_p)$ by Corollary \ref{p-norm}. 
\end{example}

\begin{example}
Let $\bQ(\rho)$ be a quadratic field. Then if $f \colon B(G) \to \bQ^{\times}$ satisfies $f(\Theta) \in \bQ^{\times 2}$ for all $\rho$-relations $\Theta$, one has $f \sim_{\rho} 1$.
\end{example}

\begin{example}
Let $E / \bQ$ be an elliptic curve, $G = \Gal(F / \bQ)$ for $F / \bQ$ a Galois extension. For $H \leq G$, the function $\psi \colon H \mapsto C(E / F^H)$ extends to a multiplicative function on the Burnside ring. Given a representation $\rho$ of $G$, one can ask when $\psi \sim_{\rho} 1$.
\end{example}

\subsection{D-local functions}\label{D-loc}

For our application of functions on the Burnside ring to {\color{red} reference something in intro}, we introduce some of the concepts and notation introduced in \cite[Section 2.iii]{reg-const}.

We are interested in functions on the Burnside ring that are number-theoretic in nature, where we take $G$ to be a Galois group. Often, these functions are \textit{local}. For example, let $G = \Gal(F / K)$ and $D$ the decomposition group at the place $v$. For $H \leq G$, the number of primes in $L = F^{H}$ above $v$ are in one-to-one correspondence with the double cosets $D \backslash H / G$. We can use the function $f \colon B(G) \to \bQ[x]^{\times}$ given by $H \mapsto  x^{| H \backslash G / D|}$ to describe the number of places above $v$ in any intermediate extension of $F / K$. But if we let $g \colon B(D) \to \bQ[x]^{\times}$ be defined by $H \mapsto x$, then 
        \[ f(H) = g\left(\Res_D H\right) = \prod_{x \in H \backslash G / D} g(D \cap H^{x^{-1}}) .\]
Therefore the value of $f$ on any $G$-set $X$ only depends on the structure of $X$ as a $D$-set. 

Such functions motivate the following definition:

\begin{defn}(\cite[Definition 2.33]{reg-const})\label{D-loc-fn}
    If $D \leq G$, we say a function $f$ on $B(G)$ is \textbf{$D$-local} if there is a function $f_D$ on $B(D)$ such that $f(H) = f_D(\Res_D H)$ for $H \leq G$.
    If this is the case, we write
    \[ f = (D, f_D). \]
\end{defn}

\begin{example}\label{tama-ex}
    For $G = \Gal(F / K)$, $v$ a place of $K$ with decomposition group $D$, the function
    \[ H \mapsto \prod_{w | v} c_w(E / F^{H}) \]
    is $D$-local, where $E$ is an elliptic curve over $K$ and $c_w$ is the local Tamagawa number. 
\end{example}

Let $I \triangleleft D$ be the inertia subgroup of the place $v$, so $D / I$ is cyclic. If a prime $w$ in $F^H$ corresponds to the double coset $HxD$, then its decomposition and inertia groups in $F / F^H$ are $H \cap D^x$ and $H \cap I^x$ respectively.
The ramification degree and residue degree of $w$ over $K$ are given by $e_w = \frac{|I|}{|H \cap I^x|}$ and $f_w = \frac{[D \colon I]}{[H \cap D^x \colon H \cap I^x]}$. We will consider functions that depend on $e$ and $f$, and so introduce the following:

\begin{defn}\cite[Definition 2.35]{reg-const}\label{D-I-fn}
    Suppose $I \triangleleft D < G$ with $D / I$ cyclic, and $\psi(e,f)$ is a function of $e, f \in \bN$. Define a function on $B(G)$ by 
    \[ \left(D, I, \psi\right) \colon \quad H \mapsto \prod_{x \in H\backslash G / D} \psi\left(\frac{|I|}{|H \cap I^x|}, \frac{[D \colon I]}{[H \cap D^x \colon H \cap I^x]}\right). \]
    This is a $D$-local function on $B(G)$ with
    \[ (D, I, \psi) = \left(D, U \mapsto \psi(\frac{|I|}{|U \cap I}, \frac{|D|}{|UI|})\right). \]
\end{defn}

\begin{example}
    If $E / K$ has split multiplicative reduction at $v$ with $c_v(E / K) = n$, then $c_w(E / F^H) = e_w n$ for a place $w$ of $F^H$ above $v$. In this case the function in example \ref{tama-ex} is $(D, I, e)$. 
\end{example}

\begin{example}\label{trivial-on-brauer}
    Let $\rho = 0$. If $W$ is a group of odd order, then $(W, W, e) \sim 1$ as functions to $\bQ^{\times} / \bQ^{\times 2}$.
    More generally if $D$ has odd order and $I \triangleleft D$ then $(D, I, e) \sim_{\rho} 1$. {\color{red} explain and reference}
\end{example}

{\color{red} Question: do i need to be worried about m when i restrict a representation? do i need to adjust my definition of rho relations?}
\begin{prop}
    Let $I \triangleleft D \leq G$ with $D / I$ cyclic. Let $\rho \in R(G)$ Then
    \begin{enumerate}
        \item If $f = (D, f_D)$ and $f_D$ is trivial on $(\Res_D \rho)$-relations, then $f$ is trivial on $\rho$-relations.
        \item {\color{red} maybe more things to say here. See e.g. \cite[Theorem 2.36]{reg-const}}  
    \end{enumerate}
\end{prop}


%Our fudge factors $C(E / F)$ are defined locally; one has $C(E / F) = \prod_v c_v(E / F) \cdot |\omega / \omega_{v, \min}|$. Here $v$ runs over finite places of $F$, $\omega$ is a global minimal differential for $E / \bQ$, and $\omega_{v, \min}$ is a minimal differential at $v$.
%Considering the function $H \mapsto C(E / F^H)$, and writing $C_p(E / F^H) =\prod_{v | p} c_v(E / F)\cdot |\omega / \omega_{v, \min}|$ one has
%\[ \sum_{i} n_i H_i \mapsto \prod_i C(E / F^{H_i})^{n_i} = \prod_{p} C_p(E / F^H)^{n_i}. \]
%Therefore, our function is the product of local functions for each $p$. Since $C_p(E / F^H)$ depends on $e_w$, $f_w$ for $w | p$, we are motivated to define the following:
%{\color{red} try make thick brackets}
%\begin{example}
%For semi-stable reduction, we're considering $\psi(e, f) = e$ (the Tamagawa number). For the $d_v$ terms in the case of additive potentially good reduction at p ($p$ not equal to $2$ or $3$), we consider $\psi(e, f) = p^{f \floor{e n /12}}$, where $n \in \{2,3,4,6,9,10\}$.
%\end{example}




