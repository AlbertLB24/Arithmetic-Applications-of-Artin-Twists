\subsection{Functions on the Burnside ring and norm relations}\label{sec-norm-rels}

Consider a multiplicative function $f \colon \B(G) \to A$, where $A$ is an abelian group. As in \cite{reg-const}, we say that $f$ is \textbf{representation theoretic} if $f$ is trivial on Brauer relations. This implies that for a $G$-set $X$, $f$ only depends on the representation $\bC[X]$. In other words, there exists a map $g \colon \R(G) \to A$ such that $f(H) = g(\bC[G / H])$ for all $H \leq G$. 

\begin{example}
  Let $\lambda \in \bR^{\times}$ and consider the function $H \mapsto \lambda^{[G : H]}$. This is trivial on Brauer relations;  if $\sum_i n_i H_i$ is a Brauer relation then $\lambda^{\sum_i n_i [G : H]} = \lambda^{\dim( \oplus_i \bC[G / H_i]^{\oplus n_i})} = 1$.
\end{example}

Let $\rho \in \R(G)$. Let $\Theta \in \B(G)$ be a $\rho$-relation, with $\bC[G / \Theta] \simeq \repnorm{\rho}^{\oplus m}$. If we have a function $f \colon \B(G) \to \bQ^{\times}$, we may then ask whether $f(\Theta) \in \fieldnorm{\rho}$. 

%Take a multiplicative function on the Burnside ring of the form $\psi \colon \B(G) \to \bQ^{\times}$. Given $\rho \in R_{\bC}(G)$ we can extend such functions from the Burnside ring to $\overline{\psi} \colon \B(G) \to \bQ^{\times} /\fieldnorm{\rho}$. {\color{red} motivate this a bit better?}

\begin{defn}
Let $\rho \in \R(G)$, $\Theta$ a $\rho$-relation, and $f \colon \B(G) \to \bQ^{\times}$. If $f(\Theta) \in \fieldnorm{\rho}$, then we call $\Theta$ a \textbf{norm relation} for $f$. 
If $f(\Theta) \in \fieldnorm{\rho}$ for every $\rho$-relation in $\B(G)$, then we say $f$ is \textbf{trivial on $\rho$-relations}.

    %If $\Theta \in \ker \overline{\psi}$, then $\psi(\Theta)$ is the norm of an element from $\bQ(\rho)^{\times}$. We call an instance of this a \textbf{norm relation}.
\end{defn}

%\begin{defn}
% We say two functions $\psi$, $\psi'$ are \textbf{$\rho$-equivalent}, written $\psi \sim_{\rho} \psi'$, if $\overline{\psi /\psi'}$ is trivial on all $\rho$-relations. Equivalently, $\psi(\Theta) / \psi'(\Theta)$ is a norm relation for all $\rho$-relations $\Theta$. 
%\end{defn}

%If a function $f$ satisfies $f \sim_{\rho} 1$, we say $f$ is trivial on $\rho$-relations. 

\begin{example}
    Let $G = C_p$ for $p$ a prime. Let $\rho$ be a character of degree $p$, so $\bQ(\rho) = \bQ(\zeta_p)$. There is a unique $\rho$-relation given by $\Theta = C_1 - C_p$. Let $\psi \colon \B(G) \to \bQ^{\times}$ be given by $\psi(H) = [G \colon H]$. Then $\psi(\Theta)$ is a norm relation, as $\psi(\Theta) = p \in \fieldnorm{\rho}$ is the norm of $1 - \zeta_p$.
\end{example}

In general, showing that a $\rho$-relation $\Theta$ is a norm relation for $f$ does not imply that this is the case for all possible $\rho$-relations. Under certain circumstances however, we can.

\begin{prop}\label{min-to-all}
    Let $\rho \in \R(G)$ and $f \colon \B(G) \to \bQ^{\times}$. Suppose that $f(\Psi) \in \fieldnorm{\rho}$ for every Brauer relation $\Psi \in \B(G)$. 
    Let $\Theta \in \B(G)$ be a $\rho$-relation, with $\bC[G / \Theta] = \repnorm{\rho}^{\oplus m}$, where $m$ is the order of $\repnorm{\rho}$ in $\C(G)$. 
    If $\Theta$ is a norm relation for $f$, then $f$ is trivial on $\rho$-relations. 
\end{prop}

\begin{proof}
    Consider an arbitrary $\rho$-relation $\Theta'$ such that $\bC[G / \Theta'] = \repnorm{\rho}^{\oplus l}$ for some $l \geq 1$. Then $m \mid l$ and $\Psi = \Theta' - \frac{l}{m}\Theta$ is a Brauer relation. Thus
    \[ f(\Theta') = f(\Psi)\cdot f(\Theta)^{\frac{l}{m}} \in \fieldnorm{\rho} \]
    and so $f$ is trivial on all $\rho$-relations.
\end{proof}

\begin{example}
    Let $G = C_n$. Then any function $f \colon \B(G) \to \bQ^{\times}$ is trivial on Brauer relations since $G$ does not have any. Let $\varphi_d$ be an irreducible complex character of $G$ of order $d \mid n$. Thus to conclude that $f$ is trivial on $\varphi_d$-relations, it is enough to show that the $\varphi_d$-relation constructed in Example \ref{cyclic-relns} is a norm relation for $f$.
\end{example}

%\begin{example}
%Let $\bQ(\rho)$ be a quadratic field. Then if $f \colon \B(G) \to \bQ^{\times}$ satisfies $f(\Theta) \in \bQ^{\times 2}$ for all $\rho$-relations $\Theta$, one has $f \sim_{\rho} 1$.
%\end{example}

\begin{example}
Let $E / \bQ$ be an elliptic curve, $G = \Gal(F / \bQ)$ for $F / \bQ$ a Galois extension. The function $\psi \colon H \mapsto C_{E / F^H}$ (\color{red} ref) for $H \leq G$ extends to a multiplicative function on the Burnside ring. In later sections we will investigate when this is trivial on $\rho$-relations, for $\rho \in \R(G)$. 
\end{example}

%\begin{prop}
   % Let $G = C_n$. Let $\rho \in \R(G)$ and consider the function $g \colon \B(G) \to \bQ^{\times}$ given by $H \mapsto [G : H]$. Then $g$ is trivial on $\rho$-relations.
%\end{prop}

%\begin{proof}
 %   Let $k$ be the minimum positive integer such that $\bQ(\rho) \subset \bQ(\zeta_k)$. Then $k = 2^{a}q_1\cdots q_t$ where $q_i$ are odd primes and $a \in \{ 0, 1, 2, 3\}$.
  %  Observe that $k \mid n$ since all characters of $G$ are realized over $\bQ(\zeta_n)$. Since $\C(G) = 1$ and $G$ has no Brauer relations, by Proposition \ref{min-to-all} it suffices to show that $\Theta \in \fieldnorm{\rho}$ for $\Theta \in \B(G)$ such that $\bC[G / \Theta] = \repnorm{\rho}$.
   % Let $\repnorm{\rho} = \sum_{d \mid n} a_d \chi_d$ where $a_d \in \bZ$ and $\chi_d$ are the basis of irreducible rational representations of $C_n$ defined in Example \ref{cyclic-relns}. %(recall $\chi_{d} = \repnorm{\varphi_{d}}$ with $\bQ(\varphi_{d}) = \bQ(\zeta_{d})$ form a basis for the irreducible rational representations of $C_n$). 
    %Let $\Psi_{d} = \sum_{d' | d}\mu(d / d')\cdot C_{n / d}$ so that $\bC[\Psi_{d}] = \chi_{d}$, as observed in the example. Then $\bC[G / \Theta] \simeq \bC[\sum_{d | n } a_{d} \Psi_{d}]$ which implies that $\Psi = \sum_{d | n } a_{d} \Psi_{d}$.

    %Evaluating $g$ on $\Psi_{d}$ is trivial unless $d = q^a$ for some $q$ prime, $a \geq 1$. Indeed, if $d = p_1^{e_1} \cdots p_r^{e_r}$ , with $r \geq 2$ and $e_i \geq 1$, then
    %\[ \prod_{d' \mid d} (d')^{\mu(d / d')} = \prod_{j_1, \ldots j_r \in \{0,1\}^r } \left(p_1^{e_1 - j_1} \cdots p_r^{e_r - j_r}\right)^{\# j_i = 1} = \prod_{i = 1}^r \left(\frac{p_i^{e_i}}{p_i^{e_i - 1}}\right)^{\sum_{ j = 0}^{r - 1} \binom{r-1}{j} (-1)^j} = 1. \]
    %On the other hand,
    %\[ \prod_{d' \mid q^a} (d')^{\mu(q^a / d')} = q .\]
    
    %The irreducible representations of $C_n$ over $\bQ(\rho)$ are given by the orbits of the complex irreducible characters of $C_n$ acted upon by $H = \Gal (\bQ(\zeta_n) / \bQ(\rho))$. Consider $d \mid n$ with $(k, d) = 1$. Recall that $\chi_d = \repnorm{\varphi_d}$, where $\bQ(\varphi_d) = \bQ(\zeta_d)$. Let $B = \Gal(\bQ(\zeta_n) / \bQ(\zeta_d))$. Then $\bQ(\rho) \cap \bQ(\zeta_d) = \bQ$, so $BH = \Gal(\bQ(\zeta_n) / \bQ)$. The orbit of $\varphi_{d}$ under $H$ is fixed by $BH$, hence is rational. It follows that $\langle \rho, \varphi_{d} \rangle = \langle \rho^{\sigma} , \varphi_{d} \rangle$ for all $\sigma \in \Gal(\bQ(\rho) / \bQ)$. Thus $[\bQ(\rho) : \bQ]$ divides $a_d$, and so $g(a_d \Psi_d) = d(\Psi_d)^{a_d} \in \bQ^{\times [\bQ(\rho) : \bQ]} \subset \fieldnorm{\rho}$.

    %Therefore 
    %\[ g(\Psi) = 2^{a_2} \cdot q_1^{a_{q_1}} \ldots q_t^{a_{q_t}}  \pmod {\fieldnorm{\rho}} \]   

%\end{proof}


\subsubsection{D-local functions}\label{D-loc}

We are interested in functions on the Burnside ring that are number-theoretic in nature, where we take $G$ to be a Galois group. Often, these functions are \textit{local}. For example, let $F / K$ be a finite Galois extension of number fields and let $G = \Gal(F / K)$. Let $D$ be the decomposition group at a place $v$ of $K$. For $H \leq G$, the number of primes in $L = F^{H}$ above $v$ are in one-to-one correspondence with the double cosets $H \backslash G / D$. We can use the function $f \colon \B(G) \to \bQ^{\times}$ given by $H \mapsto  \lambda^{| H \backslash G / D|}$ (for $\lambda \not= \pm 1$) to describe the number of places above $v$ in any intermediate extension of $F / K$. But if we let $g \colon \B(D) \to \bQ^{\times}$ be defined by $H \mapsto \lambda$, then 
        \[ f(H) = g\left(\Res_D H\right) = \prod_{x \in H \backslash G / D} g(D \cap H^{x^{-1}}) .\]
Therefore the value of $f$ on any $G$-set $X$ only depends on the structure of $X$ as a $D$-set.
Such functions motivate the following definition:

\begin{defn}(\cite[Definition 2.33]{reg-const})\label{D-loc-fn}
    If $D \leq G$, we say a function $f$ on $\B(G)$ is \textbf{$D$-local} if there is a function $f_D$ on $\B(D)$ such that $f(H) = f_D(\Res_D H)$ for $H \leq G$.
    If this is the case, we write
    \[ f = (D, f_D). \]
\end{defn}

\begin{example}\label{tama-ex}
    For $G = \Gal(F / K)$, $v$ a place of $K$ with decomposition group $D$, the function
    \[ H \mapsto \prod_{w | v} c_w(E / F^{H}) \]
    is $D$-local, where $E$ is an elliptic curve over $K$ and $c_w$ is the local Tamagawa number. 
\end{example}

Let $I \triangleleft D$ be the inertia subgroup of the place $v$, so $D / I$ is cyclic. If a prime $w$ in $F^H$ corresponds to the double coset $HxD$, then its decomposition and inertia groups in $F / F^H$ are $H \cap D^x$ and $H \cap I^x$ respectively.
The ramification degree and residue degree of $w$ over $K$ are given by $e_w = \frac{|I|}{|H \cap I^x|}$ and $f_w = \frac{[D : I]}{[H \cap D^x : H \cap I^x]}$. We will consider functions that depend on $e$ and $f$, and so introduce the following:

\begin{defn}\cite[Definition 2.35]{reg-const}\label{D-I-fn}
    Suppose $I \triangleleft D < G$ with $D / I$ cyclic, and $\psi(e,f)$ is a function of $e, f \in \bN$. Define a function on $\B(G)$ by 
    \[ \left(D, I, \psi\right) \colon \quad H \mapsto \prod_{x \in H\backslash G / D} \psi\left(\frac{|I|}{|H \cap I^x|}, \frac{[D : I]}{[H \cap D^x : H \cap I^x]}\right). \]
    This is a $D$-local function on $\B(G)$ with
    \[ (D, I, \psi) = \left(D, U \mapsto \psi\left(\frac{|I|}{|U \cap I|}, \frac{|D|}{|UI|}\right)\right). \]
\end{defn}

\begin{example}
    If $E / K$ has split multiplicative reduction at $v$ with $c_v(E / K) = n$, then $c_w(E / F^H) = e_w n$ for a place $w$ of $F^H$ above $v$. In this case the function in Example \ref{tama-ex} is $(D, I, e n)$. 
\end{example}

%\begin{example}\label{trivial-on-brauer}
%    Let $\rho = 0$. If $W$ is a group of odd order, then $(W, W, e) \sim 1$ as functions to $\bQ^{\times} / \bQ^{\times 2}$.More generally if $D$ has odd order and $I \triangleleft D$ then $(D, I, e) \sim_{\rho} 1$. {\color{red} explain and reference}
%\end{example}

%\begin{prop}
%    {\color{red} these aren't right... i need to consider the fields...}
%    Let $I \triangleleft D \leq G$ with $D / I$ cyclic. Let $\rho \in \R(G)$ Then
%    \begin{enumerate}
%        \item If $f = (D, f_D)$ and $f_D$ is trivial on $(\Res_D \rho)$-relations, then $f$ is trivial on $\rho$-relations.
%        \item Let $N \trianglelefteq G$, and $f$ a function on $\B(G)$ such that $f(H) = f_{G / N}(N H / N)$ for some function $f_{G / N}$ on $\B(G / N)$. If $f_{G / N}$ is trivial on $\rho^N$-relations (viewed in $\R(G / N)$ ), then $f$ is trivial on $\rho$-relations.
%    \end{enumerate}
%\end{prop}

One would like to be able to say that if $f = (D, f_D)$ and $f_D$ is trivial on $(\Res_D \rho)$-relations, then $f$ is trivial on $\rho$-relations. But as defined, one cannot conclude this when $[\bQ(\rho) : \bQ(\Res_D \rho) ] > 1$. Under some conditions however, when $\bQ(\Res_D \rho) = \bQ$, one can automatically conclude that $f$ is trivial on $\rho$-relations, as in the following.

\begin{prop}\label{rational-res}
Let $D \leq G$. Consider $\rho \in \R(G)$ with $[\bQ(\rho) : \bQ] = n$, where multiplication by $n$ is injective on $\C(D)$. Consider $f = (D, f_D)$ a $D$-local function on $\B(G)$. 
Suppose that $f_D(\Psi) \in \fieldnorm{\rho}$ for every Brauer relation $\Psi \in \B(D)$.
Then, if $\bQ(\Res_D \rho) = \bQ$, $f$ is trivial on $\rho$-relations.
%\colon \B(G) \bQ^{\times}$ a $D$-local function be a function that is trivial on Brauer relations. We also show that if $\bQ(\Res_{D_p} \rho) = \bQ$, then $(T_{\fP \mid p} \cdot D_{\fP \mid p})(\Theta) \in \bQ^{\times 2}$.
\end{prop}

\begin{proof}
    Let $\Theta$ be a $\rho$-relation, with $\bC[G / \Theta] \simeq \repnorm{\rho}^{\oplus m}$. Then $$\bC[D / \Res_D \Theta] \simeq \repnorm{\Res_D \rho}^{\oplus m n}.$$ 
    The condition on $n$ ensures that there exists $\Theta' \in \B(D)$ with $\bC[D / \Theta'] \simeq \repnorm{\Res_D \rho}^{\oplus m}.$ Then $\Psi = \Res_D\Theta - n\Theta'$ is a Brauer relation for $D$. Thus
    \[ f(\Theta) = f_D(\Res_D \Theta) = f_D(\Psi)f_D(n\Theta') = f_D(\Psi)f_D(\Theta')^n \in \fieldnorm{\rho}, \]
    since $f_D(\Psi) \in \fieldnorm{\rho}$ and $\bQ^{\times n} \subset \fieldnorm{\rho}$. 
\end{proof}

\begin{prop}\label{const-fns}
     Let $D \leq G$ with $D$ of odd order, and $f = (D, \alpha)$ a $D$-local function on $\B(G)$, where $\alpha \in \bQ^{\times}$ is constant. Let $\rho \in \R(G)$ with $[\bQ(\rho) : \bQ ]$ even. Then for a $\rho$-relation $\Theta$, $f(\Theta) \in \bQ^{\times 2}$. 
\end{prop}

\begin{proof}
    The function $(D, \alpha)$ on $\B(G)$ sends $H \leq G$ to $\alpha^{| H \backslash G / D|}$. Let $\Theta = \sum_i n_i H_i$ be a $\rho$-relation with $\bC[G / \Theta] \simeq \repnorm{\rho}^{\oplus m}$ for some $m \geq 1$. One has $(D, \alpha) = \alpha^{\sum_i n_i \cdot | H_i \backslash G / D|}$. We show that $\sum_i n_i \cdot | H_i \backslash G / D |$ is even.

    One has $\Res_D \Theta = \sum_i n_i \sum_{x \in H_i \backslash G / D} D \cap H_i^{x^{-1}}$, and $\bC[D / \Res_D \Theta]$ has even dimension, since it is isomorphic to $\sum_{\Gal(\bQ(\rho) / \bQ)} \Res_D \rho^{\sigma}$ and $\Gal(\bQ(\rho) / \bQ)$ has even order. The dimension is $$\sum_i n_i \sum_{x \in H_i \backslash G / D} [D : D \cap H_i^{x^{-1}} ].$$ Since each $[D : D \cap H_i^{x^{-1}} ]$ is odd, this implies there are an even number of terms in the summation, i.e. that $\sum_i n_i \cdot | H_i \backslash G / D|$ is even. 

\end{proof}

%Our fudge factors $C(E / F)$ are defined locally; one has $C(E / F) = \prod_v c_v(E / F) \cdot |\omega / \omega_{v, \min}|$. Here $v$ runs over finite places of $F$, $\omega$ is a global minimal differential for $E / \bQ$, and $\omega_{v, \min}$ is a minimal differential at $v$.
%Considering the function $H \mapsto C(E / F^H)$, and writing $C_p(E / F^H) =\prod_{v | p} c_v(E / F)\cdot |\omega / \omega_{v, \min}|$ one has
%\[ \sum_{i} n_i H_i \mapsto \prod_i C(E / F^{H_i})^{n_i} = \prod_{p} C_p(E / F^H)^{n_i}. \]
%Therefore, our function is the product of local functions for each $p$. Since $C_p(E / F^H)$ depends on $e_w$, $f_w$ for $w | p$, we are motivated to define the following:
%{\color{red} try make thick brackets}
%\begin{example}
%For semi-stable reduction, we're considering $\psi(e, f) = e$ (the Tamagawa number). For the $d_v$ terms in the case of additive potentially good reduction at p ($p$ not equal to $2$ or $3$), we consider $\psi(e, f) = p^{f \floor{e n /12}}$, where $n \in \{2,3,4,6,9,10\}$.
%\end{example}




