As we discussed in Section \ref{sec_pos_rank}, our motivation is to use Theorem \ref*{thm_positive_rank} to predict points of infinite order for families of elliptic curves. However, in Sections \ref{sec_cyclic} and \ref{sec_odd} we prove that in the theorem will never make such a prediction with the group is either cyclic or has odd order. In other words, in such cases, and after fixing some representation $\rho$ of the Galois group, the product
\begin{equation}\label{eqn_localprod}
    \frac{\prod_i C_{E/F_i}}{\prod_j C_{E/F_j'}}
\end{equation}
arising from a $\rho$-relation is always a norm from every subfield $\QQ(\sqrt{D})\subseteq\QQ(\rho)$. The aim of this section is to introduce useful notation {\color{red} Are we finally introducing notation in this section?} and important results that will be used throughout in the next sections, where the main results of the document are proven. The results are number theoretic in nature, and we exhibit two technical consequences of these results on elliptic curves.

\subsection{Expressing Local Data as Functions on the Burnside Ring}

We extend the notation introduced in Notation \ref{not_contr} by defining functions on $\B(G)$. 

\begin{notation}\label{not_contr_fns}
    If $G = \Gal(F / K)$ then for $\fp \in K$ we define functions $T_{\fP \mid \fp}$, $D_{\fP \mid \fp}$ and $C_{\fP \mid \fp}$ on $\B(G)$ by 
    \[ T_{\fP \mid \fp}(H) = T_{\fP \mid \fp}(E / F^H), \quad D_{\fP \mid \fp}(H) = D_{\fP \mid \fp}(E / F^H), \quad C_{\fP \mid \fp}(H) = C_{\fP \mid \fp}(E / F^H), \]
    as defined in Notation \ref{not_contr}.
    Note that if $H$, $H'$ are conjugate then $F^H$, $F^{H'}$ are isomorphic, and so the values of these functions are constant on conjugate subgroups, hence they are well-defined. When $K = \bQ$ we write $p$ instead of $\fp$. 
    
    Define the global contributions $C \colon \B(G) \to \bQ^{\times}$, $T \colon \B(G) \to \bQ^{\times}$ and $D \colon \B(G) \to \bQ^{\times}$ by 
    \[ C(H) = C_{E / F^H}, \quad T(H) = T(E / F^H), \quad D(H) = D(E / F^H). \] 
    Of course one then has $C = T \cdot D = \prod_{\fp} C_{\fP \mid \fp}$, ranging over the primes $\fp$ of $K$. 
\end{notation}

\begin{rem}\label{Rem-C-D-loc}
Note that $C_{\fP \mid p}$ is a $D_p$-local function. Indeed, suppose $D_p = \Gal(F_w / \bQ_p)$, where $F_w$ denotes the completion of $F$ with respect to a place $w$ lying above $p$. For a number field $K$ and place $v$, define $$C_v(E / K) = c_v(E / K) \cdot \left| \omega / \omega_v^{\min} \right|_v.$$ We use the same notation if $K$ is a local field (then the $v$ subscript holds no meaning).
One has
\begin{equation*}
    C_{\fP \mid p} = (D_p, C_v)
\end{equation*}
where $C_v$ is a function on $\B(D_p)$ sending $H \mapsto C_v(E / F_w^H)$.
\end{rem}

The following proposition describes these functions in the language introduced in Section \ref{sec-norm-rels} for each reduction type of $E / \bQ$. We do not attempt to write a formula for $T_{\fP \mid p}$ in the case of additive reduction, computing this involves using Lemma \ref{lem_add_tam}.

\begin{prop}\label{prop_local_fns}
    Let $E / \bQ$ be an elliptic curve, $G = \Gal(F / \bQ)$ and $p$ a prime of $\bQ$. Let $n = v_p(\Delta_E)$. Consider the functions $C_{\fP \mid p}$, $T_{\fP \mid p}$, and $D_{\fP \mid p}$ on $\B(G)$ defined above. Then,
    \begin{enumerate}[(i)]
        \setlength\itemsep{0em}
        \item If $E / \bQ_p$ has good reduction, $C_{\fP \mid p} = 1$,
        \item If $E / \bQ_p$ has split multiplicative reduction then $C_{\fP \mid p} = T_{\fP \mid p} = (D_p, I_p, e n)$,
        \item If $E / \bQ_p$ has non-split multiplicative reduction, 
        $C_{\fP \mid p} = T_{\fP \mid p} = \left(D_p, I_p,
        \left\{\begin{smallmatrix}
            2   & 2 \mid en, 2 \nmid f,  \\
            en   &  2 \mid f, \\
            1   & \text{else}
        \end{smallmatrix}\right.\right),$ 
        \item If $E / \bQ_p$ has potentially good reduction and $p \not= 2, 3$, $D_{\fP \mid p} = (D_p, I_p, p^{f \floor{e n /12}})$, 
        \item If $E / \bQ_p$ has potentially multiplicative reduction and $p \not= 2, 3$, $D_{\fP \mid p} = (D_p, I_p, p^{f \floor{e / 2}})$.
    \end{enumerate}  
\end{prop} 
 
\begin{proof}
    \
    \begin{enumerate}[(i)]
        \setlength\itemsep{0em}
        \item Clear. 
        \item Lemma \ref{lem_Dterms}(i) implies $D_{\fP \mid p} = 1$. If $K' / \bQ_p$ is a finite extension of ramification degree $e$, then $E / K'$ has split multiplicative reduction of type $\I_{en}$, which has Tamagawa number $en$ by Lemma \ref{lem_mult_tam}.
        \item As for split, $D_{\fP \mid p} = 1$. The description follows from applying Proposition \ref{prop_semi_red} (iii) (non-split becomes split when the residue degree is even), and Lemma \ref{lem_mult_tam}. 
        \item Follows from Lemma \ref{lem_Dterms}(ii),
        \item Follows from Lemma \ref{lem_Dterms}(iii).
    \end{enumerate}
\end{proof}
%An immediate consequence of this notation is the fact that 
%$$C_{E/F}=\prod_{\pp}C_{\mathfrak{P}\mid \pp}(F/K);$$
%that is, we can calculate $C_{E/F}$ by calculating the contribution locally at each prime of $K$. 
%{\color{red} also important to mention at some point that if the reduction is semistable, then the terms in a norm relation coming from the discriminant also vanish. Probably this would have to be introduced later.}

\begin{rem}\label{rephrase-thm}
    We rephrase Theorem \ref{thm_positive_rank} in the language introduced in $\S$\ref{sec-norm-rels}. 
    Replacing $\rho$ by the sum of its conjugates by elements of $ \Gal(\bQ(\rho) / \bQ(\sqrt{D}))$, we may assume that $\bQ(\rho) = \bQ(\sqrt{D})$. Note that this does not affect the order of $\rho$ in $\C(G)$, nor the set of $\rho$-relations (since $\repnorm{\rho}$ is unchanged). 
    
    Let $\Theta$ be a $\rho$-relation with $\bC[G / \Theta] = \repnorm{\rho}^{\oplus m}$. Let $C \colon \B(G) \to \bQ^{\times}$ be the function sending $H \mapsto C_{E / F^H}$. The theorem then states that, if $\Theta$ is not a norm relation for $C$ when $m$ is odd, or if $C(\Theta) \not\in (\bQ^{\times})^2$ for $m$ even, then $ \rk E / F > 0$. 
\end{rem}

\subsection{Number Theoretic Results}

In Sections \ref{sec_cyclic} and Sections \ref{sec_odd}, we will require some number theoretic results, and in this subsection we discuss some of them. The remaining ones can be found in the Appendix, and we encourage the reader to visit them when they are needed later on. Here we describe the quadratic subfields of certain cyclotomic extensions and prove a couple of results that are naturally phrased in terms of local fields and that have direct consequences on Type II and $\mathrm{II}^*$ elliptic curves.

\begin{lemma}\label{lem_subfields}
    Let $q$ be an odd rational prime, $n,m$ a positive integers and let $q^*=(-1)^{(q-1)/2}q$. Then the following holds.

    \begin{table}[!ht]
        \centering
        \begin{tabular}{|l|l|l|}
        \hline
        Cyclotomic field                    & Conditions & Quadratic subfields                   \\ \hline
        $\QQ(\zeta_{q^n})$                  & any $n$    & $\QQ(\sqrt{q^*})$            \\ \hline
        \multirow{3}{*}{$\QQ(\zeta_{2^m})$} & $m=1$      & none                                  \\ \cline{2-3} 
                                            & $m=2$      & $\QQ(i)$                              \\ \cline{2-3} 
                                            & $m\geq3$   & $\QQ(i),\QQ(\sqrt{2}),\QQ(\sqrt{-2})$ \\ \hline
        \multirow{3}{*}{$\QQ(\zeta_{2^mq^n})$}  & $m=1$, any $n$      & $\QQ(\sqrt{q^*})$     \\ \cline{2-3} 
                                            & $m=2$, any $n$      & $\QQ(i),\QQ(\sqrt{q}),\QQ(\sqrt{-q})$                              \\ \cline{2-3}
                                            & $m\geq 3$, any $n$      & $\QQ(i),\QQ(\sqrt{2}),\QQ(\sqrt{-2}),\QQ(\sqrt{q}),\QQ(\sqrt{-q}),\QQ(\sqrt{2q}),\QQ(\sqrt{-2q})$                              \\ 
                                             \hline
        \end{tabular}
        \end{table}

\end{lemma}

\begin{proof}
    Firstly, we remark that the discriminant of the field $\QQ(\sqrt{D})$, with $D$ squarefree is
    \begin{equation}
        \Delta(\QQ(\sqrt{D}))=
        \begin{cases}
            D \ \quad\text{  if } D\equiv1\pmod{4},\\
            4D \quad\text{ if } D\equiv2,3\pmod{4}.
        \end{cases}
    \end{equation}
    In addition, we also recall that $\QQ(\zeta_N)/\QQ$ is a Galois extension with $\Gal(\QQ(\zeta_N)/\QQ)=(\ZZ/N\ZZ)^*$ and that a rational prime $r$ ramifies in $\QQ(\zeta_N)/\QQ$ if and only if $r\mid N$. The result follows by combining these two properties with the Galois correspondence, as we show now.

    If $q$ is odd, then $\Gal(\QQ(\zeta_{q^n})/\QQ)=(\ZZ/q^n\ZZ)^*=C_{q^{n-1}(q-1)}$ is a cyclic group of even order, and therefore $\QQ(\zeta_{q^n})$ has one unique quadratic subfield, which can only ramify at $q$. If $q\equiv1\pmod{4}$, then the only such field is $\QQ(\sqrt{q})$ and if $q\equiv3\pmod{4}$ the only such field is $\QQ(\sqrt{-q})$. This proves the first row. 

    Since $\QQ(\zeta_2)=\QQ$ and $\QQ(\zeta_4)=\QQ(i)$, the second and third row are immediate. For $m\geq3$, $\Gal(\QQ(\zeta_{2^m})/\QQ)=(\ZZ/2^m\ZZ)^*=C_2\times C_2^{m-2}$ and therefore $\QQ(\zeta_{2^m})$ has three quadratic subfields that can only ramify at $2$. Again, it is easy to check that the only such fields are $\QQ(i)$, $\QQ(\sqrt{2})$ and $\QQ(\sqrt{-2})$, as desired. Alternatively, one can also show that $\zeta_8=(1+i)/\sqrt{2}$, which also implies the result. This proves the third row.

    The remaining rows are essentially a combination of the results we have already shown. We note that $\Gal(\QQ(\zeta_{2q^n})/\QQ)=(\ZZ/2q^n\ZZ)^*$ is cyclic while 
    $$\Gal(\QQ(\zeta_{2^mq^n})/\QQ)=(\ZZ/2^mq^n\ZZ)^*=(\ZZ/2^m\ZZ)^*\times(\ZZ/q^n\ZZ)^*=C_2\times C_{2^{m-2}}\times C_{p^{n-1}(p-1)}.$$
    Hence, $\QQ(\zeta_{2^mp^n})$ has one unique quadratic subfield if $m=1$ which must be $\QQ(\sqrt{p^*})$, three quadratic subfields if $m=2$, which must be $\QQ(i),\QQ(\sqrt{p}),\QQ(\sqrt{-p})$, and seven quadratic subfields if $m\geq 3$. Since $\QQ(\zeta_8),\QQ(\zeta_q)\subseteq\QQ(\zeta_{2^mq^n})$, it follows that $\QQ(\sqrt{D})\subseteq\QQ(\zeta_{2^mq^n})$ for $D\in\{-1,\pm2,\pm q,\pm 2q\}$. These are seven distinct quadratic fields, so we are done.
\end{proof}

We now state and prove the local field theory results. The first gives a necessary divisibility condition on primes ramifying in finite extensions of number fields.

\begin{prop}\label{prop_totally_ramified}
    Let $F/\QQ_p$ be a finite extension with residue field $\kappa$. Then there exists a tame, totally ramified Galois cyclic extension $F_n$ of degree $n$ over $F$ if and only if $n\mid|\kappa^*|$.
\end{prop}

\begin{proof}
    Assume first that $n\mid|\kappa^*|$. Then $x^n - 1$ splits in $\kappa$, and so by Hensel's lemma, since $p \nmid n$, $Q_p(\zeta_n) \subseteq F$. Therefore, if $\pi \in F$ is a uniformizer, the extension $F_n = F(\pi^{1 / n})$ is the splitting field of $x^n - \pi$. Hence $F_n / F$ is Galois, and totally tamely ramified. By Kummer theory, this is a cyclic extension.

    Conversely, any tamely totally ramified extension of $F$ of degree $n$ is of the form $F(\pi^{1 / n})$ (\cite[Theorem 11.9]{Sun1}). Such an extension is Galois if and only if $\bQ(\zeta_n) \subseteq F$, which, together with the condition that $p \nmid n$, is equivalent to $n \mid |\kappa^*|$.
    %Let $\pi$ be a normalizer of $F$ and consider $F_n=F(\pi^{1/n})$. We claim that $F_n$ satisfies the desired properties. Since $n\mid|\kappa^*|$, $\kappa$ contains all $n$-th roots of unity and therefore the polynomial $x^n-1$ factors into linear terms in $\kappa[x]$. The divisibility condition above implies $\Char\kappa\nmid n$ and hence by Hensel's Lemma $x^n-1$ also factors into linear terms in $F[x]$. In other words, $\QQ_p(\zeta_n)\subseteq F$ and therefore $F_n$ is the splitting field of the polynomial $x^n-\pi$. This shows that $F_n/F$ is a tame, totally ramified Galois extension, and the map 
    %\begin{align*}
    %    \psi: \Gal(F_n/F)&\longrightarrow \mu_n\cong C_n\\
    %    \sigma &\longmapsto \frac{\sigma(\pi^{1/n})}{\pi^{1/n}}
    %\end{align*}
    %is an isomorphism of groups, which proves that the extension is cyclic of degree $n$.
    %Conversely, suppose that $F_n/F$ is a tame, totally ramified cyclic extension of degree $n$. Any such field extension is generated by the $n$-th root of some uniformizer $\pi$ of $F$ (see \cite[Theorem 11.10]{Sun1}), and therefore $F_n=F(\pi^{1/n})$. The polynomial $x^n-\pi$ is Einstein over $F$, and therefore irreducible over $F$. Since $F_n/F$ is assumed to be Galois, all roots of $x^n-\pi$ lie in $F_n$. In particular, $\QQ(\zeta_n)\subseteq F_n$. Since $\Char\kappa\nmid n$, it follows that $\kappa$ also contains all $n$-th roots of unity, proving that $n\mid|\kappa^*|$ as desired. 
\end{proof}

The second result gives an explicit description of $C_4$ with equal ramification index and residual degree.

\begin{lemma}\label{lem_localC4}
    Let $F/K$ be a finite Galois extension of local fields of characteristic $0$ with residual characteristic distinct from $2$ or $3$ and such that $\Gal(F/K)=C_4$. If the ramification index and residual degree are both $2$, then $F=K(\sqrt{u},\sqrt{v\pi})$ where $\pi$ is a uniformizer of $K$, $u$ is a non-square unit of $K$ and $v$ is a non-square unit of $K(\sqrt{u})$.
    In particular, if $\varpi$ is a uniformizer of $F$, then $\varpi^2/\pi$ is a non-square unit of $F$.
\end{lemma}

\begin{proof}
    %We first recall a fundamental property of local fields. If $K$ is any local field of characteristic $0$ and residual characteristic distinct from $2$, then $K$ has three quadratic extensions, two of which are ramified and one unramified. 
    Let $L=F^{C_2}$ be the unique intermediate field of $F/K$. Since $F$ is also the fixed field by inertia, then $F/L$ is ramified while $L/K$ is unramified. We recall that $K$ has a unique unramified extension of any degree, and the unramified quadratic extension is generated by any $\sqrt{u}$ of any non-square unit of $K$. Hence $L=K(\sqrt{u})$ for some non-square unit, and note that $\pi$ is a uniformizer of $L$ too. Since $F/L$ is ramified, $F$ is either generated by $\sqrt{\pi}$ or $\sqrt{v\pi}$ for some non-square unit $v$ of $L$. If $F=L(\sqrt{\pi})$, then $F$ contains all three quadratic extensions of $K$, in which case $\Gal(F/K)=C_2\times C_2$, a contradiction. Hence, necessarily, $F=L(\sqrt{v\pi})=K(\sqrt{u},\sqrt{v\pi})$ for some non-square unit $v$ of $L$.

    To prove the last statement, note that $\sqrt{v\pi}$ is a uniformizer of $F$ and $(\sqrt{v\pi})^2/\pi=v$ is a non-square unit of $F$. Since any two uniformizers are equal up to multiplication by units, the result follows. 
\end{proof}

\subsection{Type II and \texorpdfstring{$\mathrm{II}^*$}{TEXT}  Elliptic Curves}\label{sec_type2EC}

As mentioned earlier, we now prove two technical consequences of these results about the behaviour of Type II or $\mathrm{II}^*$ elliptic curves over local fields $K$. We advise the reader to skip the proofs by now and revisit them when these results are used later.

\begin{lemma}\label{lem_nottwo}
    Let $p\geq 5$ be a rational prime and $F_\fP/K_\fp/\QQ_p$ be finite extensions with $F_\fP/K_\fp$ Galois, ramified and $\Gal(F_\fP/K_\fp)=C_3$. Let $$E/\QQ_p:y^2=x^3+Ax+B$$ be a minimal Weierstrass equation at $\pp$ with potentially good reduction. Let $n=\nu_\pp(\Delta)$ be the valuation of the minimal discriminant. If $\gcd(n,12)=2$, then $\sqrt{\Delta}\in K_\pp$.
\end{lemma}

\begin{proof}
    The condition that $E$ has additive reduction is equivalent to $A,B\in\pp$, and the condition on ramification implies that $3\mid N(\pp)-1$ by Proposition \ref{prop_totally_ramified}. In addition, by Lemma \ref{lem_Dterms}(b), we know that $\nu_\pp(\Delta)<12$, so we need to consider two cases: $n=2$ and $n=10$, and we consider them separately. By Hensel's Lemma, $\sqrt{\Delta}\in K_\pp$ is equivalent to $\sqrt{\Delta}\in \kappa_\pp$ where $\kappa_\pp$ is the residue field of $K_\pp$. Recall that when $E$ has this simple expression, $\Delta=-16(4A^3+27B^2)$.


    \textbf{Case $n=2$:}

    In this case, $\nu_\pp(-4A^3-27B^2)=2$ and this implies that $\nu_\pp(B)=1$. Note that we also have that $A,B\in p\ZZ_p$ and therefore $\nu_p(B)=1$ and $\nu_p(-4A^3-27B^2)=2$. Let $\FF_p=\ZZ_p/p\ZZ_p$ be the residue field of $\QQ_p$. Then 
    $$\frac{-4A^3-27B^2}{p^2}\equiv -3\left(\frac{3B}{p}\right)^2\pmod{p},$$
    and hence $\sqrt{\Delta}\in\FF_p$ if and only if $\sqrt{-3}\in K_\pp$. 
    If $p\equiv1\pmod{3}$, then
    $$\left(\frac{-3}{p}\right)=\left(\frac{p}{3}\right)=1,$$
    and hence $\sqrt{\Delta}\in \FF_p\subseteq \kappa_\pp$. If $p\equiv 2\pmod{3}$, then from the condition that $3\mid N(\pp)-1$, it follows that the extension $\kappa_\pp/\FF_p$ has even degree. By the uniqueness of extensions of finite fields, it follows that $\sqrt{\Delta}\pmod{\pp}\in\kappa_\pp$ as desired.

    \textbf{Case $n=10$:} 

    In this case, $\nu_\pp(-4A^3-27B^2)=10$. When $E$ is defined by this simple expression, then $c_4=-48A$ and since $E$ is assumed to have potentially good reduction, $\nu_\pp(j)=\nu_\pp(A^3/\Delta)=3\nu_\pp(A)-10\geq 0$. Hence, $\nu_\pp(A^3)\geq12$ which implies that $\nu_\pp(-27B^2)=10$ or, equivalently, that $\nu_\pp(B)=5$. This means that $\nu_p(B)=5$ if $K_\pp/\QQ_p$ is unramified or $\nu_p(B)=1$ if $K_\pp/\QQ_p$ has ramification index $2$. In the latter case, we have that $v_p(4A^3+27B^2)=2$ and we are back to the case $n=2$. So assume that $K_\pp/\QQ_p$ is unramified. Then
    $$\frac{-4A^3-27B^2}{p^{10}}\equiv -3\left(\frac{3B}{p^5}\right)^2\pmod{p},$$
    and therefore $\sqrt{\Delta}\in\FF_p$ if and only if $\sqrt{-3}\in K_\pp$. The remaining of the proof is identical to the case $n=2$.
\end{proof}

\begin{lemma}\label{lem_notthree}
    Let $p\geq 5$ be a rational prime and $F_\fP/K_\fp/\QQ_p$ be finite extensions with $F_\fP/K_\fp$ Galois, $\Gal(F_\fP/K_\fp)=C_4$ and ramification index and residual degree equal to $2$. Let $$E/\QQ_p:y^2=x^3+Ax+B$$ be a minimal Weierstrass equation at $\pp$ with potentially good reduction. Let $n=\nu_\pp(\Delta)$ be the valuation of the minimal discriminant. If $\gcd(n,12)=2$, then $\sqrt{B}\not\in F_\fP$.
\end{lemma}

\begin{proof}
    By Lemma \ref{lem_localC4}, we know that $F_\fP=K_\pp(\sqrt{u},\sqrt{v\pi})$ where $\pi$ is a uniformizer of $K$, $u$ is a non-square unit of $K$ and $v$ is a non-square unit of $K(\sqrt{u})$. We also note that $\pi$ is a uniformizer of $K(\sqrt{u})$.
    Similarly to the previous proof, we need to consider the case $n=2$ and $n=10$. 

    \textbf{Case $n=2$:}

    In this case, $\nu_\pp(B)=1$ and therefore $B=\mu\pi$ for some unit $\mu$ of $K$. Since the extension $K(\sqrt{u})/K$ is unramified of degree $2$, $\mu=\lambda^2$ for some unit $\lambda$ of $K(\sqrt{u})$. Therefore, if $\varpi$ is a uniformizer of $F_\fP$, then $B/\varpi^2=\lambda^2\pi/\varpi^2$ is a non-square unit by Lemma \ref{lem_localC4}. In particular, $\sqrt{B}\not\in F_\fP$.

    \textbf{Case $n=10$:}
    This case is solved similarly. Following the same argument as in Lemma \ref{lem_nottwo}, it follows that $\nu_\pp(B)=5$ and hence $B=\mu\pi^5$ where $\mu$ is a unit in $K$ and $\mu=\lambda^2$ for some unit $\lambda$ in $K(\sqrt{u})$. Hence,
    $$\frac{B}{\varpi^{10}}=\frac{\lambda^2\pi^5}{\varpi^{10}}=\left(\frac{\lambda\pi^2}{\varpi^4}\right)^2\frac{\pi}{\varpi^2}$$
    is a non-square unit of $F_\fP$, which implies that $\sqrt{B}\not\in F_\fP$.
\end{proof}
