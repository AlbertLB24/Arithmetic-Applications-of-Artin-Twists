\subsection{Notation and Preliminary Results}

We extend the notation introduced in Notation \ref{not_contr} by defining functions on $\B(G)$. 

\begin{notation}\label{not_contr_fns}
    If $G = \Gal(F / K)$ then for $\fp \in \bQ$ we define functions $T_{\fP \mid \fp}$, $D_{\fP \mid \fp}$ and $C_{\fP \mid \fp}$ on $\B(G)$ by 
    \[ T_{\fP \mid \fp}(H) = T_{\fP \mid \fp}(E / F^H), \quad D_{\fP \mid \fp}(H) = D_{\fP \mid \fp}(E / F^H), \quad C_{\fP \mid \fp}(H) = C_{\fP \mid \fp}(E / F^H), \]
    as defined in Notation \ref{not_contr}.
    Note that if $H$, $H'$ are conjugate then $F^H$, $F^{H'}$ are isomorphic, and so the values of these functions are constant on conjugate subgroups, hence they are well-defined. When $K = \bQ$ we write $p$ instead of $\fp$. 
    
    Define the global contributions $C \colon \B(G) \to \bQ^{\times}$, $T \colon \B(G) \to \bQ^{\times}$ and $D \colon \B(G) \to \bQ^{\times}$ by 
    \[ C(H) = C_{E / F^H}, \quad T(H) = T(E / F^H), \quad D(H) = D(E / F^H). \] 
    Of course one then has $C = T \cdot D = \prod_{\fp} C_{\fP \mid \fp}$, ranging over the primes $\fp$ of $K$. 
\end{notation}

\begin{rem}\label{Rem-C-D-loc}
Note that $C_{\fP \mid p}$ is a $D_p$-local function. Indeed, suppose $D_p = \Gal(F_w / \bQ_p)$, where $F_w$ denotes the completion of $F$ with respect to a place $w$ lying above $p$. For a number field $K$ and place $v$, define $$C_v(E / K) = c_v(E / K) \cdot \left| \omega / \omega_v^{\min} \right|_v.$$ We use the same notation if $K$ is a local field (then the $v$ subscript holds no meaning).
One has
\begin{equation*}
    C_{\fP \mid p} = (D_p, C_v)
\end{equation*}
where $C_v$ is a function on $\B(D_p)$ sending $H \mapsto C_v(E / F_w^H)$.
\end{rem}

The following proposition describes these functions in the language introduced in Section \ref{sec-norm-rels} for each reduction type of $E / \bQ$. We do not attempt to write a formula for $T_{\fP \mid p}$ in the case of additive reduction, computing this involves using Lemma \ref{lem_add_tam}.

\begin{prop}\label{prop_local_fns}
    Let $E / \bQ$ be an elliptic curve, $G = \Gal(F / \bQ)$ and $p$ a prime of $\bQ$. Let $n = v_p(\Delta_E)$. Consider the functions $C_{\fP \mid p}$, $T_{\fP \mid p}$, and $D_{\fP \mid p}$ on $\B(G)$ defined above. Then,
    \begin{enumerate}[(i)]
        \setlength\itemsep{0em}
        \item If $E / \bQ_p$ has good reduction, $C_{\fP \mid p} = 1$,
        \item If $E / \bQ_p$ has split multiplicative reduction then $C_{\fP \mid p} = T_{\fP \mid p} = (D_p, I_p, e n)$,
        \item If $E / \bQ_p$ has non-split multiplicative reduction, 
        $C_{\fP \mid p} = T_{\fP \mid p} = \left(D_p, I_p,
        \left\{\begin{smallmatrix}
            2   & 2 \mid en, 2 \nmid f,  \\
            en   &  2 \mid f, \\
            1   & \text{else}
        \end{smallmatrix}\right.\right),$ 
        \item If $E / \bQ_p$ has potentially good reduction and $p \not= 2, 3$, $D_{\fP \mid p} = (D_p, I_p, p^{f \floor{e n /12}})$, 
        \item If $E / \bQ_p$ has potentially multiplicative reduction and $p \not= 2, 3$, $D_{\fP \mid p} = (D_p, I_p, p^{f \floor{e / 2}})$.
    \end{enumerate}  
\end{prop} 
 
\begin{proof}
    \
    \begin{enumerate}[(i)]
        \setlength\itemsep{0em}
        \item Clear. 
        \item Lemma \ref{lem_Dterms}(i) implies $D_{\fP \mid p} = 1$. If $K' / \bQ_p$ is a finite extension of ramification degree $e$, then $E / K'$ has split multiplicative reduction of type $\I_{en}$, which has Tamagawa number $en$ by Lemma \ref{lem_mult_tam}.
        \item As for split, $D_{\fP \mid p} = 1$. The description follows from applying Proposition \ref{prop_semi_red} (iii) (non-split becomes split when the residue degree is even), and Lemma \ref{lem_mult_tam}. 
        \item Follows from Lemma \ref{lem_Dterms}(ii),
        \item Follows from Lemma \ref{lem_Dterms}(iii).
    \end{enumerate}
\end{proof}
%An immediate consequence of this notation is the fact that 
%$$C_{E/F}=\prod_{\pp}C_{\mathfrak{P}\mid \pp}(F/K);$$
%that is, we can calculate $C_{E/F}$ by calculating the contribution locally at each prime of $K$. 
%{\color{red} also important to mention at some point that if the reduction is semistable, then the terms in a norm relation coming from the discriminant also vanish. Probably this would have to be introduced later.}

\begin{rem}\label{rephrase-thm}
    We rephrase Theorem \ref{thm_positive_rank} in the language introduced in $\S$\ref{sec-norm-rels}. 
    Replacing $\rho$ by the sum of its conjugates by elements of $ \Gal(\bQ(\rho) / \bQ(\sqrt{D}))$, we may assume that $\bQ(\rho) = \bQ(\sqrt{D})$. Note that this does not affect the order of $\rho$ in $\C(G)$, nor the set of $\rho$-relations (since $\repnorm{\rho}$ is unchanged). 
    
    Let $\Theta$ be a $\rho$-relation with $\bC[G / \Theta] = \repnorm{\rho}^{\oplus m}$. Let $C \colon \B(G) \to \bQ^{\times}$ be the function sending $H \mapsto C_{E / F^H}$. The theorem then states that, if $\Theta$ is not a norm relation for $C$ when $m$ is odd, or if $C(\Theta) \not\in (\bQ^{\times})^2$ for $m$ even, then $ \rk E / F > 0$. 
\end{rem}