\subsection{Number Theoretic Results}

In Sections \ref{sec_cyclic} and Sections \ref{sec_odd}, we will require some number theoretic results, and in this subsection we discuss some of them. The remaining ones can be found in the Appendix, and we encourage the reader to visit them when they are needed later on. Here we describe the quadratic subfields of certain cyclotomic extensions and prove a couple of results that are naturally phrased in terms of local fields and that have direct consequences on Type II and $\mathrm{II}^*$ elliptic curves.

\begin{lemma}\label{lem_subfields}
    Let $q$ be an odd rational prime, $n,m$ a positive integers and let $q^*=(-1)^{(q-1)/2}q$. Then the following holds.

    \begin{table}[!ht]
        \centering
        \begin{tabular}{|l|l|l|}
        \hline
        Cyclotomic field                    & Conditions & Quadratic subfields                   \\ \hline
        $\QQ(\zeta_{q^n})$                  & any $n$    & $\QQ(\sqrt{q^*})$            \\ \hline
        \multirow{3}{*}{$\QQ(\zeta_{2^m})$} & $m=1$      & none                                  \\ \cline{2-3} 
                                            & $m=2$      & $\QQ(i)$                              \\ \cline{2-3} 
                                            & $m\geq3$   & $\QQ(i),\QQ(\sqrt{2}),\QQ(\sqrt{-2})$ \\ \hline
        \multirow{3}{*}{$\QQ(\zeta_{2^mq^n})$}  & $m=1$, any $n$      & $\QQ(\sqrt{q^*})$     \\ \cline{2-3} 
                                            & $m=2$, any $n$      & $\QQ(i),\QQ(\sqrt{q}),\QQ(\sqrt{-q})$                              \\ \cline{2-3}
                                            & $m\geq 3$, any $n$      & $\QQ(i),\QQ(\sqrt{2}),\QQ(\sqrt{-2}),\QQ(\sqrt{q}),\QQ(\sqrt{-q}),\QQ(\sqrt{2q}),\QQ(\sqrt{-2q})$                              \\ 
                                             \hline
        \end{tabular}
        \end{table}

\end{lemma}

\begin{proof}
    Firstly, we remark that the discriminant of the field $\QQ(\sqrt{D})$, with $D$ squarefree is
    \begin{equation*}
        \Delta(\QQ(\sqrt{D}))=
        \begin{cases}
            D \ \quad\text{  if } D\equiv1\pmod{4},\\
            4D \quad\text{ if } D\equiv2,3\pmod{4}.
        \end{cases}
    \end{equation*}
    In addition, we also recall that $\QQ(\zeta_N)/\QQ$ is a Galois extension with $\Gal(\QQ(\zeta_N)/\QQ)=(\ZZ/N\ZZ)^*$ and that a rational prime $r$ ramifies in $\QQ(\zeta_N)/\QQ$ if and only if $r\mid N$. The result follows by combining these two properties with the Galois correspondence, as we show now.

    If $q$ is odd, then $\Gal(\QQ(\zeta_{q^n})/\QQ)=(\ZZ/q^n\ZZ)^*=C_{q^{n-1}(q-1)}$ is a cyclic group of even order, and therefore $\QQ(\zeta_{q^n})$ has one unique quadratic subfield, which can only ramify at $q$. If $q\equiv1\pmod{4}$, then the only such field is $\QQ(\sqrt{q})$ and if $q\equiv3\pmod{4}$ the only such field is $\QQ(\sqrt{-q})$. This proves the first row. 

    Since $\QQ(\zeta_2)=\QQ$ and $\QQ(\zeta_4)=\QQ(i)$, the second and third row are immediate. For $m\geq3$, $\Gal(\QQ(\zeta_{2^m})/\QQ)=(\ZZ/2^m\ZZ)^*=C_2\times C_2^{m-2}$ and therefore $\QQ(\zeta_{2^m})$ has three quadratic subfields that can only ramify at $2$. Again, it is easy to check that the only such fields are $\QQ(i)$, $\QQ(\sqrt{2})$ and $\QQ(\sqrt{-2})$, as desired. Alternatively, one can also show that $\zeta_8=(1+i)/\sqrt{2}$, which also implies the result. This proves the third row.

    The remaining rows are essentially a combination of the results we have already shown. We note that $\Gal(\QQ(\zeta_{2q^n})/\QQ)=(\ZZ/2q^n\ZZ)^*$ is cyclic while 
    $$\Gal(\QQ(\zeta_{2^mq^n})/\QQ)=(\ZZ/2^mq^n\ZZ)^*=(\ZZ/2^m\ZZ)^*\times(\ZZ/q^n\ZZ)^*=C_2\times C_{2^{m-2}}\times C_{p^{n-1}(p-1)}.$$
    Hence, $\QQ(\zeta_{2^mp^n})$ has one unique quadratic subfield if $m=1$ which must be $\QQ(\sqrt{p^*})$, three quadratic subfields if $m=2$, which must be $\QQ(i),\QQ(\sqrt{p}),\QQ(\sqrt{-p})$, and seven quadratic subfields if $m\geq 3$. Since $\QQ(\zeta_8),\QQ(\zeta_q)\subseteq\QQ(\zeta_{2^mq^n})$, it follows that $\QQ(\sqrt{D})\subseteq\QQ(\zeta_{2^mq^n})$ for $D\in\{-1,\pm2,\pm q,\pm 2q\}$. These are seven distinct quadratic fields, so we are done.
\end{proof}

We now state and prove the local field theory results. The first gives a necessary divisibility condition on primes ramifying in finite extensions of number fields.

\begin{prop}\label{prop_totally_ramified}
    Let $F/\QQ_p$ be a finite extension with residue field $\kappa$. Then there exists a tame, totally ramified Galois cyclic extension $F_n$ of degree $n$ over $F$ if and only if $n\mid|\kappa^*|$.
\end{prop}

\begin{proof}
    Assume first that $n\mid|\kappa^*|$. Then $x^n - 1$ splits in $\kappa$, and so by Hensel's lemma, since $p \nmid n$, $Q_p(\zeta_n) \subseteq F$. Therefore, if $\pi \in F$ is a uniformizer, the extension $F_n = F(\pi^{1 / n})$ is the splitting field of $x^n - \pi$. Hence $F_n / F$ is Galois, and totally tamely ramified. By Kummer theory, this is a cyclic extension.

    Conversely, any tamely totally ramified extension of $F$ of degree $n$ is of the form $F(\pi^{1 / n})$ (\cite[Theorem 11.9]{Sun1}). Such an extension is Galois if and only if $\bQ(\zeta_n) \subseteq F$, which, together with the condition that $p \nmid n$, is equivalent to $n \mid |\kappa^*|$.
    %Let $\pi$ be a normalizer of $F$ and consider $F_n=F(\pi^{1/n})$. We claim that $F_n$ satisfies the desired properties. Since $n\mid|\kappa^*|$, $\kappa$ contains all $n$-th roots of unity and therefore the polynomial $x^n-1$ factors into linear terms in $\kappa[x]$. The divisibility condition above implies $\Char\kappa\nmid n$ and hence by Hensel's Lemma $x^n-1$ also factors into linear terms in $F[x]$. In other words, $\QQ_p(\zeta_n)\subseteq F$ and therefore $F_n$ is the splitting field of the polynomial $x^n-\pi$. This shows that $F_n/F$ is a tame, totally ramified Galois extension, and the map 
    %\begin{align*}
    %    \psi: \Gal(F_n/F)&\longrightarrow \mu_n\cong C_n\\
    %    \sigma &\longmapsto \frac{\sigma(\pi^{1/n})}{\pi^{1/n}}
    %\end{align*}
    %is an isomorphism of groups, which proves that the extension is cyclic of degree $n$.
    %Conversely, suppose that $F_n/F$ is a tame, totally ramified cyclic extension of degree $n$. Any such field extension is generated by the $n$-th root of some uniformizer $\pi$ of $F$ (see \cite[Theorem 11.10]{Sun1}), and therefore $F_n=F(\pi^{1/n})$. The polynomial $x^n-\pi$ is Einstein over $F$, and therefore irreducible over $F$. Since $F_n/F$ is assumed to be Galois, all roots of $x^n-\pi$ lie in $F_n$. In particular, $\QQ(\zeta_n)\subseteq F_n$. Since $\Char\kappa\nmid n$, it follows that $\kappa$ also contains all $n$-th roots of unity, proving that $n\mid|\kappa^*|$ as desired. 
\end{proof}

The second result gives an explicit description of $C_4$ with equal ramification index and residual degree.

\begin{lemma}\label{lem_localC4}
    Let $F/K$ be a finite Galois extension of local fields of characteristic $0$ with residual characteristic distinct from $2$ or $3$ and such that $\Gal(F/K)=C_4$. If the ramification index and residual degree are both $2$, then $F=K(\sqrt{u},\sqrt{v\pi})$ where $\pi$ is a uniformizer of $K$, $u$ is a non-square unit of $K$ and $v$ is a non-square unit of $K(\sqrt{u})$.
    In particular, if $\varpi$ is a uniformizer of $F$, then $\varpi^2/\pi$ is a non-square unit of $F$.
\end{lemma}

\begin{proof}
    %We first recall a fundamental property of local fields. If $K$ is any local field of characteristic $0$ and residual characteristic distinct from $2$, then $K$ has three quadratic extensions, two of which are ramified and one unramified. 
    Let $L=F^{C_2}$ be the unique intermediate field of $F/K$. Since $F$ is also the fixed field by inertia, then $F/L$ is ramified while $L/K$ is unramified. We recall that $K$ has a unique unramified extension of any degree, and the unramified quadratic extension is generated by any $\sqrt{u}$ of any non-square unit of $K$. Hence $L=K(\sqrt{u})$ for some non-square unit, and note that $\pi$ is a uniformizer of $L$ too. Since $F/L$ is ramified, $F$ is either generated by $\sqrt{\pi}$ or $\sqrt{v\pi}$ for some non-square unit $v$ of $L$. If $F=L(\sqrt{\pi})$, then $F$ contains all three quadratic extensions of $K$, in which case $\Gal(F/K)=C_2\times C_2$, a contradiction. Hence, necessarily, $F=L(\sqrt{v\pi})=K(\sqrt{u},\sqrt{v\pi})$ for some non-square unit $v$ of $L$.

    To prove the last statement, note that $\sqrt{v\pi}$ is a uniformizer of $F$ and $(\sqrt{v\pi})^2/\pi=v$ is a non-square unit of $F$. Since any two uniformizers are equal up to multiplication by units, the result follows. 
\end{proof}