\subsection{Properties of Arithmetic Terms}

The arithmetic terms we just described satisfy some important properties that allow us compute them in practice. We list them all in the following lemma.

\begin{lemma}
    Let $E/K$ be an elliptic curve over a number field, $F/K$ a finite field extension of degree $d$. Let $\pp$ be a finite place of $K$, with $\mathfrak{P}\mid\pp$ a place above it in $F$, and $\omega_\pp$ and $\omega_\mathfrak{P}$ minimal differentials for $E/K_\pp$ and $E/F_\mathfrak{P}$ respectively.
    \begin{enumerate}
        %\item If $F/K$ is Galois, then $\Sel_n(E/K)$ is a subgroup of $\Sel_n(E/F)$ for all $n$ coprime to $d$.
        \item For $P,Q\in E(K)$, their Néron-Tate height pairings over $K$ and $F$ are related by $\langle P,Q\rangle_F=\langle P,Q\rangle_K$.
        \item If $\rk E/F=\rk E/K$, then $\Reg_{E/F}=\frac{d^{rk E/K}}{n^2}\Reg_{E/K}$, where $n$ is the index of $E(K)$ in $E(F)$.
        \item If $E/K_\pp$ has good reduction, then $c_\pp=1$. If $E/K_\pp$ has multiplicative reduction of Kodaira type $I_n$ then $n=\ord_\pp\Delta_{E,\pp}^{\min}$ and $c_\pp=n$ if the reduction is split, and $c_\pp=1$ (resp, $2$) if the reduction is non-split and $n$ is odd (resp, even).
        \item If $E/K_\pp$ has good or multiplicative reduction, then $|\omega_\pp/\omega_\mathfrak{P}|_\mathfrak{P}=1$.
        \item If $E/K_\mathfrak{P}$ has potentially good reduction and the residue characteristic is not $2$ or $3$, then 
        $$\left|\frac{\omega_\pp}{\omega_\mathfrak{P}}\right|_\mathfrak{P}=q^{\floor{\frac{e_{F/K}\ord_\pp\Delta_{E,\pp}^{\min}}{12}}},$$
        where $q$ is the size of the residue field at $\mathfrak{P}$.
        \item If $\pp$ has odd residue characteristic, $E/K_\pp$ has potentially multiplicative reduction and $F_\mathfrak{P}/K_\pp$ has even ramification degree, then $E/F_\mathfrak{P}$ has multiplicative reduction.
        \item Multiplicative reduction becomes split after a quadratic unramified extension.
    \end{enumerate}
\end{lemma}