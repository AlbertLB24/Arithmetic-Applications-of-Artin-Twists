\subsection{A BSD Analogue for Artin Twists}\label{sec_BSDArtin}

A natural question to ask at this point is whether there is a conjectural analogue to the above for the Artin twists of $L$-functions. The analogue of $\BSD1$ is known in this case, which is directly compatible with Artin formalism.

\begin{conj}[BSD1 for Twists]
    Let $E/\QQ$ be an elliptic curve, $\rho$ an Artin representation and $F$ any Galois extension over $\QQ$ such that $\rho$ factors through $G=\Gal(F/\QQ)$. Then
    $$\ord_{s=1}L(E,\rho,s)=\langle\rho,E(F)_\CC\rangle,$$
    where $E(F)_{\bC} = E(F) \otimes_{\bZ} \bC$ is viewed as a $G$-representation, and $\langle \cdot, \cdot \rangle$ is the usual inner product of characters of $G$.
\end{conj}

Unfortunately, a conjectural analogue for $\BSD2$ is not known. The problem is the lack of an analogue for the term $\BSD(E/F)$ as above. However, there is indeed an important analogue of the term $\mathcal{L}(E/F)$ in this setting.

\begin{notation}\cite[Definition 12]{DEW1}
    Let $E/\QQ$ be an elliptic curve and $\rho$ an Artin representation over $\QQ$. We define
    $$\mathcal{L}(E,\rho)=\lim_{s\to1}\frac{L(E,\rho,s)}{(s-1)^r}\cdot\frac{\sqrt{\mathfrak{f}_\rho}}{\Omega_+(E)^{d^+(\rho)}|\Omega_-(E)|^{d^-(\rho)}\omega_\rho},$$
    where $r = \ord_{s=1} L(E, \rho, s)$ is the order of the zero at $s = 1$, $\mathfrak{f}_\rho$ is the conductor of $\rho$, and $d^{\pm}(\rho)$ are the dimensions of the $\pm1$-eigenspaces of complex conjugation in its action on $\rho$.
\end{notation}

For an elliptic curve $E / \bQ$ and Galois extension $F / \bQ$, Artin formalism allows one to factor $L(E / F, s)$ as a product of $L$-functions twisted by Artin representations. One would like to similarly factorize the leading term $\BSD(E / F)$ according to Artin representations. Specifically, one would like the following to hold:

%Even though the precise conjectural expression of the $\BSD(E,\rho)$ is not known, they conjecturally satisfy many important properties. The next conjecture lists some of these properties.

\begin{conj}{\cite[Conjecture 4]{DEW1}}\label{conj_4}
    Let $E/\QQ$ be an elliptic curve. 
    For every Artin representation $\rho$ over $\bQ$ there exists an invariant $\BSD(E, \rho) \in \bC^{\times}$ with the following properties. 
    %and assume that $L(E, \rho, s)$ has an analytic continuation to $\bC$ for all Artin representations $\rho$ over $\bQ$.
    Let $\rho$ and $\tau$ be Artin representations over $\bQ$ that factor through $G = \Gal(F/\QQ)$ for some finite Galois extension $F / \bQ$. Then 
    \begin{enumerate}[label={\bfseries C\arabic*.}]
        \setlength\itemsep{0em}
        \item $\mathrm{BSD}(E/F)=\mathrm{BSD}(E,\Ind_{F/\QQ}\trivial)$ for a number field $F$ (and $\Sha_{E/F}$ is finite).
        \item $\mathrm{BSD}(E,\rho\oplus\tau)=\mathrm{BSD}(E,\rho)\mathrm{BSD}(E,\tau)$.
       %\item $\mathrm{BSD}(E,\rho)=\mathrm{BSD}(E,\rho^*)\cdot(-1)^{r}\omega_{E,\rho}\omega_\rho^{-2}$, where $r=\langle\rho,E(K)_\CC\rangle$.
        %\item If $\rho$ is self-dual, then $\mathrm{BSD}(E,\rho)\in\RR$ and $\sign\ \mathrm{BSD}(E,\rho)=\sign\ \omega_\rho$.
    \end{enumerate}       
        If $\langle\rho,E(F)_\CC\rangle=0$, then moreover:
    \begin{enumerate}[label={\bfseries C\arabic*.}]
        \setcounter{enumi}{2}
       \item $\BSD(E,\rho)\in\QQ(\rho)^{\times}$ and $\BSD(E,\rho^{\fg})=\BSD(E,\rho)^{\fg}$ for all $\fg\in\Gal(\QQ(\rho)/\QQ)$. \footnote{$\bQ(\rho)$ is the abelian extension of $\bQ$ obtained by adjoining $\{\Tr \rho (g) \colon g \in G \}$ to $\bQ$, and $\rho^{\fg}$ is the Artin representation over $\bQ$ with $\Tr \rho^{\fg} = \fg \circ \Tr \rho$. }
        %\item If $\rho$ is a non-trivial primitive Dirichlet character of order $d$, and either the conductors of $E$ and $\rho$ are coprime or $E$ is semistable and has no non-trivial isogenies over $\QQ$, then $\BSD(E,\rho)\in\ZZ[\zeta_d]$. 
    \end{enumerate}
\end{conj}

The great advantage of the above conjecture is that it is free of $L$-functions since only the `arithmetic' $\BSD(E/F)$ terms appear. The authors of \cite{DEW1} justify posing this conjecture by studying $\mathcal{L}(E, \rho)$, and proving the following.

\begin{thm} \cite[Corollary 25]{DEW1}
   Let $E / \bQ$ be an elliptic curve, $\rho$, $\tau$ Artin representations over $\bQ$. Suppose that for any Artin representations $\psi$ over $\bQ$, the $L$-function $L(E, \psi, s)$ has analytic continuation to $\bC$. Then,
   \begin{itemize}[(i)]
    \setlength\itemsep{0em}
    \item $\mathcal{L}(E, \Ind_{K / \bQ} \trivial) = \BSD(E / K)$ for a number field $K$, assuming BSD holds for elliptic curves over number fields,
    \item $\mathcal{L}(E, \rho \oplus \tau) = \mathcal{L}(E, \rho)\mathcal{L}(E, \tau)$,
    \item If $L(E, \rho, 1) \not= 0$, then $\mathcal{L}(E, \rho) \in \bQ(\rho)^{\times}$ and $\mathcal{L}(E, \rho^{\fg}) = \mathcal{L}(E, \rho)^{\fg}$ for all $\fg \in \Gal(\bQ(\rho) / \bQ)$, assuming that $L(E, \rho, s)$ satisfies Deligne's period conjecture (\cite{Deligne}).
   \end{itemize}
   
\end{thm}



