\subsection{BSD and the Arithmetic Terms}

The Birch-Swinnerton-Dyer conjecture states the following.

\begin{conj}[BSD]
    Let $E$ be an elliptic curve defined over a number field $K$. Then 
    \begin{enumerate}[label={\bfseries  BSD\arabic*.}]
        \item The rank of the Mordell-Weil group of $E$ over $K$ equals the order of vanishing of the $L$-function; that is,
        $$\ord_{s=1}L(E/K,s)=\rk E/K.$$
        \item The leading term of the Taylor series at $s=1$ of the $L$-function is given by 
        \begin{equation}\label{BSD_2}
            \lim_{s\to1}\frac{L(E/K,s)}{(s-1)^r}\cdot\frac{\sqrt{|\Delta_K|}}{\Omega_+(E)^{r_1+r_2}|\Omega_-(E)|^{r_2}}=\frac{\Reg_{E/K}|\Sha_{E/K}|C_{E/K}}{|E(K)_{\tors}|^2}.
        \end{equation}
    \end{enumerate}
\end{conj}

We briefly explore the arithmetic invariants that appear as part of the statement of BSD2. Some of these invariants depend only on the number field $K$. These are the discriminant $\Delta_K$ of $K$ and the numbers $r_1$ and $r_2$, corresponding to the number of real and complex embeddings of $K$. A basic formula states that if $n=[K:\QQ]$, then $r_1+2r_2=n$. 

The other factors are arithmetic values related to the elliptic curve $E$. Importantly, here we assume that $E$ has rational coefficients.

\begin{enumerate}
    \item \textbf{Periods: } For elliptic curves $E$ defined over $\QQ$, there is a conjugation map $E\to E$, $P\mapsto\bar{P}$. We then define $E(\CC)^+=\{P\in E:\bar{P}=P\}=E(\RR)$ and $E(\CC)^-=\{P\in E:\bar{P}=-P\}$. Then the $\pm$-periods of $E$ are 
    $$\Omega_+(E)=\int_{E(\CC)^+}\omega \quad\text{and}\quad\Omega_-(E)=\int_{E(\CC)^-}\omega,$$
    and orientation chosen so that $\Omega_+(E)\in\RR_{>0}$ and $\Omega_-(E)\in i\RR_{>0}$ .
    \item \textbf{Torsion:} $|E(K)_{\tors}|$ is the size of the torsion subgroup of $E(K)$.
    \item \textbf{Regulator:} To properly define the regulator one neeeds to carefully construct the canonical height $\hat{h}:E(\bar{K})\rightarrow\RR^+$, which rougly evaluates the `arithmetic complexity' of a given point $P\in E(\bar{K})$. We refer the reader to \cite[Chapter VIII: \S4, \S5, \S6 and \S9]{S1} for a complete discussion of this topic. This map satisfies many important properties (as listed in \cite[Chapter VIII, Theorem 9.3]{S1}), among which is the fact that $\hat{h}$ is a quadratic form; in particular, the pairing
    \begin{align*}
        \langle\cdot,\cdot\rangle&:E(\bar{K})\times E(\bar{K})\longmapsto\RR\\
        \langle P,Q\rangle&=\hat{h}(P\oplus Q)-\hat{h}(P)-\hat{h}(Q)
    \end{align*}
    is bilinear. Then the regulator is the volume of $E(K)/E(K)_{\tors}$ computed using the quadratic form $\hat{h}$. In other words, let $P_1,\ldots,P_r$ be generators of the group $E(K)/E(K)_{\tors}$. Then $$\Reg_{E/K}=\det(\langle P_i,P_j\rangle)_{1\leq i,j\leq r}$$
    if $r\geq1$ and $\Reg_{E/K}=1$ if $r=0$.
    \item \textbf{Tate-Shafarevich group:} This is the most mysterious group and it is commonly defined using Galois cohomology as
    $$\Sha_{E/K}=\ker\left[H^1(K,E)\rightarrow\prod_{\pp}H^1(K_\pp,E)\right],$$
    where $H^1(F,E):=H^1(G_F,E(\bar{F}))$ and the implicit map is induced by the inclusions $G_{K_\pp}\emb G_K$. One can interpret $H^1(F,E)$ as `homogeneous spaces' of $E$ over $K$ up to equivalence. A homogeneous space over $K$ is trivial if and only if it has a $K$-rational point, so a non-trivial element of $\Sha_{E/F}$ is a homogeneous space that has points locally in every $K_\pp$ but has no $K$-rational point.

    \item \textbf{Local data:} The term $C_{E/K}$ is defined in terms of local data as 
    $$C_{E/K}=\prod_{\pp}c_\pp(E/K)\left|\frac{\omega}{\omega_\pp^{\min}}\right|_\pp=\prod_{\pp}c_\pp(E/K)\left|\frac{\Delta_{E,\pp}^{\min}}{\Delta_E}\right|_\pp^{\frac{1}{12}}.$$
    where $c_\pp(E/K)$ are the Tamagawa numbers of $E/K$ at a prime $\pp$ of $K$, as discussed in \S\ref*{subs_tamagawa}. To discuss the second term, fix some finite place $\pp$ of $K$ and let $p\ZZ=\pp\cap\QQ$. By assumption, $\Delta_E$ is a minimal discriminant at $p$, but this may not be a minimal discriminant at $\pp$ if $K_\pp/\QQ_p$ is ramified, and we denote $\Delta_{E,\pp}^{\min}$ as the minimal discriminant of $E/K$ at $\pp$. The following result is really helpful to compute these terms in explicit examples.
    
    \begin{lemma}\label{lem_Dterms}
        Let $E$ be an elliptic curve over a number field $K$, and let $F/K$ be a finite extension. Let $\pp$ be a prime in $K$ and $\fP$ a prime in $L$ above $\pp$. Let $\Delta_\pp$ and $\Delta_\fP$ be the minimal discriminants for $E/K_\pp$ and $E/F_\fP$, respectively. Then the following holds.
        \begin{enumerate}
            \item If $\pp$ is unramified at $F/K$ or if $E$ has good or multiplicative reduction at $\pp$, then $\Delta_\pp=\Delta_\fP$, so $|\Delta_\fP/\Delta_\fp|=1$.
            \item If the residual characteristic is distinct from $2$ or $3$, and $E$ has potentially good reduction, then $v_\pp(\Delta_\pp)<12$ and the same holds for $\fP$. In particular, 
            $$\left|\frac{\Delta_\fP}{\Delta_\fp}\right|^{1/12}=q^{\floor{\frac{e_{\fP\mid\fp}\nu_\fp(\Delta_\fP)}{12}}},$$
            where $e_{\fP\mid\fp}$ is the ramification index, and $q$ is the size of the residue field at $\fP$.
        \end{enumerate}
    \end{lemma}

\end{enumerate}

We will spend some time computing the local terms for families of elliptic curves, so we introduce some more notation that will be used throughout. 

\begin{notation}\label{not_contr}
    Let $E$ be an elliptic curve defined over $\QQ$ and let $F/K$ be a finite extension of number fields. For each finite place $\pp$ of $K$, we write 
    $$T_{\fP\mid\pp}(E/F)=\prod_{\fP\mid\pp}c_\fP(E/F)\quad\text{and}\quad D_{\fP\mid\pp}(E/F)=\prod_{\fP\mid\pp}\left|\frac{\Delta_{E,\fP}^{\min}}{\Delta_E}\right|_\fP^{\frac{1}{12}},$$
    and we also write 
    $$C_{\fP\mid \pp}(E/F)=T_{\fP\mid \pp}(E/F)D_{\fP\mid \pp}(E/F)$$
    for the contribution of $\pp$ inside $F$, and
    where the product is taken over the primes $\fP$ of $F$ above $\pp$. 
\end{notation}

An immediate consequence of this notation is the fact that 
$$C_{E/F}=\prod_{\pp}C_{\fP\mid \pp}(E/F),$$ where the product ranges over all primes of $K$.We can therefore calculate $C_{E/F}$ by computing the contribution locally at each prime of $K$. Another important observation is that if $E$ has good reduction over $\pp$, then $C_{\fP\mid\pp}(F/K)=1$ for any finite extension $F$ of $K$. 
%{\color{red} also important to mention at some point that if the reduction is semistable, then the terms in a norm relation coming from the discriminant also vanish. Probably this would have to be introduced later.}

We remark that the way we have organised the terms in \eqref{BSD_2} is not arbitrary, and in fact we give specific notation to both sides of the equation. 

\begin{notation}
    Let $E/\QQ$ be an elliptic curve and $K$ a number field. We define 
    $$\mathcal{L}(E/K)=\lim_{s\to1}\frac{L(E/K,s)}{(s-1)^r}\cdot\frac{\sqrt{|\Delta_K|}}{\Omega_+(E)^{r_1+r_2}|\Omega_-(E)|^{r_2}}$$
    and
    $$\BSD(E/K)=\frac{\Reg_{E/K}|\Sha_{E/K}|C_{E/K}}{|E(K)_{\tors}|^2}.$$
\end{notation}


