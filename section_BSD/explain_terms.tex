\subsection{BSD and the Arithmetic Terms}\label{sec-explain-terms}

The Birch-Swinnerton-Dyer conjecture states the following.

\begin{conj}[BSD]
    Let $E$ be an elliptic curve defined over a number field $K$. Then 
    \begin{enumerate}[label={\bfseries  BSD\arabic*.}]
        \item The rank of the Mordell-Weil group of $E$ over $K$ equals the order of vanishing of the $L$-function; that is,
        $$\ord_{s=1}L(E/K,s)=\rk E/K = r.$$
        \item The leading term of the Taylor series at $s=1$ of the $L$-function is given by 
        \begin{equation}\label{BSD_2}
            \lim_{s\to1}\frac{L(E/K,s)}{(s-1)^r}\cdot\frac{\sqrt{|\Delta_K|}}{\Omega_+(E)^{r_1+r_2}|\Omega_-(E)|^{r_2}}=\frac{\Reg_{E/K}|\Sha_{E/K}|C_{E/K}}{|E(K)_{\tors}|^2}.
        \end{equation}
    \end{enumerate}
\end{conj}

We briefly explore the arithmetic invariants that appear as part of the statement of BSD2. Some of these invariants depend only on the number field $K$. These are the discriminant $\Delta_K$ of $K$ and the numbers $r_1$ and $r_2$, corresponding to the number of real and complex embeddings of $K$. A basic formula states that if $n=[K:\QQ]$, then $r_1+2r_2=n$. 

The other factors are arithmetic values related to the elliptic curve $E$. Importantly, here we assume that $E$ has rational coefficients.

\begin{enumerate}
    \item \textbf{Periods: } For elliptic curves $E$ defined over $\QQ$, there is a conjugation map $E\to E$, $P\mapsto\bar{P}$. We then define $E(\CC)^+=\{P\in E:\bar{P}=P\}=E(\RR)$ and $E(\CC)^-=\{P\in E:\bar{P}=-P\}$. Then the $\pm$-periods of $E$ are 
    $$\Omega_+(E)=\int_{E(\CC)^+}\omega \quad\text{and}\quad\Omega_-(E)=\int_{E(\CC)^-}\omega,$$
    and orientation chosen so that $\Omega_+(E)\in\RR_{>0}$ and $\Omega_-(E)\in i\RR_{>0}$ .
    \item \textbf{Torsion:} $|E(K)_{\tors}|$ is the size of the torsion subgroup of $E(K)$.
    \item \textbf{Regulator:} To properly define the regulator one needs to carefully construct the canonical height $\hat{h}:E(\bar{K})\rightarrow\RR^+$, which roughly evaluates the `arithmetic complexity' of a given point $P\in E(\bar{K})$. We refer the reader to \cite[Chapter VIII: \S4, \S5, \S6 and \S9]{S1} for a complete discussion of this topic. This map satisfies many important properties (as listed in \cite[Chapter VIII, Theorem 9.3]{S1}), among which is the fact that $\hat{h}$ is a quadratic form; in particular, the pairing
    \begin{align*}
        \langle\cdot,\cdot\rangle&:E(\bar{K})\times E(\bar{K})\longmapsto\RR\\
        \langle P,Q\rangle&=\hat{h}(P\oplus Q)-\hat{h}(P)-\hat{h}(Q)
    \end{align*}
    is bilinear. Then the regulator is the volume of $E(K)/E(K)_{\tors}$ computed using the quadratic form $\hat{h}$. In other words, let $P_1,\ldots,P_r$ be generators of the group $E(K)/E(K)_{\tors}$. Then $$\Reg_{E/K}=\det(\langle P_i,P_j\rangle)_{1\leq i,j\leq r}$$
    if $r\geq1$ and $\Reg_{E/K}=1$ if $r=0$.
    \item \textbf{Tate-Shafarevich group:} This is the most mysterious group and it is commonly defined using Galois cohomology as
    $$\Sha_{E/K}=\ker\left[H^1(K,E)\rightarrow\prod_{\pp}H^1(K_\pp,E)\right],$$
    where $H^1(F,E):=H^1(G_F,E(\bar{F}))$ and the implicit map is induced by the inclusions $G_{K_\pp}\emb G_K$. One can interpret $H^1(F,E)$ as `homogeneous spaces' of $E$ over $K$ up to equivalence. A homogeneous space over $K$ is trivial if and only if it has a $K$-rational point, so a non-trivial element of $\Sha_{E/F}$ is a homogeneous space that has points locally in every $K_\pp$ but has no $K$-rational point.

    \item \textbf{Local data:} The term $C_{E/K}$ is defined in terms of local data as 
    \begin{equation}\label{fudge-factor}
    C_{E/K}=\prod_{\pp}c_\pp(E/K)\left|\frac{\omega}{\omega_\pp^{\min}}\right|_\pp
    \end{equation}
    where $c_\pp(E/K)$ are the Tamagawa numbers of $E/K$ at a prime $\pp$ of $K$, as discussed in \S\ref*{subs_tamagawa}.
    Here $\omega$ is the global minimal differential for $E / \bQ$, and $\omega_{\fp}^{\min}$ is the minimal differential at $\fp$. By $\omega / \omega_{\fp}^{\min}$ one means any scalar $\lambda \in K^{\times}$ with $\omega = \lambda \omega_{\fp}^{\min}$. In terms of minimal discriminants, one has
    \[ \left| \frac{\omega}{\omega_{\fp}^{\min}} \right|_{\fp}^{-12} = \left|\frac{\Delta_E}{\Delta_{E, \pp}^{\min}} \right|_{\fp}, \]
    where $\Delta_{E, \pp}^{\min}$ is the minimal discriminant at $\fp$.    
\end{enumerate}

The following result will be helpful to compute the $C_{E / K}$ terms in explicit examples.

    \begin{lemma}\label{lem_Dterms}
        Let $E$ be an elliptic curve over a number field $K$, $F/K$ a finite extension. Let $\pp$ be a prime in $K$ and $\fP$ a prime in $F$ above $\pp$. Let $q$ be the size of the residue field of $K$ at $\fp$. %Denote by $e_{\fP \mid \fp}$, $f_{\fP \mid \fp}$ the ramification index and residue degree of $F_{\fP} / K_{\fp}$. 

        Let $\Delta_\pp$, $\omega_{\pp}$ and $\Delta_\fP$, $\omega_{\fP}$ be the minimal discriminants and differentials for $E/K_\pp$ and $E/F_\fP$, respectively. Then the following holds.
        \begin{enumerate}[(i)]
            \setlength\itemsep{0em}
            \item If $\pp$ is unramified at $F/K$ or if $E$ has good or multiplicative reduction at $\pp$, then the minimal model of $E / K_{\fp}$ and $E / F_{\fP}$ coincide so $| \omega_{\fp} / \omega_{\fP} |_{\fP} = 1$. 
            
            \item If the residual characteristic is distinct from $2$ or $3$, and $E$ has potentially good reduction, then $v_\pp(\Delta_\pp)<12$ and the same holds for $\fP$. In particular, 
            $$\left|\frac{\omega_{\fp}}{\omega_{\fP}}\right|_{\fP} = q^{f_{\fP \mid \fp}\cdot\floor{\frac{e_{\fP\mid\fp}\nu_\fp(\Delta_\fp)}{12}}}.$$
            \item If the residual characteristic is distinct from $2$ or $3$, and $E / K_{\fp}$ has potentially multiplicative reduction then 
            $$ \left|\frac{\omega_{\fp}}{\omega_{\fP}}\right|_{\fP} = q^{f_{\fP \mid \fp} \cdot \floor{\frac{e_{\fP \mid \fp}}{2}}}.$$
        \end{enumerate}
    \end{lemma}

\begin{proof}[Proof Sketch]
    Let $e = e_{\fP \mid \fp}$, $f = f_{\fP \mid \fp}$, $\delta = v_{\fp}(\Delta_{\fp})$ and $\delta_{\fP} = v_{\fP}(\Delta_{\fP})$. Then $v_{\fP}(\Delta_{\fp}) = e n$. Thus $|\Delta_{\fp} / \Delta_{\fP} |_{\fP} = q^{f\cdot (\delta \cdot e - \delta_{\fP})}$, whence $$\left| \frac{\omega_{\fp}}{\omega_{\fP}}\right|_{\fP} = q^{f \cdot \floor{\frac{\delta \cdot e - \delta_{\fP}}{12}}}.$$ 
    \begin{enumerate}[(i)]
        \setlength\itemsep{0em}
        \setcounter{enumi}{1}
        \item If $E / K_{\fp}$ has potentially good reduction then $\delta \in \{ 2,3,4,6,8,9,10 \}$ and $\delta_{\fP} \leq 12$. By reducing to minimal Weierstrass equation for $E / F_{\fP}$ it follows that $\delta_{\fP} = \delta\cdot e - 12 \cdot \floor{\delta\cdot e / 12}$.
        
        \item Let $E / K_p$ have Kodaira type $\I_n^*$, so $\delta = 6 + n$. If $e$ is even then $E / F_{\fP}$ has Kodaira type $\I_{en}$, so $\delta_{\fP} = en$ and $\delta \cdot e - \delta_{\fP} = 6 e$.
        Else if $e$ is odd, $E / F_{\fP}$ has Kodaira type $\I_{en}^*$ so $\delta_{\fP} = 6 + en$ and $\delta\cdot e - \delta_{\fP} = 6 e - 6$. But then $\floor{(6e - 6)/12} = \floor{(e - 1)/2} = \floor{e / 2}$ since $e$ is odd.
    \end{enumerate}
\end{proof}

We will spend some time computing the local terms for families of elliptic curves, so we introduce some more notation that will be used throughout. 

\begin{notation}\label{not_contr}
    Let $E$ be an elliptic curve defined over $\QQ$ and let $F/K$ be a finite extension of number fields. For each finite place $\pp$ of $K$, we write 
    $$T_{\mathfrak{P}\mid\pp}(E/F)=\prod_{\mathfrak{P}\mid\pp}c_\mathfrak{P}(E/F)\quad\text{and}\quad D_{\mathfrak{P}\mid\pp}(E/F)=\prod_{\mathfrak{P}\mid\pp}\left|\frac{\omega_{\fp}}{\omega_{\fP}}\right|_\mathfrak{P},$$
    where the product ranges over primes $\fP$ of $F$ dividing $p$. We also write 
    $$C_{\mathfrak{P}\mid \pp}(E/F)=T_{\mathfrak{P}\mid \pp}(E/F)D_{\mathfrak{P}\mid \pp}(E/F).$$
\end{notation}

An immediate consequence of this notation is the fact that $C_{E/F}=\prod_{\pp}C_{\mathfrak{P}\mid \pp}(E / F)$.
%Observe that if $E$ has good reduction over $\pp$, then $C_{\mathfrak{P}\mid\pp}(E/F)=1$ for any finite extension $F$ of $K$. 

\begin{defn}\label{not_contr_fns}
    If $G = \Gal(F / \bQ)$ then for $p \in \bQ$ we define functions $T_{\fP \mid p}$, $D_{\fP \mid p}$ and $C_{\fP \mid p}$ on $\B(G)$ by 
    \[ T_{\fP \mid p}(H) = T_{\fP \mid p}(E / F^H), \quad D_{\fP \mid p}(H) = D_{\fP \mid p}(E / F^H), \quad C_{\fP \mid p}(H) = C_{\fP \mid p}(E / F^H). \]
    Note that if $H$, $H'$ are conjugate then $F^H$, $F^{H'}$ are isomorphic, and so the values of these functions are constant on conjugate subgroups, hence they are well-defined. Define $C \colon \B(G) \to \bQ^{\times}$ by $C \colon H \mapsto C_{E / F^H}$.  
\end{defn}
    
    Note that $C_{\fP \mid p}$ is a $D_p$-local function. Indeed, suppose $D_p = \Gal(F_w / \bQ_p)$, where $F_w$ denotes the completion of $F$ with respect to a place $w$ lying above $p$. For a number field $K$ and place $v$, define $$C_v(E / K) = c_v(E / K) \cdot \left| \omega / \omega_v^{\min} \right|_v.$$ We use the same notation if $K$ is a local field (then the $v$ subscript holds no meaning).
    One has
    \begin{equation*}
    C_{\fP \mid p} = (D_p, C_v)
    \end{equation*}
    where $C_v$ is a function on $\B(D_p)$ sending $H \mapsto C_v(E / F_w^H)$.

 The following proposition describes these functions in the language introduced in Section \ref{sec-norm-rels} for each reduction type of $E / \bQ$. We do not attempt to write a formula for $T_{\fP \mid \fp}$ in the case of additive reduction, computing this involves using Lemma \ref{lem_add_tam}.

 \begin{prop}\label{prop_local_fns}
    Let $E / \bQ$ be an elliptic curve, $G = \Gal(F / \bQ)$ and $p$ a prime of $\bQ$. Let $n = v_p(\Delta_E)$. Consider the functions $C_{\fP \mid p}$, $T_{\fP \mid p}$, and $D_{\fP \mid p}$ on $\B(G)$ defined above. Then,
    \begin{enumerate}[(i)]
        \setlength\itemsep{0em}
        \item If $E / \bQ_p$ has good reduction, $C_{\fP \mid p} = 1$,
        \item If $E / \bQ_p$ has split multiplicative reduction then $C_{\fP \mid p} = T_{\fP \mid \fp} = (D_p, I_p, e n)$,
        \item If $E / \bQ_p$ has non-split multiplicative reduction, 
        $C_{\fP \mid p} = T_{\fP \mid \fp} = \left(D_p, I_p,
        \left\{\begin{smallmatrix}
            2   & 2 \mid en, 2 \nmid f,  \\
            en   &  2 \mid f, \\
            1   & \text{else}
        \end{smallmatrix}\right.\right),$ 
        \item If $E / \bQ_p$ has potentially good reduction and $p \not= 2, 3$, $D_{\fP \mid p} = (D_p, I_p, p^{f \floor{e n /12}})$, 
        \item If $E / \bQ_p$ has potentially multiplicative reduction and $p \not= 2, 3$, $D_{\fP \mid p} = (D_p, I_p, p^{f \floor{e / 2}})$.
    \end{enumerate}  
 \end{prop} 
 
 \begin{proof}
    \
\begin{enumerate}[(i)]
    \setlength\itemsep{0em}
    \item Clear. 
    \item Lemma \ref{lem_Dterms}(i) implies $D_{\fP \mid p} = 1$. If $K' / \bQ_p$ is a finite extension of ramification degree $e$, then $E / K'$ has split multiplicative reduction of type $\I_{en}$, which has Tamagawa number $en$ by Lemma \ref{lem_mult_tam}.
    \item As for split, $D_{\fP \mid p} = 1$. The description follows from applying Proposition \ref{prop_semi_red} (iii) (non-split becomes split when the residue degree is even), and Lemma \ref{lem_mult_tam}. 
    \item Follows from Lemma \ref{lem_Dterms}(ii),
    \item Follows from Lemma \ref{lem_Dterms}(iii).
\end{enumerate}
 \end{proof}
%An immediate consequence of this notation is the fact that 
%$$C_{E/F}=\prod_{\pp}C_{\mathfrak{P}\mid \pp}(F/K);$$
%that is, we can calculate $C_{E/F}$ by calculating the contribution locally at each prime of $K$. 
%{\color{red} also important to mention at some point that if the reduction is semistable, then the terms in a norm relation coming from the discriminant also vanish. Probably this would have to be introduced later.}
We remark that the way we have organised the terms in \eqref{BSD_2} is not arbitrary, and in fact we give specific notation to both sides of the equation. 

\begin{notation}
    Let $E/\QQ$ be an elliptic curve and $K$ a number field. We define 
    $$\mathcal{L}(E/K)=\lim_{s\to1}\frac{L(E/K,s)}{(s-1)^r}\cdot\frac{\sqrt{|\Delta_K|}}{\Omega_+(E)^{r_1+r_2}|\Omega_-(E)|^{r_2}}$$
    and
    $$\BSD(E/K)=\frac{\Reg_{E/K}|\Sha_{E/K}|C_{E/K}}{|E(K)_{\tors}|^2}.$$
\end{notation}


