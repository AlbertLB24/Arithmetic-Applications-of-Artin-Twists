\subsection{BSD and the Arithmetic Terms}\label{sec-explain-terms}

The Birch-Swinnerton-Dyer conjecture states the following.

\begin{conj}[BSD]
    Let $E$ be an elliptic curve defined over a number field $F$. Then 
    \begin{enumerate}[label={\bfseries  BSD\arabic*.}]
        \item The rank of the Mordell-Weil group of $E$ over $F$ equals the order of vanishing of the $L$-function; that is,
        $$\ord_{s=1}L(E/F,s)=\rk E/F = r.$$
        \item The leading term of the Taylor series at $s=1$ of the $L$-function is given by 
        \begin{equation}\label{BSD_2}\tag{BSD2}
            \lim_{s\to1}\frac{L(E/F,s)}{(s-1)^r}\cdot\frac{\sqrt{|\Delta_F|}}{\Omega_+(E)^{r_1+r_2}|\Omega_-(E)|^{r_2}}=\frac{\Reg_{E/F}|\Sha_{E/F}|C_{E/F}}{|E(F)_{\tors}|^2}.
        \end{equation}
    \end{enumerate}
\end{conj}

We briefly explore the arithmetic invariants that appear as part of the statement of BSD2. Some of these invariants depend only on the number field $F$. These are the discriminant $\Delta_F$ of $F$ and the numbers $r_1$ and $r_2$, corresponding to the number of real and complex embeddings of $F$. A basic formula states that if $n=[F:\QQ]$, then $r_1+2r_2=n$. 

The other factors are arithmetic values related to the elliptic curve $E$. Importantly, here we assume that $E$ has rational coefficients.

\begin{enumerate}
    \item \textbf{Periods: } For elliptic curves $E$ defined over $\QQ$, there is a conjugation map $E\to E$, $P\mapsto\bar{P}$. We then define $E(\CC)^+=\{P\in E:\bar{P}=P\}=E(\RR)$ and $E(\CC)^-=\{P\in E:\bar{P}=-P\}$. Then the $\pm$-periods of $E$ are 
    $$\Omega_+(E)=\int_{E(\CC)^+}\omega \quad\text{and}\quad\Omega_-(E)=\int_{E(\CC)^-}\omega,$$
    and orientation chosen so that $\Omega_+(E)\in\RR_{>0}$ and $\Omega_-(E)\in i\RR_{>0}$ .
    \item \textbf{Torsion:} $|E(F)_{\tors}|$ is the size of the torsion subgroup of $E(F)$.
    \item \textbf{Regulator:} To properly define the regulator one needs to carefully construct the canonical height $\hat{h}:E(\bar{F})\rightarrow\RR^+$, which roughly evaluates the `arithmetic complexity' of a given point $P\in E(\bar{F})$. We refer the reader to \cite[Chapter VIII: \S4, \S5, \S6 and \S9]{S1} for a complete discussion of this topic. This map satisfies many important properties (as listed in \cite[Chapter VIII, Theorem 9.3]{S1}), among which is the fact that $\hat{h}$ is a quadratic form; in particular, the pairing
    \begin{align*}
        \langle\cdot,\cdot\rangle&:E(\bar{F})\times E(\bar{F})\longmapsto\RR\\
        \langle P,Q\rangle&=\hat{h}(P\oplus Q)-\hat{h}(P)-\hat{h}(Q)
    \end{align*}
    is bilinear. Then the regulator is the volume of $E(F)/E(F)_{\tors}$ computed using the quadratic form $\hat{h}$. In other words, let $P_1,\ldots,P_r$ be generators of the group $E(F)/E(F)_{\tors}$. Then $$\Reg_{E/F}=\det(\langle P_i,P_j\rangle)_{1\leq i,j\leq r}$$
    if $r\geq1$ and $\Reg_{E/F}=1$ if $r=0$.
    \item \textbf{Tate-Shafarevich group:} This is the most mysterious group and it is commonly defined using Galois cohomology as
    $$\Sha_{E/F}=\ker\left[H^1(F,E)\rightarrow\prod_{\pp}H^1(F_\pp,E)\right],$$
    where $H^1(F,E):=H^1(G_F,E(\bar{F}))$ and the implicit map is induced by the inclusions $G_{F_\pp}\emb G_F$. One can interpret $H^1(F,E)$ as `homogeneous spaces' of $E$ over $F$ up to equivalence. A homogeneous space over $F$ is trivial if and only if it has a $F$-rational point, so a non-trivial element of $\Sha_{E/F}$ is a homogeneous space that has points locally in every $F_\pp$ but has no $F$-rational point.

    \item \textbf{Local data:} The term $C_{E/F}$ is defined in terms of local data as 
    \begin{equation}\label{fudge-factor}
    C_{E/F}=\prod_{\pp}c_\pp(E/F)\left|\frac{\omega}{\omega_\pp^{\min}}\right|_\pp
    \end{equation}
    where $c_\pp(E/F)$ are the Tamagawa numbers of $E/F$ at a prime $\pp$ of $F$, as discussed in \S\ref*{subs_tamagawa}.
    Here $\omega$ is the global minimal differential for $E / \bQ$, and $\omega_{\fp}^{\min}$ is the minimal differential at $\fp$. By $\omega / \omega_{\fp}^{\min}$ one means any scalar $\lambda \in F^{\times}$ with $\omega = \lambda \omega_{\fp}^{\min}$. In terms of minimal discriminants, one has
    \[ \left| \frac{\omega}{\omega_{\fp}^{\min}} \right|_{\fp}^{-12} = \left|\frac{\Delta_E}{\Delta_{E, \pp}^{\min}} \right|_{\fp}, \]
    where $\Delta_{E, \pp}^{\min}$ is the minimal discriminant at $\fp$.    
\end{enumerate}

We remark that the way we have organised the terms in \eqref{BSD_2} is not arbitrary, and in fact we give specific notation to both sides of the equation. 

\begin{notation}
    Let $E/\QQ$ be an elliptic curve and $F$ a number field. We define 
    $$\mathcal{L}(E/F)=\lim_{s\to1}\frac{L(E/F,s)}{(s-1)^r}\cdot\frac{\sqrt{|\Delta_F|}}{\Omega_+(E)^{r_1+r_2}|\Omega_-(E)|^{r_2}}$$
    and
    $$\BSD(E/F)=\frac{\Reg_{E/F}|\Sha_{E/F}|C_{E/F}}{|E(F)_{\tors}|^2}.$$
\end{notation}


