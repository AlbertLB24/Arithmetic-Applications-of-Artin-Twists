The Birch--Swinnerton-Dyer conjecture classically provides a connection between the arithmetic of elliptic curves and their $L$-functions. We have already investigated the construction and main results of the `$L$-functions side', and now we turn out attention to statement of the conjecture and towards understanding the arithmetic terms present in the conjecture. Some arithmetic terms present in the conjecture are easier to describe if the elliptic curve has a global minimal Weierstrass equation. Since we will be mainly interested in elliptic curves over $\QQ$, and in view of Proposition \ref*{prop_globmin}, we will assume throughout that $E$ is an elliptic curve over $\QQ$ with \textbf{global minimal Weierstrass equation}
$$E:y^2+a_1xy+a_3y=x^3+a_2x^2+a_4x+a_6$$ 
for some $a_1,a_2,a_3,a_4,a_6\in\ZZ$, and let 
$$\omega=\frac{dx}{2y+a_1x+a_3}=\frac{dy}{3x^2+2a_2x+a_4-a_1y}$$
be the associalted \textbf{global minimal differential}.

\subsection{BSD and the Arithmetic Terms}

The Birch-Swinnerton-Dyer conjecture states the following.

\begin{conj}[BSD]
    Let $E$ be an elliptic curve defined over a number field $K$. Then 
    \begin{enumerate}[label={\bfseries  BSD\arabic*.}]
        \item The rank of the Mordell-Weil group of $E$ over $K$ equals the order of vanishing of the $L$-function; that is,
        $$\ord_{s=1}L(E/K,s)=\rk E/K.$$
        \item The leading term of the Taylor series at $s=1$ of the $L$-function is given by 
        \begin{equation}\label{BSD_2}
            \lim_{s\to1}\frac{L(E/K,s)}{(s-1)^r}\cdot\frac{\sqrt{|\Delta_K|}}{\Omega_+(E)^{r_1+r_2}|\Omega_-(E)|^{r_2}}=\frac{\Reg_{E/K}|\Sha_{E/K}|C_{E/K}}{|E(K)_{\tors}|^2}.
        \end{equation}
    \end{enumerate}
\end{conj}

We briefly explore the arithmetic invariants that appear as part of the statement of BSD2. Some of these invariants depend only on the number field $K$. These are the discriminant $\Delta_K$ of $K$ and the numbers $r_1$ and $r_2$, corresponding to the number of real and complex embeddings of $K$. A basic formula states that if $n=[K:\QQ]$, then $r_1+2r_2=n$. 

The other factors are arithmetic values related to the elliptic curve $E$. Importantly, here we assume that $E$ has rational coefficients.

\begin{enumerate}
    \item \textbf{Periods: } For elliptic curves $E$ defined over $\QQ$, there is a conjugation map $E\to E$, $P\mapsto\bar{P}$. We then define $E(\CC)^+=\{P\in E:\bar{P}=P\}=E(\RR)$ and $E(\CC)^-=\{P\in E:\bar{P}=-P\}$. Then the $\pm$-periods of $E$ are 
    $$\Omega_+(E)=\int_{E(\CC)^+}\omega \quad\text{and}\quad\Omega_-(E)=\int_{E(\CC)^-}\omega,$$
    and orientation chosen so that $\Omega_+(E)\in\RR_{>0}$ and $\Omega_-(E)\in i\RR_{>0}$ .
    \item \textbf{Torsion:} $|E(K)_{\tors}|$ is the size of the torsion subgroup of $E(K)$.
    \item \textbf{Regulator:} To properly define the regulator one neeeds to carefully construct the canonical height $\hat{h}:E(\bar{K})\rightarrow\RR^+$, which rougly evaluates the `arithmetic complexity' of a given point $P\in E(\bar{K})$. We refer the reader to \cite[Chapter VIII: \S4, \S5, \S6 and \S9]{S1} for a complete discussion of this topic. This map satisfies many important properties (as listed in \cite[Chapter VIII, Theorem 9.3]{S1}), among which is the fact that $\hat{h}$ is a quadratic form; in particular, the pairing
    \begin{align*}
        \langle\cdot,\cdot\rangle&:E(\bar{K})\times E(\bar{K})\longmapsto\RR\\
        \langle P,Q\rangle&=\hat{h}(P\oplus Q)-\hat{h}(P)-\hat{h}(Q)
    \end{align*}
    is bilinear. Then the regulator is the volume of $E(K)/E(K)_{\tors}$ computed using the quadratic form $\hat{h}$. In other words, let $P_1,\ldots,P_r$ be generators of the group $E(K)/E(K)_{\tors}$. Then $$\Reg_{E/K}=\det(\langle P_i,P_j\rangle)_{1\leq i,j\leq r}$$
    if $r\geq1$ and $\Reg_{E/K}=1$ if $r=0$.
    \item \textbf{Tate-Shafarevich group:} This is the most misterious group and it is commonly defined using Galois cohomology as
    $$\Sha_{E/K}=\ker\left[H^1(K,E)\rightarrow\prod_{\pp}H^1(K_\pp,E)\right],$$
    where $H^1(F,E):=H^1(G_F,E(\bar{F}))$ and the implicit map is induced by the inclusions $G_{K_\pp}\emb G_K$. One can interpret $H^1(F,E)$ as `homogeneous spaces' of $E$ over $K$ up to equivalence. A homogeneous space over $K$ is trivial if and only if it has a $K$-rational point, so a non-trivial element of $\Sha_{E/F}$ is a homogeneous space that has points locally in every $K_\pp$ but has no $K$-rational point.

    \item \textbf{Local data:} The term $C_{E/K}$ is defined in terms of local data as 
    $$C_{E/K}=\prod_{\pp}c_\pp(E/K)\left|\frac{\omega}{\omega_\pp^{\min}}\right|_\pp=\prod_{\pp}c_\pp(E/K)\left|\frac{\Delta_{E,\pp}^{\min}}{\Delta_E}\right|_\pp^{\frac{1}{12}}.$$
    where $c_\pp(E/K)$ are the Tamagawa numbers of $E/K$ at a prime $\pp$ of $K$, as discussed in \S\ref*{subs_tamagawa}. To discuss the second term, fix some finite place $\pp$ of $K$ and let $p\ZZ=\pp\cap\QQ$. By assumption, $\Delta_E$ is a minimal discriminant at $p$, but this may not be a minimal discriminat at $\pp$ if $K_\pp/\QQ_p$ is ramified, and we denote $\Delta_{E,\pp}^{\min}$ as the minimal discriminat of $E/K$ at $\pp$. 
    
    We remark that if $p$ is unramified at $K$, or if $E$ has semistable reduction at $p$, then $\Delta_{E,\pp}^{\min}=\Delta_E$ and the second term vanishes. 
\end{enumerate}

We will spend some time computing the local terms for families of elliptic curves, so we introduce some more notation that will be used throughout. 

\begin{notation}\label{not_contr}
    Let $E$ be an elliptic curve defined over $\QQ$ and let $F/K$ be a finite extension of number fields. For each finite place $\pp$ of $K$, we write 
    $$T_{\mathfrak{P}\mid\pp}(F/K)=\prod_{\mathfrak{P}\mid\pp}c_\mathfrak{P}(E/F)\quad\text{and}\quad D_{\mathfrak{P}\mid\pp}(F/K)=\prod_{\mathfrak{P}\mid\pp}\left|\frac{\Delta_{E,\mathfrak{P}}^{\min}}{\Delta_E}\right|_\mathfrak{P}^{\frac{1}{12}},$$
    and we also write 
    $$C_{\mathfrak{P}\mid \pp}(F/K)=T_{\mathfrak{P}\mid \pp}(F/K)D_{\mathfrak{P}\mid \pp}(F/K)$$
    for the contribution of $\pp$ inside $F$, and
    where the product is taken over the primes $\mathfrak{P}$ of $F$ above $\pp$. 
\end{notation}

An immediate consequence of this notation is the fact that 
$$C_{E/F}=\prod_{\pp}C_{\mathfrak{P}\mid \pp}(F/K);$$
that is, we can calculate $C_{E/F}$ by calculating the contribution locally at each prime of $K$. Another important observation is that if $E$ has good reduction over $\pp$, then $C_{\mathfrak{P}\mid\pp}(F/K)=1$ for any finite extension $F$ of $K$. 
%{\color{red} also important to mention at some point that if the reduction is semistable, then the terms in a norm relation coming from the discriminant also vanish. Probably this would have to be introduced later.}

We remark that the way we have organised the terms in \eqref{BSD_2} is not arbitrary, and in fact we give specific notation to both sides of the equation. 

\begin{notation}
    Let $E/\QQ$ be a number field and $K$ a number field. We define 
    $$\mathcal{L}(E/F)=\lim_{s\to1}\frac{L(E/K,s)}{(s-1)^r}\cdot\frac{\sqrt{|\Delta_K|}}{\Omega_+(E)^{r_1+r_2}|\Omega_-(E)|^{r_2}}$$
    and
    $$\BSD(E/F)=\frac{\Reg_{E/K}|\Sha_{E/K}|C_{E/K}}{|E(K)_{\tors}|^2}.$$
\end{notation}




%\subsection{Properties of Arithmetic Terms}

The arithmetic terms we just described satisfy some important properties that allow us compute them in practice. We list them all in the following lemma.

\begin{lemma}
    Let $E/K$ be an elliptic curve over a number field, $F/K$ a finite field extension of degree $d$. Let $\pp$ be a finite place of $K$, with $\mathfrak{P}\mid\pp$ a place above it in $F$, and $\omega_\pp$ and $\omega_\mathfrak{P}$ minimal differentials for $E/K_\pp$ and $E/F_\mathfrak{P}$ respectively.
    \begin{enumerate}
        \item If $F/K$ is Galois, then $\Sel_n(E/K)$ is a subgroup of $\Sel_n(E/F)$ for all $n$ coprime to $d$.
        \item For $P,Q\in E(K)$, their Néron-Tate height pairings over $K$ and $F$ are related by $\langle P,Q\rangle_F=\langle P,Q\rangle_K$.
        \item If $\rk E/F=\rk E/K$, then $\Reg_{E/F}=\frac{d^{rk E/K}}{n^2}\Reg_{E/K}$, where $n$ is the index of $E(K)$ in $E(F)$.
        \item If $E/K_\pp$ has good reduction, then $c_\pp=1$. If $E/K_\pp$ has multiplicative reduction of Kodaira type $I_n$ then $n=\ord_\pp\Delta_{E,\pp}^{\min}$ and $c_\pp=n$ if the reduction is split, and $c_\pp=1$ (resp, $2$) if the reduction is non-split and $n$ is odd (resp, even).
        \item If $E/K_\pp$ has good or multiplicative reduction, then $|\omega_\pp/\omega_\mathfrak{P}|_\mathfrak{P}=1$.
        \item If $E/K_\mathfrak{P}$ has potentially good reduction and the residue characteristic is not $2$ or $3$, then 
        $$\left|\frac{\omega_\pp}{\omega_\mathfrak{P}}\right|_\mathfrak{P}=q^{\floor{\frac{e_{F/K}\ord_\pp\Delta_{E,\pp}^{\min}}{12}}},$$
        where $q$ is the size of the residue field at $\mathfrak{P}$.
        \item If $\pp$ has odd residue characteristic, $E/K_\pp$ has potentially multiplicative reduction and $F_\mathfrak{P}/K_\pp$ has even ramification degree, then $E/F_\mathfrak{P}$ has multiplicative reduction.
        \item Multiplicative reduction becomes split after a quadratic unramified extension.
    \end{enumerate}
\end{lemma}

\subsection{A BSD Analogue for Artin Twists}\label{sec_BSDArtin}

A natural question to ask at this point is whether there is a conjectural analogue to the above for the Artin twists of $L$-functions. The analogue of $\BSD1$ is known in this case, which is directly compatible with Artin formalism.

\begin{conj}[BSD1 for Twists]
    Let $E/\QQ$ be an elliptic curve, $\rho$ an Artin representation and $F$ any Galois extension over $\QQ$ such that $\rho$ factors through $G=\Gal(F/\QQ)$. Then
    $$\ord_{s=1}L(E,\rho,s)=\langle\rho,E(F)_\CC\rangle,$$
    where $E(F)_{\bC} = E(F) \otimes_{\bZ} \bC$ is viewed as a $G$-representation, and $\langle \cdot, \cdot \rangle$ is the usual inner product of characters of $G$.
\end{conj}

Unfortunately, a conjectural analogue for $\BSD2$ is not known. The problem is the lack of an analogue for the term $\BSD(E/F)$ as above. However, there is indeed an important analogue of the term $\mathcal{L}(E/F)$ in this setting.

\begin{notation}\cite[Definition 12]{DEW1}
    Let $E/\QQ$ be an elliptic curve and $\rho$ an Artin representation over $\QQ$. We define
    $$\mathcal{L}(E,\rho)=\lim_{s\to1}\frac{L(E,\rho,s)}{(s-1)^r}\cdot\frac{\sqrt{\mathfrak{f}_\rho}}{\Omega_+(E)^{d^+(\rho)}|\Omega_-(E)|^{d^-(\rho)}\omega_\rho},$$
    where $r = \ord_{s=1} L(E, \rho, s)$ is the order of the zero at $s = 1$, $\mathfrak{f}_\rho$ is the conductor of $\rho$, and $d^{\pm}(\rho)$ are the dimensions of the $\pm1$-eigenspaces of complex conjugation in its action on $\rho$.
\end{notation}

For an elliptic curve $E / \bQ$ and Galois extension $F / \bQ$, Artin formalism allows one to factor $L(E / F, s)$ as a product of $L$-functions twisted by Artin representations. One would like to similarly factorize the leading term $\BSD(E / F)$ according to Artin representations. Specifically, one would like the following to hold:

%Even though the precise conjectural expression of the $\BSD(E,\rho)$ is not known, they conjecturally satisfy many important properties. The next conjecture lists some of these properties.

\begin{conj}{\cite[Conjecture 4]{DEW1}}\label{conj_4}
    Let $E/\QQ$ be an elliptic curve. 
    For every Artin representation $\rho$ over $\bQ$ there exists an invariant $\BSD(E, \rho) \in \bC^{\times}$ with the following properties. 
    %and assume that $L(E, \rho, s)$ has an analytic continuation to $\bC$ for all Artin representations $\rho$ over $\bQ$.
    Let $\rho$ and $\tau$ be Artin representations over $\bQ$ that factor through $G = \Gal(F/\QQ)$ for some finite Galois extension $F / \bQ$. Then 
    \begin{enumerate}[label={\bfseries C\arabic*.}]
        \setlength\itemsep{0em}
        \item $\mathrm{BSD}(E/F)=\mathrm{BSD}(E,\Ind_{F/\QQ}\trivial)$ for a number field $F$ (and $\Sha_{E/F}$ is finite).
        \item $\mathrm{BSD}(E,\rho\oplus\tau)=\mathrm{BSD}(E,\rho)\mathrm{BSD}(E,\tau)$.
       %\item $\mathrm{BSD}(E,\rho)=\mathrm{BSD}(E,\rho^*)\cdot(-1)^{r}\omega_{E,\rho}\omega_\rho^{-2}$, where $r=\langle\rho,E(K)_\CC\rangle$.
        %\item If $\rho$ is self-dual, then $\mathrm{BSD}(E,\rho)\in\RR$ and $\sign\ \mathrm{BSD}(E,\rho)=\sign\ \omega_\rho$.
    \end{enumerate}       
        If $\langle\rho,E(F)_\CC\rangle=0$, then moreover:
    \begin{enumerate}[label={\bfseries C\arabic*.}]
        \setcounter{enumi}{2}
       \item $\BSD(E,\rho)\in\QQ(\rho)^{\times}$ and $\BSD(E,\rho^{\fg})=\BSD(E,\rho)^{\fg}$ for all $\fg\in\Gal(\QQ(\rho)/\QQ)$. 
        %\item If $\rho$ is a non-trivial primitive Dirichlet character of order $d$, and either the conductors of $E$ and $\rho$ are coprime or $E$ is semistable and has no non-trivial isogenies over $\QQ$, then $\BSD(E,\rho)\in\ZZ[\zeta_d]$. 
    \end{enumerate}
\end{conj}

The great advantage of the above conjecture is that it is free of $L$-functions since only the `arithmetic' $\BSD(E/F)$ terms appear. The authors of \cite{DEW1} justify posing this conjecture by studying $\mathcal{L}(E, \rho)$, and proving the following.

\begin{thm}\cite[Corollary 25]{DEW1}
   Let $E / \bQ$ be an elliptic curve, $\rho$, $\tau$ Artin representations over $\bQ$. Suppose that for any Artin representations $\psi$ over $\bQ$, the $L$-function $L(E, \psi, s)$ has analytic continuation to $\bC$ and satisfies Deligne's period conjecture (\cite{Deligne}). Further assume that $\BSD$ holds for elliptic curves over number fields. Then Conjecture \ref{conj_4} holds if one takes $\BSD(E, \rho) = \mathcal{L}(E, \rho)$. 
   %\begin{itemize}[(i)]
   % \setlength\itemsep{0em}
   % \item $\mathcal{L}(E, \Ind_{K / \bQ} \trivial) = \BSD(E / K)$ for a number field $K$, assuming BSD holds for elliptic curves over number fields,
   % \item $\mathcal{L}(E, \rho \oplus \tau) = \mathcal{L}(E, \rho)\mathcal{L}(E, \tau)$,
   % \item If $L(E, \rho, 1) \not= 0$, then $\mathcal{L}(E, \rho) \in \bQ(\rho)^{\times}$ and $\mathcal{L}(E, \rho^{\fg}) = \mathcal{L}(E, \rho)^{\fg}$ for all $\fg \in \Gal(\bQ(\rho) / \bQ)$, assuming that $L(E, \rho, s)$ satisfies Deligne's period conjecture (\cite{Deligne}).
   %\end{itemize}  
\end{thm}



