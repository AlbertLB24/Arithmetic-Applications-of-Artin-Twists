\subsection{Local Polynomials and L-functions}
We now briefly discuss how to attach analytic objects to Artin and $\ell$-adic representations. These objects are usually described for local fields. Then, one constructs global objects attached to number fields by completing them at their finite places, obtaining the local information and then combining it appropriately. 

To begin, let $K$ be a local field and let $p$ be the characteristic of the residue field $\kappa$. Let $\rho:G_K\to\GL(V)$ be an Artin or $\ell$-adic representation such that $\ell\neq p$ (this is an important technical assumption that we will not discuss further). It is a fundamental result in algebraic number theory that the natural map 
$$\epsilon:\Gal(\bar{K}/K)\longrightarrow\Gal(\bar{\kappa}/\kappa)$$
is surjective, and $I_K:=\ker\epsilon$ is denoted as the inertia group of $K$. Therefore, we have a short exact sequence
$$0\longrightarrow I_K\longrightarrow \Gal(\bar{K}/K)\xlongrightarrow{\epsilon} \Gal(\bar{\kappa}/\kappa)\longrightarrow 0.$$

In addition, the map $\phi\in\Gal(\bar\kappa/\kappa)$ such that $\phi(x)=x^p$ is a topological generator of $\Gal(\bar\kappa/\kappa)$ and any preimage of $\phi$ under $\epsilon$ is called a Frobenius element $\Frob_K$, which is well-defined up to $I_K$. Furthermore, the space of inertia-invariants 
$$V^{I_K}:=\{v\in V:\rho(g)v=v\text{ for all }g\in I_K\}$$
is naturally a $G_K/I_K$ representation, which we denote $\rho^{I_K}$. In this setting, $\rho^{I_K}(\Frob_K)$ is therefore well-defined. We are now ready to define the local polynomial attached to $\rho$.

\begin{defn}
    Let $K$ be a local field and let $p$ the characteristic of its residue field. If $\rho$ is an Artin or $\ell$-adic representation such that $\ell\neq p$, then the local polynomial attached to $\rho$ is
    $$P(\rho,T):=\det\left(I-T\cdot\rho^{I_K}\left(\Frob_K^{-1}\right)\right).$$  
\end{defn}

If $K$ is instead a number field, the idea is to consider all finite places of $K$ and consider all the local polynomials attached to all local completions of $K$ to build the corresponding L-function. More concretely, let $\rho:G_K\to\GL(V)$ be an Artin or $\ell$-adic representation, let $\pp$ be a finite place of $K$ and let $K_\pp$ be the corresponding completion. Since $G_{K_\pp}=\Gal(\bar{K_\pp}/K_\pp)$ is naturally a subgroup of $G_K$, we can restrict $\rho$ to $\Res_\pp\rho:G_{K_\pp}\to\GL(V)$ and then calculate the corresponding local polynomial as long as $\pp$ and $\ell$ are coprime. If $\rho$ is an Artin representation, this allows us to construct the associated $L$-function.

\begin{defn}
    Let $K$ be a number field and $\rho$ an Artin representation over $K$. If $\pp$ is a finite place of $K$, we denote the local polynomial at $\pp$ as 
    $$P_\pp(\rho,T):=P(\Res_\pp\rho,T).$$
    The associated $L$-function to $\rho$ is 
    $$L(\rho,s):=\prod_{\pp\text{ prime}}\frac{1}{P_\pp(\rho,N(\pp)^{-s})}.$$
\end{defn}

However, if $\rho$ is an $\ell$-adic representation, constructing a global object is harder, since the above method does not yield information at the finite places $\pp$ that divide $\ell$. This motivates the following important definition.

\begin{defn}
    Let $\{\rho_\ell\}_{\ell}$ be a family of $\ell$-adic representations for each prime $\ell$. We then say that $\{\rho_\ell\}_\ell$ is a \textbf{weakly compatible system of $\ell$-adic representations} if for every finite place $\pp$ of $K$ and rational primes $\ell,\ell'$ not divisible $\pp$, 
    $$P_\pp(\rho_\ell,T)=P_\pp(\rho_{\ell'},T).$$
\end{defn}

When $\{\rho_\ell\}_\ell$ is a weakly compatible system of $\ell$-adic representations, the local polynomial $P_\pp(\rho_\ell,T)$ can be computed using any $\ell$ not divisible by $\pp$. This also allows us to define the $L$-function in this context.

\begin{defn}
    Let $K$ be a number field and let $\{\rho_\ell\}_\ell$ be a weakly compatible system of $\ell$-adic representations. Then the $L$-function attached to the system is 
    $$L(\{\rho_\ell\}, s)=\prod_{\pp\text{ prime}}\frac{1}{P_\pp(\{\rho_\ell\},N(\pp)^{-s})}.$$
\end{defn}
