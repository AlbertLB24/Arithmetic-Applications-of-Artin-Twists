\subsection{Artin Representations and  \texorpdfstring{$\ell$}{TEXT}-adic Representations} \label{subsection_reps}

We begin by recalling the notion of an Artin representation.

\begin{defn}
    Let $K$ be a number field or a local field with characteristic $0$. An \textbf{Artin representation} $\rho$ over $K$ is a complex finite-dimensional vector space $V$ together with a homomorphism $\rho:G_K\to\GL(V)=\GL_n(\CC)$ such that there is some finite Galois extension $F/K$ with $\Gal(\bar{K}/F)\subseteq\ker\rho$. In other words, $\rho$ factors through $\Gal(F/K)$ for some finite extension $F$ of $K$.
\end{defn}

Hence, an Artin representation can be equivalently viewed as a finite dimensional representation of $\Gal(F/K)$ where $F$ is some finite Galois extension of $K$. Throughout the document, we will use both notions and refer to either of them as Artin representations. Which notion we refer to is always clear from context.

\begin{rem}
    The condition above that $\Gal(\bar{K}/F)\subseteq\ker\rho$ is equivalent to $\ker\rho$ being open in $G_K$. This condition is clearly equivalent to $\rho$ being continous with respect the discrete topology on $\GL_n(\CC)$. Interestingly, the profinite topology of $G_K$ has an surprising consequence: this condition is also equivalent to continuity with respect to the usual complex topology on $\GL_n(\CC)$. Necessity is clear, and the proof of sufficiency relies on the fact that under the complex topology, `small' neighbourhoods of the identity in $\GL(V)=\GL_n(\CC)$ do not contain any non-trivial subgroups. Hence, if $\phi:G_K\to\GL(V)$ is continous with respect to the complex topology and $U$ is such a neighbourhood in $\GL(V)$, then $\phi^{-1}(U)\subseteq\ker\phi$ and $\phi^{-1}(U)$ is open, showing that $\ker\phi$ is open too. Hence, Artin representations are simply continous group homomorphisms $\rho:G_k\to\GL_n(\CC)$.
\end{rem}

Next, we define the notion of an $\ell$-adic representation, which will be needed to define the $L$-function of an elliptic curve. This is the local analogue of an Artin representation.

\begin{defn}
    Let $K$ be a number field or a local field of characteristic $0$. A \textbf{continuous $\ell$-adic representation} $\rho$ over $K$ is a continous homomorphism $\rho:G_K\to\GL_n(F)$ where $F$ is a finite extension of $\QQ_\ell$ and $\GL_n(F)$ is equipped with the $\ell$-adic topology.
\end{defn}

\begin{rem} \label{rem_cont_ladic}
    The topologies on $\GL_n(\CC)$ and $\GL_n(\QQ_\ell)$ are very different, and in particular and $\ell$-adic representation may not have an open kernel. Instead, continouity is equivalent to the following condition: for every $m\geq1$, there is some finite field extension $F_m$ of $K$ such that for all $g\in\Gal(\bar{K}/F_m)$, $\rho(g)\equiv \mathrm{Id}_n\pmod{\ell^m}$.
\end{rem}

Given an Artin representation $\rho$, one can view it as homomorphism $\rho:G_K\to\GL_n(\bar{\QQ})$ and since it factors through a finite quotient, we can realise it as $\rho:G_K\to\GL_n(F)$ for some number field $F$. Hence, if $\ell$ is any rational prime and $\mathfrak{l}$ is a prime in $F$ above $\ell$, then one can realise $\rho$ as an $\ell$-adic representation $$\rho:G_K\longrightarrow\GL_n(F_\mathfrak{l}),$$
which is continous since $\rho$ factors through a finite quotient. Furthermore, Artin and $\ell$-adic representations over $K$ have more structure; namely, one can take \textbf{direct sums} and \textbf{tensor products}.

We describe the construction for Artin representations, since the $\ell$-adic case is completely analogous. Suppose we have two Artin representations $\rho_1,\rho_2$ over $K$, and by the discussion on the preceeding paragraph we can realise them as maps $\rho_i:G_K\to\GL_{n_i}(L_i),\ i=1,2$ where $L_1$ and $L_2$ are number fields. If we let $L=L_1L_2$, then the natural maps $\rho_1\oplus\rho_2:G_K\to\GL_{n_1+n_2}(L)$ and $\rho_1\otimes\rho_2:G_K\to\GL_{n_1n_2}(L)$ are both Artin representations. One can also show that this construction is also well-defined up to equivalence.

Finally, we discuss the notion of an induced Artin representation. Suppose $L$ is a finite field extension of $K$ of degree $d$ and let $\rho:G_L\to\GL(V)$ be an Artin representation. Then $G_L$ is naturally a subgroup of $G_K$ of index $d$, and therefore we can construct $\Ind_{G_L}^{G_K}\rho$ in the usual way. This turns out to be an Artin representation of $K$: if $F$ be a number field so that $\rho$ factors through $\Gal(F/L)$, then $\Ind_{G_L}^{G_K}\rho$ will factor through $\Gal(F/K)$. Furthermore, the corresponding representation over $\Gal(F/K)$ will be equivalent to $\Ind_{\Gal(F/L)}^{\Gal(F/K)}\rho$ where $\rho$ is now viewed as a representation of $\Gal(F/L)$. Hence, the notion of induction is naturally compatible with this process of passing through finite quotients. Therefore, and to simplify notation, we will write $\Ind_{L/K}\rho$ for the induced Artin representation, and it will always be clear from context the implicit field $F$.

%Then one can define the vector space 
%$$X=\{f:G_K\to V:f(\tau\sigma)=\rho(\tau)f(\sigma), \tau\in G_L, \sigma\in G_K\}$$
%and equip it with a $G_K$ action given by $(\sigma\cdot f)(\pi)=f(\pi\sigma)$.