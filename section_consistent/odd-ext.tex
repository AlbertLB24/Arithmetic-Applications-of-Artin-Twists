\subsection{Norm relations in odd order extensions}

{\color{red} add some motivation (justification) for why I'm proving this result. The point is that the test for positive rank provided by root number computations never says anything in odd order extensions. If we expect the norm relations test to be weaker than root numbers, then nor should this test.}

\begin{thm}\label{odd-exts}
 Let $E / \bQ$ be an elliptic curve, $F / \bQ$ be an extension of odd order with Galois group $G$. 
 
Assume that $E$ has semistable reduction at $2$ and $3$. 
Take any representation $\rho \in \R(G)$ with quadratic subfield $\bQ(\sqrt{D}) \subset \bQ(\rho)$ and relation
\begin{equation*}\label{odd-exp} \repnorm{\rho}^{\oplus m} =
 \left(\bigoplus_{\mathfrak{g}\in\Gal(\QQ(\rho)/\QQ)}\rho^{\mathfrak{g}}\right)^{\oplus m }=\bigoplus_i\Ind_{F_i/\QQ}\mathds{1}\ominus\bigoplus_j\Ind_{F'_j/\QQ}\mathds{1}
\end{equation*}
 as in Theorem \ref{thm_positive_rank}. Then
 \[ \frac{\prod_i C_{E/F_i}}{\prod_j C_{E/F_j'}}  \in 
    \begin{cases}
        N_{\bQ(\sqrt{D}) / \bQ}(\bQ(\sqrt{D})^{\times}) & m \ \text{odd}, \\
        \bQ^{\times 2} & m \ \text{even}.
    \end{cases} \] 
    In other words, one cannot use Theorem \ref{thm_positive_rank} to conclude that $E / F$ must have positive rank. 
\end{thm}

As in Remark \ref{rephrase-thm}, we may replace $\rho$ by the sum of its conjugates by elements of $ \Gal(\bQ(\rho) / \bQ(\sqrt{D}))$, we may assume that $\bQ(\rho) = \bQ(\sqrt{D})$. We take $D \in \bZ \ \{0,1 \}$ to be squarefree.

The product of terms we are computing is $C(\Theta)$, where $C \colon \B(G) \to \bQ^{\times}$ is given by $C \colon H \mapsto C_{E / F^H}$, and $\Theta$ is any $\rho$-relation.
We break up the function $C$ into $C = \prod_p C_{\fP \mid p} \prod_p = T_{\fP \mid p} \cdot D_{\fP \mid p}$, ranging over primes $p \in \bQ$,
as defined in Notation \ref{not_contr_fns}.
Then one has
\begin{equation*}\label{Dp-loc}
T_{\fP \mid p} \cdot D_{\fP \mid p} = (D_p, C_v)
\end{equation*}
where $C_v$ is a function on $\B(D_p)$ sending $H \mapsto C_v(E / F_w^H)$. The following result will allow us to apply some results from $\S$\ref{sec-norm-rels}.

\begin{thm}\label{odd-c-brauer}
    When $D_p$ has odd order, $C_v(\Psi) \in \bQ^{\times 2}$ for any Brauer relation $\Psi \in \B(D_p)$. 
\end{thm}

\begin{proof}
    This follows from \cite[Theorem 2.47]{reg-const} and \cite[Theorem 3.2  (Tam)]{reg-const}.
\end{proof}

\begin{cor}
It is enough to prove Theorem \ref{odd-exts} when $m$ is the order of $\repnorm{\rho}$ in $\C(G)$. Thus we need to prove that, given any $
\Theta \in \B(G)$ such that $\bC[\Theta] \simeq \repnorm{\rho}^{\oplus m}$, $\Theta$ is a norm relation for the function $C$. 
\end{cor}

\begin{proof}
    Since $\bQ(\rho)$ is quadratic, we have $\bQ^{\times 2} \subset \fieldnorm{\rho}$. As $G$ is odd, any choice of $D_p$ is odd. It follows from Theorem \ref{odd-c-brauer} that $C(\Psi) \in \bQ^{\times 2}$ for all Brauer relations $\Psi \in \B(G)$. Therefore by Proposition \ref{min-to-all}, it is enough to prove Theorem \ref{odd-exts} when $m$ is the order of $\repnorm{\rho}$ in $\C(G)$. Then $m$ divides $|G|$, hence is odd.
\end{proof}

Let $\tau$ be the generator of $\Gal(\bQ(\sqrt{D}) / \bQ)$.
Let $\ff$ be the smallest integer such that $\bQ(\sqrt{D}) \subset \bQ(\zeta_{\ff})$. Then $\ff \mid |G|$, hence is odd. By Remark \ref{conductor}, $\ff = |D|$ and that $D \equiv 1 \pmod 4$. The following shows that it is only of interest to consider decomposition groups of exponent divisible by $\ff$.

\begin{cor}\label{rational-res-2}
    Let the exponent of $D_p$ be $b$. If $\ff \nmid b$, then $(T_{\fP \mid p} \cdot D_{\fP \mid p})(\Theta) \in \bQ^{\times 2}$.
\end{cor}

\begin{proof}
    Note that $\bQ(\Res_{D_p}{\rho}) \subset \bQ(\zeta_b) \cap \bQ(\rho) \subset \bQ(\zeta_b)$. Then $\bQ(\rho) \subset \bQ(\zeta_b) \implies \ff \mid b$ by minimality of $\ff$. Since $\ff \nmid b$, we have $\bQ(\rho) \not\subset \bQ(\zeta_b)$, so $\bQ(\Res_{D_p} \rho) = \bQ$. The corollary then follows from Proposition \ref{rational-res} and Theorem \ref{odd-c-brauer}, noting that since $\C(D_p)$ is odd, multiplication by $2$ is injective. 
\end{proof}

Fix $\Theta = \sum_i n_i H_i \in \B(G)$ with $\bC[\Theta] \simeq \repnorm{\rho}^{\oplus m}$. We prove that at each prime $p$, $(T_{\fP \mid p}\cdot D_{\fP \mid p})(\Theta) \in \fieldnorm{\rho}$.  As observed, this depends on $D_p$ and $I_p$. As we deal with each local factor individually, we argue that one can take $D_p = I_p$.

\begin{lemma}\label{tam-up-to-square}
    Let $E / K$ be an elliptic curve. Let $K' / K$ be an extension of number fields odd degree, unramified at the place $v$ of $K$. Then $C_w(E / K') \equiv C_v(E / K) \mod \bQ^{\times 2}$ for any place $w$ of $K'$ with $ w \mid v$. 
\end{lemma}

\begin{proof}
This is automatic for good reduction and split multiplicative reduction. It is also clear for non-split multiplicative reduction since the residue degree cannot be even (so the reduction type remains non-split at $w$). For additive reduction, see \cite[Lemma 3.12]{reg-const}.
\end{proof}

\begin{lemma}\label{DeqI}
    At a prime $p$, we may assume that $D_p = I_p$ when computing $(T_{\fP \mid p} \cdot D_{\fP \mid p})(\Theta)$. 
\end{lemma}

\begin{proof}
Let $p$ have residue degree $f_p$. Let $L / \bQ$ be a Galois extension of degree $f_p$ with cyclic Galois group, such that $p$ is inert in $L$. Further ensure that $F \cap L = \bQ$. Then $\Gal(FL / L) = G$. Let $F_i = F^{H_i}$ and $L_i = F_i L$.

Let $v$ be a place over $p$ in $F_i$. The extension $L_i / F_i$ is Galois, so $v$ is either split or inert in $L_i$.
We claim that $C_v(E / F_i) \equiv \prod_{w | v} C_w(E / L_i) \mod \bQ^{\times 2}$. Indeed, the number of terms in the product on the right is odd, and by Lemma \ref{tam-up-to-square} $C_v(E / F_i) \equiv C_w(E / L_i) \mod \bQ^{\times 2}$. 
Letting $T_{\fP \mid p}'$ and $D_{\fP \mid p}'$ be functions on $\B(G)$ defined as in (\ref{not_contr_fns}) but with $\bQ$, $F$ replaced by $L$, $FL$, we see that $(T_{\fP \mid p} \cdot D_{{\fP \mid p}})(\Theta) \equiv (T_{\fP \mid p}' \cdot D_{\fP \mid p})(\Theta) \mod \bQ^{\times 2}$. 
Thus it is equivalent to do our computation in $FL / L$, but here $p$ has residue degree $1$.
\end{proof}

To prove Theorem \ref{odd-exts}, we proceed by considering separately each reduction type.

\subsubsection*{Good reduction}
If $E / \bQ$ has good reduction at $p$, it has good reduction at all primes lying above $p$ in subfields of $F$. Hence the Tamagawa number is always one, as well as $\left|\omega / \omega_{\fP}^{\min}\right|_{\fP}$ for any finite place $\fP \mid p$ in an intermediate field. Therefore $T_{\fP \mid p} \cdot D_{\fP \mid p} = 1$ as a  function on $\B(G)$.

\subsubsection*{Multiplicative reduction}

If $E / \bQ_p$ has multiplicative reduction, then as in the good reduction case one has $\left|\omega / \omega_{\fP}^{\min}\right|_{\fP} = 1$ for any place $\fP \mid p$ in an intermediate field. Thus $D_{\fP \mid p} = 1$.
For $T_{\fP \mid p}$, we consider non-split/split reduction separately.
\vspace{1em}

\noindent\underline{\textit{Non-split multiplicative reduction}}

Let $E / \bQ_p$ have non-split multiplicative reduction. Since $D_p = I_p$, all primes above $p$ have residue degree $1$. Then the reduction at places above $p$ remains non-split in all intermediate subfields.
It follows that 
\[ T_{\fP \mid p} = (D_p, \alpha) \]
where $\alpha$ is the constant function on $B(D_p)$ with $\alpha \in \{1, 2\}$, depending on $\ord_p(\Delta)$ being even or odd. By Proposition \ref{const-fns}, it follows that $T_{\fP \mid p}(\Theta) \in \bQ^{\times 2}$.\\  

\noindent\underline{\textit{Split multiplicative reduction}}

Now suppose $E / \bQ_p$ has split multiplicative reduction. The reduction type remains split at all places above $p$ within sub-extensions of $F / \bQ$. Let $\ord_p(\Delta) = n$. Then 
\[ T_{\fP \mid p} = (D_p, D_p, en). \]
Since the $n$ factor is constant, $(D_p, D_p, en)(\Theta) \equiv (D_p, D_p, e)(\Theta) \mod \bQ^{\times 2}$ by Proposition \ref{const-fns} .

We have $D_p = I_p = P_p \ltimes C_l$, where $P_p \triangleleft I_p$ is wild inertia, and $C_l = I_p / P_p$ is the tame quotient. $C_l$ is a cyclic group, with $l \mid p^f - 1 = p - 1$. By Corollary \ref{rational-res-2}, we may consider such $D_p$ with exponent  $p^u l$ for some $u \geq 0$ such that $\ff \mid p^u l$.

Now, $(D_p, D_p, e)(\Theta)$ is the product of ramification indices at primes above $p$. We separate the $p$-part and tame part of this expression.
Recall that the ramification index of a place $w$ above $p$ corresponding to the double coset $H_i x D_p$ has ramification degree $e_w = \frac{|I_p|}{|H_i \cap I_p^x|} =\frac{|I_p|}{|I_p \cap H^{x^{-1}}|}$.
This is the dimension of the permutation representation $\bC[D_p / D_p \cap H^{x^{-1}}]$.
Let  $D_p \cap H^{x^{-1}} = P' \ltimes C_a$ where $P' \leq P$ and $a | l$. Then the ramification index is $\frac{|P|}{|P'|}\cdot \frac{l}{a}$. 

Taking fixed points under wild inertia, one has the following isomorphism of $D_p$-representations  $$\bC[D_p / D_p \cap H^{x^{-1}}]^{P_p} \simeq \bC[D_p / P_p (D_p \cap H^{x^{-1}})] \simeq \bC[D_p / P_p \ltimes C_a].$$ This permutation representation has dimension $\frac{l}{a}$, so we've killed off the $p$-part. 
Then $$\bC[\Res_{D_p} \Theta]^{P_p} \simeq \left(\Res_{D_p} \rho^{\oplus m} \oplus \tau\left(\Res_{D_p}\rho^{\oplus m}\right)\right)^{P_p},$$
and we can consider these as representations of $D_p / P_p = C_l$.

Let $\Psi = P_p \cdot \Res_{D_p}\Theta / P_p \in B(C_l)$.
Consider the function $g$ on $B(C_l)$ with $H \mapsto [C_l : H] = \dim \bC[C_l / H]$. 
It follows from the above discussion that $(D_p, D_p, e)(\Theta)$ differs from $g(\Psi)$ up to a factor of $p$. Note that if $p = 2$ then $P_p = 1$ since $|P_p| \mid |G|$ which is odd. So we only need to consider this factor of $p$ for $p$ odd.
Crucially, this factor of $p$ doesn't matter:

\begin{lemma}
    Let $K = \bQ(\sqrt{D})$, with $\ff$ the smallest positive integer such that $K \subset \bQ(\zeta_{\ff})$. Suppose that $\ff$ is odd. Let $\ff \mid p^u l $, for some odd prime $p$, $u \geq 0$ and $l$ such that $p \equiv 1 \pmod l$. Then $p$ is the norm of an element from $K^{\times}$.
\end{lemma}

\begin{proof}
    Since $\ff$ is odd, one has $D = \prod_{q | \ff} q^*$, the product being taken over primes dividing $\ff$. Note that if $q \not= p$, then since $q \mid l$, we have $p \equiv 1 \pmod l \implies p \equiv 1 \pmod q$. By Theorem \ref{p-one-mod-disc},  $p$ is the norm of a principal fractional ideal of $K$. If $K$ is imaginary, then $p$ is the norm of an element of $K$. Else, we invoke Theorem \ref{p-norm-elem-1} or Theorem \ref{p-norm-elem-2}.
    %We show that $p$ has residue degree $1$ in the extended genus field $E^{+} = K(\{\sqrt{q^*} \colon q | \ff \})$ of $K$ ({\color{red} cf. appendix}).
    %If $q \not= p$ then $q \mid l$, so $p \equiv 1 \pmod l$. Therefore $p$ splits in any quadratic subfield of $E^{+}$ of discriminant not divisible by $p$. Else, $p$ ramifies in any quadratic subfield with discriminant divisible by $p$. Thus it is clear that $p$ has residue degree $1$ in $E^{+}$, hence also in the genus field $E$, and it follows from theorem \ref{p-principal} that $p$ is the norm of a principal ideal.  Else, we invoke theorem \ref{minus-one-norm}.
\end{proof}

Therefore we only need to worry about the tame part of our ramification indices. Let $\phi = (\Res_{D_p} \rho)^{P_p}$, viewed as a representation on $D_p / P_p = C_l$. Then $\Psi$ is a $\phi$-relation. If $\ff \nmid l$ then $\bQ(\phi) = \bQ$. Therefore $\bC[C_l / \Psi] \simeq \phi^{\oplus 2}$, implying that $\Psi = 2\Psi'$ for some $\Psi' \in \B(C_l)$ with $\bC[C_l / \Psi'] = \phi$. Then $g(\Psi) = g(\Psi')^2 \in \fieldnorm{\rho}$. Otherwise, suppose that $\bQ(\phi) = \bQ(\rho)$. It follows from Proposition \ref{index-fn-trivial} that $g(\Psi) \in \fieldnorm{\rho}$.
%has rational character. Then, arguing as in Lemma \ref{rational-res-2}, $g(\Psi) \in \bQ^{\times 2}$ {\color{red} say more?}.%Hence we may assume that $\ff \mid l$ and that $\bQ(\phi) = \bQ(\rho) = K$.

%\begin{prop}\label{semi-stable-gd}
%    One has $g(P_p \cdot \Res_{D_p}\Theta / P_p) = g(\Psi) \in \fieldnorm{\rho}$.   
%\end{prop}

%\begin{proof}
%    Write $\phi^{\oplus m} \oplus \tau(\phi^{\oplus m}) = \bC[C_l / \Psi] = \sum_{l' \mid l}{a_{l'}} \chi_{l'}$ where $a_{l'} \in \bZ$ and $\chi_{l'}$ are defined in Example \ref{cyclic-relns} (recall $\chi_{l'} = \repnorm{\varphi_{l'}}$ with $\bQ(\varphi_{l'}) = \bQ(\zeta_{l'})$ and these form a basis for the irreducible rational representations of $C_l$). Let $\Psi_{l'} = \sum_{l'' | l'}\mu(l' / l'')\cdot C_{l / l''}$ so that $\bC[\Psi_{l'}] = \chi_{l'}$, as observed in the example. Then $\bC[\Psi] \simeq \bC[\sum_{l' | l } a_{l'} \Psi_{l'}]$ which implies that $\Psi = \sum_{l' | l } a_{l'} \Psi_{l'}$ since cyclic groups have no Brauer relations.

 %   Evaluating $g$ on $\Psi_{l'}$ is trivial unless $l' = q^a$ for some $q$ prime, $a \geq 1$. Indeed, if $l' = p_1^{e_1} \cdots p_r^{e_r}$ , with $r \geq 2$ and $e_i \geq 1$, then
 %   \[ \prod_{l'' \mid l'} (l'')^{\mu(l' / l'')} = \prod_{j_1, \ldots j_r \in \{0,1\}^r } \left(p_1^{e_1 - j_1} \cdots p_r^{e_r - j_r}\right)^{\# j_i = 1} = \prod_{i = 1}^r \left(\frac{p_i^{e_i}}{p_i^{e_i - 1}}\right)^{\sum_{ j = 0}^{r - 1} \binom{r-1}{j} (-1)^j} = 1. \]
 %   On the other hand,
  %  \[ \prod_{l' \mid q^a} (l')^{\mu(q^a / l')} = q .\]
    
   % We claim that $\ff \nmid l'$ implies $a_{l'}$ is even. The irreducible representations of $C_l$ over $\bQ(\phi)$ are given by the orbits of the complex irreducible characters of $C_l$ acted upon by $H = \Gal (\bQ(\zeta_l) / \bQ(\phi))$. If $ \ff \nmid l'$ then $\bQ(\phi) \not\subset \bQ(\zeta_{l'})$, so that $B = \Gal(\bQ(\zeta_l) / \bQ(\zeta_{l'})) \not\leq H$. Then $\bQ(\phi) \cap \bQ(\zeta_{l'}) = \bQ$ so $BH = \Gal(\bQ(\zeta_l) / \bQ)$. The orbit of $\varphi_{l'}$ under $H$ is fixed by $BH$, hence is rational. It follows that $\langle \phi, \varphi_{l'} \rangle = \langle \tau(\phi) , \varphi_{l'} \rangle$ so that $a_{l'}$ is even.  

  %  Thus we can only possibly get something interesting if $\ff = q$ is a prime. But then $q$ is a norm from $\bQ(\sqrt{q^*})$ by Corollary \ref{p-norm}. 
%\end{proof}

\subsubsection*{Additive reduction}

Now suppose that $E / \bQ_p$ has additive reduction. In this case, assume that $p \geq 5$. We have $D_p = P_p \ltimes C_l$ with $ l \mid p - 1$.  
%is at worst tamely ramified in $F / \bQ$. This ensures that $D_p = I_p = C_l$ is cyclic, and $l \mid p - 1$. 
Once again we may assume that $\ff \mid p^k l$ where $p^k l $ is the exponent of $D_p$ by Corollary \ref{rational-res-2}.

Let $\delta = \ord_p(\Delta_E)$. Consider a place $\fP$ of $F^H$ over $p$ with ramification degree  $e_{\fP}$ over $\bQ$. Then $\Delta_E$ has valuation $n e_{\fP}$ with respect to $\fP$. Then $\left| \Delta_E / \Delta_{E, \fP}^{\min} \right|_{\fP} = p^{-(\delta\cdot e_{\fP} - \delta_H)}$, where $\delta_H = \ord_{\fP}(\Delta_{E, \fP}^{\min})$.
Recall that
\[ \left| \frac{\omega}{\omega_{\fP}^{\min}} \right|_{\fP}^{-12} = \left| \frac{\Delta_E}{\Delta_{E, \fP}^{\min}} \right|_{\fP} .\] 
Therefore $\left| \omega / \omega_{\fP}^{\min} \right|_{\fP} = p^{\floor{\frac{\delta\cdot e_{\fP} - \delta_H}{12}}}$.

Suppose that $E / \bQ_p$ has Kodaira type $I_n^*$, so $\delta = 6 + n$. For a finite extension $K' / \bQ_p$ with ramification degree $e$, $E / K'$ has Kodaira type $I_{en}^*$ if $e$ is odd, and type $I_{en}$ if $n$ is even. Thus in odd degree extensions the reduction type will stay potentially multiplicative. Then $\delta \cdot e_{\fP} - \delta_H = 6 e_{\fP}$.

If $E / \bQ_p$ has potentially good reduction then $\delta \in \{2,3,4,6,8,9,10 \}$. $E$ also has potentially good reduction at the place $\fP$ in $F^H$. Hence $\delta_H \leq 12$ and it follows that $\delta_H = \delta \cdot e_\fP - 12 \cdot \floor{\delta \cdot e_\fP /12}$.

In conclusion, 
\[ D_{\fP \mid p} = 
    \begin{cases}
        (D_p, D_p,\ p^{\floor{e_ /2}}) & \text{if } E \text{ has potentially multiplicative reduction}, \\
        (D_p, D_p,\ p^{\floor{\delta \cdot e/12}}) & \text{if } E \text { has potentially good reduction}.
    \end{cases}
    \]

In either case, $D_{\fP \mid p}(\Theta) \in N_{\bQ(\rho) / \bQ}(\bQ(\rho)^{\times})$. Indeed, this takes values $1$ or $p$ in $\bQ^{\times} / \bQ^{\times 2}$. But $p \equiv 1 \pmod l$ implies $p \equiv 1 \pmod \ff$ so that $p$ is the norm of a principal ideal in $\bQ(\rho)$, and hence the norm of an element, by Corollary \ref{p-one-mod-disc} and Theorem \ref{p-norm-elem-1}.

%\[ \left|\frac{\Delta_{E}}{\Delta_{E, w}^\min} \right|_w = p^{f_w 12 \cdot \floor{e_w n / 12}} \implies 
%       \left|\frac{\omega}{\omega_{w}^\min} \right|_w = p \]
\vspace{1em}

For the Tamagawa number computations, since $p \geq 5$ we may write $E / \bQ_p$ as $E \colon y^2 = x^3 + A x + B$ and use the description from \cite{reg-const} for computing Tamagawa numbers, as detailed in Lemma \ref{tamagawa-num}. The discriminant of $E / \bQ_p$ is $\Delta = -16(4A^3 + 27 B^2)$, and we write $\delta = v_p(\Delta)$. {\color{red} maybe add a small reminder}

Firstly, suppose that $E / \bQ_p$ has reduction type $I_{n}^{*}$. 
Write $D_p = \Gal(F_w / \bQ_p)$. Since we assume $D_p = I_p$, i.e. the residue degree is one, it follows that any subextension $L'$ of $F_{w} / \bQ_p$ satisfies $\sqrt{B} \in L' \iff \sqrt{B} \in \bQ_p$ and $\sqrt{\Delta} \in L' \iff \sqrt{B} \in \bQ_p$. 
Therefore $T_{\fP \mid p} = (D_p, \alpha)$ where $\alpha \in \{2, 4\}$. But then $(D_p, \alpha)(\Theta) \in \bQ^{\times 2}$ by Proposition \ref{const-fns}.

Now suppose that $E / \bQ_p$ has potentially good reduction. Observe that in a totally ramified extension of degree coprime to $12$, the Tamagawa number remains the same. Hence for $D_p$ odd, if $3 \nmid |D_p|$ it follows that $T_{\fP \mid p} = (D_p, \alpha)$ for some constant $
\alpha$ and so $T_{\fP \mid p}(\Theta) \in \bQ^{\times 2}$.

Thus we assume that $3 \mid |D_p|$. Since we assumed $p \geq 5$, we have $D_p = I_p = P_p \ltimes C_l$ with $3 \mid l$ and $p \equiv 1 \pmod l$. 
If we have type $III$ or $III^*$ or $I_0^*$ then the Tamagawa number is still unchanged in any totally ramified extension of odd degree extension, even when the degree is divisible by $3$. We will treat the other cases separately: 

\vspace{1em}

\noindent\underline{\textit{Type $II$ and $II^*$ reduction:}}

Suppose $\delta = 2$, that is we have Type $II$ reduction. If $L' / \bQ_p$ is an odd degree extension that is divisible by $3$, then $E / L'$ has reduction type $I_0^*$. By Lemma \ref{tamagawa-num} the Tamagawa number of $E / L'$ then depends on whether $\sqrt{\Delta} \in \bQ_p$. Since we have additive reduction, we know that $p \mid A$, $p \mid B$. Moreover, $\delta = 2$ implies $2 = v_p(\Delta) = v_p(27 B^2 + 4A^3)$, so that $v_p(B) = 1$. Then, $\Delta = p^2\cdot \alpha$, and $\alpha \equiv -27\cdot\square \pmod p$. Therefore $\sqrt{\Delta} \in \bQ_p \iff -3$ is a square $\pmod p$. But this is the case; we assumed $p \equiv 1 \pmod l$, so $p \equiv 1 \pmod 3$. Then $\legendre{-3}{p} = \legendre{p}{3} = 1$. Therefore the Tamagawa number will be $1$ or $4$, which is a square.
On the other hand if $L' / \bQ_p$ is an extension of odd degree then the reduction type over $L'$ is $II$ or $II^*$ and the Tamagawa number is $1$. It follows that $T_{\fP \mid p}(\Theta)$ is a product of square terms, so is itself square. 

If $\delta = 10$, then $E / L'$ has reduction type $I_0^*$ whenever $3 \mid [ L' : \bQ_p]$. 
We show again that one must have $\sqrt{\Delta} \in \bQ_p$. Recall that $E / \bQ_p$ has potentially good reduction if and only if its $j$-invariant is integral, i.e. $v_p(j) \geq 0$ (cf. \cite[\S VII.5, Proposition 5.5]{S1}). One has $j = -1728(4A^3) / \Delta$ and so $3 v_p(A) \geq v_p(\Delta) = 10 \implies 3 v_p(A) > v_p(\Delta) \implies v_p(\Delta) = 2v_p(B)$, so $v_p(B) = 5$. Thus $\Delta = p^{10} \cdot \alpha$ with $\alpha \equiv -27 \cdot \square \pmod p$, and we conclude as above.    

\vspace{1em}

\noindent\underline{\textit{Type $IV$ and $IV^*$ reduction:}}

Now, if $E /\bQ_p$ has additive reduction of type $IV$ or $IV^*$, it attains good reduction over any totally ramified cyclic extension of degree divisible by $3$. This could result with $3$ coming up an odd number of times in $T_{\fP \mid p}(\Theta)$, when $\sqrt{B} \not\in \bQ_p$. 
%We show that for both types, one has $\sqrt{B} \in \bQ_p$. 
%Indeed, if $\delta = 4$, then $v_p(B) = 2$, and $v_p(A) \geq 2$. 
%\vspace{1em}
%In summary, 
%\begin{equation}
 %   \prod_{d ' \mid d} C(E / F_{\fp}^{D_{d'}})^{\mu(d / d')}
  %  = 
   % \begin{cases}
    %    1 & 3 \nmid d, \\
    %   1 & 3 \mid d, \delta \in \{0, 3, 6, 9\}, \\
    %    1 \cdot \square & 3 \mid d, \delta \in \{2, 10\}, \\
    %    3^a \cdot\square, a \in \{0,1\} & 3 \mid d, \delta \in \{4,8\}.
    %\end{cases}
%\end{equation}
%\begin{rem}
%   There's no reason why we can't get 3; see elliptic curve 441b1 with additive reduction at $7$ of type IV and Tamagawa number equal to $3$
%\end{rem}

%\textbf{If $D_p = C_l$ then we are able to finish our argument.} As in the proof of Proposition \ref{semi-stable-gd}, there exists $a_{l'} \in \bZ$ such that $\Res_{D_p}\Theta = \sum_{l' \mid l} a_{l'} \Psi_{l'}$ where $\Psi_{l'} \in \B(G)$ is such that $\bC[\Psi_{l'}] \simeq \chi_{l'}$, as in Example \ref{cyclic-relns}.

Recall from the proof of Proposition \ref{const-fns} that $\Res_{D_p} \Theta = \sum_i n_i \sum_{x \in H_i \backslash G / D_p} D_p \cap H^{x^{-1}}$, with $\sum_i n_i | H_i \backslash G / D_p|$ even. If $D_p = \Gal(F_w / \bQ_p)$, then the number of subextensions divisible by $3$ (i.e. the number of subextensions where we obtain good reduction ) corresponds to the number of subgroups with index divisible by $3$ in $\Res_{D_p}\Theta$. We compute this number to determine $\ord_3(T_{\fP \mid p}(\Theta))$ modulo squares.

Similarly to the split multiplicative case, we may pass to the tame quotient $C_p / P_p = C_l$. Indeed 

\[ 3 \mid  [D_p : D_p \cap H^{x^{-1}}] = \dim \bC [ D_p / D_p \cap H^{x^{-1}}] \iff 3 \mid \dim \bC[ D_p / D_p \cap H^{x^{-1}} ]^{P_p},\] 
since $3 \nmid|P_p|$. Therefore we may compute the number of subgroups divisible by $3$ in $\Psi = P_p \cdot \Res_{D_p} \Theta / P_p \in \B(C_l)$.  Let $h \colon \B(C_l) \to \bQ^{\times}$ be the function given by $H \mapsto \begin{cases} 3 & 3 \mid [C_l : H], \\ 1 & 3 \nmid [C_l : H]. \end{cases}$

\begin{prop}
   Suppose that $E / \bQ_p$ has additive reduction of Type $IV$ or $IV^*$, with $c_v(E / \bQ_p) = 3$.  Then $T_{\fP \mid p}(\Theta) \equiv h(\Psi) \mod \bQ^{\times 2}$ and $T_{\fP \mid p}(\Theta) \in \fieldnorm{\rho}$. 
\end{prop}

\begin{proof}
The fact that $T_{\fP \mid p}(\Theta) \equiv h(\Psi) \mod \bQ^{\times 2}$ has been observed above.
Let $\psi_3$ be an irreducible character of $D_p$ of order $3$. One has that $ \langle \Ind_{C_{l / l'}}^{C_l} \trivial , \psi_3 \rangle =  1$ when $3 \mid l'$ and is zero when  $3 \nmid l'$. 
Thus $$h(\Psi) = 3^{\langle \bC[C_l / \Psi], \psi_3 \rangle}.$$ As in the proof of Proposition \ref{index-fn-trivial}, write $\bC[C_l / \Psi] = \sum_{l' \mid l} a_{l'} \chi_{l'}$, where $\chi_{l'}$ is an irreducible rational character, of $C_l$ with kernel of index $l'$. Observe that $\langle \chi_{l'}, \psi_3 \rangle = 0$ unless $l' = 3$, in which case it is $1$. Therefore $h(\Psi) \equiv 3^{a_3} \mod \bQ^{\times 2}$. In the proof of Proposition \ref{index-fn-trivial}, we showed that $a_3$ is even unless $\ff \mid 3$,  i.e. that $\bQ(\rho) = \bQ(\sqrt{-3})$. But then $3$ is a norm in $\bQ(\rho)$. Thus we see that in all cases $T_{\fP \mid p}(\Theta) \in N_{\bQ(\rho) / \bQ}(\bQ(\rho)^{\times})$. 
\end{proof}

We have observed that for all reduction types of $E / \bQ_p$, one always has $T_{\fP \mid p} \cdot D_{\fP \mid p}(\Theta) \in \fieldnorm{\rho}$, and so $C(\Theta) \in \fieldnorm{\rho}$, completing the proof of Theorem \ref{odd-exts}.

\qed

{\color{red} corollary about restricting to odd order decomposition groups??}
%However, it turns out we will only get $3$ occurring oddly when $d = 3$. Indeed, one has that $\langle \Ind_{D_{d'}}^D \trivial, \psi_3 \rangle = 1$ if $3 \mid d'$, and $0$ if $3 \nmid d'$, where $\psi_3$ is an irreducible character of $D$ of order $3$. Therefore one sees that the number of places with ramification degree divisible by $3$ cancels unless $d = 3$. Indeed, $\langle \chi_d , \psi_3 \rangle = 0$ unless $d = 3$, 
%in which case it is $1$. Therefore (\ref{tam-contrib}) can only be non-square when $d = 3$. {\color{red} then conclude why this is fine}

%\begin{lemma}
%    Consider $M / L$ a field extension. Let $E / L$ be an elliptic curve, $v$ a finite place of $L$ and $w$ a finite place of $M$ with $w \mid v$. Let $\omega_v$ and $\omega_w$ be the minimal differentials for $E / L_v$ and $E / M_w$ respectively. 
    
%    Then, if $E / K_v$ has potentially good reduction and the residue characteristic is not $3$ or $2$, one has
    
%    \[ \left|\frac{\omega_v}{\omega_w} \right|_{w} = q^{\floor{\frac{e_{F / K} \cdot \ord_v(\Delta_{E, v}^{\min})}{12}}}, \]
%    where $q$ is the size of the residue field at $w$.
%\end{lemma}
%We consider $F / \bQ$ with additive potentially good reduction at $p$ . Since $D_p = I_p$, the size of the residue field is $p$ at all intermediate extensions. Let $n = v_p(\Delta)$. Then $n \in \{2,3,4,6,8,9,10\}$.  Consider $(D_p, I_p, \psi_p)$ where $\psi_p(e,f) = p^{\floor{e n / 12}}$. Then $(D_p, I_p, \psi_p) \sim_{\rho} 1$. Indeed, this takes values $1$ or $p$ in $\bQ^{\times} / \bQ^{\times 2}$. But $p \equiv 1 \pmod l$ implies that $p$ is the norm of a principal ideal in $\bQ(\rho)$, and hence the norm of an element, by corollary \ref{p-one-mod-disc} and theorem \ref{minus-one-norm}.
%So suppose an elliptic curve $E / \bQ$ has additive reduction at $p$, with $p \geq 5$. Then we can write $E \colon y^2 = x^3 + Ax + B$. Let $D = \Gal(F_{\fp} / \bQ_p)$ be the local Galois group at $p$. Assume that $p$ is totally tamely ramified, so that $D = I = C_n$. Since there is no wild ramification, and $f = 1$, this means that $n \mid p - 1$. We consider the contribution corresponding to an irreducible rational character $\chi_d$ of $D$, given by 
%\begin{equation}\label{tam-contrib}
%\prod_{d ' \mid d} C(E / F_{\fp}^{D_{d'}})^{\mu(d / d')}.
%\end{equation}
%Observe that in a totally ramified extension of degree coprime to $12$, the Tamagawa number remains the same. If $(12, d) = 1$, then $(12, d') = 1$ for $d' \mid  d$, so the Tamagawa number is constant across subfields $F_{\fp}^{D_{d'}}$. Therefore,
%\[\prod_{d ' \mid d} C(E / F_{\fp}^{D_{d'}})^{\mu(d / d')} = C(E / \bQ_p)^{\sum_{d' \mid d} \mu(d / d')} = 1,\]
%assuming $d > 1$. 
%So we only need to worry about when $3 \mid d$. 