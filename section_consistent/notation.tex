\subsection{Notation and Preliminary Results}

In previous sections, we have discussed the behaviour of Tamagawa numbers of elliptic curves that allows us to compute them over finite extensions of number fields. The following result will be helpful to compute the local terms arising from the minimal differential in explicit examples.

    \begin{lemma}\label{lem_Dterms}
        Let $E$ be an elliptic curve over a number field $K$, $F/K$ a finite extension. Let $\pp$ be a prime in $K$ and $\fP$ a prime in $F$ above $\pp$. Let $q$ be the size of the residue field of $K$ at $\fp$. %Denote by $e_{\fP \mid \fp}$, $f_{\fP \mid \fp}$ the ramification index and residue degree of $F_{\fP} / K_{\fp}$. 

        Let $\Delta_\pp$, $\omega_{\pp}$ and $\Delta_\fP$, $\omega_{\fP}$ be the minimal discriminants and differentials for $E/K_\pp$ and $E/F_\fP$, respectively. Then the following holds.
        \begin{enumerate}[(i)]
            \setlength\itemsep{0em}
            \item If $\pp$ is unramified at $F/K$ or if $E$ has good or multiplicative reduction at $\pp$, then the minimal model of $E / K_{\fp}$ and $E / F_{\fP}$ coincide so $| \omega_{\fp} / \omega_{\fP} |_{\fP} = 1$. 
            
            \item If the residual characteristic is distinct from $2$ or $3$, and $E$ has potentially good reduction, then $v_\pp(\Delta_\pp)<12$ and the same holds for $\fP$. In particular, 
            $$\left|\frac{\omega_{\fp}}{\omega_{\fP}}\right|_{\fP} = q^{f_{\fP \mid \fp}\cdot\floor{\frac{e_{\fP\mid\fp}\nu_\fp(\Delta_\fp)}{12}}}.$$
            \item If the residual characteristic is distinct from $2$ or $3$, and $E / K_{\fp}$ has potentially multiplicative reduction then 
            $$ \left|\frac{\omega_{\fp}}{\omega_{\fP}}\right|_{\fP} = q^{f_{\fP \mid \fp} \cdot \floor{\frac{e_{\fP \mid \fp}}{2}}}.$$
        \end{enumerate}
    \end{lemma}

\begin{proof}[Proof Sketch]
    Let $e = e_{\fP \mid \fp}$, $f = f_{\fP \mid \fp}$, $\delta = v_{\fp}(\Delta_{\fp})$ and $\delta_{\fP} = v_{\fP}(\Delta_{\fP})$. Then $v_{\fP}(\Delta_{\fp}) = e n$. Thus $|\Delta_{\fp} / \Delta_{\fP} |_{\fP} = q^{f\cdot (\delta \cdot e - \delta_{\fP})}$, whence $$\left| \frac{\omega_{\fp}}{\omega_{\fP}}\right|_{\fP} = q^{f \cdot \floor{\frac{\delta \cdot e - \delta_{\fP}}{12}}}.$$ 
    \begin{enumerate}[(i)]
        \setlength\itemsep{0em}
        \setcounter{enumi}{1}
        \item If $E / K_{\fp}$ has potentially good reduction then $\delta \in \{ 2,3,4,6,8,9,10 \}$ and $\delta_{\fP} \leq 12$. By reducing to minimal Weierstrass equation for $E / F_{\fP}$ it follows that $\delta_{\fP} = \delta\cdot e - 12 \cdot \floor{\delta\cdot e / 12}$.
        
        \item Let $E / K_p$ have Kodaira type $\I_n^*$, so $\delta = 6 + n$. If $e$ is even then $E / F_{\fP}$ has Kodaira type $\I_{en}$, so $\delta_{\fP} = en$ and $\delta \cdot e - \delta_{\fP} = 6 e$.
        Else if $e$ is odd, $E / F_{\fP}$ has Kodaira type $\I_{en}^*$ so $\delta_{\fP} = 6 + en$ and $\delta\cdot e - \delta_{\fP} = 6 e - 6$. But then $\floor{(6e - 6)/12} = \floor{(e - 1)/2} = \floor{e / 2}$ since $e$ is odd.
    \end{enumerate}
\end{proof}

Following this result, we introduce some notation that will be very useful to compute the local factors $C_{E/F}$ while using the representation theoretic language discussed in Section \ref{sec_norm}.

\begin{notation}\label{not_contr}
    Let $E$ be an elliptic curve defined over $\QQ$ and let $F/K$ be a finite extension of number fields. For each finite place $\pp$ of $K$, we write the \textbf{local contribution of $\pp$} as 
    %$$T_{\mathfrak{P}\mid\pp}(E/F)=\prod_{\mathfrak{P}\mid\pp}c_\mathfrak{P}(E/F)\quad\text{and}\quad D_{\mathfrak{P}\mid\pp}(E/F)=\prod_{\mathfrak{P}\mid\pp}\left|\frac{\Delta_{E,\mathfrak{P}}^{\min}}{\Delta_E}\right|_\mathfrak{P}^{\frac{1}{12}},$$
    $$T_{\mathfrak{P}\mid\pp}(E/F)=\prod_{\mathfrak{P}\mid\pp}c_\mathfrak{P}(E/F),\quad D_{\mathfrak{P}\mid\pp}(E/F)=\prod_{\mathfrak{P}\mid\pp}\left|\frac{\Delta_{E,\mathfrak{P}}^{\min}}{\Delta_E}\right|_\mathfrak{P}^{\frac{1}{12}}$$ 
    and $C_{\fP\mid\fp}(E/F)=T_{\fP\mid\fp}(E/F)D_{\fP\mid\fp}(E/F)$, where the product ranges over primes $\fP$ of $F$ over $\pp$. 

    We are ultimately interested in computing the global contributions from the terms above, so we also introduce notation for them. With the same notation as above, we denote the \textbf{global contribution over $F$} of the Tamagawa numbers and the discriminant terms as 
    $$T(E/F)=\prod_\pp T_{\fP\mid\pp}(E/F)=\prod_\fP c_\fP(E/F)\quad\text{and}\quad D(E/F)=\prod_\pp D_{\fP\mid\pp}(E/F)=\prod_\fP \left|\frac{\Delta_{E,\mathfrak{P}}^{\min}}{\Delta_E}\right|_\mathfrak{P}^{\frac{1}{12}}.$$ 

\end{notation}

An immediate consequence of this notation is the fact that $C_{E/F}=T(E/F)D(E/F)$.
Observe that if $E$ has good reduction over $\pp$, then $T_{\mathfrak{P}\mid\pp}(E/F)=D_{\mathfrak{P}\mid\pp}(E/F)=1$ for any finite extension $F$ of $K$. 

\begin{defn}\label{not_contr_fns}
    If $G = \Gal(F / \bQ)$ then for $p \in \bQ$ we define functions $T_{\fP \mid p}$, $D_{\fP \mid p}$ and $C_{\fP \mid p}$ on $\B(G)$ by 
    \[ T_{\fP \mid p}(H) = T_{\fP \mid p}(E / F^H), \quad D_{\fP \mid p}(H) = D_{\fP \mid p}(E / F^H), \quad C_{\fP \mid p}(H) = C_{\fP \mid p}(E / F^H). \]
    Note that if $H$, $H'$ are conjugate then $F^H$, $F^{H'}$ are isomorphic, and so the values of these functions are constant on conjugate subgroups, hence they are well-defined. Define $C \colon \B(G) \to \bQ^{\times}$ by $C \colon H \mapsto C_{E / F^H}$.  
\end{defn}
    
Note that $C_{\fP \mid p}$ is a $D_p$-local function. Indeed, suppose $D_p = \Gal(F_w / \bQ_p)$, where $F_w$ denotes the completion of $F$ with respect to a place $w$ lying above $p$. For a number field $K$ and place $v$, define $$C_v(E / K) = c_v(E / K) \cdot \left| \omega / \omega_v^{\min} \right|_v.$$ We use the same notation if $K$ is a local field (then the $v$ subscript holds no meaning).
One has
\begin{equation*}
    C_{\fP \mid p} = (D_p, C_v)
\end{equation*}
where $C_v$ is a function on $\B(D_p)$ sending $H \mapsto C_v(E / F_w^H)$.

The following proposition describes these functions in the language introduced in Section \ref{sec-norm-rels} for each reduction type of $E / \bQ$. We do not attempt to write a formula for $T_{\fP \mid \fp}$ in the case of additive reduction, computing this involves using Lemma \ref{lem_add_tam}.

\begin{prop}\label{prop_local_fns}
    Let $E / \bQ$ be an elliptic curve, $G = \Gal(F / \bQ)$ and $p$ a prime of $\bQ$. Let $n = v_p(\Delta_E)$. Consider the functions $C_{\fP \mid p}$, $T_{\fP \mid p}$, and $D_{\fP \mid p}$ on $\B(G)$ defined above. Then,
    \begin{enumerate}[(i)]
        \setlength\itemsep{0em}
        \item If $E / \bQ_p$ has good reduction, $C_{\fP \mid p} = 1$,
        \item If $E / \bQ_p$ has split multiplicative reduction then $C_{\fP \mid p} = T_{\fP \mid \fp} = (D_p, I_p, e n)$,
        \item If $E / \bQ_p$ has non-split multiplicative reduction, 
        $C_{\fP \mid p} = T_{\fP \mid \fp} = \left(D_p, I_p,
        \left\{\begin{smallmatrix}
            2   & 2 \mid en, 2 \nmid f,  \\
            en   &  2 \mid f, \\
            1   & \text{else}
        \end{smallmatrix}\right.\right),$ 
        \item If $E / \bQ_p$ has potentially good reduction and $p \not= 2, 3$, $D_{\fP \mid p} = (D_p, I_p, p^{f \floor{e n /12}})$, 
        \item If $E / \bQ_p$ has potentially multiplicative reduction and $p \not= 2, 3$, $D_{\fP \mid p} = (D_p, I_p, p^{f \floor{e / 2}})$.
    \end{enumerate}  
\end{prop} 
 
\begin{proof}
    \
    \begin{enumerate}[(i)]
        \setlength\itemsep{0em}
        \item Clear. 
        \item Lemma \ref{lem_Dterms}(i) implies $D_{\fP \mid p} = 1$. If $K' / \bQ_p$ is a finite extension of ramification degree $e$, then $E / K'$ has split multiplicative reduction of type $\I_{en}$, which has Tamagawa number $en$ by Lemma \ref{lem_mult_tam}.
        \item As for split, $D_{\fP \mid p} = 1$. The description follows from applying Proposition \ref{prop_semi_red} (iii) (non-split becomes split when the residue degree is even), and Lemma \ref{lem_mult_tam}. 
        \item Follows from Lemma \ref{lem_Dterms}(ii),
        \item Follows from Lemma \ref{lem_Dterms}(iii).
    \end{enumerate}
\end{proof}
%An immediate consequence of this notation is the fact that 
%$$C_{E/F}=\prod_{\pp}C_{\mathfrak{P}\mid \pp}(F/K);$$
%that is, we can calculate $C_{E/F}$ by calculating the contribution locally at each prime of $K$. 
%{\color{red} also important to mention at some point that if the reduction is semistable, then the terms in a norm relation coming from the discriminant also vanish. Probably this would have to be introduced later.}

\begin{rem}\label{rephrase-thm}
    We rephrase Theorem \ref{thm_positive_rank} in the language introduced in $\S$\ref{sec-norm-rels}. 
    Replacing $\rho$ by the sum of its conjugates by elements of $ \Gal(\bQ(\rho) / \bQ(\sqrt{D}))$, we may assume that $\bQ(\rho) = \bQ(\sqrt{D})$. Note that this does not affect the order of $\rho$ in $\C(G)$, nor the set of $\rho$-relations (since $\repnorm{\rho}$ is unchanged). 
    
    Let $\Theta$ be a $\rho$-relation with $\bC[G / \Theta] = \repnorm{\rho}^{\oplus m}$. Let $C \colon \B(G) \to \bQ^{\times}$ be the function sending $H \mapsto C_{E / F^H}$. The theorem then states that, if $\Theta$ is not a norm relation for $C$ when $m$ is odd, or if $C(\Theta) \not\in (\bQ^{\times})^2$ for $m$ even, then $ \rk E / F > 0$. 
\end{rem}