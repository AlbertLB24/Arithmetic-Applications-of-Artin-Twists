\subsection{Class field theory}
\subsubsection{Genus field}

In this section we introduce the concept of a genus field, as well as properties that will be useful for us.

Let $K$ be a number field. The \textbf{ideal class group} $\Cl_K = I_K / P_K$ is the group of fractional ideals quotiened{\color{red}?} by principal ideals.
For an ideal $\fp$, we let $[\fp]$ denote its class in $\Cl_K$.

The \textbf{extended ideal class group} is the group $\Cl_K^{+} = I_K / P_K^{+}$, where
$P_K^{+}$ denotes the subgroup of principal ideals with totally positive generator, i.e. ideals $\alpha \cO_K$ where $\sigma(\alpha) > 0$ for all real embeddings $\sigma \colon K \emb \bR$.

Note that $\Cl_K^{+}$ is the ray class group for the modulus $\fm$ of $K$ consisting of the product of all real places. The corresponding ray class field is known as the \textbf{extended Hilbert class field}, which we'll denote as $H^{+}$. This is the maximal extension of $K$ that is unramified at all finite primes. Let $H$ be the usual Hilbert Class field of $K$. Then one has $H \subset H^{+}$. Moreover, the index can be described in terms of the structure of $K$:

\begin{thm}[Janusz 3. Extended Class group]
    Let $r$ be the number of real primes of $K$. Let $U_K$, $U_K^{+}$ the group of units and totally positive units of $K$ respectively, Then 
    \[ [H^{+} \colon H] = 2^r [U_K \colon U_K^{+}]^{-1} .\]
\end{thm}
Observe that if $K$ has no real places, then $H^{+} = H$. For quadratic fields, the index depends on the norm of a fundamental unit:

\begin{cor}
    Let $K = \bQ(\sqrt{D})$ with $D$ a square-free positive integer. Let $\epsilon$ be a fundamental unit of $K$. Then $[H^{+} \colon H] = 1$ or $2$, according as $N_{K / \bQ}\left(\epsilon\right) = -1$ or $1$. 
\end{cor}


Fix $K = \bQ(\sqrt{D})$ for $D$ a squarefree integer. The (extended) Hilbert class field of $K$ need not be abelian over $\bQ$ (note that it is Galois over $\bQ$ by uniqueness of the (extended) Hilbert class field). However it can be convenient to consider the maximal subfield of $H$ that is Galois over $\bQ$. 

\begin{defn}
    For any abelian extension $K / \bQ$, the \textbf{genus field} of $K$ over $\bQ$ is the largest abelian extension $E$ of $\bQ$ contained in $H$. The \textbf{extended genus field } is the largest abelian extension $E^{+}$ of $\bQ $ contained in $H^{+}$.
\end{defn}

Let $\sigma \in \Gal(H^{+} / \bQ)$ be such that $\sigma|_{K}$ generates $\Gal(K / \bQ)$. $E$ has the following properties:

\begin{thm}\cite[Something 3.3]{Janusz} \label{janusz-3.3}
    \begin{enumerate}
        \item $\Gal(H / E)$ is isomorphic to the subgroup of $C_K$ generated by the ideal classes of the form $[\sigma(\fU)\fU^{-1}]$, $\fU \in I_K$. 
        \item $\Gal(H / E) \simeq (C_K)^2$. 
    \end{enumerate}
\end{thm}

Note that this says that every class $[\sigma(\fU) \fU^{-1}]$ is a square in $C_K$.
This allows us to deduce the following:

\begin{thm}\label{p-principal}
Let $p$ be a prime in $\bQ$. If the inertial degree of $p$ in $E / \bQ$ is $1$, then $p$ is the norm of a principal ideal in $K$. 
\end{thm} 

\begin{proof}
It's clear by inspection that $\Gal(E / K) = \Cl_K / \left(\Cl_K\right)^2$ is the maximal quotient of exponent $2$. Let $\fp$ be a prime of $K$ lying over $p$. Then $N_{K / \bQ}(\fp) = p$ and $\fp$ splits in $E$, so that $[\fp] \in (\Cl_K)^2$. Thus by theorem \ref{janusz-3.3} there is a fractional ideal $\fU$ of $I_K$ such that 
$[\fp] = [\sigma(\fU)\fU^{-1}]$. Observe that $N_{K / \bQ}(\sigma(\fU)\fU^{-1}) = 1$. It follows that $[\fp]^n$ is represented by a fractional ideal of norm $p$ for all $n$. Since $\Cl_K$ is finite, this implies there is a principal fractional ideal in $K$ of norm $p$. 
\end{proof}

The extended genus field $E^{+}$ is easier to describe than $E$.

\begin{thm} 
Suppose the discriminant of $K / \bQ$ has $t$ prime divisors. Then $C_K / (C_K)^2$ has order $2^{t-1}$ if $D < 0$ or if $D > 0$ and a unit of $K$ has norm $-1$. Otherwise, if $D > 0$ and all units of $K$ have norm $1$, it has order $2^{t - 2}$.
\end{thm} 

\begin{thm} 
Let the discriminant of $K$ be $\Delta$ and suppose $|\Delta| = p_1 p_2 \cdots p_t$ where $p_2, \ldots p_t$ are odd primes, and $p_1$ is either odd or a power of $2$. Then the extended genus field of $K$ is 
    \[ E^{+} = \bQ(\sqrt{D}, p_2^*, \ldots p_t^*) = K(p_2^*, \ldots p_t^*), \] 
where 
\[ \begin{cases}
    p_i^* = \sqrt{p_i} & \mathrm{if }\ p_i \equiv 1 \pmod 4, \\
    p_i^* = \sqrt{-p_i} & \mathrm{if }\ p_i \equiv 3 \pmod 4
\end{cases}\]
\end{thm} 

\begin{cor}\label{p-one-mod-disc}
    Let $p$ be a prime in $\bQ$, $K = \bQ(\sqrt{D})$ with discriminant $\Delta$ such that $|\Delta| = p_1 p_2 \cdots p_t$ as above. If $p \equiv 1 \pmod {|\Delta|}$, then $p$ is the norm of a fractional principal ideal in $K$. It is also the norm of a fractional principal ideal in $K' = \bQ({\sqrt{pD}})$. {\color{red} maybe I need $\Delta$ to be odd.}
\end{cor}

\begin{proof}
    Any prime above $p$ in $K$ splits in $E^{+}$, hence also in $E$ (in particular it has residue degree $1$). Similarly for $K'$, the inertial degree of $p$ in its extended genus field is $1$, and so in its genus field also.
\end{proof}

We want to understand when $p$ is the norm of an element. Note that if $H = H^{+}$, then $p$ being the norm of an ideal guarantees that it is the norm of an element. If $-1$ is a norm in our field then we are also fine. 

\begin{thm}\label{minus-one-norm}
Let $K = \bQ(\sqrt{D})$ and suppose that all odd primes dividing $D$ are congruent to $1 \pmod 4$. Then $-1$ is the norm of an element of $K^{\times}$. 
\end{thm}

\begin{proof}
{\color{red} write.}
\end{proof}

Note that $-1$ being the norm of an element in $K$ does not ensure that $-1$ is the norm of a unit in $K$. The smallest counter-example is $K = \bQ(\sqrt{34})$. The element $\frac{5}{3} + \sqrt{34}$ has norm $-1$, but there is no unit with norm $-1$. 

The following two results are needed in the body of this report.

\begin{thm}\label{p-norm-elem-1}
    Let $K = \bQ(\sqrt{D})$ and let $k$ be the minimal cyclotomic field such that $K \subset \bQ(\zeta_k)$. Suppose that $k$ is odd and $K$ is real.  If $p$ is a prime such that $p \equiv 1 \pmod {|\Delta|}$, then $p$ is the norm of an element from $K$. 
\end{thm}

\begin{proof}
    Note that $k$ being odd implies $D$ is odd. We know that $p$ is the norm of a principal fractional ideal of $K$ by corollary \ref{p-one-mod-disc}. Therefore there exists integers $x$, $y$, $z$ such that $\pm p z^2 = x^2 - Dy^2$. Suppose firstly that all primes dividing $D$ are congruent to $1 \pmod 4$. Then there is an element of $K^{\times}$ of norm $-1$ by theorem \ref{minus-one-norm}. Hence we can find an element of norm $p$.

    Otherwise, there exists a prime $q \mid D$ such that $q \equiv 3 \pmod 4$. Reducing mod $q$, we have
    $ \pm p = \square$. Since $p \equiv 1 \pmod q$, it is a square $\pmod q$. But $-1$ is not a square mod $q$, hence our sign must have been $+$ and so $p$ is the norm of an element from $K^{\times}$.
\end{proof}

\begin{thm}\label{p-norm-elem-2}
    Let $K = \bQ(\sqrt{D})$ and let $k$ be the minimal cyclotomic field such that $K \subset \bQ(\zeta_k)$. Suppose that $k$ is odd and $K$ is real. Let $p$ be a prime such that $p \mid D$ and $p \equiv 1 \pmod q$ for all other primes $q \mid D$. Then $p$ is the norm of an element from $K$. 
\end{thm}

\begin{proof}
    By corollary \ref{p-one-mod-disc}, we know that $p$ is the norm of a principal fractional ideal of $K$. The rest of the argument is analogous to the previous proof.
\end{proof}


\begin{prop}
$\bQ(\sqrt{p^*})$ has odd narrow class number.
\end{prop}    

\begin{cor}\label{p-norm}
The prime $p \in \bQ$ is the norm of an element in $\bQ(\sqrt{p^*})^{\times}$.
\end{cor}