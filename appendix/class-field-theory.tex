\subsection{Class field theory and genus fields}

In this section we introduce the concept of a genus field, as well as properties that will be useful for us.

Let $K$ be a number field. The \textbf{ideal class group} $\Cl_K = I_K / P_K$ is the group of fractional ideals modulo principal fractional ideals.
For an ideal $\fp$, we let $[\fp]$ denote its class in $\Cl_K$.

The \textbf{extended ideal class group} is the group $\Cl_K^{+} = I_K / P_K^{+}$, where
$P_K^{+}$ denotes the subgroup of principal fractional ideals with totally positive generator, i.e. ideals $\alpha \cO_K$ where $\sigma(\alpha) > 0$ for all real embeddings $\sigma \colon K \emb \bR$.

Note that $\Cl_K^{+}$ is the ray class group for the modulus $\fm$ of $K$ consisting of the product of all real places. The corresponding ray class field is known as the \textbf{extended Hilbert class field}, which we'll denote as $H^{+}$. This is the maximal extension of $K$ that is unramified at all finite primes. Let $H$ be the usual Hilbert Class field of $K$. Then one has $H \subset H^{+}$. Moreover, the index can be described in terms of the structure of $K$:

\begin{thm}\cite[Chapter VI, Section 3, Theorem 3.1]{Janusz}
    Let $r$ be the number of real primes of $K$. Let $U_K$, $U_K^{+}$ the group of units and totally positive units of $K$ respectively, Then 
    \[ [H^{+} \colon H] = 2^r [U_K \colon U_K^{+}]^{-1} .\]
\end{thm}
Observe that if $K$ has no real places, then $H^{+} = H$. For quadratic fields, the index depends on the norm of a fundamental unit:

\begin{cor}
    Let $K = \bQ(\sqrt{D})$ with $D$ a square-free positive integer. Let $\epsilon$ be a fundamental unit of $K$. Then $[H^{+} \colon H] = 1$ or $2$, according as $N_{K / \bQ}\left(\epsilon\right) = -1$ or $1$. 
\end{cor}


Fix $K = \bQ(\sqrt{D})$ for $D$ a squarefree integer. The (extended) Hilbert class field of $K$ need not be abelian over $\bQ$ (note that it is Galois over $\bQ$ by uniqueness of the (extended) Hilbert class field). However it can be useful to consider the maximal subfield of $H$ that is abelian over $\bQ$. 

\begin{defn}
    For any abelian extension $K / \bQ$, the \textbf{genus field} of $K / \bQ$ is the largest abelian extension $L / \bQ$ contained in $H$. The \textbf{extended genus field } is the largest abelian extension $L^{+} / \bQ $ contained in $H^{+}$.
\end{defn}

Let $\sigma \in \Gal(H^{+} / \bQ)$ be such that $\sigma|_{K}$ generates $\Gal(K / \bQ)$. $L$ has the following properties:

\begin{thm}\cite[Ch. VI, $\S$3, Theorem 3.3]{Janusz}\label{janusz-3.3}
    Let $K = \bQ(\sqrt{D})$.
    \begin{enumerate}
        \item $\Gal(H / L)$ is isomorphic to the subgroup of $C_K$ generated by the ideal classes of the form $[\sigma(\fU)\fU^{-1}]$, $\fU \in I_K$. 
        \item $\Gal(H / L) \simeq (C_K)^2$. 
    \end{enumerate}
\end{thm}

\begin{proof}[Proof sketch of 1.]
    Let $G = \Gal(H / \bQ)$. Then $L = H^{[G, G]}$ is the fixed field of the commutator subgroup of $G$. The Artin map induces an isomorphism $\varphi \colon C_K \to C \subset G$ with $\varphi(C_K) \simeq C = \Gal(H / K)$. One shows that $\varphi([\sigma(\fU)\fU^{-1}]) \in [G, C]$ and conversely that every commutator element in $[G, G]$ can be expressed as $\varphi([\sigma(\fU)\fU^{-1}])$ for some $\fU \in I_K$. 
\end{proof}

%Note that this says that every class $[\sigma(\fU) \fU^{-1}]$ is a square in $C_K$.
This allows us to deduce the following:

\begin{thm}\label{p-principal}
Let $p$ be a prime in $\bQ$. If the residue degree of $p$ in $L / \bQ$ is $1$, then $p$ is the norm of a principal fractional ideal in $K$. 
\end{thm} 

\begin{proof}
Let $\varphi \colon C_K \to \Gal(H / K)$ be the isomorphism induced by the Artin map. By Theorem \ref{janusz-3.3}, $\Gal(L / K) = \Cl_K / \left(\Cl_K\right)^2$ is abelian. Let $\fp$ be a prime of $K$ lying over $p$. Then $N_{K / \bQ}(\fp) = p$ and $\fp$ has residue degree $1$ in $L$. It follows that $\varphi([\fp])|_{L} = \Id$ so that $\varphi([\fp]) \in \Gal(H / L)$. Thus by Theorem \ref{janusz-3.3} there is a fractional ideal $\fU$ of $I_K$ such that 
$[\fp] = [\sigma(\fU)\fU^{-1}]$. Observe that $N_{K / \bQ}(\sigma(\fU)\fU^{-1}) = 1$. It follows that we can represent $[\fp]^n$ by a fractional ideal of norm $p$ for all $n \geq 1$. Since $\Cl_K$ is finite, this implies there is a principal fractional ideal in $K$ of norm $p$. 
\end{proof}

Observe that $C_K / (C_K)^2$ is an abelian group of exponent $2$. The following theorem describes its order:

\begin{thm}\cite[Ch VI, \S3, Theorem 3.9]{Janusz}
Suppose the discriminant of $K / \bQ$ has $t$ prime divisors. Then $C_K / (C_K)^2$ has order $2^{t-1}$ if $D < 0$ or if $D > 0$ and a unit of $K$ has norm $-1$. Otherwise, if $D > 0$ and all units of $K$ have norm $1$, it has order $2^{t - 2}$.
\end{thm} 

\begin{rem}\label{conductor}
    The conductor of $K / \bQ$ is a particular modulus for $K / \bQ$ ({\color{red} ref}). We denote the finite part of it by $\ff \in \bZ_{>0}$ . Then $\ff$ is the smallest positive integer such that $K \subset \bQ(\zeta_{\ff})$. For quadratic fields, one has that $\ff = |\Delta|$ where $\Delta$ is the discriminant of $K$ ({\color{red} ref}). Thus
\[ \ff = |\Delta| =  \begin{cases} |D| & D \equiv 1 \pmod 4, \\ 4|D| & D \not\equiv 1 \pmod 4. \end{cases} \]
\end{rem}

Our introduction of the extended genus field $L^{+}$ is mostly because it is easier to describe than $L$.

\begin{thm}\label{extended-genus}\cite[Ch VI, \S3, Theorem 3.10]{Janusz}
Let the discriminant of $K = \bQ(\sqrt{D})$ be $\Delta$ and suppose $|\Delta| = p_1 p_2 \cdots p_t$ where $p_2, \ldots p_t$ are odd primes, and $p_1$ is either odd or a power of $2$. Then the extended genus field of $K$ is 
    \[ L^{+} = \bQ(\sqrt{D}, \sqrt{p_2^*}, \ldots \sqrt{p_t^*}) = K(\sqrt{p_2^*}, \ldots \sqrt{p_t^*}), \] 
where 
\[ \begin{cases}
    p_i^* = p_i & \mathrm{if }\ p_i \equiv 1 \pmod 4, \\
    p_i^* = -p_i & \mathrm{if }\ p_i \equiv 3 \pmod 4
\end{cases}\]
\end{thm} 

\vspace{1em}

\begin{cor}\label{p-one-mod-disc}
    Let $p$ be an odd prime in $\bQ$, $K = \bQ(\sqrt{D})$ with discriminant $\Delta$ such that $|\Delta| = p_1 p_2 \cdots p_t$, as in Theorem \ref{extended-genus}. If $p \equiv 1 \mod {|\Delta|}$, then $p$ is the norm of a principal fractional ideal in $K$. 
    It is also the norm of a principal fractional ideal in $K' = \bQ({\sqrt{p^*D}})$.
\end{cor}

\begin{proof}
   Let $L^+ = \bQ(\sqrt{D}, \sqrt{p_2^*}, \ldots \sqrt{p_t^*})$ be the extended genus field of $K$, and $L$ the genus field. If $p$ has residue degree $1$ in $L^+ / \bQ$, then it has residue degree $1$ in $L / \bQ$, and the first result follows by Theorem \ref{p-principal}. 

   Note that $p$ splits in the quadratic subfields $\bQ(\sqrt{p_i^*})$ for $i = 2, \ldots t$, since $\legendre{p_i^*}{p} = \legendre{p}{p_i} = 1$, as $p \equiv 1 \pmod {\Delta} \implies p \equiv 1 \pmod {p_i}$ for $i = 2, \ldots t$. To show that $p$ splits in $L^+$ we just need to show that it also splits in $K$. 
   
   First suppose $D \equiv 1 \pmod 4$. Write $D = \prod_{i=1}^t p_i^*$. Then 
    \[ \legendre{D}{p} = \prod_{i = 1}^t \legendre{p_i^*}{p} = \prod_{i = 1}^t \legendre{p}{p_i} = 1. \]
    Thus $p$ splits in $K$.

    Now consider that $D \not\equiv 1 \pmod 4$. First assume $D \equiv 3 \pmod 4$. Write $D = -\prod_{i=2}^t p_i^*$. Then
    \[ \legendre{D}{p} = \legendre{-1}{p} \prod_{i = 2}^t \legendre{p_i^*}{p} = \legendre{-1}{p} = 1, \] 
    since $p \equiv 1 \pmod {|\Delta|}$ and $4 \mid |\Delta| \implies p \equiv 1 \pmod 4$. 

    Now assume that $2 \mid D$ so that $8 \mid |\Delta|$ and $p \equiv 1 \pmod 8$. Then $D = \pm 2 \prod_{i = 2}^t p_i^*$. Thus
    \[ \legendre{D}{p} = \legendre{\pm 1}{p} \legendre{2}{p}\prod_{i = 2}^t \legendre{p_i^*}{p} = 1 .\] 

    Let $L'^{+}$ be the extended genus field of $K'$. Now $p$ ramifies in $K'$, $\bQ(\sqrt{p^*})$. Using the calculations above, it is either split or ramified in all quadratic subfields of $L'^{+}$, and so has residue degree $1$ in $L'^{+}$, and the result follows. 
\end{proof}

We want to understand when $p \in N_{K / \bQ}(K^{\times})$. If $p$ is the norm of a principal fractional ideal in $K$, then $\pm p \in N_{K / \bQ}(K^{\times})$. If $K$ is imaginary, one must have $p \in N_{K / \bQ}(K^{\times})$. We can also arrive to the same conclusion when $K$ is real and $-1 \in N_{K / \bQ}(K^{\times})$. 

\begin{thm}\label{minus-one-norm}
Let $K = \bQ(\sqrt{D})$ with $D >0$ squarefree and suppose that all odd primes dividing $D$ are congruent to $1 \pmod 4$. Then $-1$ is the norm of an element of $K^{\times}$. 
\end{thm}

\begin{proof}
The condition on $D$ ensures that there exists $x$, $y \in \bQ$ such that $D = x^2 + y^2$. Therefore $-1 = (x / y)^2 - D(1/ y)^2$ so that $-1$ is the norm of the element $\frac{x}{y} + \frac{1}{y} \sqrt{D}$.
\end{proof}

Note that $-1$ being the norm of an element in $K$ does not ensure that $-1$ is the norm of a unit in $K$. The smallest counter-example is $K = \bQ(\sqrt{34})$. The element $\frac{5}{3} + \frac{1}{3}\sqrt{34}$ has norm $-1$, but there is no unit with norm $-1$. 

\begin{thm}\label{p-norm-elem-1}
    Let $K = \bQ(\sqrt{D})$ be a real quadratic field.  If $p$ is an odd prime such that $p \equiv 1 \pmod {|\Delta|}$, then $p$ is the norm of an element from $K$. 
\end{thm}

\begin{proof}
    We know that $p$ is the norm of a principal fractional ideal of $K$ by Corollary \ref{p-one-mod-disc}. 
    Therefore there exists integers $x$, $y$, $z$ such that $\pm p z^2 = x^2 - Dy^2$. 
    
    Suppose firstly that all odd primes dividing $D$ are congruent to $1 \pmod 4$. Then there is an element of $K^{\times}$ of norm $-1$ by Theorem \ref{minus-one-norm}. Hence we can find an element of norm $p$.

    Otherwise, there exists a prime $q \mid D$ such that $q \equiv 3 \pmod 4$. Reducing mod $q$, we have
    $ \pm p = \square$. Since $p \equiv 1 \pmod q$, it is a square $\pmod q$. But $-1$ is not a square mod $q$, hence our sign must have been $+$ and so $p$ is the norm of an element from $K^{\times}$.
\end{proof}


\begin{thm}\label{p-norm-elem-2}
    Let $K = \bQ(\sqrt{D})$ be a real quadratic field. Let $p$ be an odd prime such that $p \mid D$ and $p \equiv 1 \pmod {|\Delta| / p}$. Then $p$ is the norm of an element from $K$. 
\end{thm}

\begin{proof}
    By Corollary \ref{p-one-mod-disc}, we know that $p$ is the norm of a principal fractional ideal of $K$. The rest of the argument is analogous to the previous proof.
\end{proof}  

\begin{cor}\label{p-norm}
The odd prime $p \in \bQ$ is the norm of an element in $\bQ(\sqrt{p^*})^{\times}$.
\end{cor}

\begin{thm}\label{thm_class_number}
    Let $h(D)$ be the class number of $\QQ(\sqrt{D})$ and let $p$ be a rational prime. Then the following holds.
    \begin{itemize}
        \item If $p\equiv1\pmod{4}$, then $h(p)$ is odd and $h(-p)$ is even.
        \item If $p\equiv3\pmod{4}$, then $h(p)$ and $h(-p)$ are both odd.
    \end{itemize}
\end{thm}

