\subsection{Compatibility in odd order extensions}

In this section we prove the following:

\begin{thm}
    Let $E / \bQ$ be an elliptic curve. Let $F / \bQ$ be an extension of \textbf{odd order} with Galois group $G$. Suppose that the primes of additive reduction of $E$ are at worst tamely ramified in $F / \bQ$ (and $\geq 5$). 
    
    Then for any representation $\rho$ of $G$ and any expression as in (\ref{rho-reln}), the corresponding ratio of Tamagawa numbers is a norm from any quadratic subfield of $\bQ(\rho)$.
\end{thm}

If $\bQ(\rho) = \bQ$ there is nothing to prove. 
If $[\bQ(\rho) \colon \bQ] > 1$ then this index is even. Indeed, since $G$ has odd order, there is an element $\sigma \in \Aut(\bQ(\rho) / \bQ)$ that acts by conjugation (i.e. is of order $2$). Therefore there is a quadratic subfield $\bQ(\sqrt{d}) \subset \bQ(\rho)$. Choose any such quadratic subfield. Replacing $\rho$ by the sum of its conjugates by elements of $ \Gal(\bQ(\rho) / \bQ(\sqrt{d}))$, we may assume that $\bQ(\rho) = \bQ(\sqrt{d})$. Let $\tau$ be the generator of $\Gal(\bQ(\sqrt{d}) / \bQ)$. Let $m$ be the smallest integer such that $\bQ(\sqrt{d}) \subset \bQ(\zeta_m)$. Then $m$ divides the exponent of $G$, hence is odd.

We prove that each $(D_p, I_p, \psi_p)$  satisfies $(D_p, I_p, \psi_p) \sim_{\rho} 1$. Since we deal with each local factor individually, we may assume that $D_p = I_p$ by theorem \ref{DeqI}. 

\subsubsection*{Good reduction}
If $E / \bQ$ has good reduction at $p$, then it has good reduction at all primes lying above $p$ in subfields of $F$. Thus $C_p(E / F_i) = 1$ for each subfield $F_i \subset F$, so that $(D_p, I_p, \psi_p) = 1$ on $\rho$-relations. 

\subsubsection*{Multiplicative reduction}

Let $p$ be a prime of muliplicative reduction. First suppose that this reduction is non-split. Since $D_p = I_p$, all primes above $p$ have residue degree $1$. Thus the reduction type remains non-split at primes above $p$. Therefore $\psi_p = 1$ or $2$, depending on $\ord_p(\Delta)$ being even or odd. 

We prove a more general lemma that constant functions are trivial on $\rho$-relations.

\begin{lemma}
Let $G$, $\rho$ be as above. If $(D_p, I_p, \psi_p)$ is such that $\psi_p$ is constant, then $(D_p, I_p, \psi_p) \sim_{\rho} 1$.  
\end{lemma}   

\begin{proof}
    Let $\psi_p = \alpha$. Then $(D_p, I_p, \psi_p)$ sends $H \leq G$ to $\alpha^{| H \backslash G / D_p|}$. Thus if $\Theta = \sum_i n_i H_i$ is a $\rho$-relation, $(D_p, I_p, \psi_p)(\Theta) = \alpha^{ \sum_i n_i \cdot | H_i \backslash G / D_p|}$. We show that $\sum_i n_i \cdot | H_i \backslash G / D_p|$ is even. 

    %Since $G$ is odd, each $| H_i \backslash G / D_p|$ is odd. On the other hand, $\rho \oplus \tau(\rho)$ has even dimension. Therefore $\sum_i |n_i|$ is even. 
    One has $\Res_D \Theta = \sum_i n_i \sum_{x \in H_i \backslash G / D} D \cap H^{x^{-1}}$ and the permutation representation $\bC[\Res_D \Theta]$ of $D$ is isomorphic to $\Res_D (\rho \oplus \tau(\rho))$. In particular the dimension of this permutation representation is even. The dimension is $\sum_i n_i \sum_{x \in H_i \backslash G / D} [D \colon D \cap H^{x^{-1}} ]$. Since each $[D \colon D \cap H^{x^{-1}} ]$ is odd, this implies there are an even number of terms in the summation, i.e. that $\sum_i n_i \cdot | H_i \backslash G / D_p|$ is even. 
    
\end{proof}

Now suppose that $p$ has split multiplicative reduction. Then $\psi_p(e, f) = e$.
The following result shows that if $\bQ(\Res_{D_p}\rho ) = \bQ$, then $(D_p, I_p, \psi_p) \sim_{\rho} 1$\dots

\begin{lemma}
    Let $G, \rho$ be as above, $D_p \leq G$. Let the exponent of $D_p$ be $b$. If $m \nmid b$, then $(D_p, I_p, \psi_p) \sim_{\rho} 1$. 
\end{lemma}

\begin{proof}
    Note that $\bQ(\Res_{D_p}{\rho}) \subset \bQ(\zeta_b) \cap \bQ(\rho) \subset \bQ(\zeta_b)$. Then $\bQ(\rho) \subset \bQ(\zeta_b) \implies m \mid b$ by minimality of $m$. Thus, if $m \nmid b$, one has $\bQ(\rho) \not\subset \bQ(\zeta_b)$ and so $\bQ(\Res_{D_p} \rho) = \bQ$. 

    Then, $\Res_{D_p} \rho = \tau (\Res_{D_p} \rho)$. Let $\Theta$ be a $\rho$-relation. It follows that every rational representation that is a summand of $\bC[\Res_{D_p} \Theta]$ arises with multiplicity two. 
    Thus, there is a $(\Res_{D_p}\rho$)-relation $\Theta ' $ such that $\bC[\Res_{D_p} \Theta] \simeq \bC[2 \Theta']$. This means $\Psi = \Theta - 2 \Theta'$ is a Brauer relation for $D_p$. Therefore, $(D_p, I_p, \psi_p)(\Theta) = (D_p, I_p, \psi_p)(\Psi) \cdot (D_p, I_p, \psi_p)(2 \Theta') = (D_p, I_p, e)(\Psi)  \cdot (D_p, I_p, \psi_p)(\Theta')^2 = 1$ as as function to $\bQ^{\times} / \bQ^{\times}$. Indeed $(D_p, I_p, e) = 1$ as a function to $\bQ^{\times} / \bQ^{\times}$ on Brauer relations, as per example \ref{trivial-on-brauer}.
\end{proof}

We have $D_p = I_p = P_p \ltimes C_l$, where $P_p \triangleleft I_p$ is wild inertia, and $C_l = I_p / P_p$ is the tame quotient. $C_l$ is a cyclic group, with $l \mid p^f - 1 = p - 1$. By the previous result, it is only of interest to consider decomposition groups $D_p = P_p \ltimes C_l$, with $m \mid p^u l$ for some $u \geq 0$. 

In this case, $(D_p, I_p, \psi_p)(\Theta)$ is the product of ramification indices at primes above $p$. We seperate the $p$-part and tame part of this expression.
Recall that the ramification index of a place $w$ above $p$ corresponding to the double coset $H_i x D_p$ has ramification degree $\frac{|I_p|}{|H_i \cap I_p^x|} =\frac{|I_p|}{|I_p \cap H^{x^{-1}}|}$.
This is the dimension of the permutation representation $\bC[D_p / D_p \cap H^{x^{-1}}]$.
Let  $D_p \cap H^{x^{-1}} = P' \ltimes C_k$ where $P' \leq P$ and $k | l$. Then the ramification index is $\frac{|P|}{|P'|}\cdot \frac{l}{k}$. 


Consider taking fixed points $\bC[D_p / D_p \cap H^{x^{-1}}]^{P_p} \simeq \bC[D_p / P_p (D_p \cap H^{x^{-1}})] \simeq \bC[D_p / P_p \ltimes C_k]$. Now this has dimension $\frac{l}{k}$, so we've killed off the $p$-part. 

We have $\bC[\Res_{D_p} \Theta]^{P_p} \simeq \left(\Res_{D_p} \rho \oplus \tau(\rho)\right)^{P_p}$. 




Therefore, the place is tamely ramified if and only if $P_p \subset I \cap H^{x^{-1}}$. 


\subsubsection*{dv terms in additive reduction}


\subsubsection*{Tamagawa numbers in additive reduction}

We use the following description of Tamagawa numbers. %Tim and Vlad reg consts Lemma 3.19

\begin{lemma}
    Let $K' /K / \bQ_p$ be finite extensions and $p \geq 5$. Let $E / K$ be an elliptic curve with addtive reduction; 
    \[ E \colon y^2 = x^3 + Ax + B, \]
    with discriminant $\Delta = -16(4 A^3 + 27 B^2)$. Let $\delta = v_K(\Delta)$, and $e = e_{K' / K}$.

    If $E$ has potentially good reduction, then 
        \[
        \begin{array}{l l l l}
            \gcd(\delta e, 12) = 2 & \implies & c_v(E / K') = 1, & \quad (II, II^*) \\
            \gcd(\delta e, 12) = 3 & \implies & c_v(E / K') = 2, & \quad (III, III^*) \\
            \gcd(\delta e, 12) = 4 & \implies & c_v(E / K') = \begin{cases} 1, & \sqrt{B} \notin K'
                                \\ 3, & \sqrt{B} \in K' \end{cases}, & \quad (IV, IV^*) \\
            \gcd(\delta e, 12) = 6 & \implies & c_v(E / K') = \begin{cases} 2, & \sqrt{\Delta} \notin K'
                \\ 1 \ \text{or} \ 4, & \sqrt{\Delta} \in K' \end{cases}, & \quad (I_0^*) \\
            \gcd(\delta e, 12) = 12 & \implies & c_v(E / K') = 1. & \quad (I_0)
        \end{array}
        \]
    Moreover, the extensions $K'(\sqrt{B}) / K'$ and $K'(\sqrt{\Delta}) / K'$ are unramified.
\end{lemma}

So suppose an elliptic curve $E / \bQ$ has additive reduction at $p$, with $p \geq 5$. Then we can write $E \colon y^2 = x^3 + Ax + B$. Let $D = \Gal(F_{\fp} / \bQ_p)$ be the local Galois group at $p$. Assume that $p$ is totally tamely ramified, so that $D = I = C_n$. Since there is no wild ramification, and $f = 1$, this means that $n \mid p - 1$. We consider the contribution corresponding to an irreducible rational character $\chi_d$ of $D$, given by 
\begin{equation}\label{tam-contrib}
\prod_{d ' \mid d} C(E / F_{\fp}^{D_{d'}})^{\mu(d / d')}.
\end{equation}

Observe that in a totally ramified extension of degree coprime to $12$, the Tamagawa number remains the same. If $(12, d) = 1$, then $(12, d') = 1$ for $d' \mid  d$, so the Tamagawa number is consant accross subfields $F_{\fp}^{D_{d'}}$. Therefore,
\[\prod_{d ' \mid d} C(E / F_{\fp}^{D_{d'}})^{\mu(d / d')} = C(E / \bQ_p)^{\sum_{d' \mid d} \mu(d / d')} = 1,\]
assuming $d > 1$. 

So we only need to worry about when $3 \mid d$. If we have type $III$ or $III^*$ or $I_0^*$ then the Tamagawa number is still unchanged in any totally ramified cyclic extension of degree dividing $d$. We will treat the other cases seperately: 

\vspace{1em}

\noindent\underline{\textit{Type $II$ and $II^*$ reduction:}}

Firstly, suppose that $\delta = 2$, that is we have Type $II$ reduction. If $3 \mid d'$ then $E / F_{\fp}^{D_{d'}}$ has type $I_0^*$ reduction. The Tamagawa number then depends on whether $\sqrt{\Delta} \in \bQ_p$. Since we have additive reduction, we know that $p \mid A$, $p \mid B$. Moreover, $\delta = 2$ implies that $v_p(B) = 1$. Then, $\Delta = p^2\cdot \alpha$, and $\alpha \equiv -27\cdot\square \pmod p$. Therefore $\sqrt{\Delta} \in \bQ_p \iff -3$ is a square $\pmod p$. But this is the case; we assumed $p \equiv 1 \pmod n$, so $p \equiv 1 \pmod 3$. Therefore the Tamagawa number will be $1$ or $4$, which is a square.
If $3 \nmid d'$ then the reduction type over $ F_{\fp}^{D_{d'}}$ is $II$ or $II^*$. Then the Tamagawa number is $1$. Thus in total, we get a square contribution from (\ref{tam-contrib}).

If $\delta = 10$, then $E / F_{\fp}^{D_{d'}}$ has reduction type $I_0^*$ whenever $3 \mid d'$. Once more, $v_p(A), v_p(B) \geq 1$, and $v_p(\Delta) = 10 = \min(3 v_p(A), 2 v_p(B))$ {\color{red} maybe this is suss} $\implies v_p(B) = 5$. Therefore we get $\Delta = p^{10} \alpha$ with $\alpha \equiv -27\cdot\square \pmod p$, and we conclude as above.

\vspace{1em}

\noindent\underline{\textit{Type $IV$ and $IV^*$ reduction:}}

Now, if $E /\bQ_p$ has additive reduction of type $IV$ or $IV^*$, it attains good reduction over any totally ramified cyclic extension of degree divisible by $3$. This could result with $3$ coming up an odd number of times in our Tamagawa number product, when $\sqrt{B} \not\in \bQ_p$. 

%We show that for both types, one has $\sqrt{B} \in \bQ_p$. 
%Indeed, if $\delta = 4$, then $v_p(B) = 2$, and $v_p(A) \geq 2$. 
\vspace{1em}
In summary, 
\begin{equation}
    \prod_{d ' \mid d} C(E / F_{\fp}^{D_{d'}})^{\mu(d / d')}
    = 
    \begin{cases}
        1 & 3 \nmid d, \\
        1 & 3 \mid d, \delta \in \{0, 3, 6, 9\}, \\
        1 \cdot \square & 3 \mid d, \delta \in \{2, 10\}, \\
        3^a \cdot\square, a \in \{0,1\} & 3 \mid d, \delta \in \{4,8\}.
    \end{cases}
\end{equation}

\begin{rem}
   There's no reason why we can't get 3; see elliptic curve 441b1 with additive reduction at $7$ of type IV and Tamagawa number equal to $3$) 
\end{rem}


However, it turns out we will only get $3$ occuring oddly when $d = 3$. Indeed, one has that $\langle \Ind_{D_{d'}}^D \trivial, \psi_3 \rangle = 1$ if $3 \mid d'$, and $0$ if $3 \nmid d'$, where $\psi_3$ is an irreducible character of $D$ of order $3$. Therefore one sees that the number of places with ramification degree divisible by $3$ cancels unless $d = 3$. Indeed, $\langle \chi_d , \psi_3 \rangle = 0$ unless $d = 3$, 
in which case it is $1$. Therefore (\ref{tam-contrib}) can only be non-square when $d = 3$.
