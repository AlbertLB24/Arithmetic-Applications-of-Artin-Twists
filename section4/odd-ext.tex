\subsection{Norm relations in odd order extensions}

{\color{red} add some motivation (justification) for why I'm proving this result. The point is that the test for positive rank provided by root number computations never says anything in odd order extensions. If we expect the norm relations test to be weaker than root numbers, then nor should this test.}

\begin{thm}\label{odd-exts}
 Let $E / \bQ$ be an elliptic curve, $F / \bQ$ be an extension of \textbf{odd order} with Galois group $G$. 
 
Suppose that the primes of additive reduction of $E$ are at worst tamely ramified in $F / \bQ$ (and $\geq 5$). 
Take any representation $\rho \in R(G)$ with quadratic subfield $\bQ(\sqrt{D}) \subset \bQ(\rho)$ and relation
\begin{equation*}\label{odd-exp}
 \left(\bigoplus_{\mathfrak{g}\in\Gal(\QQ(\rho)/\QQ)}\rho^{\mathfrak{g}}\right)^m=\bigoplus_i\Ind_{F_i/\QQ}\mathds{1}\ominus\bigoplus_j\Ind_{F'_j/\QQ}\mathds{1}
\end{equation*}
 as in theorem \ref{thm_positive_rank}. Then
 \[ \frac{\prod_i C_{E/F_i}}{\prod_j C_{E/F_j'}}  \in 
    \begin{cases}
        N_{\bQ(\sqrt{D}) / \bQ}(\bQ(\sqrt{D})^{\times}) & m \ \text{odd}, \\
        \bQ^{\times 2} & m \ \text{even}.
    \end{cases} \] 
    In other words, one cannot use theorem \ref{thm_positive_rank} to conclude that $E / F$ must have positive rank. 
\end{thm}
Let us break up the function $C \colon H \mapsto C_{E / F^H}$ into $C = \prod_p  \fc_p \cdot d_p$ where
\begin{equation}\label{c-and-d}
         \fc_p(H) = \prod_{v | p}  c_v(E / F^H) , \qquad d_p(H) = \prod_{v | p} \left|\frac{\omega}{\omega_{v}^{\min}}\right|_v, 
\end{equation}
the product ranging over all finite places of $F^H$ dividing $p$.
Observe that $\fc_p$ and $d_p$ are $D_p$-local functions. Recall the notation that for a number field $K$ and place $v$, $C_v(E / K) = c_v(E / K) \cdot \left| \omega / \omega_v^{\min} \right|_v.$ 
Then
\begin{equation}\label{Dp-loc}
\fc_p \cdot d_p = (D_p, f_p)
\end{equation}
where $f_p$ is a function on $B(D_p)$ with $D_p \cap H^{x^{-1}} \mapsto C_v(E / F^H)$ when $HxD_p$ is a double coset representative corresponding to the place $v$ of $F^H$. {\color{red} say this better. write $D_p$ as a local Galois group and say more generally how this fn is defined}

%We will observe throughout that if $v$ is a place of $F^H$ corresponding to the double coset $HxD_{p}$, then $C_v(E / F^H)$ is a function of $e_v$ and $f_v$. Thus we have
%\[ \fc_p \cdot d_p = (D_p, I_p, C_v) \]

We first argue that it is enough to prove this theorem when $m$ is the order of $\widetilde{\rho}$ in $C(G)$. 

\begin{lemma}
    Consider the set-up as in theorem \ref{odd-exts}. If the statement of the theorem holds when $m$ is the order of $\widetilde{\rho}$ in $C(G)$, then it holds for all possible $m$.
\end{lemma}

\begin{proof}
    By definition, if $m$ is the order of $\widetilde{\rho}$ in $C(G)$ then there exists $\Theta \in B(G)$ such that $\bC[\Theta] \simeq \widetilde{\rho}^{\oplus m}$.
    Now consider $n$ and $\Psi \in B(G)$ such that $\bC[\Psi] \simeq \widetilde{\rho}^{\oplus n}$. Clearly $m \mid n$.  Then $\Phi = \Psi - \frac{n}{m}\Theta$ is a Brauer relation.

    At this point, we use that the function $C$ on $B(G)$, viewed as a function to $\bQ^{\times} / \bQ^{\times 2}$, is trivial on Brauer relations. For odd order groups this follows from \cite[Theorem 2.47]{reg-const} and \cite[Theorem 3.2  (Tam)]{reg-const}, looking locally at each $C_p$. 
    Then $C(\Psi) = C(\Theta)^{n / m} \cdot C(\Phi)$ so $C(\Psi) \equiv C(\Theta)^{n / m} \mod \bQ^{\times 2}$. It follows that if $C(\Theta)$ satisfies the conditions of theorem \ref{odd-exts}, then so does $C(\Psi)$.
\end{proof}

Taking $m$ to be the order of $\widetilde{\rho}$ in $C(G)$, we have that $m$ divides $|G|$, hence is odd. Therefore we need to prove that, given any $
\Theta \in B(G)$ such that $\bC[\Theta] \simeq \widetilde{\rho}^{\oplus m}$, the expression $C(\Theta)$ is the norm of an element from $\bQ(\sqrt{D})^{\times}$.
Replacing $\rho$ by the sum of its conjugates by elements of $ \Gal(\bQ(\rho) / \bQ(\sqrt{D}))$, we may assume that $\bQ(\rho) = \bQ(\sqrt{D})$. Note that this does not affect $m$. 

We record
\begin{table}[H]
         \vspace{-1em}
         \setlength\itemsep{0em}
        \centering
\begin{tabular}{l l}
    $\tau$ & the generator of $\Gal(\bQ(\sqrt{D}) / \bQ)$, \\
    $k$ & the smallest integer such that $\bQ(\sqrt{D}) \subset \bQ(\zeta_k)$. Then $k \mid |G|$, hence is odd.
\end{tabular}
\vspace{-1em}
\end{table}
Fix $\Theta = \sum_i n_i H_i \in B(G)$ with $\bC[\Theta] \simeq \widetilde{\rho}^{\oplus m}$. We prove that at each prime $p$, $\fc_p(\Theta)$ and $d_p(\Theta)$ are the norms of elements from $\bQ(\rho)^{\times}$.  These depends on $D_p$ and $I_p$ ; the decomposition and inertia group respectively at $p$. As we deal with each local factor individually, we argue that one can take $D_p = I_p$.

\begin{lemma}\label{tam-up-to-square}
    Let $E / K$ be an elliptic curve. Let $K' / K$ be an extension of number fields odd degree, unramified at the place $v$ of $K$. Then $C_w(E / K') \equiv C_v(E / K) \mod \bQ^{\times 2}$ for any place $w$ of $K'$ with $ w \mid v$. 
\end{lemma}

\begin{proof}
This is automatic for good reduction and split multiplicative reduction. It is also clear for non-split multiplicative reduction since the residue degree cannot be even (so the reduction type remains non-split at $w$). For additive reduction, see \cite[Lemma 3.12]{reg-const}.
\end{proof}

\begin{lemma}\label{DeqI}
    At a prime $p$, we may assume that $D_p = I_p$ when computing $(\fc_p \cdot d_p)(\Theta)$. 
\end{lemma}

\begin{proof}
Let $p$ have residue degree $f_p$. Let $L / \bQ$ be a Galois extension of degree $f_p$ with cyclic Galois group, such that $p$ is inert in $L$. Further ensure that $F \cap L = \bQ$. Then $\Gal(FL / L) = G$. Let $F_i = F^{H_i}$ and $L_i = F_i L$.

Let $v$ be a place over $p$ in $F_i$. The extension $L_i / F_i$ is Galois, so $v$ is either split or inert in $L_i$.
We claim that $C_v(E / F_i) \equiv \prod_{w | v} C_w(E / L_i) \mod \bQ^{\times 2}$. Indeed, the number of terms in the product on the right is odd, and by lemma \ref{tam-up-to-square} $C_v(E / F_i) \equiv C_w(E / L_i) \mod \bQ^{\times 2}$. 
Letting $\fc_p'$ and $d_p'$ be functions on $B(G)$ defined as in (\ref{c-and-d}) but with $\bQ$, $F$ replaced by $L$, $FL$, we see that $(\fc_p \cdot d_p)(\Theta) \equiv (\fc_p' \cdot d_p')(\Theta) \mod \bQ^{\times 2}$. 
Thus it is equivalent to do our computation in $FL / L$, but here $p$ has residue degree $1$.
\end{proof}

We also show that if $\bQ(\Res_{D_p} \rho) = \bQ$, then $(\fc_p \cdot d_p)(\Theta) \in \bQ^{\times}$.

\begin{lemma}\label{rational-res}
    Let the exponent of $D_p$ be $b$. If $k \nmid b$, then $(\fc_p \cdot d_p)(\Theta) \in \bQ^{\times}$. 
\end{lemma}

\begin{proof}
    Note that $\bQ(\Res_{D_p}{\rho}) \subset \bQ(\zeta_b) \cap \bQ(\rho) \subset \bQ(\zeta_b)$. Then $\bQ(\rho) \subset \bQ(\zeta_b) \implies k \mid b$ by minimality of $k$. Since $k \nmid b$, we have $\bQ(\rho) \not\subset \bQ(\zeta_b)$, so $\bQ(\Res_{D_p} \rho) = \bQ$ and $\Res_{D_p} \rho = \tau (\Res_{D_p} \rho)$. 
    
    Now $\bC[\Res_{D_p} \Theta] \simeq (\Res_{D_p} \rho)^{\oplus 2 m} \in \Perm(D_p)$. Since $C(D_p) = \Char_{\bQ}(D_p) / \Perm(D_p)$ has odd order, it follows that $(\Res_{D_p} \rho)^{\oplus m} \in \Perm(D_p)$.  
    Therefore there is $\Theta' \in B(D_p)$ such that $\bC[\Theta'] \simeq (\Res_{D_p} \rho)^{\oplus m}$. Then $\Psi = (\Res_{D_p}\Theta) - 2 \Theta'$ is a Brauer relation for $D_p$. Then 
    \[(\fc_p \cdot d_p)(\Theta) \ {\ov{(\ref{Dp-loc})}{=}} \ f_p(\Res_{D_p}\Theta)
    = f_p(\Psi) \cdot f_p(\Theta')^2 \in \bQ^{\times 2}. \]
    Again we are using \cite[Theorem 2.47]{reg-const} and \cite[Theorem 3.2]{reg-const} which imply that $f_p(\Psi) \in \bQ^{\times 2}$ when $\Psi$ is a Brauer relation for $D_p$ (since $D_p$ is odd).   
\end{proof}


To prove theorem \ref{odd-exts}, we proceed by considering separately each reduction type.

\subsubsection*{Good reduction}
If $E / \bQ$ has good reduction at $p$,it has good reduction at all primes lying above $p$ in subfields of $F$. Hence the Tamagawa number is always one, as well as $\left|\omega / \omega_{v}^{\min}\right|_v$ for any place $v \mid p$ in an intermediate field. Therefore $\fc_p$, $d_p = 1$ as functions on $B(G)$.

\subsubsection*{Multiplicative reduction}

If $E / \bQ_p$ has multiplicative reduction, then as in the good reduction case one has $\left|\omega / \omega_{v}^{\min}\right|_v = 1$ for any place $v \mid p$ in an intermediate field. Thus $d_p = 1$.
For $\fc_p$, we consider non-split/split reduction separately.
\vspace{1em}

\noindent\underline{\textit{Non-split multiplicative reduction}}

Let $E / \bQ_p$ have non-split multiplicative reduction. Since $D_p = I_p$, all primes above $p$ have residue degree $1$. Then the reduction at places above $p$ remains non-split in all intermediate subfields.
It follows that 
\[ \fc_p = (D_p, \alpha) \]
where $\alpha$ is the constant function on $B(D_p)$ with $\alpha \in \{1, 2\}$, depending on $\ord_p(\Delta)$ being even or odd. 
We prove a more general lemma that $D_p$-local constant functions are trivial on $\rho$-relations.

\begin{lemma}\label{const-fns}
Let $G$, $\rho$ be as above. The function $(D_p, \alpha)$ for $\alpha \in \bQ^{\times}$ satisfies $(D_p, \alpha)(\Theta) \in \bQ^{\times}$.  
\end{lemma}   

\begin{proof}
    The function $(D_p, \alpha)$ on $B(G)$ sends $H \leq G$ to $\alpha^{| H \backslash G / D_p|}$. Thus if $\Theta = \sum_i n_i H_i$ is a $\rho$-relation, $(D_p, \alpha)(\Theta) = \alpha^{ \sum_i n_i \cdot | H_i \backslash G / D_p|}$. We show that $\sum_i n_i \cdot | H_i \backslash G / D_p|$ is even. 

    %Since $G$ is odd, each $| H_i \backslash G / D_p|$ is odd. On the other hand, $\rho \oplus \tau(\rho)$ has even dimension. Therefore $\sum_i |n_i|$ is even. 
    One has $\Res_{D_p} \Theta = \sum_i n_i \sum_{x \in H_i \backslash G / D_p} D_p \cap H^{x^{-1}}$ and the permutation representation $\bC[\Res_{D_p} \Theta]$ of $D_p$ is isomorphic to $\Res_{D_p} (\rho^{\oplus m} \oplus \tau(\rho^{\oplus m}))$. In particular the dimension is even. The dimension is $$\sum_i n_i \sum_{x \in H_i \backslash G / D_p} [D_p \colon D_p \cap H^{x^{-1}} ].$$ Since each $[D_p \colon D_p \cap H^{x^{-1}} ]$ is odd, this implies there are an even number of terms in the summation, i.e. that $\sum_i n_i \cdot | H_i \backslash G / D_p|$ is even. 
    
\end{proof}

\noindent\underline{\textit{Split multiplicative reduction}}

Now suppose $E / \bQ_p$ has split multiplicative reduction. The reduction type remains split at all places above $p$ within sub-extensions of $F / \bQ$. Let $\ord_p(\Delta) = n$. Then 
\[ \fc_p = (D_p, D_p, en). \]
Since the $n$ factor is constant, $(D_p, D_p, en)(\Theta) \equiv (D_p, D_p, e)(\Theta) \mod \bQ^{\times 2}$ by lemma \ref{const-fns} .

We have $D_p = I_p = P_p \ltimes C_l$, where $P_p \triangleleft I_p$ is wild inertia, and $C_l = I_p / P_p$ is the tame quotient. $C_l$ is a cyclic group, with $l \mid p^f - 1 = p - 1$. By lemma \ref{rational-res}, it is only of interest to consider such $D_p$ with exponent  $p^u l$ for some $u \geq 0$ such that $k \mid p^u l$.

Now, $(D_p, I_p, e)(\Theta)$ is the product of ramification indices at primes above $p$. We separate the $p$-part and tame part of this expression.
Recall that the ramification index of a place $w$ above $p$ corresponding to the double coset $H_i x D_p$ has ramification degree $e_w = \frac{|I_p|}{|H_i \cap I_p^x|} =\frac{|I_p|}{|I_p \cap H^{x^{-1}}|}$.
This is the dimension of the permutation representation $\bC[D_p / D_p \cap H^{x^{-1}}]$.
Let  $D_p \cap H^{x^{-1}} = P' \ltimes C_a$ where $P' \leq P$ and $a | l$. Then the ramification index is $\frac{|P|}{|P'|}\cdot \frac{l}{a}$. 

Taking fixed points under wild inertia, one has $$\bC[D_p / D_p \cap H^{x^{-1}}]^{P_p} \simeq \bC[D_p / P_p (D_p \cap H^{x^{-1}})] \simeq \bC[D_p / P_p \ltimes C_a].$$ This permutation representation has dimension $\frac{l}{a}$, so we've killed off the $p$-part. 
Then $$\bC[\Res_{D_p} \Theta]^{P_p} \simeq \left(\Res_{D_p} \rho^{\oplus m} \oplus \tau\left(\Res_{D_p}\rho^{\oplus m}\right)\right)^{P_p},$$
and we can consider these as representations of $D_p / I_p = C_l$.

Consider the function $g$ on $B(C_l)$ with $H \mapsto [C_l \colon H] = \dim \bC[C_l / H]$. 
It follows that $(D_p, D_p, e)(\Theta)$ differs from $g(P_p \cdot \Res_{D_p}\Theta / P_p)$ up to a factor of $p$. 

%$(C_l, C_l, e)$ evaluated at $P_p \cdot \Res_{D_p}\Theta / P_p$ equals $(D_p, D_p, \psi_p)$ evaluated at $\Res_{D_p} \Theta$ modulo squares up to (possibly) a factor of $p$.

Crucially, this factor of $p$ doesn't matter:

\begin{lemma}
    Let $K = \bQ(\sqrt{D})$ be a quadratic field, contained in the minimal cyclotomic field $\bQ(\zeta_k)$ with $k$ odd. Let $k \mid p^u l $, for some $u \geq 0$ and $l$ such that $p \equiv 1 \pmod l$. Then $p$ is the norm of an element from $K^{\times}$.
\end{lemma}

\begin{proof}
    Since $k$ is odd, it is clear that $D = \prod_{q | k} q^*$, the product being taken over primes dividing $k$. Note that if $q \not= p$, then since $q \mid l$, we have $p \equiv 1 \pmod l \implies p \equiv 1 \pmod q$. By theorem \ref{p-one-mod-disc},  $p$ is the norm of a principal fractional ideal of $K$. If $K$ is imaginary, then $p$ is the norm of an element of $K$. Else, we invoke theorem \ref{p-norm-elem-1} or theorem \ref{p-norm-elem-2}.
    %We show that $p$ has inertial degree $1$ in the extended genus field $E^{+} = K(\{\sqrt{q^*} \colon q | k \})$ of $K$ ({\color{red} cf. appendix}).
    %If $q \not= p$ then $q \mid l$, so $p \equiv 1 \pmod l$. Therefore $p$ splits in any quadratic subfield of $E^{+}$ of discriminant not divisible by $p$. Else, $p$ ramifies in any quadratic subfield with discriminant divisible by $p$. Thus it is clear that $p$ has inertial degree $1$ in $E^{+}$, hence also in the genus field $E$, and it follows from theorem \ref{p-principal} that $p$ is the norm of a principal ideal.  Else, we invoke theorem \ref{minus-one-norm}.
\end{proof}

Thus, we only need to worry about the tame part of our ramification indices. If $k \nmid l$, then $\phi = (\Res_{D_p} \rho)^{P_p}$ (viewed as a representation on $D_p / P_p$) has rational character. Therefore, arguing as in lemma \ref{rational-res}, $g(P_p \cdot \Res_{D_p}\Theta / P_p) \in \bQ^{\times 2}$ {\color{red} say more?}.
Therefore we may assume that $k \mid l$ and that $\bQ(\phi) = \bQ(\rho) = K$.

\begin{prop}
    Let $k \mid l$. Then $g(P_p \cdot \Res_{D_p}\Theta / P_p) \in N_{K / \bQ}(K^{\times})$.   
\end{prop}

\begin{proof}
    Let $\Psi = P_p \cdot \Res_{D_p}\Theta / P_p$, so that $\phi^{\oplus m} \oplus \tau(\phi^{\oplus m}) = \bC[\Psi] = \sum_{l' \mid l}{a_{l'}} \chi_{l'}$ where $a_{l'} \in \bZ$ and $\chi_{l'}$ are defined in example \ref{cyclic-relns}. Let $\Psi_{l'} = \sum_{l'' | l'}\mu(l' / l'')\cdot C_{l / l''}$ so that $\bC[\Psi_{l'}] = \chi_{l'}$, as observed in the example. Then $\bC[\Psi] \simeq \bC[\sum_{l' | l } a_{l'} \Psi_{l'}]$ which implies that $\Psi = \sum_{l' | l } a_{l'} \Psi_{l'}$ since cyclic groups have no Brauer relations.

    Evaluating $g$ on $\Psi_{l'}$ is trivial unless $l' = q^a$ for some $q$ prime, $a \geq 1$. Indeed, if $l' = p_1^{e_1} \cdots p_r^{e_r}$ , with $r \geq 2$ and $e_i \geq 1$, then
    \[ \prod_{l'' \mid l'} (l'')^{\mu(l' / l'')} = \prod_{j_1, \ldots j_r \in \{0,1\}^r } \left(p_1^{e_1 - j_1} \cdots p_r^{e_r - j_r}\right)^{\# j_i = 1} = \prod_{i = 1}^r \left(\frac{p_i^{e_i}}{p_i^{e_i - 1}}\right)^{\sum_{ j = 0}^{r - 1} \binom{r-1}{j} (-1)^j} = 1. \]
    On the other hand,
    \[ \prod_{l' \mid q^a} (l')^{\mu(q^a / l')} = q .\]
    
    We claim that $k \nmid l'$ implies $a_{l'}$ is even. The irreducible representations of $C_l$ over $\bQ(\phi)$ are given by the orbits of the complex irreducible characters of $C_l$ acted upon by $H = \Gal (\bQ(\zeta_l) / \bQ(\phi))$. One has $\chi_{l'} = \widetilde{\varphi_{l'}}$ where $\bQ(\varphi_{l'}) = \bQ(\zeta_{l'})$. If $ k \nmid l'$ then $\bQ(\phi) \not\subset \bQ(\zeta_{l'})$, so that $B = \Gal(\bQ(\zeta_l) / \bQ(\zeta_{l'})) \not\leq H$. Then $\bQ(\phi) \cap \bQ(\zeta_{l'}) = \bQ$ so $BH = \Gal(\bQ(\zeta_l) / \bQ)$. The orbit of $\varphi_{l'}$ under $H$ is fixed by $BH$, hence is rational. It follows that $\langle \phi, \varphi_{l'} \rangle = \langle \tau(\phi) , \varphi_{l'} \rangle$ so that $a_{l'}$ is even.  

    Thus we can only possibly get something interesting if $k = q$ is a prime. But then $q$ is a norm from $\bQ(\sqrt{q^*})$ by corollary \ref{p-principal}. 
\end{proof}

\subsubsection*{Additive reduction}

Now suppose that $E / \bQ_p$ has additive reduction. In this case, assume that $p \geq 5$ is at worst tamely ramified in $F / \bQ$. This ensures that $D_p = I_p = C_l$ is cyclic, and $l \mid p - 1$.

Let $\delta = \ord_p(\Delta_E)$. Consider a place $w$ of $F^H$ over $p$ with ramification degree  $e_w$over $\bQ$. Then $\Delta_E$ has valuation $n e_w$ with respect to $w$. Then $\left| \Delta_E / \Delta_{E, w}^{\min} \right|_w = p^{-(\delta\cdot e_w - \delta_H)}$, where $\delta_H = \ord_w(\Delta_{E, w}^{\min})$.
Recall that
\[ \left| \frac{\omega}{\omega_{w}^{\min}} \right|_{w}^{-12} = \left| \frac{\Delta_E}{\Delta_{E, w}^{\min}} \right|_w .\] 
Therefore $\left| \omega / \omega_{w}^{\min} \right|_w = p^{\floor{\frac{\delta\cdot e_w - \delta_H}{12}}}$.

Suppose that $E / \bQ_p$ has Kodaira type $I_n^*$, so $\delta = 6 + n$. For a tamely ramified finite extension $K' / \bQ_p$ with ramification degree $e$, $E / K'$ has Kodaira type $I_{en}^*$ if $e$ is odd, and type $I_{en}$ if $n$ is even. Thus in odd degree extensions the reduction type will stay potentially multiplicative. Then $\delta \cdot e_w - \delta_H = 6 e_w$.

If $E / \bQ_p$ has potentially good reduction then $\delta \in \{2,3,4,6,8,9,10 \}$. $E$ also has potentially good reduction at the place $w$ in $F^H$. It follows that $\delta_H = \delta \cdot e_w - 12 \cdot \floor{\delta \cdot e_w /12}$.

In conclusion, 
\[ d_p = 
    \begin{cases}
        (D_p, D_p,\ p^{\floor{e /2}}) & \text{if } E \text{ has potentially multiplicative reduction}, \\
        (D_p, D_p,\ p^{\floor{\delta \cdot e_w/12}}) & \text{if } E \text { has potentially good reduction}.
    \end{cases}
    \]

{\color{red} now we say how in any case this $p$ doesn't matter... i should try not assume tame here}

%\[ \left|\frac{\Delta_{E}}{\Delta_{E, w}^\min} \right|_w = p^{f_w 12 \cdot \floor{e_w n / 12}} \implies 
%       \left|\frac{\omega}{\omega_{w}^\min} \right|_w = p \]

For the Tamagawa number computations we consider potentially multiplicative and potentially good separately.


\noindent\underline{\textit{Potentially multiplicative reduction}}

{\color{red} TO DO}
\vspace{10em}


\noindent\underline{\textit{Potentially good reduction}}

\begin{lemma}
    Consider $M / L$ a field extension. Let $E / L$ be an elliptic curve, $v$ a finite place of $L$ and $w$ a finite place of $M$ with $w \mid v$. Let $\omega_v$ and $\omega_w$ be the minimal differentials for $E / L_v$ and $E / M_w$ respectively. 
    
    Then, if $E / K_v$ has potentially good reduction and the residue characteristic is not $3$ or $2$, one has
    
    \[ \left|\frac{\omega_v}{\omega_w} \right|_{w} = q^{\floor{\frac{e_{F / K} \cdot \ord_v(\Delta_{E, v}^{\min})}{12}}}, \]
    where $q$ is the size of the residue field at $w$.
\end{lemma}

We consider $F / \bQ$ with additive potentially good reduction at $p$ . Since $D_p = I_p$, the size of the residue field is $p$ at all intermediate extensions. Let $n = v_p(\Delta)$. Then $n \in \{2,3,4,6,8,9,10\}$.  Consider $(D_p, I_p, \psi_p)$ where $\psi_p(e,f) = p^{\floor{e n / 12}}$. Then $(D_p, I_p, \psi_p) \sim_{\rho} 1$. Indeed, this takes values $1$ or $p$ in $\bQ^{\times} / \bQ^{\times 2}$. But $p \equiv 1 \pmod l$ implies that $p$ is the norm of a principal ideal in $\bQ(\rho)$, and hence the norm of an element, by corollary \ref{p-one-mod-disc} and theorem \ref{minus-one-norm}.

The Tamagawa numbers take a little more work. We use the following description of Tamagawa numbers. %Tim and Vlad reg consts Lemma 3.19

\begin{lemma}
    Let $K' /K / \bQ_p$ be finite extensions and $p \geq 5$. Let $E / K$ be an elliptic curve with additive reduction; 
    \[ E \colon y^2 = x^3 + Ax + B, \]
    with discriminant $\Delta = -16(4 A^3 + 27 B^2)$. Let $\delta = v_K(\Delta)$, and $e = e_{K' / K}$.

    If $E$ has potentially good reduction, then 
        \[
        \begin{array}{l l l l}
            \gcd(\delta e, 12) = 2 & \implies & c_v(E / K') = 1, & \quad (II, II^*) \\
            \gcd(\delta e, 12) = 3 & \implies & c_v(E / K') = 2, & \quad (III, III^*) \\
            \gcd(\delta e, 12) = 4 & \implies & c_v(E / K') = \begin{cases} 1, & \sqrt{B} \notin K'
                                \\ 3, & \sqrt{B} \in K' \end{cases}, & \quad (IV, IV^*) \\
            \gcd(\delta e, 12) = 6 & \implies & c_v(E / K') = \begin{cases} 2, & \sqrt{\Delta} \notin K'
                \\ 1 \ \text{or} \ 4, & \sqrt{\Delta} \in K' \end{cases}, & \quad (I_0^*) \\
            \gcd(\delta e, 12) = 12 & \implies & c_v(E / K') = 1. & \quad (I_0)
        \end{array}
        \]
    Moreover, the extensions $K'(\sqrt{B}) / K'$ and $K'(\sqrt{\Delta}) / K'$ are unramified.
\end{lemma}

So suppose an elliptic curve $E / \bQ$ has additive reduction at $p$, with $p \geq 5$. Then we can write $E \colon y^2 = x^3 + Ax + B$. Let $D = \Gal(F_{\fp} / \bQ_p)$ be the local Galois group at $p$. Assume that $p$ is totally tamely ramified, so that $D = I = C_n$. Since there is no wild ramification, and $f = 1$, this means that $n \mid p - 1$. We consider the contribution corresponding to an irreducible rational character $\chi_d$ of $D$, given by 
\begin{equation}\label{tam-contrib}
\prod_{d ' \mid d} C(E / F_{\fp}^{D_{d'}})^{\mu(d / d')}.
\end{equation}

Observe that in a totally ramified extension of degree coprime to $12$, the Tamagawa number remains the same. If $(12, d) = 1$, then $(12, d') = 1$ for $d' \mid  d$, so the Tamagawa number is constant across subfields $F_{\fp}^{D_{d'}}$. Therefore,
\[\prod_{d ' \mid d} C(E / F_{\fp}^{D_{d'}})^{\mu(d / d')} = C(E / \bQ_p)^{\sum_{d' \mid d} \mu(d / d')} = 1,\]
assuming $d > 1$. 

So we only need to worry about when $3 \mid d$. If we have type $III$ or $III^*$ or $I_0^*$ then the Tamagawa number is still unchanged in any totally ramified cyclic extension of degree dividing $d$. We will treat the other cases separately: 

\vspace{1em}

\noindent\underline{\textit{Type $II$ and $II^*$ reduction:}}

Firstly, suppose that $\delta = 2$, that is we have Type $II$ reduction. If $3 \mid d'$ then $E / F_{\fp}^{D_{d'}}$ has type $I_0^*$ reduction. The Tamagawa number then depends on whether $\sqrt{\Delta} \in \bQ_p$. Since we have additive reduction, we know that $p \mid A$, $p \mid B$. Moreover, $\delta = 2$ implies that $v_p(B) = 1$. Then, $\Delta = p^2\cdot \alpha$, and $\alpha \equiv -27\cdot\square \pmod p$. Therefore $\sqrt{\Delta} \in \bQ_p \iff -3$ is a square $\pmod p$. But this is the case; we assumed $p \equiv 1 \pmod n$, so $p \equiv 1 \pmod 3$. Therefore the Tamagawa number will be $1$ or $4$, which is a square.
If $3 \nmid d'$ then the reduction type over $ F_{\fp}^{D_{d'}}$ is $II$ or $II^*$. Then the Tamagawa number is $1$. Thus in total, we get a square contribution from (\ref{tam-contrib}).

If $\delta = 10$, then $E / F_{\fp}^{D_{d'}}$ has reduction type $I_0^*$ whenever $3 \mid d'$. Once more, $v_p(A), v_p(B) \geq 1$, and $v_p(\Delta) = 10 = \min(3 v_p(A), 2 v_p(B))$ {\color{red} maybe this is suss} $\implies v_p(B) = 5$. Therefore we get $\Delta = p^{10} \alpha$ with $\alpha \equiv -27\cdot\square \pmod p$, and we conclude as above.

\vspace{1em}

\noindent\underline{\textit{Type $IV$ and $IV^*$ reduction:}}

Now, if $E /\bQ_p$ has additive reduction of type $IV$ or $IV^*$, it attains good reduction over any totally ramified cyclic extension of degree divisible by $3$. This could result with $3$ coming up an odd number of times in our Tamagawa number product, when $\sqrt{B} \not\in \bQ_p$. 

%We show that for both types, one has $\sqrt{B} \in \bQ_p$. 
%Indeed, if $\delta = 4$, then $v_p(B) = 2$, and $v_p(A) \geq 2$. 
\vspace{1em}
In summary, 
\begin{equation}
    \prod_{d ' \mid d} C(E / F_{\fp}^{D_{d'}})^{\mu(d / d')}
    = 
    \begin{cases}
        1 & 3 \nmid d, \\
        1 & 3 \mid d, \delta \in \{0, 3, 6, 9\}, \\
        1 \cdot \square & 3 \mid d, \delta \in \{2, 10\}, \\
        3^a \cdot\square, a \in \{0,1\} & 3 \mid d, \delta \in \{4,8\}.
    \end{cases}
\end{equation}

\begin{rem}
   There's no reason why we can't get 3; see elliptic curve 441b1 with additive reduction at $7$ of type IV and Tamagawa number equal to $3$
\end{rem}

However, it turns out we will only get $3$ occurring oddly when $d = 3$. Indeed, one has that $\langle \Ind_{D_{d'}}^D \trivial, \psi_3 \rangle = 1$ if $3 \mid d'$, and $0$ if $3 \nmid d'$, where $\psi_3$ is an irreducible character of $D$ of order $3$. Therefore one sees that the number of places with ramification degree divisible by $3$ cancels unless $d = 3$. Indeed, $\langle \chi_d , \psi_3 \rangle = 0$ unless $d = 3$, 
in which case it is $1$. Therefore (\ref{tam-contrib}) can only be non-square when $d = 3$. {\color{red} then conclude why this is fine}
