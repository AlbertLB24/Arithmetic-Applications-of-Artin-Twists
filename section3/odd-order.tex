\subsection{Compatibility in odd order extensions}

In this section we work towards proving the following:

\begin{thm}
    Let $F / \bQ$ be a Galois extension of odd degree, with $G = \Gal(F / \bQ)$. 
    Consider a semistable elliptic curve $E / \bQ$ with good reduction at primes that are wildly ramified in $F / \bQ$.
    
    Then, for any $\rho \in R_{\bC}(G)$ with $[\bQ(\rho)\colon \bQ] > 1$, the function $f \colon B(G) \to
    \bQ^{\times} / N_{\bQ(\rho) / \bQ}(\bQ(\rho)^{\times})$ sending $H \mapsto C(E / F^{H})$ satisfies $f \sim_{\rho} 1$.
\end{thm}

We look at a relation of the form
\begin{equation}\label{reln} 
    \sum_i n_i \Ind_{H_i}^G \trivial \simeq \rho \oplus \tau(\rho),
\end{equation}
where $\rho$ is a character of $G$ with $\Gal(\bQ(\rho) / \bQ) = \langle \tau \rangle $ of size $2$. In particular we let $m$ denote the minimal positive integer such that $\bQ(\rho) \subset \bQ(\zeta_m)$. The sum on the left is over subgroups $H_i \subseteq G$. 

If I consider $\Res_D(\rho)$ where $D$ is a decomposition group of exponent $k$, then for $\Res_D(\rho)$ to be non-rationally valued, one needs $m |k$. Note that in the context of norm relations, if $\Res_D(\rho) = \Res_D(\tau(\rho))$, then we always get squares.

So now suppose that $D = I = C_n$ with $m | n$. Applying $\Res_D$ to (\ref{reln}), we get
\begin{equation}\label{reln-resD}
\sum_i n_i \sum_{x \in H_i \ G / D} \Ind_{D \cap x^{-1}H_i x}^D \trivial \simeq \Res_D \rho \oplus \tau(\Res_D \rho).
\end{equation}

Since both sides are now rationally valued, we can write this as $\sum_{d |n} a_d \cdot \chi_d$ where $a_d \in \bZ$ and $\{ \chi_d \colon d|n \}$ form a basis for the irreducible rational-valued representations of $D$. Explicitly, $\chi_d$ is the sum of the Galois conjugates of an irreducible complex character of $D$ with field of values $\bQ(\zeta_d)$ and kernel of index $d$ (which we'll write as $D_d$ ).

We can write each $\chi_d$ in terms of permutation modules:
\begin{equation}\label{chi-d} \chi_d = \sum_{d' | d} \mu(d' / d)\Ind_{D_d'}^D \trivial .\end{equation}
Substituting this into  $\sum_{d |n} a_d \cdot \chi_d$ gives an expression for the LHS of (\ref{reln}).
In particular, if we have a $D$-local function, we can evaluate it on each $\chi_d$-relation as in (\ref{chi-d}). Then the total expression is the product of these, raised to $a_d$.
