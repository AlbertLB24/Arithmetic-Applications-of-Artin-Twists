\subsection{Compatibility in odd order extensions}

Consider $G = \Gal(F / \bQ)$ with odd order, and $E / \bQ$ an elliptic curve with good reduction at wildly ramified primes in $F / \bQ$. Consider a relation of the form
\begin{equation}\label{reln} 
    \Theta = \sum_i n_i \Ind_{H_i}^G \trivial \simeq \rho \oplus \tau(\rho),
\end{equation}
where $\rho$ is a character of $G$ with $\Gal(\bQ(\rho) / \bQ) = \langle \tau \rangle $ of size $2$. Let $m$ denote the minimal positive integer such that $\bQ(\rho) \subset \bQ(\zeta_m)$. The sum on the left is over subgroups $H_i \leq G$. 

Observe that $\Res_D\rho$, where $D$ is a decomposition group of exponent $n$, is rationally-valued when $m \nmid n$. In the context of norm relations, if $\Res_D\rho = \Res_D\tau(\rho)$, then we always get squares. Thus the interesting case is when $m \mid n$. 

We understand $\Res_D$ best when $D$ is cyclic. Let $D = C_n$ with $m \mid n$. Applying $\Res_D$ to (\ref{reln}), we get
\begin{equation}\label{reln-resD}
\sum_i n_i \sum_{x \in H_i \backslash G / D} \Ind_{D \cap x^{-1}H_i x}^D \trivial \simeq \Res_D \rho \oplus \tau(\Res_D \rho).
\end{equation}

Since both sides are now rationally valued, we can write this as $\sum_{d \mid n} a_d \cdot \chi_d$ where $a_d \in \bZ$ and $\chi_d$ are defined in Example \ref{cyclic-relns}. Writing each $\chi_d$ in terms of permutation representations as in the Example, one obtains an expression for the LHS of (\ref{reln}) (since cyclic groups have no Brauer relations, this is on the nose).
Therefore, if $\psi(\Theta) = \prod_p \psi(\Res_{D_p} \Theta)$, we have $\psi(\Res_{D_p} \Theta) = \prod_{d \mid n} \psi(\chi_d)^{a_d}$ whenever $D_p = C_n$.

As such, we'd like to be able to reduce to cyclic decomposotion groups. As we only assume bad reduction at tamely ramified primes in $F / \bQ$, one has that $I$ is cyclic. 
It turns out that we may assume that $D = I$ when $[D \colon I]$ is odd. 

\begin{lemma}\label{DeqI}
    In an odd degree unramified extension, Tamagawa numbers change only up to squares. In particular, if $[D \colon I]$ is odd, then $(D, I, \psi) \sim_{\rho} (I, I, \psi)$ for any $\rho$ with $\bQ(\rho)$ even, where $\psi(e,f)$ is the Tamagawa number. 
\end{lemma}

\begin{proof}
    Yadada
\end{proof}

\subsubsection{Semistable reduction}

In this subsection we work towards proving the following:

\begin{thm}
    Let $F / \bQ$ be a Galois extension of odd degree, with $G = \Gal(F / \bQ)$. 
    Consider a semistable elliptic curve $E / \bQ$ with good reduction at primes that are wildly ramified in $F / \bQ$.
    
    Then, for any $\rho \in R_{\bC}(G)$ with $[\bQ(\rho)\colon \bQ] > 1$, the function $f \colon B(G) \to
    \bQ^{\times} / N_{\bQ(\rho) / \bQ}(\bQ(\rho)^{\times})$ sending $H \mapsto C(E / F^{H})$ satisfies $f \sim_{\rho} 1$.
\end{thm}

\subsubsection{dv terms in additive reduction}


\subsubsection{Tamagawa numbers in additive reduction}

We use the following description of Tamagawa numbers. %Tim and Vlad reg consts Lemma 3.19

\begin{lemma}
    Let $K' /K / \bQ_p$ be finite extensions and $p \geq 5$. Let $E / K$ be an elliptic curve with addtive reduction; 
    \[ E \colon y^2 = x^3 + Ax + B, \]
    with discriminant $\Delta = -16(4 A^3 + 27 B^2)$. Let $\delta = v_K(\Delta)$, and $e = e_{K' / K}$.

    If $E$ has potentially good reduction, then 
        \[
        \begin{array}{l l l l}
            \gcd(\delta e, 12) = 2 & \implies & c_v(E / K') = 1, & \quad (II, II^*) \\
            \gcd(\delta e, 12) = 3 & \implies & c_v(E / K') = 2, & \quad (III, III^*) \\
            \gcd(\delta e, 12) = 4 & \implies & c_v(E / K') = \begin{cases} 1, & \sqrt{B} \notin K'
                                \\ 3, & \sqrt{B} \in K' \end{cases}, & \quad (IV, IV^*) \\
            \gcd(\delta e, 12) = 6 & \implies & c_v(E / K') = \begin{cases} 2, & \sqrt{\Delta} \notin K'
                \\ 1 \ \text{or} \ 4, & \sqrt{\Delta} \in K' \end{cases}, & \quad (I_0^*) \\
            \gcd(\delta e, 12) = 12 & \implies & c_v(E / K') = 1. & \quad (I_0)
        \end{array}
        \]
    Moreover, the extensions $K'(\sqrt{B}) / K'$ and $K'(\sqrt{\Delta}) / K'$ are unramified.
\end{lemma}

So suppose an elliptic curve $E / \bQ$ has additive reduction at $p$, with $p \geq 5$. Then we can write $E \colon y^2 = x^3 + Ax + B$. Let $D = \Gal(F_{\fp} / \bQ_p)$ be the local Galois group at $p$. Assume that $p$ is totally tamely ramified, so that $D = I = C_n$. Since there is no wild ramification, and $f = 1$, this means that $n \mid p - 1$. We consider the contribution corresponding to an irreducible rational character $\chi_d$ of $D$, given by 
\begin{equation}\label{tam-contrib}
\prod_{d ' \mid d} C(E / F_{\fp}^{D_{d'}})^{\mu(d / d')}.
\end{equation}

Observe that in a totally ramified extension of degree coprime to $12$, the Tamagawa number remains the same. If $(12, d) = 1$, then $(12, d') = 1$ for $d' \mid  d$, so the Tamagawa number is consant accross subfields $F_{\fp}^{D_{d'}}$. Therefore,
\[\prod_{d ' \mid d} C(E / F_{\fp}^{D_{d'}})^{\mu(d / d')} = C(E / \bQ_p)^{\sum_{d' \mid d} \mu(d / d')} = 1,\]
assuming $d > 1$. 

So we only need to worry about when $3 \mid d$. If we have type $III$ or $III^*$ or $I_0^*$ then the Tamagawa number is still unchanged in any totally ramified cyclic extension of degree dividing $d$. We will treat the other cases seperately: 

\vspace{1em}

\noindent\underline{\textit{Type $II$ and $II^*$ reduction:}}

Firstly, suppose that $\delta = 2$, that is we have Type $II$ reduction. If $3 \mid d'$ then $E / F_{\fp}^{D_{d'}}$ has type $I_0^*$ reduction. The Tamagawa number then depends on whether $\sqrt{\Delta} \in \bQ_p$. Since we have additive reduction, we know that $p \mid A$, $p \mid B$. Moreover, $\delta = 2$ implies that $v_p(B) = 1$. Then, $\Delta = p^2\cdot \alpha$, and $\alpha \equiv -27\cdot\square \pmod p$. Therefore $\sqrt{\Delta} \in \bQ_p \iff -3$ is a square $\pmod p$. But this is the case; we assumed $p \equiv 1 \pmod n$, so $p \equiv 1 \pmod 3$. Therefore the Tamagawa number will be $1$ or $4$, which is a square.
If $3 \nmid d'$ then the reduction type over $ F_{\fp}^{D_{d'}}$ is $II$ or $II^*$. Then the Tamagawa number is $1$. Thus in total, we get a square contribution from (\ref{tam-contrib}).

If $\delta = 10$, then $E / F_{\fp}^{D_{d'}}$ has reduction type $I_0^*$ whenever $3 \mid d'$. Once more, $v_p(A), v_p(B) \geq 1$, and $v_p(\Delta) = 10 = \min(3 v_p(A), 2 v_p(B))$ {\color{red} maybe this is suss} $\implies v_p(B) = 5$. Therefore we get $\Delta = p^{10} \alpha$ with $\alpha \equiv -27\cdot\square \pmod p$, and we conclude as above.

\vspace{1em}

\noindent\underline{\textit{Type $IV$ and $IV^*$ reduction:}}

Now, if $E /\bQ_p$ has additive reduction of type $IV$ or $IV^*$, it attains good reduction over any totally ramified cyclic extension of degree divisible by $3$. This could result with $3$ coming up an odd number of times in our Tamagawa number product, when $\sqrt{B} \not\in \bQ_p$. 

%We show that for both types, one has $\sqrt{B} \in \bQ_p$. 
%Indeed, if $\delta = 4$, then $v_p(B) = 2$, and $v_p(A) \geq 2$. 
\vspace{1em}
In summary, 
\begin{equation}
    \prod_{d ' \mid d} C(E / F_{\fp}^{D_{d'}})^{\mu(d / d')}
    = 
    \begin{cases}
        1 & 3 \nmid d, \\
        1 & 3 \mid d, \delta \in \{0, 3, 6, 9\}, \\
        1 \cdot \square & 3 \mid d, \delta \in \{2, 10\}, \\
        3^a \cdot\square, a \in \{0,1\} & 3 \mid d, \delta \in \{4,8\}.
    \end{cases}
\end{equation}

\begin{rem}
   There's no reason why we can't get 3; see elliptic curve 441b1 with additive reduction at $7$ of type IV and Tamagawa number equal to $3$) 
\end{rem}


However, it turns out we will only get $3$ occuring oddly when $d = 3$. Indeed, one has that $\langle \Ind_{D_{d'}}^D \trivial, \psi_3 \rangle = 1$ if $3 \mid d'$, and $0$ if $3 \nmid d'$, where $\psi_3$ is an irreducible character of $D$ of order $3$. Therefore one sees that the number of places with ramification degree divisible by $3$ cancels unless $d = 3$. Indeed, $\langle \chi_d , \psi_3 \rangle = 0$ unless $d = 3$, 
in which case it is $1$. Therefore (\ref{tam-contrib}) can only be non-square when $d = 3$.
