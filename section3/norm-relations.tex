Recall that in Section \ref{rep}, we associated to $\rho \in R_{\bC}(G)$ the character
    \[ \widetilde{\rho} = \sum_{\sigma \in \Gal(\bQ(\rho) / \bQ)} m(\rho) \rho^{\sigma} \quad \in R_{\bQ}(G) .\]
 Call $\Theta = \sum_i n_i H_i \in B(G)$ a \textbf{$\rho$-relation} if 
    \[ \sum_i n_i \Ind_{H_i}^G \trivial \simeq \widetilde{\rho}. \]
Given such a $\rho$, consider functions $\psi \colon B(G) \to \bQ^{\times} / N_{\bQ(\rho) / \bQ}(\bQ(\rho)^\times)$ (written multiplicatively). We say two functions $\psi$, $\psi'$ are \textbf{$\rho$-equivalent}, written $\psi \sim_{\rho} \psi'$, if $\psi / \psi'$ is trivial on all $\rho$-relations. 

If $\Theta \in \ker \psi$, then $\psi(\Theta)$ is the norm of an element from $\bQ(\rho)^{\times}$. We call an instance of this a \textbf{norm relation}.
In particular, when $\psi \sim_{\rho} 1$, then we obtain a norm relation for all $\rho$-relations $\Theta$. 

\begin{rem}
    If $\rho = 0$ then we call functions $\psi \sim_{\rho} 1$ \textbf{representation theoretic}. These have been studied in {\color{red} cite}.
\end{rem}

\begin{example}
  Take $\rho = 0$, and $V$ a representation of $G$. The function $\psi(H) = \dim V^H$ satisfies $\psi \sim_{\rho} 1$ as $\dim V^H = \langle \Res_{H} V , \trivial_H \rangle = \langle V, \Ind_{H}^G \trivial \rangle$ by Frobenius reciprocity.
\end{example}

\begin{example}
    Let $G = C_p$ for $p$ a prime. Let $\rho$ be a character of degree $p$. There is a unique $\rho$-relation given by $\Theta = C_1 - C_p$. Let $\psi(H) = [G \colon H]$. Then $\psi(\Theta) = p$, which is a norm from $\bQ(\sqrt{p^*}) \subset \bQ(\zeta_p)$ by Corollary \ref{p-norm}. 
\end{example}

\begin{example}\label{cyclic-relns}
Let $G = C_n$. For each $d \mid n$, let $\chi_d = \widetilde{\varphi_d}$, where $\varphi_d$ is an irreducible complex character of $G$ with field of values $\bQ(\zeta_d)$ and kernel of index $d$.
Then $\{ \chi_d \colon d\mid n \}$ form a basis for the irreducible rational-valued representations of $G$. Note that $\Ind_{C_{n/ d}}^G \trivial$ is the direct sum of irreducible complex representations of $G$ contain $C_{n / d}$ in their kernel. Thus, $\Ind_{C_{n/ d}}^G \trivial \simeq \sum_{d' \mid d} \chi_{d'}$. Applying M\"{o}bius inversion, we obtain the unqiue $\varphi_d$-relation for each $d \mid n$:
\[ \chi_d = \sum_{d' \mid d} \mu(d / d') \cdot \Ind_{C_{n/ d}}^G \trivial. \]
\end{example}

\begin{example}
Let $E / \bQ$ be an elliptic curve, $G = \Gal(F / \bQ)$ for $F / \bQ$ a Galois extension. For $H \leq G$, the function $\psi \colon H \mapsto C(E / F^H)$ extends to a multiplicative function on the Burnside ring. Given a representation $\rho$ of $G$, one can ask when $\psi \sim_{\rho} 1$.
\end{example}

\subsection{D-local functions}\label{D-loc}

{\color{red} Maybe just add in definition of D-local function, and explain all this way better. Maybe also some parts of Theorem 2.36 in the reg consts paper (the parts that translate).}

(This is taken from section 2.3 of Vlad and Tim's regulator constants paper.)

Consider $G = \Gal(F / \bQ)$ and intermediate field $F^H$ for $H < G$. Let $p$ be a prime with decomposition group $D$ in $G$. 
Then the primes above $p$ in $F^H$ correspond to double cosets $H\backslash G/ D$. If a prime $w$ in $F^H$ coresponds to the double coset $HxD$, then its decomposition and inertia groups in $F / F^H$ are $H \cap D^x$ and $H \cap I^x$ respectively. In partiular, the ramification degree and residue degree over $\bQ$ are given by $e_w = \frac{|I|}{|H \cap I^x|}$ and $f_w = \frac{[D \colon I]}{[H \cap D^x \colon H \cap I^x]}$. 

Our fudge factors $C(E / F)$ are defined locally; one has $C(E / F) = \prod_v c_v(E / F) \cdot |\omega / \omega_{v, \min}|$. Here $v$ runs over finite places of $F$, $\omega$ is a global minimal differential for $E / \bQ$, and $\omega_{v, \min}$ is a minimal differential at $v$.
Considering the function $H \mapsto C(E / F^H)$, and writing $C_p(E / F^H) =\prod_{v | p} c_v(E / F)\cdot |\omega / \omega_{v, \min}|$ one has

\[ \sum_{i} n_i H_i \mapsto \prod_i C(E / F^{H_i})^{n_i} = \prod_{p} C_p(E / F^H)^{n_i}. \]
Therefore, our function is the product of local functions for each $p$. Since $C_p(E / F^H)$ depends on $e_w$, $f_w$ for $w | p$, we are motivated to define the following:

\begin{defn}\label{D-I-fn}
    Suppose $I \triangleleft D < G$ with $D / I$ cyclic, and $\psi(e,f)$ is a function of $e, f \in \bN$. Define
    \[ \left(D, I, \psi\right) \colon \quad H \mapsto \prod_{x \in H\backslash G / D} \psi\left(\frac{|I|}{|H \cap I^x|}, \frac{[D \colon I]}{[H \cap D^x \colon H \cap I^x]}\right). \]
    Then, this is a function on the Burnside ring.
\end{defn}

\begin{example}
For semi-stable reduction, we're considering $\psi(e, f) = e$ (the Tamagawa number). For the $d_v$ terms in the case of additive potentially good reduction at p ($p$ not equal to $2$ or $3$), we consider $\psi(e, f) = p^{f \floor{e n /12}}$, where $n \in \{2,3,4,6,9,10\}$.
\end{example}



