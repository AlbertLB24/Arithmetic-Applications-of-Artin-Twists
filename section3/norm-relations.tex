\subsection{Norm relations}

Recall that in Section \ref{rep}, we associated to $\rho \in R_{\bC}(G)$ the character
    \[ \tilde{\rho} = \sum_{\sigma \in \Gal(\bQ(\rho) / \bQ)} m(\rho) \rho^{\sigma} .\]
 We call $\sum_i n_i H_i \in B(G)$ a \textbf{$\rho$-relation} if 
    \[ \sum_i n_i \Ind_{H_i}^G \trivial \simeq \tilde{\rho}. \]
Given such a $\rho$, consider functions $\psi \colon B(G) \to \bQ^{\times} / N_{\bQ(\rho) / \bQ}(\bQ(\rho)^\times)$ (written multiplicatively). We say two functions $\psi$, $\psi'$ are equivalent, written $\psi \sim_{\rho} \psi'$, if $\psi / \psi'$ is trivial on all $\rho$-relations. 

\begin{rem}
    If $\rho = 0$ then we call functions $\psi \sim_{\rho} 1$ \textbf{representation theoretic}. These have been studied in {\color{red} cite}.
\end{rem}


\begin{example}
    Consider $G = C_2 \times C_2$.
\end{example}


\subsubsection{D-local functions}\label{D-loc}

This is taken from section 2.3 of Vlad and Tim's regulator constants paper.

Consider $G = \Gal(F / \bQ)$ and intermediate field $F^H$ for $H < G$. Let $p$ be a prime with decomposition group $D$ in $G$. 
Then the primes above $p$ in $F^H$ correspond to double cosets $H\ G/ D$. If a prime $w$ in $F^H$ coresponds to the double coset $HxD$, then its decomposition and inertia groups in $F / F^H$ are $H \cap D^x$ and $H \cap I^x$ respectively. In partiular, the ramification degree and residue degree over $\bQ$ are given by $e_w = \frac{|I|}{|H \cap I^x|}$ and $f_w = \frac{[D \colon I]}{[H \cap D^x \colon H \cap I^x]}$. 

Since we consider many local functions which depend on $e$ and $f$, we introduce the following definition:

\begin{defn}
    Suppose $I \triangleleft D < G$ with $D / I$ cyclic, and $\psi(e,f)$ is a function of $e, f \in \bN$. Define
    \[ \left(D, I, \psi\right) \colon \quad H \mapsto \prod_{x \in H\backslash G / D} \psi\left(\frac{|I|}{|H \cap I^x|}, \frac{[D \colon I]}{[H \cap D^x \colon H \cap I^x]}\right). \]
    Then, this is a function on the Burnside ring.
\end{defn}

For example, for semi-stable reduction, we're considering $\psi(e, f) = e$ (the Tamagawa number). For the $d_v$ terms in the case of additive potentially good reduction at p ($p$ not equal to $2$ or $3$), we consider $\psi(e, f) = p^{f \floor{e n /12}}$, where $n \in \{2,3,4,6,9,10\}$.




